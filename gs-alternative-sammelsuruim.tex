%% -*- coding:utf-8 -*-
\chapter{Alternatives}
\label{chap-alternatives}%





\section{Binary branching structures and linearization domains}

Works by \citet{Reape94a} paved the way for linearization grammars that assume binary-branching
structures with flat linearization domains. A verb and its arguments are located in the same linearization
domain and can -- despite not belonging to the same local tree -- be ordered according to linearization 
restrictions. These kinds of models have been used by \citet{Kathol95a,Kathol2000a} and myself \citep{Mueller95c,Mueller99a,Mueller2002b}.
For a discussion of the advantages of linearization grammars from a computational linguistic perspective, see \citew{Mueller2004b}.

These analyses face the same drawbacks as those discussed in the previous section:
it is difficult to motivate the fact that multiple fronted constituents form a single 
constituent. \citet{Kathol97a} argues for a non"=abstract syntax as it would otherwise
not be possible to explain how children acquire language. He directly links the linear order of
constituents to clause type. If one wishes to avoid null elements, this sort of approach is doomed
to fail as there are topic drop constructions in which the initial position is not filled but the
sentences nevertheless have the same clause type as cannonical V2 clauses.  A  grammatical model which focuses exclusively 
on surface order can deal with neither these facts nor the data from cases of supposed multiple fronting
\citep{Mueller2004e}. If, then, it is  not possible to develop a grammar which complies with Kathol's concepts, which
still adequately explains the data, one cannot simply use considerations of learnability to argue in
their favour. Kathol's surface"=based syntax will be discussed in more detail in Chapter~\ref{chap-empty}.

Furthermore, linearization analyses with binary-branching structures have the disadvantage of
not being able to explain why dative objects as well as accusative objects can occur with the verb
in the prefield. The examples in (\mex{1}) show that there are different fronting possibilities with the same verb 
depending on how one fills the argument slots:\todostefan{more detailed explanation}
\eal
\label{bsp-acc-dat-pvp}
\ex 
\gll Den Wählern erzählen sollte man diese Geschichte nicht.\\
	 the voters tell should one this story not\\
\glt `One shouldn't tell this story to the voters.'
\ex 
\gll Märchen erzählen sollte man den Wählern nicht.\\
	 fairy.tales tell should one the voters not\\
\glt `One shouldn't tell the voters fairy tales.'
\zl

In linearization grammars, one has to combine the arguments of a head in a fixed sequence as the combination
order is independent from the surface order. If one were to allow arbitrary combination orders, one would yield
false ambiguities. If one assumes a \subcatl of the form $\langle$ NP[{\it nom}], NP[{\it acc}],
NP[{\it dat}] $\rangle$ for \emph{erzählen} `tell', then it is only possible to analyse (\mex{0}a).
(\mex{0}b) cannot be analyzed since \emph{Märchen} `fairy tale' can only be combined with \emph{erzählen} after it has 
been combined with the dative object. 


In order to solve this problem, \citet[\page 242]{Kathol2000a} suggests not specifying a 
fixed order for these objects in the \subcatl. The sentences (\mex{0}) would be able to
be analyzed, on the other hand a sentence such as (\mex{1}) would have two possible analyses:
\ea
\gll dass er den Wählern Märchen erzählt\\
	 that he the voters fairy.tales told\\
\glt `that he told the voters fairy tales/tall stories'
\z
As the order of constituents in the clause is independent from the elements in \subcatl in linearization
grammars, it is possible to derive every possible surface order for every possible order in \subcatl
and thereby generating unwanted spurious ambiguities.

The data in (\ref{bsp-acc-dat-pvp}) are unproblematic for the analysis presented here as the head-argument schema
on page \,\pageref{schema-bin} allows the combination of arguments and their heads in any order.

Does this mean that we should abandon linearization grammars for more restrictive grammar models? Before jumping to
this conclusion, one has to demonstrate how we can deal with phenomena which have been analyzed using discontinuous
constituents. An example of this is so-called `intermediate placement', where a verb is ordered in the middle of a
government chain (\mex{1}) (see (\citealp[\page 182]{dBE83a}\ia{den Besten}\ia{Edmondson};
Meurers, \citeyear[Chapter~3.2.2]{Meurers97a}; \citeyear[Chapter~3.2.1.3]{Meurers2000b})).
\ea
\gll daß er das Examen bestehen wird können.\\
     that he the exam pass  will be.able\\
\glt `that he will be able to pass the exam'
\z
An analysis of this phenomenon that makes do without discontinuous constituents was suggested by \citet{Meurers2000b}.

Discontinuous constituents have also been assumed by \citew{Mueller2002b} for a co-indexation analysis of
depictive predicates. If we assume that \emph{nackt} `naked' and \emph{hilft} `helps' form a discontinuous
constituent, then the subject of \emph{nackt} can be co-indexed with the subject of \emph{hilft} in the argument
structure.
\ea
\gll weil er nackt der Frau hilft\\
	 because he naked the woman helps\\
\glt `because he is helping the woman (while) naked'
\z



As pointed out by \citet[\page 87--88]{Kaufmann95a}, examples such as (\mex{0}) are already problematic if one assumes that
the subject of a depictive predicate is co-indexed with some element from the argument structure of the verb and furthermore
that information about argument structure is only available in lexical elements. These data can however be satisfactorily 
explained if we assume that the subject of a depictive predicate is co-indexed with an element from the \subcatl (of a projection)
of the head. It follows on from this analysis that depictive predicates can only refer to elements to the left of themselves
\citep[Chapter~4.1.4.1]{Mueller2002b}. The elements to the right of a depictive predicate are of course already combined with
the head and therefore no longer present in \subcatl of the projection in question.
Apparent exceptions to this sequence are actually instances of I"=topicalization.\footnote{
		Note that analyses which assume a different branching structure for (i) than for (\mex{0}) are
		incompatible with this kind of analysis of depictive predicates since the antecedent for
		\emph{nackt} is not longer present in the \subcatl of \emph{hilft er}.
		\ea
		\gll Hilft er nackt der Frau?\\
		     helps he naked the woman\\
		\glt `Is he helping the woman naked?'
		\z
		For a discussion of these kinds of suggestions see Section~\ref{crysmann}.%
}\todostefan{echt? Nicht arg-st projection? Abbildung}
For the details of this analysis, see \citew{Mueller2004c}.

A further phenomenon which I suggested for the use of discontinuous constituents are 
Werner's dialect data \citep[\page 349, 355]{Werner94a} in (\mex{1}):
\eal
\label{bsp-sonneberg-partikel-modal}
\ex\iw{abplagen}
\gll so ham  sich die Leut  oumüßploug\label{bsp-abmussplagen}\\
%     so have \textsc{refl} the people \textsc{part}.must.plague\\ 
    so have  \textsc{refl} the people \textsc{part}.must.struggle\\
\glt `People had to struggle so much'
\ex\iw{anhören}
\gll Wos  da       sich          ölles       aahotmüßhör!\\
     what this.one \textsc{refl} everything  \textsc{part}.must.listen.to\\
%\glt `What he had to listen to!'
\glt `All these things he had to listen to!'
\ex
\gll wall    e  in  Brander vollstn    ümhotwöllstimm\\
    because he the Brander completely \partic.has.want.to.tune\\
\glt `because he wanted to change Brander's mind completely'
\zl
In these data from the Franconian and Thüringen dialects, the verb particle is left of
the verbal complex, separated from its base verb by superordinate verbs. \citet{BvN98a} have analyzed
parallel data from Dutch using flat structures: all verbs, their arguments and verb particles are 
realised in the same local tree and can therefore be ordered according to linearization rules. In
\citew{Mueller2002b}, I developed an analysis of the data in (\mex{0}), which assumes a binary-branching
structure and allows discontinuous verbal complexes. \emph{Anhören} in (\mex{0}b) is analyzed as a discontinuous 
particle verb.

If we want to avoid positing both completely flat structures and discontinuous constituents,
then we have to resort to an analysis which raises the particle inside the verbal complex.
The difference between Standard German and the Franconian/""Thüringen dialect can be accounted
for under the assumption that verb particles in Standard German always have to be lexical
elements (\textsc{lex}+), whereas they have the status of a maximal projection in the discussed
dialects -- that is, they have to be \textsc{lex}$-$.

There is, therefore, no area of the description of German sentence structure where linearization
grammars are superior to traditional grammars. However, there are some aspects (those discussed above), 
where they are mostly certainly inferior. For this reason, the analysis presented here is preferred
over them. 


\section{To do: Wetta}

\citet{Wetta2011a}


\section{Multiple frontings as multiple frontings}

\citet{Mueller2000d}



\citet{Kathol95a,Kathol97a,Kathol2000a,Kathol2001a} developed a theory of German clause types that
is based on the Topological Fields model known from descriptive linguistics 
\citep{Drach37,Reis80a,Hoehle86,Askedal86}. He suggests relating the clause type of sentences 
to serialization patterns of overtly realized material. \citet{Kathol97a} refutes
CP/IP analyses of German clause structure on the basis of learnability arguments and
argues for a non-abstract syntax, i.e.\ a syntax where surface
order plays a crucial role and the reference to abstract syntactic objects such as
functional heads is avoided in favor of observationally accessible properties (p.\,89).

In this paper, I show that an entirely surface-based conception of syntax is not tenable and
that Kathol's proposal faces problems with certain elliptical constructions.

In the first section, I very briefly repeat his key assumptions. In Section~\ref{sec-problems}, I will
discuss certain declarative sentences that do not fit the
pattern suggested by Kathol. I then suggest an analysis that
does not rely on the surface order of constituents for the classification of clause types,
but on the relations expressed by immediate dominance schemata.

\section{Constructional Constraints and Topological Fields}

The examples in (\mex{1}) show various linearization patterns that are
attested in German clauses:
\eal
\ex
\gll daß Lisa eine Blume gepflanzt hat\\
     that Lisa a flower planted has\\
\glt `that Lisa planted a flower.'

\ex 
\gll was Lisa  gepflanzt hat\\
     what Lisa planted has\\
\ex
\gll Hat Lisa eine Blume gepflanzt?\\
     has Lisa a flower planted\\
\glt `Did Lisa plant a flower?'
\ex 
\gll Eine Blume hat Lisa gepflanzt.\\
     a flower   has Lisa planted\\
\glt `Lisa planted a flower.'
\zl
(\mex{0}a) is an example for sentences that are introduced by a complementizer
and (\mex{0}b) is an example for embedded interrogative sentences. Both sentences
are verb-final. (\mex{0}c--d) are verb-initial sentences: (\mex{0}c) is a yes/no question
and (\mex{0}d) is a declarative sentence. Declarative sentences usually differ from
yes/no questions in that one constituent fills the position before the finite
verb.

\subsection{Topological Fields, Linearization Rules, and Uniqueness Constraints}


\citet[p.\,50]{Kathol2001a} gives the following devision into topological fields
for the sentences in (\mex{1}):
\begin{table*}[htbp]
\resizebox{\linewidth}{!}{
\begin{tabular}{@{}|l|l|l|l|l|@{}}\hline
       & Vorfeld         & linke          & Mittelfeld      & rechte \\
       & `initial field' & Satzklammer    & `middle field'  & Satzklammer\\
       &                 & `left bracket' &                 & `right bracket'\\
       & 1               & 2              & 3               & 4\\\hline
Vfinal &                 & daß            & Lisa eine Blume & gepflanzt hat\\\hline
       &                 & was            & Lisa            & gepflanzt hat\\\hline
V1     &                 & hat            & Lisa eine Blume & gepflanzt\\\hline
V2     & eine Blume      & hat            & Lisa            & gepflanzt\\\hline
\end{tabular}}
\end{table*}

\noindent
This is the classical terminology with additional labels 1--4 to refer to the respective
positions.

He then formulates the following linearization constraints:
\ea
Topological Linear Precedence Constraint\\
1 $<$ 2 $<$ 3 $<$ 4
\z
\ea
Topological Uniqueness Conditions\\
a. 1 $<$ 1\\
b. 2 $<$ 2
\z
The first constraint ensures that all elements that are assigned to the field
1 are serialized before 2 and so on. The second is a trick from the GPSG literature
\citep*[p.\,55]{GKPS85a} to rule out multiple occurrences of elements assigned to the
fields 1 or 2. Since constraint (\mex{0}a) requires that all elements with 
the field 1 have to precede the other elements assigned to field 1 
this constraint is necessarily violated if there is more than one element assigned to 1.

\subsection{A Hierarchy of Clause Types}

Kathol follows \citet{Reape90a,Reape92a,Reape94a}, who introduced linearization
domains into the HPSG framework. Daughters which are combined by the usual dominance
schemata may be non-adjacent. The daughters are inserted into a domain list named \textsc{dom}.
The elements of this list may be permuted in any order provided no LP constraint is violated.
The order of the elements corresponds to the surface order. This makes it possible to
assign both sentences in (\mex{1}) the dominance structure in (\mex{2}).
\eal
\ex der Mann das Buch liest
\ex Liest der Mann das Buch?
\zl
\ea
{}[\sub{V} der Mann [\sub{V} das Buch liest]]
\z
The sentences differ only in the order of the elements in their linearization domain. In
the analysis of (\mex{-1}a) the verb is serialized finally and in the analysis of (\mex{-1}b)
it is serialized initially.

\citet{Kathol2001a} defines clause types with reference to elements in the constituent
order domains. He assumes that all clauses are subtypes of the following three
types:
\eal
\label{clause-types}
\ex\label{v1-clause-type} {\it V1-clause\/} $\to$ \ms{
                                                     S[\textit{fin}]\\
                                                     dom & \liste{ \ms{ \textit{2}\\ 
                                                                        V[{\it fin\/}]\\
                                                                      }, \ldots }
                                                  }

\ex\label{v2-clause-type} {\it V2-clause\/} $\to$ \ms{
                                                     S[\textit{fin}]\\
                                                     dom & \liste{ \ms{  \textit{1} }, 
                                                                   \ms{  \textit{2}\\ 
                                                                         V[{\it fin\/}] }, \ldots }
                                                     }
\ex\label{subord-clause-types} {\it subord-clause\/} $\to$ \ms{
                                                           S[\textit{fin}]\\
                                                           dom & \sliste{ \ldots, \ms{ \textit{2}\\ 
                                                                                       \textsc{head}
                                                                                       $\neg$ V[{\it
                                                                                           fin\/}]\\ },
                                                             \ldots }
                                                           }
\zl
These types impose restrictions on possible orderings of elements in constituent
order domains or stipulate that finite verbs may not appear in the field 2 in subordinated
clauses. The first type states that a verb first clause has a finite verb as the
first element in its domain list and the second states that there is an element in 1 (the \textit{Vorfeld}) 
before the finite verb in 2.

Kathol cross-classifies the types in (\mex{0}) with the types \type{declarative}, \type{wh"=interrogative}, and
\type{polar}. He provides the hierarchy shown in Figure~\vref{fig-clausal-types}.
% For this paper it is sufficient to know that {\it v2-clause} and {\it declarative} have a common
% subtype and that {\it v1-clause\/} and {\it polar\/} have a common subtype. {\it v2-clause} and {\it polar\/}
% and {\it v1-clause\/} and {\it declarative} are incompatible, respectively.


\begin{figure}
\centering
\oneline{%
%% \begin{forest}
%% typehierarchy
%% [\type{finite-clause}
%%   [\type{internal-syntax}
%%     [\type{root}
%%       [\type{v2}
%%         [\type{r-wh-int}, tier=leaf]
%%         [\type{r-decl}, tier=leaf]]
%%       [\type{v1}
%%         [\type{r-pol-int}, tier=leaf]]]
%%     [\type{subord} 
%%       [\type{s-wh-int}, tier=leaf]
%%       [\type{s-pol-int}, tier=leaf]
%%       [\type{s-decl}, tier=leaf]]]
%%   [\type{clausality} 
%%     [\type{inter}
%%       [\type{wh}]
%%       [\type{polar}, tier=polar]]
%%     [\type{decl}, tier=polar]]]
%% \end{forest}
\begin{tabular}{cccccc}
    & & \mynode{fc}{\type{finite-clause}}\\[3ex]
\multicolumn{2}{c}{\mynode{is}{\framebox{\type{internal-syntax}}}} &  & &\mynode{cl}{\framebox{\type{clausality}}}\\[2ex]
\mynode{root}{\type{root}} & \mynode{subord}{\type{subord}} &     & \multicolumn{2}{c}{\mynode{inter}{\type{inter}}}\\[2ex]
\mynode{v2}{\type{v2}}   & \mynode{v1}{\type{v1}}     &     & \mynode{wh}{\type{wh}} & \mynode{polar}{\type{polar}} & \mynode{decl}{\type{decl}}\\[6ex]
\mynode{r-wh-int}{\type{r-wh-int}} & \mynode{r-decl}{\type{r-decl}} & \mynode{r-pol-int}{\type{r-pol-int}} & \mynode{s-wh-int}{\type{s-wh-int}} & \mynode{s-pol-int}{\type{s-pol-int}} & \mynode{s-decl}{\type{s-decl}}\\
\end{tabular}
\begin{tikzpicture}[overlay,remember picture,shorten <=2pt,shorten >=2pt] 
\draw
(fc.south)--(is.north)(fc.south)--(cl.north)
(is.south)--(root.north)(is.south)--(subord.north)
(root.south)--(v1.north)(root.south)--(v2.north)
(v2.south)--(r-wh-int.north)(v2.south)--(r-decl.north)
(v1.south)--(r-pol-int.north)
(cl.south)--(inter.north)(cl.south)--(decl.north)
(inter.south)--(wh.north)(inter.south)--(polar.north)
(wh.south)--(r-wh-int.north)(wh.south)--(s-wh-int.north)
(decl.south)--(r-decl.north)(decl.south)--(s-decl.north)
(polar.south)--(r-pol-int.north)(polar.south)--(s-pol-int.north)
(subord.south)--(s-wh-int.north)(subord.south)--(s-pol-int.north)(subord.south)--(s-decl.north);
\end{tikzpicture}
}
\caption{\label{fig-clausal-types}Clausal Types}
\end{figure}
Such a linearization-based approach to clause type determination would be very attractive
if there were a one-by-one mapping from the surface order of constituents to clause types,
but as I will show in Section~\ref{sec-problems}, this is not the case.


\subsection{Competition of Complementizer and Finite Verb}

Kathol follows ideas by \citet{Thiersch78a} and \citet{denBesten83a} and assumes that
the complementizer and the finite verb compete for the position in the left sentence bracket. 
If no complementizer is present, the verb may move into the left sentence bracket. 
If the left sentence bracket is occupied, it has to stay in the right sentence bracket.

Kathol enters verbs into the lexicon with a specification of the potential topological fields they may appear
in. He specifies finite verbs for the fields 2 or 4. Complementizers are always located in the left sentence bracket:
They are specified for 2.
%Because of the type constraint in (\ref{subord-clause-types}),
%the finite verb cannot appear in 2 in subordinated clauses. 
If a linearization domain contains a complementizer, the Topological Uniqueness Condition b ensures that
no other element can be serialized in 2, hence the field 4 is the only option for the finite verb.

\section{Problematic Aspects of this Approach}
\label{sec-problems}

In the following section I want to discuss four problematic aspects of this proposal.



\subsection{Topic Drop}

While the cases of copula ellipsis can be found in novels, news papers, magazines, and
everyday speech, a construction, which is called {\it Vor\-feld\-ellip\-se\/} or Topic Drop or Pronoun Zap is more
restricted to a certain register/style. 
\citet{Huang84}, \citet{Fries88b}, and \citet{Hoffmann97a}
discuss this construction in some detail.
Topic drop is also problematic for Kathol's approach:
Sentences with Topic Drop look like polar questions at the surface. If an obligatory
complement is dropped, the sentence is distinguishable from questions since the complement is
missing in the \textit{Mittelfeld} (\mex{1}a). If optional complements or adjuncts are dropped, the form of the sentence
is absolutely identical to the form of yes/no questions (\mex{1}b).
\eal
\ex 
\gll Hab' ich auch gekannt.\\
     have I   also known\\
\glt `I also knew him/her/it.'
\ex
\gll Hab' ich auch gegessen.\\
     have I   also eaten\\
\glt `I also eat him/her/it.' or (with different intonation) `Did I also eat?' 
\zl
Such topic drop utterances and polar questions differ
only in intonation and not in the sequence of elements.

In order to save the clause type determination one could assume a phonologically empty element
in the \textit{Vorfeld}. As was discussed in Section~\ref{sec-verbless-clauses},
Kathol explicitly rejects empty elements.

Alternatively one could stipulate just one more type that constrains the domain list
to contain a slashed verb, as was suggested by a reviewer of HPSG 2002. While this is technically
possible, the commonalities of sentences with a filled \emph{Vorfeld} and those that are
the result of Topic Drop would not be captured.


% However, \citet[p.\,{\bf check}]{Kathol95a} explicitly designs the relational constraint
% that computes \textsc{domain} values in a way that excludes phonetically empty material from linearization
% domains. In \citep[p.\,38]{Kathol2001a} he argues against empty elements so that for him the necessity to
% stipulate an empty element seems to be an unwanted consequence of his proposal.



\subsection{Sentential Complements}

\citet[p.\,152]{Kathol2000a} assumes that in (\mex{1}) the V2 clauses in brackets
are complement clauses:
\eal
\ex
\gll Otto glaubt   [die Erde sei flach].\\
     Otto believes ~the earth is flat\\
\glt `Otto believes that the earth is flat.'
\ex\label{noun-scomp}
\gll die Überzeugung /  der Glaube / \ldots{} [die Russen würden nicht in Polen eingreifen]\footnotemark\\
     the conviction  {} the belief {} {}    ~the Russians would not in Poland intervene\\
\footnotetext{
\citew[p.\,287]{Reis85b}.
}
\glt `the conviction/belief/\ldots that the Russians would not intervene in Poland.'
\zl
On page 153 he formulates a Head-V2-Complement Schema that combines a head that takes
a finite unmarked clause as complement with that complement. The schema restricts the
clause type of the complement to be {\it root-decl\/}, i.e., a sentence with the verb
in second position. Kathol's clausal types are subtypes of the type {\it sign\/}. They refer
to the domain values of a sign which are represented at the outermost level of a feature structure
and therefore the clause types could not be subtypes of {\it synsem\/}
or other types inside of the feature structures contained under \textsc{synsem} and hence
the clause type of complements cannot be selected by governing heads. Therefore
Kathol is forced to encode this combinatorial property in the immediate dominance schemata.
In order to avoid spurious ambiguities Kathol has to restrict the general head argument
schema so that it does not apply when the Head-V2-Complement Schema applies.

A grammar that uses sufficient subcategorization information and one head argument schema 
instead of stipulating several special sche\-ma\-ta is more general than what is suggested by Kathol and
should therefore be regarded the better alternative.



\subsection{Multiple Constituents in the \emph{Vorfeld}}

As far as learnability and non-abstractness are concerned the following data
pose a problem for Kathol:\footnote{
  (\mex{1}b--c) are quoted from \citew{Mueller2002c}.%
}
\eal
\ex
\gll [Nichts] [mit  derartigen     Entstehungstheorien] hat es natürlich zu tun, wenn \ldots\footnotemark\\
     ~nothing ~with those.kinds.of creation.theories    has it of.course  to do   when\\
\glt `Of course it has nothing to do with that kind of creation theory when \ldots'%
\label{bsp-nichts-mit-derartigen}
\footnotetext{
        K. Fleischmann, {\em Verbstellung und Relieftheorie\/}, München, 1973, p.\,72.
        quoted from \citep[p.\,135]{vdVelde78a}.
        }
\ex 
\gll [Trocken] [durch   die Stadt] kommt man am     Wochenende auch mit  der BVG.\footnotemark\\
     ~dry      ~through the town   comes one at.the weekend    also with the BVG\\
\footnotetext{
        taz berlin, 10.07.1998, p.\,22
      }\label{bsp-trocken-durch-die-stadt}
\glt `The BVG (Berlin public transport system) will also get you about town on the weekend without getting wet.'%
\ex\label{bsp-alle-traeume}
\gll [Alle Träume] [gleichzeitig]  lassen sich nur  selten verwirklichen.\footnotemark\\
     ~all  dreams   ~simultaneously let   self only rarely realize\\
\footnotetext{
        Brochure from Berliner Sparkasse, 1/1999
        }
\glt `All dreams can seldom be realized at once.'
\zl
These examples seem to violate the V2 constraint. In a purely surface-based model without
any abstract entities, there is no way to explain sentences like (\mex{0}). One could stipulate
constructions that combine the elements before the finite verb so that they form a constituent and
the V2 constraint is saved. However, the data discussed in \citealp{Mueller2003b} shows that various 
combinations of material in the \emph{Vorfeld} are possible. For instance, we have a depictive
secondary predicate and a directional PP argument in (\ref{bsp-trocken-durch-die-stadt}) and
an argument and an adverbial in (\ref{bsp-alle-traeume}).
This means that the stipulation of several
constructions would be necessary in order to provide the correct meaning for the combination
of material infornt of the finite verb. 
%% A construction that assigns constituent status to
%% \emph{Trocken durch die Stadt} and anotherone that assigns constituent status to
%% \emph{Alle Träume gleichzeitig} would be needed.

If one uses one abstract entity, an empty verbal head as suggested
by \citet{Mueller2002c}, a stipulation of several constructions would be unnecessary. The
empty verbal head is related to a verb in the remaining clause by a non-local dependency,
which constraints the elements that can appear together in the Vorfeld and makes possible a compositional
assignment of meaning to the sentence.
\citet{Mueller2002c} uses a linearization-based model of the Reape/""Kathol style
to account for verb-initial and verb-final sentences. In such a model the use of an empty
head is a stipulation. If one returns to a verb movement analysis as suggested 
for instance by \citew{KW91a,Kiss95a} the empty head that is used for verb movement in general
can also be used for the multiple fronting cases in (\mex{0}). The details of the multiple
fronting analysis for (\mex{0}) together with a verb movement analysis can be found in \citew{Mueller2005c,Mueller2005d}.
%
\nocite{Reape90a,Reape94a}



\section{Dependency structures and topological fields}




