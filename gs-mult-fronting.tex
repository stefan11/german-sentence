%% -*- Coding:utf-8 -*-
%%%%%%%%%%%%%%%%%%%%%%%%%%%%%%%%%%%%%%%%%%%%%%%%%%%%%%%%%
%%   $RCSfile: mehr-vf-lb.tex,v $
%%  $Revision: 1.10 $
%%      $Date: 2009/03/27 16:28:44 $
%%     Author: Stefan Mueller (DFKI)
%%    Purpose: 
%%   Language: LaTeX
%%%%%%%%%%%%%%%%%%%%%%%%%%%%%%%%%%%%%%%%%%%%%%%%%%%%%%%%%
%% $Log: mehr-vf-lb.tex,v $
%% Revision 1.10  2009/03/27 16:28:44  stefan
%% *** empty log message ***
%%
%% Revision 1.9  2004/08/25 22:04:17  stefan
%% erste goeteborg-Version
%%
%% Revision 1.8  2004/07/25 16:27:18  stefan
%% removed Corpus-syntax now in corpus-syntax
%%
%% Revision 1.7  2004/07/20 13:08:58  stefan
%% *** empty log message ***
%%
%%
%%
%% Revision 1.1  2004/06/08 16:42:34  stefan
%% gesplittete Version, vor ersten Veränderungen
%%
%%%%%%%%%%%%%%%%%%%%%%%%%%%%%%%%%%%%%%%%%%%%%%%%%%%%%%%%%

\newcommand{\ao}{Avgustinova und Oliva\xspace}%


\chapter{Multiple fronting}
\label{chapter-mult-front}

\footnotetext{
This chapter is based on \citew{Mueller2005d}.
}

In the brief introductory Chapter~\ref{chapter-introduction}, I mentioned that German is a V2
language. This means that declarative sentences and certain interrogative sentences are formed by
placing a constituent in front of the finite verb.
\citet{Thiersch78a}, \citet[\page 55]{denBesten83a}, \citet{Uszkoreit87a}\ia{Uszkoreit}, among others, have suggested that
verb"=second sentences are in fact verb-initial sentences from which one constituent has been extracted and placed in the prefield. 
In the case of (\mex{1}b), it would be \emph{das Buch} which has been extracted from the verb-initial clause.
\eal
\ex 
\gll Kennt er das Buch?\\
	 knows he the book\\
\glt `Does he know the book?'
\ex 
\gll Das Buch kennt er.\\
	 the book knows he\\
\glt `He knows the book.'
\zl
This chapter deals with apparent exceptions to the V2 property of German of the type exemplified in (\mex{1}):
\ea
\gll {}[Trocken] [durch die Stadt] kommt man am Wochenende auch mit der BVG.\footnotemark\\
	 \spacebr{}dry \spacebr{}through the city comes one at.the weekend also with the BVG\\
\footnotetext{
        taz berlin, 10.07.1998, p.\,22.
      }
\glt `With the BVG, you can be sure to get around town dry at the weekend.'
\z
Neither \emph{trocken} `dry' depends on \emph{durch die Stadt} `through the city' nor the other way
round. Rather both constituents depend on \emph{kommt} `comes'.

Viewing fronting as the extraction of one element has become the most established analysis up to now.
Examples in which more than one constituent occupies the prefield have been discussed from time to time in the 
more theoretical literature. To account for these data, certain analyses have been developed where the 
constituents preceding the finite verb are viewed as a single constituent, i.e. it is assumed there is 
only a single constituent in the prefield (\citealp[\page 17]{Haider82}; \citealp[\page 79]{Wunderlich84}; 
Fanselow \citeyear[\page 99--100]{Fanselow87a}; \citeyear[Chapter~3]{Fanselow93a}; \citealp[\page
  1634]{Hoberg97a}; G.\ \citealp[Chapter~5.3]{GMueller98a}).

The exceptions to this are \citet{Grubacic65a}, \citet{Lee75a}, \citet{Loetscher85a}, \citet[\page
  412]{Eisenberg94a}, \citet{Jacobs86a}, \citet{BH2001a}, and \citet{Speyer2008a}. 
\citet{Jacobs86a} and \citet{BH2001a} argue that it is necessary to assume V3 order for sentences with focus particles such as \word{nur}, \word{auch} and \word{sogar} or rather
a special position for focus particles preceding verb-second clauses.
\ea
\label{ex-v3-particles-jacobs}
\gll Nur die Harten kommen in den Garten.\\
     only the hard come into the garden\\
\glt `Only the though ones make it into the garden.'
\z
For a discussion of these suggestions, see \citew{Reis2002a,Reis2005b} and \citew{Mueller2005e}.
Jacobs also assumes that several of the so-called `sentence adverbs' can occur in sentences with V3 constituent order.
He demonstrates this with \emph{leider} `unfortunately' und \emph{vermutlich} `probably' (p.\,107, p.\,112).
The examples which will be discussed in what follows are of a different kind. \citet{Grubacic65a} offers some examples which
I will view as cases of (apparent) multiple fronting. However, some of her examples are also of the same kind as discussed by \citet{Lee75a}.
\eal
\ex 
\gll Piachi, als ihm der Stab gebrochen war, verweigerte sich hartnäckig der Absolution.\footnotemark\\
     Piachi  when him the stick broken was   refused     \self{} persistent the absolution\\
\footnotetext{
Kleist, \emph{Der Findling}, p.\,214.
}
\glt `Piachi persistently refused the absolution, when the stick was broken over him.'
\ex  
\gll Der Junge, sobald er den Alten nur verstanden hatte, nickte und sprach: o ja, sehr gern.\footnotemark\\
     the boy    once   he the old only understood had    nodded and said    o yes very gladly\\
\footnotetext{
Kleist, \emph{Der Findling}, p.\,21 I.
}
\glt `As soon as the boy understood the old man, he nodded and said: O yes, I like to do this very much.'
\ex 
\gll Und damit, ehe ich noch recht begriffen, was sie sagt, auf dem Platz, vor Erstaunen sprachlos, läßt sie mich stehen.\footnotemark\\
     nad there.with before I yet right understood what she said on the place before asonishment
     speechless let she me stand\\
\footnotetext{
Kleist, \emph{Kohlhaas}, p.\,92.
}
\glt `After this she abandons me on the place speechless and before I fully understood what she was saying.'
\zl
I do not consider Lee's examples V3"=clauses in the sense that is relevant here. Some of the
examples are parenthetical insertions and some are of the type that is discussed in
Section~\ref{sec-analyse-mfvorschlaege}. For further discussion of Lee's data, see \citew[\page 33]{Mueller2003b}.



For expository purposes, I will discuss some data in the following section where it seems (at least on the surface) that more
than one constituent precedes the finite verb. Section~\ref{sec-analyse-mf} presents the analysis of
apparent multiple frontings. In Section~\ref{sec-analyse-mfvorschlaege}, I will show that many of the multiple fronting analyses suggested thus far make predictions
that are incompatible with the data in Section~\ref{sec-phenomenon-mult-front} and additional data from German.
In Section~\ref{sec-zusammenfassung}, I draw some conclusion.


\section{The phenomenon}
\label{sec-phenomenon-mult-front}

The assumption that only a single constituent can occur before the finite verb is well established and
descriptively correct for the vast majority of German sentences. In certain circumstances, however, several
constituents, that is, multiple phrases which are not syntactically dependent on each other, can occur there
together. The following sentences are examples of the occurence of different types of constituents in the prefield.
I have ordered the examples according to the type of the fronted elements. The division into constituents is shown
by the corresponding bracketing notation. In cases where multiple divisions are possible, I have omitted the brackets.

Many of the following examples were published in a descriptive paper that appeared in \emph{Deutsche
  Sprache} \citep{Mueller2003b}. I found most of these examples by careful reading. After the
publication of the paper in 2003 I continued to collect data and made it available to the community
on my webpage \citep{Mueller2013a}. A further resource that is also available online is a database
put together by Felix \citet{Bildhauer2011a} in the DFG project \emph{Theorie und Implementation
  einer Analyse der Informationsstruktur im Deutschen unter besonderer Berücksichtigung der linken
  Satzperipherie} (MU 2822/1-1 and SFB 632, A6). He collected 3.200 examples mainly from the corpora
that are available from the Institut für Deutsche Sprache in Mannheim at \url{https://clarin.ids-mannheim.de/SFB632/A6}.\footnote{
  \citet{Winkler2014a} uses almost exclusively data from
  \citew{Mueller2003b,Mueller2005d,Mueller2013a,MBC2012a,BC2010a,Bildhauer2011a} without proper
  acknowledgment of the source. Researchers who want to cite examples properly are urged to check
  the mentioned papers before attributing data to Winkler.
}
These examples are annotated with respect to part of speech, grammatical function and information structural status.

The following examples were discussed in many German publications but until now they were not
available with glossing and translation.





\subsection{Subject and adverb}
\label{sec-subj-mf}

In (\ref{bsp-richtig-geld}), an adjective used adverbially is present in the prefield with the subject of a passive 
clause. The same is true for the construction in (\ref{bsp-alle-traeume-gleichzeitig}):
The subject has been fronted along with an adjective.
\eal
\label{bsp-mehrfach-vf-subjekt}
\ex 
\gll {}[Richtig] [Geld] wird aber nur im Briefgeschäft verdient.\label{bsp-richtig-geld}\footnotemark\\
	 \spacebr{}right \spacebr{}money is PRT only in postal.services earned\\
\footnotetext{        
	taz, 28./29.10.2000, p.\,5.
}
\glt `It's only in postal services where you earn serious money.'
\ex 
\gll {}[Alle Träume] [gleichzeitig] lassen sich nur selten verwirklichen.\label{bsp-alle-traeume-gleichzeitig}\footnotemark\\
     \spacebr{}all dreams \spacebr{}simultaneously let \textsc{refl} only seldom realized\\
\footnotetext{        
		Broschüre der Berliner Sparkasse, 1/1999.
        }
\glt `All our dreams can only seldomly be realized at the same time.'
\zl
There are examples such as (\mex{1}) where one may be tempted to count the temporal adjunct
\emph{täglich} `daily' as part of the NP, but we are not
dealing with these kinds of constructions in \pref{bsp-alle-traeume-gleichzeitig} as the adverb obviously refers to \emph{verwirklichen} `to realize'.
\ea
\gll ein weiteres Großcenter [\ldots], das mit [20.000 Besuchern täglich] zu den beliebtesten gehört.\footnotemark\\
     a further big.centre    {} that with \spacebr{}20,000 vistors daily to the most.popular belongs\\
\footnotetext{
        taz berlin, 11.10.2002, p.\,13.
}
\glt `another large scale centre, which -- with its total of 20,000 visitors daily -- counts as one of the most popular.'
\z

Note that the fronted elements in (\ref{bsp-mehrfach-vf-subjekt}) are logical objects. The fronting
of logical subjects together with other constituents does -- if we ignore examples like
(\ref{ex-v3-particles-jacobs}) which are sometimes analyzed as V3 \citep{Jacobs86a} -- not seem to be possible
(see \citew[\page 413]{Eisenberg94a}).\todostefan{Was hat Eisenberg genau gesagt? Daten?}

As \citet[\page 316]{Lenerz86a}, \citet[\page 99]{Fanselow87a}, and \citet[\page 32]{Duerscheid89a} noted, examples like (\mex{1}) are absolutely unacceptable.
\eal
\ex[*]{
\gll Ich das Wienerschnitzel habe bestellt.\footnotemark\\
     I.\nom{}   the.\acc{} wiener.schnitzel have ordered\\
\footnotetext{
         \citew[\page 316]{Lenerz86a}.
}	 
\glt `I ordered the Wiener schnitzel.'
}
\ex[*]{
\gll Einen interessanten Vortrag der Sascha dürfte gehalten haben.\footnotemark\\
     an interesting      talk    the Sascha might  hold     have\\
\footnotetext{
\citep[\page 99]{Fanselow87a}.
}
\glt `Sascha probably gave an interesting talk.'
}
\zl

\noindent
However, examples like (\mex{1}) -- which are quoted from
\citew[\page 72]{BC2010a} and \citew[\page 371]{Bildhauer2011a}, respectively, -- show that it is possible in principle:

\eal
\ex
\gll [Weiterhin] [Hochbetrieb] herrscht am Innsbrucker Eisoval.\footnotemark\\
     \spacebr{}further \spacebr{}high.traffic reigns at.the Innsbruck icerink\\
\footnotetext{
\citew[\page 72]{BC2010a}
}
\glt ‘It’s still all go at the Innsbruck icerink.’
\ex Die Kinder haben eigene Familien gegründet und wohnen alle einigermaßen in der Nähe, so daß die Jubilarin ihre 19 Enkel- und 17 Urenkelkinder häufig sehen kann.\\
\gll "`[Alle] [gleichzeitig] können mich nicht besuchen, weil ich {gar nicht} so viel Platz habe"', lacht sie.\footnotemark\\
     \hspaceThis{"`[}all simultaneously can me not visit  because I not.at.all so much space have    laughs she\\      
\footnotetext{
DeReKo corpus, V99/JAN.02701. Quoted from \citew[\page 371]{Bildhauer2011a}.
}
\glt `The children raised their own families and live close enough so that the jubilarian can see her 19 grandchildren and 17 great"=grandchildren often. It is not possible that all grandchildren and great"=grandchildren visit me simultaneously
because I do not have that much space, she says laughingly.'\todostefan{check}
\zl






\subsection{Accusative objects and prepositional phrases}

In (\mex{1}), the prefield consists of a noun phrase and a prepositional phrase.
\eal 
\gll {}[Nichts] [mit derartigen Entstehungstheorien] hat es natürlich zu tun, wenn \ldots \label{bsp-nichts-mit-derartigen}\footnotemark\\
	 \spacebr{}nothing \spacebr{}with these.kind theories.of.origin has it of.course to do if\\      
\footnotetext{
        K. Fleischmann, \emph{Verbstellung und Relieftheorie}, München, 1973, p.\,72.
        quoted from \citew[\page 135]{vdVelde78a}.
        }
\glt `It has, of course, nothing to do these kinds of theories of origin, if \ldots'
\ex 
\gll {}[Zum zweiten Mal] [die Weltmeisterschaft] errang Clark 1965 \ldots\footnotemark\\
	   \spacebr{}to.the second time \spacebr{}the world.championship won Clark 1965 {}\\
\footnotetext{
        \citep*[\page 162]{Benes71}
      }\label{bsp-zum-zweiten-mal-die-Weltmeisterschaft}
\glt `Clark won the world championship for the second time in 1965.'
\ex\label{die-kinder-nach-stuttgart}
\gll {}[Die Kinder] [nach Stuttgart] sollst du bringen.\footnotemark\\
     \spacebr{}the children \spacebr{}to Stuttgart should you bring\\
\footnotetext{
        \citep[\page 81]{Engel70a}
    }
\glt `You should take the children to Stuttgart.'
\zl



\noindent
In (\ref{bsp-nichts-mit-derartigen}), we are dealing with \emph{cohesion}\footnote{%
			See \citew[\page 77]{Bech55a} for more on the term cohesion.
}: The word \emph{nichts} `nothing' is a semantic fusion of  \emph{nicht} `not'
and \emph{etwas} `something'.  \emph {etwas} is the accusative object. The \emph{mit} PP is a
complement of \emph{zu tun haben} `to do have'.
The PP \emph{zum zweiten Mal} `for the second time' in \pref{bsp-zum-zweiten-mal-die-Weltmeisterschaft} is, on the other hand,
an adjunct.


\begin{comment}
Reviewer: raus

\citet[\page 69]{Fanselow93a} diskutiert das folgende Beispiel:
\ea
In Hamburg eine Wohnung hätte er sich besser nicht suchen sollen.
\z
Bei diesem Beispiel handelt es sich aber wahrscheinlich im NP-interne Voranstellung, wie sie
\zb von \citet[\page 68]{Fortmann96a-unread-gekauft} für (\mex{1}) in Erwägung gezogen wird:
\ea
Mit der Bahn eine Reise ist nicht geplant.
\z
\citet[\page 133]{Abb94} analysiert solche Voranstellungen als DP-interne Topikalisierungen.
Er ordnet auch folgende Beispiele als umgangsprachlich möglich ein:
\eal
\ex Übermorgen das Spiel gegen Kaiserslautern würde ich gern live sehen.
\ex Der die Karten hat, der Mann, soll gleich kommen.
\ex An der Wand das Bild kommt mir bekannt vor.
\zl
Bei (\mex{0}b) sieht man besonders deutlich, daß es sich nicht um eine Mehrfachbesetzung
des Vorfelds handeln kann, da der Relativsatz ja allein nicht vorfeldfähig ist. Solche Beispiele
sollen in diesem Aufsatz nicht behandelt werden.
\end{comment}

% M89/910.39160: Mannheimer Morgen, 21.10.1989, Politik; Jürgen Möllemanns Hindernislauf
% PS: Sicher nicht ganz unwahr ist, daß Möllemann seit Mai vorsichtshalber für jede Kabinettssitzung einen Fotografen einbestellt hat, der den großen Augenblick für die Nachwelt im Bild festhalten soll.




\subsection{Accusative objects and adverbs}

In (\mex{1}), we are dealing with sentences where the accusative object occurs in initial position together with an adverb 
or an adjective used as an adverb.
\eal
\label{bsp-mehrfach-vf-adv-acc}
\ex 
\gll {}[Gezielt] [Mitglieder] [im Seniorenbereich] wollen die Kendoka allerdings nicht werben.\label{bsp-gezielt-mitglieder}\footnotemark\\
       \spacebr{}specifically \spacebr{}members \spacebr{}in pensioner.area want the Kendoka PRT not gain\\
\footnotetext{
        taz, 07.07.1999, p.\,18
      }
\glt `The kendoka are not looking to gain members specifically in the pensioner demographic.'
\ex
\gll {}[Dauerhaft] [mehr Arbeitsplätze] gebe es erst, wenn sich eine Wachstumsrate von  mindestens 2,5 Prozent über einen Zeitraum von drei oder vier Jahren halten lasse.\footnotemark\\ 
       \spacebr{}constantly \spacebr{}more jobs gives it first when \textsc{refl} a growth.rate of  at.least 2.5 percent over a time.period of three or four years hold let\\
\footnotetext{
        taz, 19.04.2000, p.\,5. %taz Nr. 6123 vom 19.4.2000 Seite 5
}\label{bsp-dauerhaft-mehr-arbeitsplaetze}
\glt `In the long run, there will only be more jobs available when a growth rate of at least 2.5 percent 
can be maintained over a period of three of four years.'	
\ex 
\gll {}[Kurz] [die Bestzeit] hatte der Berliner Andreas Klöden [\ldots] gehalten.\footnotemark\\
	 \spacebr{}briefly \spacebr{}the best.time had the Berliner Andreas Klöden {} held\\
\footnotetext{
        Märkische Oderzeitung, 28./29.07.2001, p.\,28.
}\label{bsp-kurz-die-bestzeit}     
\glt `Andreas Klöden from Berlin had briefly held the best time.'
%% \ex 
%% \gll {}[Noch entschiedener] [prädikativen Charakter] hat das Adj., wenn [\ldots]\footnotemark\\
%% 	 \spacebr{}still more.deciding \spacebr{}predicative character has the adj. if\\
%% \footnotetext{
%%         In the main text of \citew[\page 52]{Paul1919a}.
%%     }\label{bsp-praedikativen-charakter}
%% \glt  `Even more decidingly, the adjective has a predicative character, if \ldots'
\zl



In (\ref{bsp-gezielt-mitglieder}), the prefield is possibly even occupied by three elements since it is more likely that the
prepositional phrase refers to \emph{werben} `to solicit' rather than to \emph{Mitglieder} `members'. The sentence does not have the interpretation
that they want to gain `members in the pensioner demographic' but rather that the people who the advertising measures
are trying to attract are in fact seniors -- that is, they are advertising to the `demographic of seniors'.

The example (\ref{bsp-gezielt-mitglieder}) cannot be analyzed in the same way that \citet{Jacobs86a} suggested for
sentences such as (\mex{1}) since \emph{gezielt} `specifically' only has scope over \emph{werben} `to solicit' but not over the modal verb.
\ea
\label{bsp-vermutlich}
\gll {}[Vermutlich] [Brandstiftung] war die Ursache für ein Feuer in einem Waschraum in der Heidelberger Straße.\footnotemark\\
	   \spacebr{}supposedly \spacebr{}arson was the cause for a fire in a washroom in the Heidelberger Street\\
\footnotetext{
Mannheimer Morgen, 04.08.1989, Lokales; Pflanzendieb.
}
\glt `Arson was supposedly the cause of a fire in a washroom in the Heidelberger Straße.'
\z
In Jacob's analysis, \emph{gezielt} `specifically' would be connected to the rest of the sentence
and one  would therefore get a structure where the adverbial has scope over the modal verb.


\subsection{Präpositinalobjekt und Adverb}

(\mex{1}) shows an example of a fronting of an adverb together with a prepositional object:
\ea
\gll {}[Besonders] [an Profil] gewinnt Kathrin Passig allerdings in der Auseinandersetzung mit Jonathans Franzens technikkritischen Essays.\footnotemark\\
  \spacebr especially \spacebr at profile wins Kathrin Passing but in the argument with Jonathans Franzen's tecnics.critically essays\\ 
\footnotetext{
 taz 20./21.07.2019, p.\,16
}
\glt `Kathrin Passig gains profile especially in the competition with those essays of Jonathans
Franzen that are critical of technology.' 
\z



\subsection{Dative objects and prepositional phrases}

(\mex{1}) is an example of simultaneous fronting of a dative object and a prepositional object.
\begin{sloppypar}
\ea\iw{gratulieren}
\gll {}[Der Universität] [zum Jubiläum] gratulierte auch Bundesminister Dorothee Wilms, die in den fünfziger Jahren in Köln studiert hatte.\footnotemark\\
	   \spacebr{}the university \spacebr{}to.the anniversary congratulated also state.minister Dorothee Wilms who in the fifties years in Cologne studied had\\
\footnotetext{
        Kölner Universitätsjournal, 1988, p.\,36, quoted from \citew[\page 87]{Duerscheid89a}.
}
\glt `State minister Dorothee Wilms -- who studied in Cologne in the 1950s -- also congratulated the university on its anniversary.' 
\z
\end{sloppypar}


\subsection{Dative and accusative object}
\label{sec-dat-acc-vf}

The following examples are constructed examples from the literature that show that dative NPs can be
fronted together with accusative NPs:
\eal
\ex 
\gll Der Maria einen Ring glaube ich nicht, dass er je schenken wird.\footnotemark\\
	 the Maria a ring believes I not that he ever give will\\
\footnotetext{
\citew[\page 67]{Fanselow93a}.
}
\glt `I dont think that he would ever give Maria a ring.'
\ex 
\gll Ihm den Stern hat Irene gezeigt.\footnotemark\\
	 him the star has Irene shown\\
\footnotetext{
  \citew[\page 412]{Eisenberg94a}.%
}
\glt `Irene showed him the star.'
\ex 
\gll (Ich glaube) Kindern Bonbons gibt man besser nicht.\footnotemark\\
     \hspaceThis{(}I think children candy gives one better not\\
\footnotetext{
        G.\ \citew[\page 260]{GMueller98a}.
}
\glt `I think it's better not to give candy to children.'
\zl
\ea 
\gll Studenten einem Lesetest unterzieht er des öfteren.\\
     students a reading.test subjects.to he the often\\
\glt `He often makes his students do a reading comprehension test.'	
\z
(\mex{0}) is due to Anette Frank (p.\,c.\ 2002).

I discussed these sentences in \citew{Mueller2005d}. Back then I did not have any attested
examples apart from the one in (\mex{1}), which involves an idiom. 
\ea
\label{bsp-zeitgeist} 
\gll {}[Dem Zeitgeist] [Rechnung] tragen im unterfränkischen Raum die privaten, städtischen und kommunalen Musikschulen.\footnotemark\\
      \spacebr{}the.\dat{} Zeitgeist \spacebr{}account carry in.the lower.Franconian area the private, urban and communal music.schools\\
\footnotetext{
        Fränkisches Volksblatt, quoted from Spiegel, 24/2002, p.\,234.
    }
\z
But a more systematic corpus exploration by \citet{Bildhauer2011a} resulted in attested examples like the one in (\mex{1}a). (\mex{1}b) was
found by chance by Arne Zeschel.
\eal
\ex
\label{ex-dem-saft-eine-kraeftige-farbe}
\gll Dem Saft eine kräftige Farbe geben Blutorangen.\footnotemark\\
     the.\dat{} juice a.\acc{}   strong   color give blood.oranges\\
\footnotetext{
\citet{BC2010a} found this example in the \emph{Deutsches Referenzkorpus} (DeReKo), hosted at Institut
für Deutsche Sprache, Mannheim: \url{http://www.ids-mannheim.de/kl/projekte/korpora}
}
\glt `Blood oranges give the juice a strong color.'
\ex\label{bsp-ihnen-für-heute}
\gll {}[Ihnen] [für heute] [noch] [einen schönen Tag] wünscht Claudia Perez.\footnotemark\\
  \spacebr{}you.\dat{} \spacebr{}for today \spacebr{}still \spacebr{}a.\acc{} nice day wishes Claudia Perez\\
\footnotetext{
  Claudia Perez, Länderreport, Deutschlandradio.
}%
\glt `Claudia Perez wishes you a nice day.'
\zl
(\mex{1}) also from \citew[\page 369]{Bildhauer2011a} again involves an idioimatic example:
\ea
\gll {}[Den Kölnern] [einen Bärendienst] erwies nach etwas mehr als einer Stunde ausgerechnet Nationalspieler Podolski,
der wegen einer Fußblessur zunächst auf der Bank Platz nehmen musste.\footnotemark\\
       \spacebr{}the inhabitants.of.Cologne \spacebr{}a disservice did after some more than an
       hour of.all.people national.player Podolski who because.of his foot.wound initially on the bench
       seat take must\\
\footnotetext{
  \url{http://www.haz.de/Nachrichten/Sport/Fussball/Uebersicht/FC-Augsburg-gelingt-Coup-gegen-acht-Koelner},
  10.02.2010.
}
\glt `Podolski did a disservice to the Cologne team after a little more than an hour since he had to
seat himself on the bench due to a foot wound.'
\z

See (\ref{ex-multiple-nps-collocation-idiom}) and (\ref{bsp-weiterhin-derjugend}) for further examples that involve the
fronting of idiom/""collocation parts or parts of support verb constructions.

These examples show that such frontings may include two NPs and hence the syntax has to account for
such structures. This does not mean that all structures involving two fronted NPs will be predicted
to be possible. For instance frontings like the one in (\mex{1}) which I discussed in
\citew{Mueller2005d} are hardly possible without a context.
\ea[?*]{
\label{ex-maria-peter-stellt-max-vor}
\gll Maria Peter stellt Max vor.\\
     Maria Peter introduces Max \particle\\
\glt `Max introduces Peter to Maria.'
}
\z
As \citet[\page 48]{Winkler2014a} suggested the markedness of examples like (\mex{0}) is probably
due to the lack of case marking of the noun phrases. Because of this it is unclear who
introduces whom to whom. The sentence greatly improves if determiners are used with nouns since the
determiners are case marked and help to identify which noun fills which grammatical role.
\eal
\ex[?]{
\gll Die Maria dem Peter stellt der Max vor.\\
     the.\acc{} Maria the.\dat{} Peter introduces the.\nom{} Max \particle\\
\glt `Max introduces Maria to Peter.'
}
\ex[?]{ 
\gll Der Maria den Peter stellt der Max vor.\\
     the.\dat{} Maria the.\acc{} Peter introduces the.\nom{} Max \particle\\
\glt `Max introduces Peter to Maria.'
}
\zl
So, the unacceptability of (\ref{ex-maria-peter-stellt-max-vor}) may be due to processing
difficulties.\todostefan{Say something about *Ich das Wienerschnitzel habe bestellt. Should the
  discussion about NPs that was in the problem section be reflected somewhere else?}


\subsection{Instrumental prepositional phrases and temporal prepositional phrases}

In (\mex{1}), there is both a temporal prepositional phrase as well as an instrumental prepositional phrase 
in the pre-field.
\ea
\label{bsp-instrument}
\gll {}[Zum letzten Mal] [mit der Kurbel] wurden gestern die Bahnschranken an zwei Übergängen im Oberbergischen Ründeroth geschlossen.\footnotemark\\
	 \spacebr{}to.the last time \spacebr{}with the crank were yesterday the train.barriers at two crossings in Oberbergisch Ründeroth closed\\
\footnotetext{
        Kölner Stadtanzeiger, 26.04.1988, p.\,28, quoted from \citew[\page 107]{Duerscheid89a}.
}
\glt `The barriers at a train station in Ründeroth, Oberberg were closed using a crank for the last time yesterday.'
\z

I have also found many other examples for most of the types of examples discussed here. Furthermore, we find multiple fronting
with adjectives used adverbially and directional/local prepositional phrases, noun phrases in copula constructions with adverbials, 
prepositional phrases in copula constructions with adverbs, predicative conjunction phrase with adverbs, directional prepositional 
phrases with adverbs as well as local prepositional phrases with adverbs. For space considerations, not all examples have been included
here. A comprehensive discussion of the data can be found in \citep{Mueller2003b}, which appeared in \emph{Deutsche Sprache}.
Further data from newspaper can be found at
\url{http://hpsg.fu-berlin.de/~stefan/Pub/mehr-vf-ds.html}. A more systematic data collection was
done in the project A~6 of the SFB~632. The database is documented in \citew{Bildhauer2011a}. The
database is hosted at the IDS Mannheim and can be accessed via \url{http://hpsg.fu-berlin.de/Resources/MVB/}.



\subsection{Support verb constructions and idiomatic usages}
\label{sec-phraseolog}

In examples \fromto{\mex{1}}{\mex{3}}, we are dealing with support verb constructions/idiomatic usages, where
either a set phrase or some fixed lexical element has been fronted together with a complement or adjunct.
In (\mex{1}), there is an element in the prefield which is not part of the phraseologism. On the
other hand, there are only parts of a phraseologism in the prefield in example (\mex{2}). The most notable feature of the examples in 
(\mex{3}) is that more than two constituents are occupying the prefield.


\eal%
\label{pvp-fvg-idioms}%
\ex 
\gll {}[Den Kürzungen] [zum Opfer] fiel auch das vierteljährlich erscheinende Magazin \emph{aktuell}, das seit Jahren als eines der kompetentesten in Sachen HIV und Aids gilt.\footnotemark\\
	  \spacebr{}the cuts \spacebr{}to.the victim fell also the quarterly appearing magazine \emph{aktuell} which since years as one of.the most.competent in things HIV and Aids counts \\
\footnotetext{
         zitty, 8/1997, p.\,36.
}
\glt `The magazine \emph{aktuell}, which appears quarterly and has for years had a reputation as being one of the most competent when it comes to HIV and Aids, has also fallen victim to the cuts.'
\ex 
\gll  {}[Eine lange Kolonialgeschichte] [hinter sich] hat das einst britische Warenhaus Lane Crawford\footnotemark\\
	   \spacebr{}a long colonial.history \spacebr{}behind \textsc{refl} has the once British warehouse Lane Crawford \\
\footnotetext{
        Polyglott-Reiseführer "`Hongkong Macau"', München 1995, p.\,28.
      }
\glt `The former British warehouse Lane Crawford has a long colonial history behind it'
\ex %Wenn es darauf ankommt, wirkt ein CDU-Politiker, der gegen die SPD-Regierung schimpft, eben einfach überzeugender. 
\gll {}[Ernsthaft] [in Schwierigkeiten] geriet Koch deshalb nur am Anfang, als es um den drohenden Irakkrieg ging.\footnotemark\\
	  \spacebr{}seriously  \spacebr{}in difficulties came Koch therefore only at.the start when it around the threatening Iraq.war went\\
\footnotetext{
        taz, 28.01.2003, p.\,6.
   }
\glt `Koch therefore only encountered serious problems at the start when dealing with the impending Iraq war.'
\ex 
\gll {}[Ihm] [zur Seite] steht als stellvertretender Vorstandschef Gerd Tenzer.\footnotemark\\
	 \spacebr{}him \spacebr{}to.the side stands as temporary committee.boss Gerd Tenzer\\
\footnotetext{
        taz, 18.07.2002, p.\,7.
}
\glt `Gerd Tenzer is on his side as temporary head of the committee.'
\ex 
\gll Sex ist je besser, desto lauter. [Am lautesten] ["`zur Sache"'] gehe es in Köln und Düsseldorf mit einem Spitzenwert von jeweils 25\,\%.\footnotemark\\
	 Sex is the better, the louder \spacebr{}at.the loudest \hspaceThis{["`}to.the thing goes it in Cologne and Düsseldorf with a top.value of each 25\,\%\\
\footnotetext{
taz, 19.04.2000, p.\,11.% April
}
\glt `When it comes to sex: the better, the louder. The loudest when ``getting down to business'' can be found in Cologne and Düsseldorf with both topping 25\,\%.'
% Marga Reis -> Gradpartikel
% \ex {}[Erst recht] [auf Touren] brachte er sie, als das Regime im Jahr darauf den 
% Liedermacher Wolf Biermann ausbürgerte und zumeist Intellektuelle gegen 
% diese Willkür protestierten.\footnote{
%         Spiegel 44/2000, p.\,272
% }\label{bsp-erst-recht-auf-touren-brachte}
\ex 
\gll {}[Damit] [im Zusammenhang] steht auch eine Eigenschaft der paarweisen Konjunkte\footnotemark\\
	 \spacebr{}with.it \spacebr{}in.the relation stands also a property of.the in.pairs conjuncts\\
\footnotetext {
        In the main text of \citew[\page 40]{Haider88a}.
}
\glt `A property of the conjuncts in pairs is also related to this.'
\ex 
\gll {}[Endgültig] [auf den TV"=Geschmack] kam Anne Will bei den olympischen Spielen 2000.\footnotemark\\
	 \spacebr{}finally \spacebr{}on the TV"=taste came Anne Will at the Olympic Games 2000\\
\footnotetext{
        taz, 16.03.2001, p.\,12.
}
\glt `Anne Will finally got a taste of television at the 2000 Olympic Games.'
% Marga Reis sagt, das sei sowas wie ein Fokuspartikel
% \ex {}[Zunehmend] [Spaß] hat Michael Jordan mit seinen Washington Wizards: [\ldots]\footnote{
%         taz, 14.12.2001, p.\,19
% }
%
\ex 
\gll {}[Stark] [unter Druck] geriet der Pharmawert Schering.\footnotemark\\
	 \spacebr{}strong \spacebr{}under pressure came the pharmaceutical Schering\\
\footnotetext{
        taz, 28./29.09.2002, p.\,9 (dpa).
}
\glt `The pharmaceutical company Schering came under extreme pressure.'
\zl
%-------------------------------------------------------------------------------------------
\eal
\ex\label{mit-den-huehnern-ins-bett}
\gll {}[Mit den Hühnern] [ins Bett] gehen sie dort.\footnotemark\\
     \spacebr{}with the chickens \spacebr{}in.the bed go they there\\
\footnotetext{
    \citew[\page 81]{Engel70a}. Engel discusses this example in connection with \pref{die-kinder-nach-stuttgart}.
    Engel views \emph{ins Bett} and \emph{nach Stuttgart} as inner frame elements
    and notes that the ability to front a constituent with an inner frame element is restricted. Engel also
	classifies adjectives in copula constructions as inner frame elements. Fronting of adjectives with dependent
	elements behaves completely normally however.
    See Section~\ref{sec-keine-mehr-vf-pvp}.%
}
\glt `They go to bed very early there.'
% {}[Mit den Hühnern] [ins Bett] gehen sie in diesem fernen Land.\footnote{
%         \citet[p.\,192]{Engel94} schreibt zu diesem Beispiel: \emph{Gelegentlich
% werden sogar mehrere -- teilweise umfangreiche -- Satzglieder mit einer infiniten Verbform
% zusammen ins Vorfeld übernommen}. In (\ref{mit-den-huehnern-ins-bett}) liegt aber keine infinite
% Verbform vor. Auf p.\,195 wird dasselbe Beispiel zusammen mit (i) diskutiert:
%         \ea
%         Ans Meer gefahren sind wir erst im September.
%         \z
% Man kann das nur so verstehen, daß angenommen wird, daß in (\ref{mit-den-huehnern-ins-bett}) eine
% zu (i) parallele Konstruktion vorliegt. Das ist die Analyse, die ich im Abschnitt~\ref{sec-analyse-mf}
% vorschlagen werde.%
%
% Mit den Hühnern ins Bett pflegt er zu gehen. Engel82a:227
%
%}
\ex 
\gll {}[Öl] [ins Feuer] goß gestern das Rote-Khmer-Radio\footnotemark\\
	 \spacebr{}oil \spacebr{}in.the fire poured yesterday the Rote-Khmer-Radio \\
\footnotetext{
        taz, 18.06.1997, p.\,8.
}
\glt `Rote-Khmer-Radio fanned the flames yesterday'
\ex\iw{setzen!das Tüpfel aufs i $\sim$}
\gll {}[Das Tüpfel] [aufs i] setze der Bürgermeister von Miami, als er am Samstagmorgen von einer schändlichen Attacke der US-Regierung sprach.\footnotemark\\
	  \spacebr{}the dot  \spacebr{}on.the i put the mayor of Miami as he on Saturday.morning from a shameful attack of.the US-government spoke\\
\footnotetext{
        taz, 25.04.2000, p.\,3. %taz Nr. 6126 vom 25.4.2000 Seite 3
    }
\glt `On Saturday morning, the icing on the cake was when the mayor of Miami spoke of the shameful attack by the US government.'
\ex 
\gll {}[Ihr Fett] [weg] bekamen natürlich auch alte und neue Regierung [\ldots]\footnotemark\\
	  \spacebr{}their fat \spacebr{}away got of.course also old and new government\\ 
\footnotetext{
        Mannheimer Morgen, 10.03.1999, Lokales; SPD setzt auf den "`Doppel-Baaß"'. %M99/903.16159 Mannheimer Morgen, 10.03.1999, Lokales; SPD setzt auf den "Doppel-Baaß"
      }
\glt `Both the old and new governments were taken to task \ldots'
\ex 
\gll {}[Den Finger] [mitten in die Wunde] legte jetzt eine findige Gruppe Internetexperten aus Österreich: [\ldots]\footnotemark\\
	 \spacebr{}the finger \spacebr{}middle in the wound laid now a clever group internet.experts from Austria\\
\footnotetext{
        taz, 04./05.11.2000, p.\,30.
}
\glt `A clever group of internet experts from Austria have now rubbed salt into the wounds \ldots'
%Partikel steht noch rechts -> gesamte Verb kann nicht in V1 stehen
% Mit rechten Dingen geht es hingegen bei den deutschen Turnieren zu.\footnote{
%         taz, 19.07.2001, p.\,19
% }
\ex 
\gll {}[Heiß] [her] geht es dagegen beim Thema "`Kundenbewertungen"'  -- einem Herzstück der Online"=Börse.\footnotemark\\
	  \spacebr{}hot \spacebr{}to.here goes it on.the.other.hand by.the topic customer.reviews {} a centrepiece of.the online"=market\\
\footnotetext{
        Spiegel, 1/2003, p.\,123.
}
\glt `On the other hand, it gets rather heated when it comes to `customer reviews' -- a crucial part of the online market.'
\ex 
%Merkwürdig: Im Konzerthaus stieg gestern Abend eine Benefiz-Gala - "Cinema for Peace". Geladen hatte Roger Moore (Ex-007, jetzt Unicef), erwartet wurde Hollywoodpersonal à la George Clooney. Am Gendarmenmarkt aber flackert seitdem, als "Zeichen gegen den Krieg", ein Ableger der "Welt-Friedensflamme". Entzündet hat sie Christopher Lee. Und was von diesem Herrn zu erwarten ist, weiß man spätestens seit "The Two Towers".
% Absatz
\gll {}[Übles] [im Schilde] führten auch zwei mit Schußwaffen ausgestattete Maskierte, die am frühen Montagmorgen eine Kneipe in Neukölln überfielen und mit den Tageseinnahmen flüchteten.\footnotemark\\
     \spacebr{}bad.things  \spacebr{}in.the shield led also two with guns equipped masked.men who on early monday.morning a pub in Neukölln held.up and with the daily.takings fled\\
\footnotetext{
        taz berlin, 11.02.2003, p.\,20.
}
\glt `Two masked men carrying guns were also up to no good as they held up a pub in Neukölln and made off with that day's takings.'
\zl


\eal
\label{drei-und-mehr-idiomatisch}
\ex 
\gll {}[Endlich] [Ruhe] [in die Sache] brachte die neue deutsche Schwulenbewegung zu Beginn der siebziger Jahre.\footnotemark\\
	  \spacebr{}finally  \spacebr{}peace  \spacebr{}in the matter brought the new German gay.movement to beginning of.the seventy years\\ 
\footnotetext{
        taz, 07.11.1996, p.\,20.
}
\glt `The new German gay movement finally brought peace to the matter in the early 70s.'
\ex
% Den Arbeitern in der Tischlerei gefällt's, auch wenn ein ganz Junger offen zugibt, daß er "den Mann nur vom Namen her kennt und ihn Politik überhaupt nicht interessiert".
\gll {}[Wenig] [mit Politik] [am Hut] hat auch der Vorarbeiter, der sich zur Aussage hinreißen läßt, "`daß der Sausgruber das falsche anhat"'.\footnotemark\\
	 \spacebr{}little  \spacebr{}with politics  \spacebr{}on.the hat has also the foreman who \textsc{refl} to.the statement carry.away lets that the Sausgruber the wrong.one wears\\
\footnotetext{
        Vorarlberger Nachrichten, 03.03.1997, p.\,A5.
}\label{bsp-wenig-mit-politik-am-hut-hat}
\glt `The foreman also cares little about politics and got so carried away he claimed that Sausgruber was wearing the wrong thing.'
\ex
\gll {}[Wenig] [mit den aktuellen Ereignissen] [im Zusammenhang] steht die Einstellung der Produktion bei der Montlinger Firma Mega-Stahl AG auf Ende November.\footnotemark\\
	 \spacebr{}little  \spacebr{}with the recent events  \spacebr{}in relation stands the
         cancellation of.the production from the Montlingen company Mega-Stahl AG on end November\\
\footnotetext{
St. Galler Tagblatt, 26.10.2001 ; Sparsam auf bessere Zeiten wartend.
}
\glt `The suspension of production until the end of November at the company Mega-Stahl AG in Montlingen has little to do with recent events.'
% \ex 
% % Ihren frühen Arbeitszeitbeginn -- Prammer startet gegen sieben Uhr -- behält sie übrigens bei, egal, wie spät es in der vorangegangenen Nacht wurde. 
% Wenig mit Sport am Hut hat auch Unterrichtsministerin Elisabeth Gehrer (VP).
% X00/JAN.03127 Oberösterreichische Nachrichten, 25.01.2000; Sauna, Schokobananen und viel zu wenig Schlaf
% P92/AUG.24714 Die Presse, 20.08.1992; Ein Kokoschka für Bürgermeister Zilk, ein Attersee für Pasterk
% Nun hängt's, verkehrt rum, in seinem Pressebüro. Wenig mit Bildern am Hut, pardon: an der Wand, hat auch Stadtrat Johann Hatzl.
\zl
%
%-------------------------------------------------------------------------------------------
%
The examples in (\mex{1}) show that the verbal part of the idiom, i.e. the functional verbal complex,
does not necessary have to be adjacent to the fronted elements.
\eal
\label{bsp-idioms-nicht-adjazent}
\ex 
\gll {}[Öl] [ins Feuer] dürfte auch die Ausstrahlung eines Interviews gießen, das die US-Fernsehstation ABC in der vergangenen Woche mit Elián führte.\label{bsp-oel-ins-feuer-duerfte}\footnotemark\\
     \spacebr{}oil \spacebr{}in.the fire may also the broadcast of.an interview pour that the US-TV.station ABC in the last week with Elián led\\
\footnotetext{
        taz, 28.03.2000, p.\,9 %28.3.2000 Seite 9.
}
\glt `The broadcast of an interview with Elián carried out last week by the US Network ABC should also fan the flames somewhat.'
\ex\iw{kommen!in Berührung $\sim$}\label{bsp-zum-ersten-mal-mit-punk}
\gll {}[Zum ersten Mal] [persönlich] [in Berührung mit Punk und New Wave] bin ich über Leute gekommen, die in meiner Lehrklasse waren.\label{bsp-zum-ersten-mal-persoenlich-in-beruehrung-bin-ich-gekommen}\footnotemark$^,$\footnotemark\\
	   \spacebr{}the first time \spacebr{}peronsally \spacebr{}in contact with Punk and New Wave be I over people come who in my vocational apprenticeship.class were\\
\footnotetext{
        Toster in an interview in Ronald Galenza und Heinz Havemeister (eds).
        {\em Wir wollen immer artig sein \ldots{} Punk, New Wave, HipHop,
        Independent"=Szene in der DDR 1980--1990\/}, Berlin:
        Schwarzkopf \& Schwarzkopf Verlag, 1999, p.\,309.
        }
\footnotetext{
If one analyzes \emph{in Berührung kommen} as a support verb construction, then one has to view the
\emph{mit} PP as an extraposed argument of the support verb construction. As a result, one would have four constituents in the pre-field in \pref{bsp-zum-ersten-mal-persoenlich-in-beruehrung-bin-ich-gekommen}.
If one were to analyze \emph{in Berührung mit Punk und New Wave} ìn contact with Punk and New Wave'
as a single prepositional phrase, (\ref{bsp-zum-ersten-mal-mit-punk}) would still have three fronted
constituents.%
}
\glt `I first came into contact with Punk and New Wave through people in the apprenticeship class.'
\ex\label{wirklich-in-bed} 
\gll {}[wirklich] [in Bedrängnis] hatte die Konkurrenz den Texaner nämlich auch gestern nicht bringen können.\footnotemark\\
       \spacebr{}really  \spacebr{}in trouble had the competition the Texan actually also yesterday not bring could\\
\footnotetext{
        taz, 24.07.2002, p.\,19.
}
\glt `In fact, the competition couldn't pin the Texan into a corner yesterday either.'
\ex\label{ein-bisschen-wasser-in-den-wein}
\gll Allerdings: [Ein bißchen Wasser] [in den Wein] muß ich schon gießen, [\ldots]\footnotemark\\
     nevertheless  \spacebr{}a bit water  \spacebr{}in the wine must I PRT pour\\
\footnotetext{
%obwohl ich nicht die Probleme mit dem Empfang habe, von dem (sic!) die Hauptstadtpresse schon nach einem Tag in fetten Schlagzeilen zu berichten weiß: [\ldots]
taz, 05.03.2003, p.\,18.
}
\glt `Nevertheless, I will have to add a bit of water to the wine.'
\zl
In the examples in (\mex{0}), the finite verb is a modal verb or a perfect auxillary verb. (\mex{1})
presents an example with the phraseologism \emph{eine gute Figur machen} `to cut a fine figure', where the finite verb occupies
the left sentential bracket but is, however, not adjacent to \emph{Figur} but rather separated from it by
the heavy \emph{bei} prepositional phrase.
\ea
\gll {}[Die beste Figur] [beim ersten Finalspiel um die Basketball"=Meisterschaft in der Berliner Max"=Schmeling"=Halle] machte ohne Zweifel Calvin Oldham.\footnotemark\\
	   \spacebr{}the best figure \spacebr{}at.the first final.game for the basketball"=championship in the Berlin Max"=Schmeling"=Halle made without doubt Calvin Oldham.\\
\footnotetext{
       taz, 22.05.2000, p.\,17.
     }
\glt `It was Calvin Oldham who, without doubt, made the biggest impression during the first round of the final of the basketball championship in the Max Schmeling Halle in Berlin.'
\z
(\ref{bsp-gezielt-mitglieder}) and %(\ref{bsp-erstmals-in-Hongkong}),
(\ref{bsp-kurz-die-bestzeit}) are examples of multiple fronting without idioms where the verb on which the
constituents are dependent is not in initial position.
Analyses which assume that multiple fronting is only possibly when the verb on which the constituents
are dependent is in initial position, are therefore inadequate.

The examples in (\mex{1}) show that it is certainly possible for two noun phrases to occupy the pre-field.
\eal
\label{ex-multiple-nps-collocation-idiom}
\ex
\label{bsp-zeitgeist-zwei} 
\gll {}[Dem Zeitgeist] [Rechnung] tragen im unterfränkischen Raum die privaten, städtischen und kommunalen Musikschulen.\footnotemark\\
      \spacebr{}the Zeitgeist \spacebr{}account carry in.the lower.Franconian area the private, urban and communal music.schools\\
\footnotetext{
        Fränkisches Volksblatt, quoted from Spiegel, 24/2002, p.\,234.
    }
\glt `The private, urban and communal music schools in the lower Franconian area account for the
\emph{zeitgeist}.'
\ex
\gll [Dem Frühling] [ein Ständchen] brachten Chöre aus dem Kreis Birkenfeld im Oberbrombacher Gemeinschaftshaus.\footnotemark\\
\hspaceThis{[}to.the spring \hspaceThis{[}a little.song brought choirs from the county Birkenfeld
    in.the Oberbrombach municipal.building\\
\footnotetext{
 \sigle{RHZ02/JUL.05073}.
}
\glt `Choirs from Birkenfeld county welcomed (the arrival of) spring with a little song in the Oberbrombach municipal building.'\label{fruehling}
\ex
\gll [Dem Ganzen] [ein Sahnehäubchen] setzt der Solist Klaus Durstewitz auf\footnotemark\\
     \hspaceThis{[}to.the everything \hspaceThis{[}a little.cream.hood puts the soloist Klaus
         Durstewitz on\\
\footnotetext{
 \sigle{NON08/FEB.08467}.
}
\glt `Soloist Klaus Durstewitz is the cherry on the cake.'\label{bsp-sahne}
\zl
See also (\ref{ex-dem-saft-eine-kraeftige-farbe}) for a non-idiomatic example.


\subsection{Fronting of three or more constituents}
\label{sec-fronting-more-than-two}

\citet[\page 11]{Luehr85a}\iadata{Lühr} presents examples with more than two fronted elements:\footnote{
	She also discusses other combinations of elements in the prefield which occur in Feuchtwanger's texts.
	She arrives, however, at the conclusion that the order of elements is a conscious style choice on the part
	of the author designed to mirror camera movements in films. The examples are rather deviant in standard German.

\citet{Lee75a} discusses several examples from Kleist where sometimes up to four constituents have been fronted.%
}

\eal
\ex
\gll Im Schnellzug, nach den raschen Handlungen und Aufregungen der Flucht und der 
      Grenzüberschreitung, nach einem Wirbel von Spannungen und Ereignissen, Aufregungen
      und Gefahren, noch tief erstaunt darüber, daß alles gut gegangen war, sank Friedrich
      Klein ganz und gar in sich zusammen.\footnotemark\\
      in.the express.train after the swift action and excitement of.the escape and the border.crossing after a whirlwind of
tensions and events commotions and danger still deeply shocked about that all good gone was sank Friedrich Klein whole and done
in REFL together\\
\footnotetext{
        Herman Hesse. Klein und Wagner. In {\em Gesammelte Werke Band 5\/}. Frankfurt/M. 1970.
}
\glt 'In the express train, following the swift events and action of the escape and the border crossing, after a whirlwind of tensions and events, commotion and danger and still deeply shocked that everything turned out well, Friedrich Klein slumped down completely into himself.'
\ex Mit seinen großen Buchstaben, quer über die letzte Schreibmaschinenseite des Gesuches,
      langsam mit rotem Stift malt Klenk: "`Abgelehnt K."'.\footnote{
        Lion Feuchtwanger. \emph{Erfolg. Drei Jahre Geschichte einer Provinz}. Frankfurt/M. 1981, p.\,114.
}\todoandrew{Übersetzung fehlt}
\zl

See also (\ref{bsp-gezielt-mitglieder}) for a further example with more than two elements in the prefield. The examples in
(\ref{drei-und-mehr-idiomatisch}) constitute idiomatic usages (support verb constructions) which also have more than two fronted
constituents.

The following examples are taken from newspapers:\footnote{
  I thank Felix Bildhauer\aimention{Felix Bildhauer} for these examples.
}
\eal
\ex\label{bsp-ebenfalls-positiv} 
\gll {}[Ebenfalls] [positiv] [auf die Kursentwicklung] wirkte sich die Ablehnung einer Zinserhöhung durch die
Bank of England aus.\footnotemark\\
\spacebr{}also \spacebr{}positive \spacebr{}on the market.trend affected \refl{} the rejection of.a
rate.hike by the Bank of England \partic\\
\footnotetext{
Tiroler Tageszeitung, 18.05.1998, Ressort: Wirtschaft; Frankfurt in fester Verfassung; I98/MAI.19710.
}
\glt `The rejection of a rate hike by the bank of England also affected the market trend positively.'
\ex 
\gll {}[Zum ersten Mal]    [ein Trikot]    [in der Bundesliga]    hat    Chen Yang angezogen, und zwar das des Aufsteigers Eintracht Frankfurt.\footnotemark\\
    \spacebr{}to.the first time \spacebr{}a jersey in the Bundesliga has Chen Yang on.put  und
    namely that the promoted.team Eintracht Frankfurt\\
\footnotetext{
Frankfurter Rundschau, 24.08.1998, S. 13, Ressort: FRANKFURTER, R98/AUG.67436.
}
\glt `Chen Yang put on a jersey in the Bundesliga for the first time, namely one of the jerseys of
the promoted team Eintracht Frankfurt.'
\ex\label{bsp-weiterhin-derjugend}
\gll {}[Weiterhin]    [der Jugend]    [das Vertrauen]    möchte    man beim KSK Klaus schenken.\footnotemark\\
     \spacebr{}still  \spacebr{}the.\dat{} youth \spacebr{}the.\acc{} trust wants one at.the KSK Klaus give.as.a.present\\
\footnotetext{
Vorarlberger Nachrichten, 26.09.1997, S. C4, Ressort: Sport; Die Ländle-Staffeln wollen Serie halten,    V97/SEP.48951.
}
\glt `People at the KSK Klaus want to continue to trust the youth.'
\zl

(\ref{bsp-ihnen-für-heute}) -- repeated here as (\mex{1}) for convenience -- is the most extreme example I know of with four constituents before the finite verb:\footnote{
  I thank Arne Zeschel\aimention{Arne Zeschel} for this example.
}
\ea\label{bsp-ihnen-für-heute-zwei}
\gll {}[Ihnen] [für heute] [noch] [einen schönen Tag] wünscht Claudia Perez.\footnotemark\\
  \spacebr{}you.\dat{} \spacebr{}for today \spacebr{}still \spacebr{}a.\acc{} nice day wishes Claudia Perez\\
\footnotetext{
  Claudia Perez, Länderreport, Deutschlandradio.
}%
\glt `Claudia Perez wishes you a nice day.'
\z



\subsection{Non-cases of multiple fronting}
\label{sec-keine-mehr-vf-pvp}

This section explores examples that were discussed in connection with multiple frontings but behave
different in important respects. Subsection~\ref{sec-pvp-is-not-mf} deals with complex \vfs that
include a verb, Subsection~\ref{sec-left-dislocation-hanging-topic} deals with left dislocation and
hanging topic and Subsection~\ref{sec-np-internal-frontings} deals with NP-internal
frontings.\todostefan{Selting93a: Linksversetzung = Mehrfache Vorfeldbesetzung}

\subsubsection{Partial verb phrase fronting}
\label{sec-pvp-is-not-mf}

In connection with cases of multiple fronting, certain examples have been discussed with supposed
cases of fronted nonfinite verbs or adjectives as well as elements dependent on them
\citep{VogelgesangDoncer2004a}. Examples of this kind of fronting are shown in (\mex{1}): 
\eal
\label{bsp-pvp}
\ex 
\gll Besonders Einsteigern empfehlen\iw{empfehlen} möchte ich Quarterdeck Mosaic, dessen gelungene grafische Oberfläche und Benutzerführung auf angenehme Weise über die ersten Hürden hinweghilft, obwohl sich die Funktionalität auch nicht zu verstecken braucht.\footnotemark\\
     especially beginners recommend want.to I Quarterdeck Mosaic whose well.designed graphic surface and user.interface on pleasant way over the first hurdles help.over although \textsc{refl} the functionality also not to hide needs\\
\footnotetext{
        c't, 9/1995, p.\,156.
}\label{bsp-besonders-einsteigern}
\glt `I would particularly recommend Quarterdeck Mosaic for beginners due to its well-designed graphic surface and user interface, which can give a helping hand over those first few hurdles. This should not, however, raise any doubts about its functionality.'
\ex 
\gll Der Nachwelt hinterlassen\iw{hinterlassen} hat sie eine aufgeschlagene \emph{Hör zu} und einen kurzen Abschiedsbrief: \ldots\footnotemark\\
     the afterworld left has she an opened \emph{Hör zu} and a short suicide.note:\\
\footnotetext{
        taz, 18.11.1998, p.\,20.
}\label{bsp-der-nachwelt-hinterlassen}
\glt `She left the rest of the world an open copy of \emph{Hör zu} and a short suicide note.'
\ex 
\gll Viel anfangen\iw{anfangen mit} konnte er damit nicht.\footnotemark\\
	 much begin could he with.it not\\
\footnotetext{
        Wochenpost, 41/1995, p.\,34.
        }
\glt `It was lost on him.'
\ex 
\gll Bei der Polizei angezeigt\iw{anzeigen} hatte das Känguruh ein Autofahrer, nachdem es ihm vor die Kühlerhaube gesprungen war und dabei fast angefahren wurde.\footnotemark\\
	 at the police reported had the kanagroo a motorist after it him before the bonnet jumped was and there.by almost run.over was\\
\footnotetext{
        taz, 18./19.01.1997, p.\,32.
      }
\glt `A motorist informed the police of the kangaroo after it had jumped in front of his car and was nearly hit.'
\ex 
\gll Aktiv am Streik beteiligt\iw{beteiligen} haben sich "`höchstens zehn Prozent"':\footnotemark\\
     active on.the strike took.part have \textsc{refl} \hspaceThis{"`}at.most ten percent\\%
\footnotetext{    
        taz, 11.12.1997, p.\,7.
}\label{bsp-aktiv-am-streik}%
\glt `Only a ``maximum of ten percent'' actively took part in the strike action:'
\zl



These kinds of constructions have been investigated extensively and there is now some consenus about the fact that
there is exactly one constituent present in the prefield.\todostefan{references}
However, one also finds suggestions like Gunkel's \citeyearpar[\page 170--171]{Gunkel2003b} to analyse sentences such as (\mex{1}) as verb-third
clauses with a flat structure. He does not, however, offer any explanation for the linearization contraints
for such clauses. If one were to analyze examples such as (\mex{1}) with a completely flat structure and with
three fronted constituents, it is not possible to explain why the constituents preceding the finite verb
act as if they also contained a middlefield, right verbal bracket and a postfield.
\ea
\gll Den Kunden sagen, daß die Ware nicht lieferbar ist, wird er wohl müssen.\\
	 the customer say that the product not deliver.able is will he PRT must\\
\glt `He will presumably have to tell the customers that the product cannot be delivered.' 
\z
On the other hand, if one assumes that the three constituents preceding the finite verb form
a verbal projection, then the individual elements in the verbal projection can be assigned to
topological fields and the order of the constituents do not require any special explanation.
See \citew[\page 82]{Reis80a}.
	
Regardless of the question whether the words preceding the finite verb have constituent status
\citep{Kathol95a} or not \citep{Gunkel2003b}, analyses which attempt to explain (\mex{0}) via local reordering
cannot account for examples such as (\mex{1}).
\eal
\ex 
\gll Das Buch gelesen glaube ich nicht, dass er hat.\footnotemark\\
     the book read believe I not that he has\\
\footnotetext{
\citep[\page 82]{Sabel2000a}.
}
\glt `I don't think that he has read the book.'
\ex 
\gll Angerufen denke ich, daß er den Fritz nicht hat.\footnotemark\\
     called think I that he the Fritz not has\\
\footnotetext{
\citep{Fanselow2002a}.
}
\glt `I don't think he has called Fritz.'
\zl
In (\mex{0}), we have elements preceding the finite verb which clearly originate in the
embedded clause and therefore cannot have reached their current position by local reordering.%
\todostefan{gb4e should use (i) in footnotes}



I have shown in \citew[\page 93--94]{Mueller2002b} that the fact that \emph{den Wagen} `the car' in (\mex{1}) bears
accusative case could not be explained if one had two independent constituents in the prefield. 
\eal 
\ex[]{
\gll Den Wagen zu reparieren wurde versucht.\\
	 the.\acc{} car to repair was tried\\
\glt `They tried to repair the car.'\todoandrew{geht denn für a: It was tried to repair the car.}%
\footnote{
      The original German sentence is actually a passive: `The car was tried to be repaired'.
	  As such a construction is impossible in English, I have translated it with an active sentence.%
}}
\ex[*]{
\gll Der Wagen     zu reparieren wurde versucht.\\
	 the.\nom{} car   to repair was tried\\
}
\zl
In constructions with the so-called `remote passive'\is{passive!remote}, the object can most certainly appear in the
nominative as is shown by (\mex{1}a).\footnote{
       Evidence for the long-distance passive from corpora can be found in \citew[\page
         136--137]{Mueller2002b} and in \citew{Wurmbrand2003a}.%
}
It is clear from (\mex{1}b) that it is possible to front the nominative NP on its own.
\eal
\ex[]{
\gll weil der Wagen zu reparieren versucht wurde\\
	 because the.\nom{} car to repair tried was\\
\glt `because they tried to repair the car'
}
\ex[]{
\gll Der Wagen wurde zu reparieren versucht.\\
	 the.\nom{} car was to repair tried\\
\glt `They tried to repair the car.'
}
\zl
The infintival construction with \emph{zu} can also be fronted on its own as shown in (\mex{1}):
\ea
\gll Zu reparieren wurde der Wagen versucht.\\
	 to repair was the car tried\\
\z
The NP \emph{der Wagen} has to bear nominative case in this kind of construction. If (\mex{-2}) were an example of fronting
the infintive and the noun phrase as a single constituent, we would also expect the nominative to be possible here, which
is in fact not what we observe.

%\subsection{Prädikative Adjektive und Adverbialien}


% Hier handelt es sich wohl um einen Spezifikator
% \ea
% Am ehesten börsentauglich dürfte der Fernverkehr sein, der schon heute auf einigen
% Strecken gute Gewinne einfährt.\footnote{
%         taz, 16.10.1999, p.\,9
% }
% }

% Im folgenden Beispiel scheint sich die PP allerdings auf das Verb \emph{zeigen} und nicht auf das
% Adjektiv zu beziehen:
% \ea
% {}[Kauf"|freudig] [im betrachteten vierten Quartal] zeigten sich aber auch Azubis/""Zivildienstleistende (21,2 Prozent)
% und Angestellte (19,6 Prozent).\footnote{
%         c't, 1/2003, p.\,69, Umfrage zum PC-Kauf zu Weihnachten 2002
% }
% \z





\subsubsection{Left dislocation and free topics}
\label{sec-left-dislocation-hanging-topic}
              
Other authors have discussed examples with left dislocation or `free topics' as cases of multiple
fronting.\todostefan{R1: add examples} These kinds of movement
have been discussed in detail by \citet{Altmann81a}. I assume that left-dislocated constituents and free topics do not move to the
prefield, but rather -- as suggested by \citet[\page 329]{Hoehle86} -- that they occupy another topological position. For this reason,
they are not relevant to the present discussion.

\subsubsection{NP-internal frontings}
\label{sec-np-internal-frontings}


\citet[\page 456]{Speyer2008a} treats examples like those in (\mex{1}) as instances of multiple fronting:
\ea
\label{ex-inzuepfners-box-der-mercedes}
\gll {}[[In Züpfners Box] [der Mercedes]] bewies, dass Züpfner zu Fuß gegangen war.\footnotemark\\
       \hspaceThis{[[}in Züpfners box \spacebr{}the Mercedes proofed that Züpfner by foot went was\\
\footnotetext{
Böll, Heinrich (1963): \emph{Ansichten eines Clowns}. Köln: Kiepenheuer \& Witsch. Quoted from
\citew[\page 456]{Speyer2008a}.
}
\glt `The Mercedes in Züpfners box was proof of Züpfner's walking.'
\z

A similar example is also discussed by \citet[\page 69]{Fanselow93a} in the context of multiple frontings:
\ea
\gll In Hamburg eine Wohnung hätte er sich besser nicht suchen sollen.\\
     in Hamburg a    flat    had  he \self{} better not search should\\
\glt `It would have been better for him not to rent/buy a flat in Hamburg.'
\z
I exclude these examples from the present discussion since they are probably best analyzed as
NP-internal frontings as suggested for instance by \citet[\page 68--69]{Fortmann96a-u} for (\mex{1}):
\ea
\gll Mit der Bahn eine Reise ist nicht geplant.\\
     with the train a journey is not planned\\
\glt `A journey by train is not planned.'
\z
\citet[\page 133]{Abb94} also treats such examples as DP-internal frontings. He remarks that the
following examples are possible in colloquial speech:
\eal
\ex 
\gll Übermorgen das Spiel gegen Kaiserslautern würde ich gern live sehen.\\
     the.day.after.tomorrow the game against Kaiserslautern would I like.to live see\\
\glt `I would like to see the game against Kaiserslautern tomorrow live.'
\ex 
\gll Der die Karten hat, der Mann, soll gleich kommen.\\
     who the tickets has the man shall soon come\\
\glt `The man with the tickets is supposed to come soon.'
\ex 
\gll An der Wand das Bild kommt mir bekannt vor.\\
     on the wall the picture comes me known \particle\\
\glt `I think I know the picture on the wall.'
\zl
The example (\mex{0}b) clearly shows that an analysis like Speyer's \citeyearpar{Speyer2008a} would
fail on such sentences since relative clauses cannot be fronted independent of the noun they modify:
\ea[*]{
\gll Der die Karten hat, soll der Mann gleich kommen.\\
     who the tickets has shall the man soon come\\
\glt Intended: `The man with the tickets is supposed to come soon.'
}
\z

% Na irgendwie gibt es ja doch Subjekte.
%Note also that subjects in apparent multiple frontings are rather rare (see Subsection~\ref{sec-subj-mf}).


\subsection{Impossible multiple frontings (Same Clause Constraint)}
\label{sec-ausgeschlossen-mvf}\label{sec-clause-mates}

As noted by Fanselow \citeyearpar[\page 99]{Fanselow87a}; \citeyearpar[\page 67]{Fanselow93a}, the constituents preceding the finite verb have to belong to
the same clause. Simultaneous fronting of several constituents from different clauses is not possible:
\eal
\label{ex-mult-front-same-verb}
\ex[]{
\gll Ich glaube  dem Linguisten nicht, einen Nobelpreis  gewonnen zu haben.\\
     I   believe the linguist   not    a     Nobel.prize won      to have\\
\glt  `I don't believe the linguist's claim that he won a Nobel prize.'
}
\ex[*]{
\gll Dem Linguisten einen Nobelpreis  glaube  ich nicht gewonnen zu haben.\\
     the linguist   a     Nobel.price believe I   not   won      to have\\
}
\ex[]{
\gll Ich habe den Mann gebeten, den Brief  in den Kasten zu werfen.\\
     I   have the man  asked    the letter in the box    to throw\\
\glt `I asked the man to post the letter in the letterbox.'
}
\ex[*]{
\gll Den Mann in den Kasten habe ich gebeten, den Brief  zu werfen.\\
     the man  in the box    have I asked      the letter to throw\\
}
\zl
This observation was verified with 3.200 examples of apparent multiple fronting that were collected by \citet{Bildhauer2011a}
in the DFG project \emph{Theorie und Implementation einer Analyse der Informationsstruktur im Deutschen unter besonderer Berücksichtigung der linken Satzperipherie} (MU 2822/1-1 and SFB 632, A6).



\subsection{Multiple frontings of idiom parts and restrictions on separate frontings}
\label{sec-idiom-parts-mf}

Many of the examples in \pref{pvp-fvg-idioms} support the claim that multiple fronting is actually fronting of a single
projection which contains part of the predicate complex. If we were to assume -- as in \citew{Mueller2000d} -- that in these
cases two independent constituents have been fronted, we would also have to assume that each of these constituents can be fronted
individually, which would be difficult to reconcile with the ungrammaticality of (\mex{1}):

\eal
\ex[*]{
\gll Ins Feuer goß gestern das Rote-Khmer-Radio Öl.\\
     in.the fire poured yesterday the Rote-Khmer-Radio oil\\
}
\ex[*]{
\gll Aufs i setze der Bürgermeister von Miami das Tüpfel, als er am Samstagmorgen von einer schändlichen 
Attacke der US-Regierung sprach.\\
     \spacebr{}on.the i put the mayor of Miami the dot  as he on Saturday.morning from a shameful attack of.the US-government spoke\\
}
\ex[*]{
\gll Weg bekamen natürlich auch alte und neue Regierung ihr Fett.\\
     away got of.course also old and new government their fat\\ 
}
% Die gehen, wenn man die Adverbien ganz wegläßt.
% \ex[*]{
% Unter Schock stehen (schwer) deshalb (schwer) zur Zeit (schwer) zwei der hervorragendsten Kräfte ihrer Branche (schwer).
% }
% \ex[*]{
% Ins Gericht gehen (hart) die Wirtschaftsforscher (hart) zudem (hart) mit der sogenannten Fluggastgebühr (hart).
% }
\ex[*]{
\gll Rechnung tragen im unterfränkischen Raum die privaten, städtischen und kommunalen Musikschulen dem Zeitgeist.\\
     account carry in.the lower.Franconian area the private, urban and communal music.schools the Zeitgeist\\
}
\zl
One would have to formulate complex constraints which would ensure that, for example,
\emph{Rechnung} `account' could
only be fronted if \emph{dem Zeitgeist} `the Zeitgeist' were also fronted. All in all, this sort of explanation would turn out
to be more complicated than one which assumes that part of a predicate complex is fronted.

\subsection{Scope of negation and fronting}

Furthermore, Fanselow notes that negation has scope over everything preceding 
the finite verb.
\eal
\ex  
\gll Nicht der Anna einen Brief hätte er schreiben sollen, sondern der Ina eine Postkarte.\\
     not the Anna a letter had he write should rather the Ina a postcard\\
\glt `He shouldn't have sent Anna a letter, but rather Ina a postcard.'
\ex 
\gll Nicht am Sonntag einen Brief hätte er schreiben sollen, sondern am Samstag seinen Vortrag für Potsdam.\\
     not on Sunday a letter had he write should rather on Saturday his presentation for Potsdam.\\
\glt `He shouldn't have written a letter on Sunday, he should have written his presentation for Potsdam on Saturday.'
\zl

The data discussed here can be easily accounted for if one assumes that the fronted elements are arguments of
an empty head or that they modify some kind of empty head. This null head has the properties of a verb in the
remaining sentence, which explains the fact that the fronted constituents cannot be dependents of different verbs.
Corresponding suggestions in this direction have been made by \citet{Fanselow93a} and \citet[\page 1634]{Hoberg97a},
although they did not work out the details of these suggestions.

As the examples in (\mex{0}) show, the negation cannot be analyzed as constituent negation. It
follows that \emph{nicht} is a separate constituent in (\mex{0}) and not part of an NP. Again this is entirely
unproblematic in approaches that assume that \emph{nicht} is part of a larger verbal constituent
\emph{nicht der Anna einen Brief} `not the Anna a letter'.

%% Note also that fronting of negation alone is rather restricted, but as \citet{Ulvestad75a} showed in
%% a careful corpus study, attested examples can be found.
%% \ea[*]{
%% \gll Nicht ging Peter einkaufen.\\
%%      not went Peter shopping\\
%% \glt `Peter didn't go shopping.'
%% }
%% \z
%% The following two examples were provided by \citet{Reis80a} and
%% \citet{Hoberg81a} and the examples in (\mex{2}) are attested examples that show that the fronting of
%% the negation is possible. The examples (\ref{ex-nicht-weht}) and (\ref{ex-nicht-etwa-haben}) can
%% also be found in \citew[\page 348]{Mueller99a}.
%% \eal
%% \ex Das alles erwähnte der Autor. Nicht hat er hingegen berücksichtigt,
%%               daß \ldots{}\footnote{
%% \citep*[\page 72]{Reis80a}
%% }
%% \ex Nicht ahnten wir, daß Franz von Papen das Spiel der Intrigen fortspann 
%%               -- \ldots{}\footnote{
%%   \citep*[\page 161]{Hoberg81a}
%% }
%% \zl
%% \eal
%% \ex Nicht aber ist der abtrennbare Teil des Verbs auch stets ein Satzglied.\footnote{
%%  Im Haupttext von \citep[\page 365]{Stechow79}.
%% }
%% \ex Noch nicht erhalten w[i]r dagegen (2).\footnote{
%% Im Haupttext von \citep[\page 440]{Stechow79}.
%% }
%% %       \item Vorläufig noch nicht schaffe ich allerdings zwei Lesarten für
%% %       "`Jedermann liebt jemanden"'.    (Im Haupttext von
%% %              \citep[S.\,464]{Stechow79})     
%% \ex Du bist, wie ich sehe, ein Mann des Buches, und nicht frommt es dir, dem Einsamen,
%%         obdachlos in Bettlerkleidung zu stromern.\footnote{
%% Michail Bulgakow, \emph{Der Meister und
%%           Margarita}. München: Deutscher Taschenbuch Verlag. 1997, S.\,420}

%% \ex\label{ex-nicht-weht} Der Wind ist auch mal aus der Puste, aber nicht weht er nur selten.\footnote{
%% taz, taz-mag, 07./08.98, S. 16. als Antwort auf die Frage: "`Was macht
%%                der Wind, wenn er nicht weht?"'
%% }
%% \ex\label{ex-nicht-etwa-haben} In der Logik dieses Modells liegt, daß die von Tovee befragten Studenten
%%               alle aus reinem Fortpflanzungsinteresse denselben "`Playmate"'-Körperbau
%%               bevorzugen. Nicht etwa haben Modeplakate oder Hochglanzmagazine den Blick
%%               der Männer auf diese Modelmaße geeicht.\footnote{
%% Spiegel 47/98, S.\,238}

%% \ex Die Athleten jedoch haben eine feine Antenne für das, was
%%                 in diesem Verband geht, und auch für das, was nicht geht.
%%               Nicht geht ein in welcher Form auch immer geratenes Mitspracherecht
%%               ging aus der Einaldung hervor: [\ldots]\footnote{
%% taz, 25.09.2003, S.\,13}

%% \ex Auch nicht hilft die später lancierte Behauptung, die SPD-Linken hätten ihn zu sehr unter Druck
%% gesetzt - die beteuern ihm nun seit Wochen ihr allergrößtes Vertrauen.\footnote{
%% taz, 08.06.2005, S.\,6}
%% \zl



\subsection{The order of fronted constituents}
\label{sec-abfolge}

As was noted by \citealp[\page 6--7]{Luehr85a}, \citet[\page 412--413]{Eisenberg94a}, and \citet[\page
  1625--]{Hoberg97a},\todostefan{Check Lühr and Hoberg} the order of the fronted constituents is relatively fixed. If the order of the
elements in \pref{bsp-alle-traeume-gleichzeitig} and \pref{bsp-dauerhaft-mehr-arbeitsplaetze} is
changed as in the following examples, the result is sentences that are degraded in aceptability:
\eal
\ex[?*]{
\gll Gleichzeitig   alle Träume lassen sich       nur selten verwirklichen.\\
     simultaneously all  dreams let    themselves only seldom realize\\
\glt `Very rarely, all dreams can be realized simultaneously.'
}
%% \ex[*]{
%% Nach der Zugehörigkeit Personen bezeichnen auch \emph{Gesellschafter}, \emph{Gewerkschafter} \ldots.
%% }
\ex[?*]{
\gll Mehr Arbeitsplätze dauerhaft   gebe es erst, wenn \ldots.\\
     more jobs          permanently gives it first when \\
}
\zl
The observation that the order in apparent multiple frontings corresponds to the unmarked order in
the \mf was verified with 3.200 examples of apparent multiple fronting that were collected by \citet{Bildhauer2011a}
in the DFG project \emph{Theorie und Implementation einer Analyse der Informationsstruktur im
  Deutschen unter besonderer Berücksichtigung der linken Satzperipherie} (MU 2822/1-1 and SFB 632,
A6).\footnote{
  The database is available at \url{https://clarin.ids-mannheim.de/SFB632/A6}.
}


These differences can be explained if one assumes that there is a single verbal projection (the
projection of a single verbal head) present in the prefield. The verbal projection contains a
middle-field, right verbal bracket occupied by the empty head, and even a postfield in certain
cases. The order of the fronted elements is therefore subject to the same restrictions that are
known for the ordering of elements in the middle-field/postfield:

\eal
\ex[]{
\gll weil sich nur selten alle Träume gleichzeitig verwirklichen lassen\\
	 because \textsc{refl} only seldom all dreams simultaneously realise let\\
\glt `because only seldom can all of your dreams be realised at the same time'
}
\ex[??]{
\gll weil sich nur selten gleichzeitig alle Träume verwirklichen lassen\\
     because \textsc{refl} only seldom simultaneously all dreams realise let\\
}
\zl
\eal
\ex[]{
\gll weil    es dauerhaft mehr Arbeitsplätze erst gebe, wenn \ldots.\\
     because it constantly more jobs PRT give if\\
\glt `because there will only be a constant supply of jobs if/when \ldots'
}
\ex[?*]{
\gll weil es mehr Arbeitsplätze dauerhaft erst gebe, wenn \ldots.\\
     because it more jobs constantly PRT give if\\
}
\zl



\subsection{Summary of the data discussion}

I have shown that various kinds of constituents can co-occur in the prefield: arguments, adjuncts and predicatives
can be fronted together with another constituent. The number of constituents preceding the finite verb is by no
means limited to two.

The sequence of the fronted elements corresponds to the order the constituents would have in the middle-field.
This supports an analysis which assumes that multiple fronting involves a complex verbal projection, which contains its own
topological fields: middlefield, right verbal bracket and postfield. The right verbal bracket is occupied by a silent
verbal head.

I showed that multiple fronting with idioms is quite common and that certain parts of phraseologisms cannot be
fronted individually. The constituent parts of a phraseologism can be realised inside this projection, but individual fronting
is not possible.

The observation that only elements from the same clause can be fronted together can also be explained by the assumption of a silent
verbal head.
 





\section{The analysis}
\label{sec-analyse-mf}\label{sec-analyse-mf-vn}\label{sec-vn}

A prerequisite for the analysis of apparent multiple frontings are the following
sub-analyses: 1) an analysis of V1-order derived by verb movement, 2) an analysis of the verbal
complex by means of argument attraction and 3) an analysis of fronting as a long-distance dependency.
These three ingredients have already been provided in Chapter~\ref{chap-german-sentence-structure}
and I will show in Subsection~\ref{sec-mult-front-complex-formation} how they interact in the
analysis of apparent multiple frontings. Section~\ref{sec-left-dislocation} discusses a potential problem with left
dislocation, Subsection~\ref{sec-extraposition} talks about extraposition in complex prefields and Subsection~\ref{sec-unwanted-traces}
deals with traces in unwanted positions.

\subsection{Multiple frontings as lexical rule and predicate complex formation}
\label{sec-mult-front-complex-formation}

It was shown in the data discussion in Section~\ref{sec-ausgeschlossen-mvf} that elements can only be fronted together
if they are dependent on the same head/predicate complex.\footnote{
		The examples from Jacobs with sentence adverbs behave differently. It is certainly
		possible that there are cases where focus particles or sentence adverbs and a constituent
		from an embedded clause occur together before the finite verb.%
}
		
%
% Ist nicht so stichhaltig, wenn man Reanalyze annimmt.
% \citet[\page 246]{BH2001a} diskutieren die Beispiele in (i).
% \eal
% \ex[]{
% Sogar [gegen die Regierung]$_i$ hat sie [eine Proklamation \_$_i$] unterzeichnet.
% }
% \ex[*]{
% {}[Eine Proklamation sogar gegen die Regierung]$_i$ hat sie  \_$_i$  unterzeichnet.
% }
% \zl
% (i.b) zeigt, daß es nicht sinnvoll ist, anzunehmen, daß \emph{sogar} die PP \emph{gegen die Regierung}
% modifiziert.



\citet{Fanselow93a} and \citet[\page 1634]{Hoberg97a} have therefore suggested positing a silent head which
can then be combined with the arguments and adjuncts which actually belong to the verb.
In what follows, I will attempt to formalize and define this analysis more precisely.  
Like Hoberg, I assume that the silent head is a part of the predicate complex and that fronting is analogous
to partial fronting of a predicate complex.
Example (\ref{bsp-zum-zweiten-mal-die-Weltmeisterschaft}) would therefore have the following structure:
\ea
\label{ex-zum-zweiten-anal}%
\gll {}[\sub{VP} [Zum zweiten Mal] [die Weltmeisterschaft] \_\sub{V} ]$_i$ errang$_j$ Clark 1965 \_$_i$ \_$_j$.\\
     {}          \spacebr{}to.the second time  \spacebr{}the world.championship {} {} won Clark 1965\\
\z
\_$_j$ represents the movement trace, which is left behind by the verb \emph{errang} in initial position.
\_$_i$ is the trace of the extraction of \emph{zum zweiten Mal die Weltmeisterschaft}
`for the second time the world's championship',  which also binds it.
\_\sub{V} stands for the silent verbal head in the prefield.
\citet[\page 69]{Fanselow93a} suggests treating this empty head in a similar way to the empty elements present
in gapping constructions and argues against a fronting analysis with verb trace with the following examples
of particle verbs:
\eal
\ex[*]{
\gll Die Anette an sollte man lieber nicht mehr rufen.\\
	 the Anette on should one rather not more call\\
\glt Intended: `It's probably better if you don't call Anette.'
}
\ex[*]{
\gll Mit dem Vortrag auf sollte er lieber hören.\\
	 with the presentation on should he rather stop\\
\glt Intended: `It would be better if he were to end his presentation.'
}
\ex[*]{
\gll Dem Minister einen Aufsichtsratsposten zu hätte er niemals schanzen sollen.\\
	 the minister a supervisory.board.post to had he never ensure should\\
\glt Intended: `He should have never made sure that the minister got a position on the supervisory board.'
}
\zl
Fanselow argues that an analysis which treats fronting of multiple constituents as including movement of a 
corresponding trace should predict that the sentences in (\mex{0}) are grammatical.\footnote{
		Although see \citew{Fanselow2003c} for an analysis of particle fronting
		as pars"=pro"=toto movement.%
}
As these sentences are clearly ungrammatical, Fanselow assumes that these kinds of movement analyses are not
adequate. However, the following examples in (\mex{1}) show that particles can indeed occur with other constituents
in the prefield.

\eal\label{ex-adv-particle}
\ex\iw{zurechtkommen}%
\gll Gut \emph{zurecht} \emph{kommt} derjenige, der das Leben mit all seinen Überraschungen annimmt und dennoch verantwortungsvoll mit sich umgeht.\label{ex-gut-zurecht}\footnotemark\\
	 good to.right comes the.one who the life with all its surprises accepts and PRT responsibly with \textsc{refl} treats\\
\footnotetext{
        Balance, broschure of TK-series for healthy living, Techniker Kran\-ken\-kas\-se. 1995.
      }
\glt `Those who accept life with all its little surprises, yet still act responsibly, are the ones who will cope best.'
\ex\iw{klarkommen} 
\gll Ich bin alleinstehende Mutter, und so gut \emph{klar} \emph{komm} ich nicht.\footnotemark\\
	 I   am single mother and so good clear come I not\\
\footnotetext{
        radio show, 02.07.2000, I would like to thank Andrew McIntyre\ia{MacIntyre@McIntyre} for this example.
    }
\glt `I am a single mother and I am really not coping that well.'
\ex 
\gll Den Atem \emph{an} \emph{hielt} die ganze Judenheit des römischen Reichs und weit hinaus über die Grenzen.\footnotemark\\
	 the breath in held the whole Jewish.people of.the Roman empire and wide further over the borders\\
\footnotetext{
        Lion Feuchtwanger, \emph{Jud Süß}, p.\,276, citied in \citew[\page 56]{Grubacic65a}.
}
\glt `The entire Jewish population of the Roman Empire held their breath and the same was true far past its borders.'
\ex\iw{umhinkönnen}%
\gll Nicht \emph{umhin} \emph{konnte} Peter, auch noch  einen Roman über das Erhabene zu schrei\-ben.\label{ex-grewendorf-nicht-umhin}\footnotemark\\
	not around could Peter also \textsc{prt} a novel over the sublime to write\\
\footnotetext{
        \citep*[\page 90]{Grewendorf90a}.
      }
\glt `Peter couldn't get around writing another novel about the sublime.'
\ex\iw{herauskommen}%
\gll Die Zeitschrift \frq Focus\flq{} hat vor einiger Zeit auch die Umweltdaten deutscher Städte miteinander verglichen. Dabei \emph{heraus} \emph{kam} u.\,a., daß Halle an der Saale die leiseste Stadt Deutschlands ist.\footnotemark\\
	 the magazine \frq Focus\flq{} has before some time also the environmental.data German cities with.eachother compared there.at out came amongst.other.things that Halle an der Saale the quietest city Germany's is\\	 
\footnotetext{
        Max Goldt, \emph{Die Kugeln in unseren Köpfen}. Munich:\ Wilhelm Heine Verlag. 1997, p.\,18.
}
\glt `Not too long ago, the magazine \emph{Focus} compared environmental data on various German cities. As a result, they found out, among other things, that Halle an der Saale was the quietest city in Germany.'
\ex\label{ex-los-damit} 
\gll \emph{Los} damit \emph{geht} es schon am 15. April.\footnotemark\\
	  off there.with goes it PRT on 15. April\\
\footnotetext{
        taz, 01.03.2002, p.\,8.
    }
\glt `The whole thing starts on the 15th April.'
\ex 
\gll Sein Vortrag wirkte [\ldots] ein wenig arrogant, nicht zuletzt wegen seiner Anmerkung, neulich habe er bei der Premiere des neuen "`Luther"'"=Films in München neben Sir Peter Ustinov und Uwe Ochsenknecht gesessen. Gut \emph{an} \emph{kommt} dagegen die Rede des Jokers im Kandidatenspiel: des Thüringer Landesbischofs Christoph Kähler (59).\footnotemark\\
    his presentation seemed {} a bit arrogant not lastly because.of his comment recently has he at the premiere of.the new \hspaceThis{"`}Luther.film in Munich next.to Sir Peter Ustinov and Uwe Ochsenknecht sat good on comes there.against the speech of.the joker in candidate.game of.the Thüring state.bishop Christoph Kähler (59)\\
\footnotetext{
        taz, 04.11.2003, p.\,3.%
}
\glt `His presentation came across somewhat arrogant. Not least because of his comment that he recently sat next to Sir Peter Ustinov and Uwe Ochsenknecht at the premiere of the new Luther film. What did get a good reception was the speech by the wild card in the election race: the Thüringen state bishop Christoph Kähler (59).' 
\ex\iw{hinzukommen}
\gll Erschwerend \emph{hinzu} \emph{kommt} der Leistungsdruck, dem auch die Research"=Abteilungen unterliegen.\label{ex-adv-non-true-particle}\\
	 difficultly there.to comes the pressure.to.perform that also the Research"=departments underlie\\
\glt `What makes it even more difficult is the pressure to perform, which the research departments are also under.'
\ex\iw{vornliegen}
\gll Immer  noch  mit  Abstand  \emph{vorn} \emph{liegt} Reiseunternehmer Kuoni.\footnotemark\\
	 always still PRT   with distance  in.front lies travel.company Kuoni\\
\footnotetext{
        \citep[\page 126]{CT75a-u}.
      }
\glt `The travel company Kuoni is always ahead by some distance.'
\ex  
\gll Den Umschwung im Jahr 1933 stellt Nolte als "`Volkserregung"' und "`Volksbewegung"' dar. (\ldots) Nicht \emph{hinzu} \emph{setzt}\iw{hinzusetzen} Nolte Zeugnisse republiktreuer Sozialdemokraten  und Zentrumsleute, die im Januar 1933 von lähmendem Entsetzen befallen (\ldots) waren.\footnotemark\\
	 the turnaround in.the year 1933 presents Nolte as excitment.of.the.people and \hspaceThis{"`}people's.movement PRT {} not here.to places\iw{hinzusetzen} Nolte testimonies loyal.to.the.repbulic social.democrats and centre.people who in January 1933 of paralyising horror struck {} were\\
\footnotetext{
        Die Zeit, 19.03.1993, p.\,82. Cited in \citew[\page 1633]{Hoberg97a}.
      }\label{ex-nicht-hinzu-setzt}
\glt `Nolte presents the turnaround in 1933 as `animation of the people' and a `people's movement'. Nolte does not include testimonies of Social Democrats and people positioned the centre of the political spectrum, who were struck by paralysing horror in January 1933.'
\zl



These data show that structures with a fronted particle cannot be ruled out in general. I assume that such structures have to be made
available by syntax in general and that there are certain stipulations for fronting which are responsible for Fanselow's examples
being ungrammatical. For more on fronting of verb particles and further data, see \citew{Mueller2002b,Mueller2002d}.

I will assume then that there is an ordinary verb trace in the prefield and will follow Hoberg in assuming that the example of fronting in (\ref{ex-zum-zweiten-anal})
should be analyzed parallel to the fronting of a partial projection of a verbal complex. Hoberg describes the idea for her analysis in a footnote and does
not go into any details. In particular, it remains unexplained how the trace in (\ref{ex-zum-zweiten-anal}) is licensed.

In what follows, I wish to delve a little deeper into the details of the analysis. I will start with a discussion of the less complex examples in (\mex{1}).
\eal
\ex\label{dass-clark} 
\gll dass Clark 1965 zum zweiten Mal die Weltmeisterschaft errungen hat\\
	 that Clark 1965 to.the second time the world.championship won has\\
\glt `that Clark won the world championship for the second time in 1965'
\ex\label{zum-zweiten-mal-hat} 
\gll {}[\sub{VP} [Zum zweiten Mal] errungen]$_i$  hat$_j$ Clark die Weltmeisterschaft 1965 \_$_i$ \_$_j$.\\
     {}          \spacebr{}to.the second time   won has Clark the world.championship 1965\\
\ex\label{zum-zweiten-mal-errungen} 
\gll {}[\sub{VP} [Zum zweiten Mal] [die Weltmeisterschaft] errungen]$_i$ hat$_j$ Clark 1965 \_$_i$ \_$_j$.\\
     {}          \spacebr{}to.the second time \spacebr{}the world.championship  won has Clark  1965\\
\zl



In (\ref{dass-clark}), the relations between the various elements should be clear. The auxiliary
verb \emph{hat} `has' selects the participle \emph{errungen} `won' and they together form a verbal complex. The arguments of the verbal complex can be permutated in the middle-field and adjuncts
can appear between the arguments. In (\ref{zum-zweiten-mal-hat}), the auxiliary is in initial position. The verb with which
\emph{hat} would have normally formed a complex is now in the prefield. The extraction trace \_$_i$ has the same arguments
as the verb in initial position, namely \emph{Clark} and \emph{die Weltmeisterschaft} `the world championship'. The verb trace \_$_j$, which corresponds to
\emph{hat} in initial position, forms a verbal complex with the extraction trace \_$_i$, which then requires these two arguments.
For this reason, \emph{Clark} and \emph{die Weltmeisterschaft} can now appear in the middle-field. In (\ref{zum-zweiten-mal-errungen}),
the extraction trace \_$_i$ corresponds to the verb phrase \emph{zum zweiten Mal die
  Weltmeisterschaft errungen} `for the second time the world championship'. When the auxiliary is
combined with this trace, it is not possible for a further complement to be attracted, since \emph{die Weltmeisterschaft} is already a
complement of \emph{errungen}. Therefore, only the subject of \emph{errungen} can appear in the middle-field.
(\ref{ex-zum-zweiten-anal}) can be explained as follows: I assume an empty verb in the prefield, which takes \emph{die Weltmeisterschaft} as
complement and \emph{zum zweiten Mal} as an adjunct. The properties of this head are determined by the other material in the main clause, i.e.
the arguments of \emph{errang} which occur in the middle-field cannot be realised in the prefield -- and adjuncts which occur in the prefield
must be compatible with the semantic properties of \emph{errang}. Sentences such as (\mex{1}) are not possible:
\eal\label{bsp-zu-viele-komplemente-adjunkte}
\ex[*]{
\gll Zum zweiten Mal die Weltmeisterschaft errang Clark 1965 die Goldmedaille.\\
	 to.the second time the world.championship won Clark 1965 the gold.medal\\
\glt Intended: `Clark won the gold medal for the second time during the world championships in 1965.'
}
\ex[*]{
\gll Drei Stunden lang die Weltmeisterschaft errang Clark 1965.\\
	 the hours long the world.championship won Clark 1965\\
\glt Intended: `Clark won the world championship for three hours in 1965.'
}
\zl



In (\mex{0}a), both \emph{die Weltmeisterschaft} `the world championship' and \emph{die
  Goldmedaille} `the gold medal' would fulfil the role of object and in (\mex{0}b),
the adjunct \emph{drei Stunden lang} is not compatible with \emph{errang}.

We can only explain this if we assume some relation between \emph{errang} (or the verb trace \_$_j$) and the extraction trace
\_$_i$ in (\ref{ex-zum-zweiten-anal}), repeated here as (\mex{1}). The extraction trace is in a filler"=gap relation to the
complex projection in the prefield. What is missing is a relation between the extraction trace \_$_i$ and the overt verb.
\ea
\label{ex-zum-zweiten-anal-zwei}%
\gll {}[\sub{VP} [Zum zweiten Mal] [die Weltmeisterschaft] \_\sub{V} ]$_i$ errang$_j$ Clark 1965 \_$_i$ \_$_j$.\\
      {}         \spacebr{}to.the second time \spacebr{}the world.championship {} {} won Clark 1965\\
\z 
It is for this reason that I suggest a lexical rule which licenses a further lexical item for each verb that is able to select a trace
with which it forms a predicate complex. The trace has to have the same valence as the original verb and all arguments which are
not realised together with the trace are attracted by the verb. (\mex{1}) shows the syntactic aspects of this lexical rule:

\eas
\label{lr-mult-vf}
Lexical rule for multiple fronting (preliminary version):\\
\begin{tabular}[t]{@{}l@{}}
\ms{
synsem$|$loc & \ibox{1} \ms{ cat$|$head & \ms[verb]{
                                          initial & \ibox{2}\\
                                          vform   & \ibox{3}\\
                                          }\\
% XCOMP<>
                           }\\
} $\mapsto$\\
\onems{
synsem$|$loc$|$cat \ms{ head & \ms[verb]{
                                          initial & \ibox{2}\\
                                          vform   & \ibox{3}\\
                                          }\\
               spr & \eliste\\
               comps & \ibox{4} $\oplus$ \sliste{ \onems{ loc$|$cat \onems{ head \ms[verb]{
                                                                                   dsl & \ibox{1}\\
                                                                                  }\\
                                                                    comps \ibox{4}\\
                                                                   }\\
                                                           lex +\\
                                                         }}\\
                       }\\
}
\end{tabular}
\zs



\noindent
The trace of the silent verbal head \_\sub{V} in \pref{ex-zum-zweiten-anal-zwei} is identical to
the trace which is responsible for verb movement in the analysis of verb-first order. The details
of verb movement are explained in Section~\ref{sec-v1}. There, I give the following entry for the
verb trace:
\ea
\label{le-verbspur2-mf}
Head movement trace as suggested by \citew[\page 207]{Meurers2000b}:
\onems{
phon \eliste\\
synsem$|$loc \ibox{1} \onems{ cat$|$head$|$dsl \ibox{1} }\\
}
\z

Figure~\vref{anal1} shows the analysis of \pref{ex-zum-zweiten-anal-zwei} when using this trace. I assume
that the ouput of the lexical rule in \pref{lr-mult-vf} forms the input of the verb-first lexical rule. The rule
for verb movement, which is also explained in detail in Section~\ref{sec-v1}, has the following
form:

\eas
\label{lr-verb-movement2-mf}
Lexical rule for verb in initial position:\\
\begin{tabular}[t]{@{}l@{}}
\ms{
synsem$|$loc & \ibox{1} \ms{ cat$|$head & \ms[verb]{ vform & fin\\
                                                     initial & $-$\\
                                             }\\
                  }\\
} $\mapsto$\\*
\onems{
synsem$|$loc \onems{ cat  \ms{ head & \ms[verb]{ vform & fin\\
                                                     initial & $+$\\
                                             }\\
                           spr   & \eliste\\
                           comps & \sliste{ \onems{ loc \onems{ cat \ms{ head & \ms[verb]{
                                                                                dsl & \ibox{1}\\
                                                                               }\\
                                                                        comps & \eliste\\
                                                                    }\\
                                                           cont \ibox{2}\\
                                                         }\\
                                              }}\\
                         }\\
                     cont \ibox{2}\\
             }\\
}
\end{tabular}
\zs



It is important that the verb trace on the far right corresponds to the right-hand side of the rule
in \pref{lr-mult-vf}. 
\begin{figure}
\resizebox{!}{\textheight-3\baselineskip}{%
%\resizebox{\textwidth}{!}{%
\begin{sideways}
\begin{forest}
sm edges
[V{[\comps \eliste]}
	[V\ibox{5}\,{[\comps \sliste{ \ibox{1} }]}
		[PP
			[zum zweiten Mal;to.the second time, roof]]
		[V{[\comps \sliste{ \ibox{1} }]}
			[\ibox{2} NP{[\textit{acc}]}
				[die Weltmeisterschaft; the world championship, roof]]
			[V{[\comps \sliste{ \ibox{1}, \ibox{2} }]}
				[\trace]]]]
	[V\feattab{\comps \eliste,\\
                   \textsc{slash} \sliste{ \ibox{5} }}
		[V{[\comps \sliste{ \ibox{6} }]}
			[V{[\comps \ibox{3} $\oplus$ \sliste{ \ibox{4} }]}, tier=np,edge label={node[midway,right]{V1-LR}}
				[V{[\comps \sliste{ \ibox{1}, \ibox{2} }]}, tier=trace,edge label={node[midway,right]{MF-LR}}
					[errang;won]]]]
		[\ibox{6} V\feattab{\comps \eliste,\\
                                    \textsc{slash} \sliste{ \ibox{5} }}
			[\ibox{1} NP{[\textit{nom}]}, tier=np
				[Clark;Clark]]
			[V\feattab{\comps \ibox{3},\\
                                   \textsc{slash} \sliste{ \ibox{5} }}
				[\ibox{4}\feattab{\textsc{loc} \ibox{5} {[\comps \ibox{3}]},\\
                                                  \textsc{slash} \sliste{ \ibox{5} }}, tier=trace
					[\trace]]
				[V{[\comps \ibox{3} \sliste{ \ibox{1} } $\oplus$ \sliste{ \ibox{4} }]}
					[\trace]]]]]]
\end{forest}
\end{sideways}
}
\caption{\label{anal1}Analysis of multiple frontings with an empty head}
\end{figure}\todostefan{Unbedingt noch H, A, CL-Anotation machen}


The verb trace in the prefield is combined with \emph{die Weltmeisterschaft} `the world
championship' as an argument and \emph{zum zweiten Mal} `for the second time' as an adjunct to form
the phrase \emph{zum zweiten Mal die Weltmeisterschaft} `for the second time the world
championship'. The entire phrase is the filler in a long-distance dependency that was
introduced by the extraction trace directly next to \emph{Clark}. The local properties of the filler
\iboxb{5} are identical to those of the extraction trace. The arguments of the extraction trace
attracted by the lexical entry for \emph{errang} are licensed by the lexical rule (\ref{lr-mult-vf})
(see \iboxt{3} in the trace for verb movement furthest to the right). Therefore, the \compsl of the
trace of verb movement and the extraction trace contain exactly those elements which cannot appear
as arguments of the verb trace in the prefield, namely \iboxt{1} in Figure~\ref{anal1}.

As we have seen from the discussion of (\ref{bsp-zu-viele-komplemente-adjunkte}), there has to be a connection between the trace in the prefield and
the verb in the remainder of the sentence. This connection is established in the same way as the connection between the verb in initial position and 
the verb trace at the end of sentence: the head feature \dsl is used to represent the required
information. Figure~\vref{anal2} shows the identity of the respective \dsl features \iboxb{7} in addition to the valence information and the \textsc{nonloc} information. 

\begin{figure}
\resizebox{!}{\textheight-3\baselineskip}{%
\begin{sideways}
\begin{forest}
sm edges
[V{[\comps \eliste]}
	[V\ibox{5}\,{[\dsl \ibox{7}, \comps \sliste{ \ibox{1} }]}
		[PP
			[zum zweiten Mal;to.the second time, roof]]
		[V{[\dsl \ibox{7}, \comps \sliste{ \ibox{1} }]}
			[\ibox{2} NP{[\textit{acc}]}
				[die Weltmeisterschaft;the world championship, roof]]
			[V\feattab{\dsl \ibox{7},\\
                                   \comps \sliste{ \ibox{1}, \ibox{2} }}
				[\trace]]]]
	[V\feattab{\comps \eliste,\\
                   \textsc{slash} \sliste{ \ibox{5} }}
		[V{[\comps \sliste{ \ibox{6} }]}
			[V{[\comps \ibox{3} $\oplus$ \sliste{ \ibox{4} }]}, tier=np,edge label={node[midway,right]{V1-LR}}
				[V{[\comps \sliste{ \ibox{1}, \ibox{2} }]}, tier=trace,edge label={node[midway,right]{MF-LR}}
					[errang;won]]]]
		[\ibox{6} V\feattab{\comps \eliste,\\
                                    \textsc{slash} \sliste{ \ibox{5} }}
			[\ibox{1} NP{[\textit{nom}]}, tier=np
				[Clark;Clark]]
			[V\feattab{\comps \ibox{3},\\
                                   \textsc{slash} \sliste{ \ibox{5} }}
				[\ibox{4} {[\begin{tabular}[t]{@{}l@{}}
                                           \textsc{loc} \ibox{5} [\begin{tabular}[t]{@{}l@{}}
                                                                       \dsl \ibox{7},\\
                                                                       \comps \ibox{3}\,],
                                                                       \end{tabular}\\
                                                   \textsc{slash} \sliste{ \ibox{5} }]
                                           \end{tabular}}, tier=trace
					[\trace]]
				[V{[\comps \ibox{3} \sliste{ \ibox{1} } $\oplus$ \sliste{ \ibox{4} }]}
					[\trace]]]]]]
\end{forest}
\end{sideways}
}
\caption{\label{anal2}Representation of valence information}
\end{figure}
The properties of the verb \emph{errang} are listed under \dsl in the \compsv of the item licensed
by the lexical rule in (\ref{lr-mult-vf}). The complement in the predicate complex \iboxb{4} is
realized by an extraction trace. The \localv of this trace \iboxb{5} is identical to the \localv
of the filler. Since \dsl is a head feature and therefore inside of the \localv, the \dsl value of
the complement of the verbal complex of \emph{errang} is identical to the \dsl value of the phrase 
\emph{zum zweiten Mal die Weltmeisterschaft}. As \dsl is a head feature, it is also ensured that the 
\dsl value is identical in all the projections of the verb trace in the prefield. In the verb trace
(\ref{le-verbspur2-mf}), the structure sharing between \local and \dsl ensures that the \compsv of the verb trace
matches the valence information under \dsl. In this way, we can ensure that the trace allows only those elements
which were required by the original verb.

The representation of meaning of the constituents in the prefield and in the trace is done in an analogous manner:
The semantic content (\iboxt{5}) in (\mex{1}) is taken over from the projection of the trace that is selected by the verb in
initial position. (\mex{1}) shows the corresponding modified lexical
rule:
\eas
\label{lr-mult-vf-zwei}
Lexical rule for multiple fronting:\\
\resizebox{\linewidth}{!}{%
\begin{tabular}{@{}l@{}}
\ms{
synsem$|$loc & \ibox{1} \onems{ cat$|$head \ms[verb]{
                                          initial & \ibox{2}\\
                                          vform   & \ibox{3}\\
                                          }\\
                              }\\
} $\mapsto$\\
\onems{
synsem$|$loc \onems{ cat \ms{ head & \ms[verb]{
                                          initial & \ibox{2}\\
                                          vform   & \ibox{3}\\
                                          }\\
                              spr & \eliste\\
                    comps & \ibox{4} $\oplus$ \sliste{\ms{ loc & \onems{ cat \ms{ head & \ms[verb]{
                                                                                            dsl & \ibox{1}\\
                                                                                           }\\
                                                                                  %spr & \eliste{}
                                                                                  %SPR ist leer,
                                                                                  %weil die finiten
                                                                                  %Verben immer
                                                                                  %leere SPR-Werte
                                                                                  %haben, deshalb
                                                                                  %kann auch über
                                                                                  %den Trace nie
                                                                                  %etwas anderes
                                                                                  %projiziert werden
                                                                                  %und wir müssen
                                                                                  %hier nichts
                                                                                  %sagen. Oder? 08.06.2015
                                                                                  comps & \ibox{4}\\
                                                                                  }\\
                                                                          cont \ibox{5}\\ 
                                                                     }\\
                                                               lex & $+$\\
                                                           }}\\
                            }\\
        cont \ibox{5}\\
      }\\
}
\end{tabular}}
\zs
Inside the trace in (\ref{le-verbspur2-mf}), a connection is made between the meaning of the original verb, which is represented
under \textsc{dsl}, and the meaning of the trace, which is represented under \local.
Figure~\vref{anal3} shows the aspects of the semantic representation with the modified lexical rule and the trace (\ref{le-verbspur2-mf}).

\begin{figure}
\resizebox{!}{\textheight-3\baselineskip}{%
\begin{sideways}
\begin{forest}
sm edges
[V{[\textsc{cont} \ibox{3}\,]}
	[V\ibox{5}\,{[\dsl \ibox{7}, \textsc{cont} \ibox{3}\, 2*erringen{(c,w)}]}
		[PP
			[zum zweiten Mal;to.the second time, roof]]
		[V{[\dsl \ibox{7}, \textsc{cont} erringen{(c,w)}]}
			[\ibox{2} NP{[\textit{acc}]}
				[die Weltmeisterschaft;the world championship, roof]]
			[V\feattab{\dsl \ibox{7},\\
                                   \textsc{cont} \ibox{2}\, erringen{(c,w)}}
				[\trace]]]]
	[V\feattab{\textsc{cont} \ibox{3},\\
                   \textsc{slash} \sliste{ \ibox{5} }}
		[V{[\comps \sliste{ \ibox{6} }, \textsc{cont} \ibox{3}\,]}
			[V{[\comps \sliste{ \ldots, \ibox{4} }, \textsc{cont} \ibox{3}\,]}, tier=np,edge label={node[midway,right]{V1-LR}}
				[V{[\textsc{cont} \ibox{2}\, erringen{(c,w)}]}, tier=trace,edge label={node[midway,right]{MF-LR}}
					[errang;won]]]]
		[\ibox{6} V\feattab{\textsc{cont} \ibox{3},\\ 
                                    \textsc{slash} \sliste{ \ibox{5} }}
			[NP{[\textit{nom}]}, tier=np
				[Clark;Clark]]
			[V\feattab{\textsc{cont} \ibox{3},\\
                                   \textsc{slash} \sliste{ \ibox{5} }}
				[\ibox{4}{[\begin{tabular}[t]{@{}l@{}}
                                           \textsc{loc} \ibox{5} [\begin{tabular}[t]{@{}l@{}}
                                                              \dsl \ibox{7},\\
                                                              \textsc{cont} \ibox{3}\,],
                                                              \end{tabular}\\
                                           \textsc{slash} \sliste{ \ibox{5}}]
                                           \end{tabular}}, tier=trace
					[\trace]]
				[V{[\comps \sliste{ \ldots, \ibox{4} }\textsc{cont} \ibox{3}\,]}
					[\trace]]]]]]
\end{forest}
\end{sideways}
}
\caption{\label{anal3}Representation of meaning contribution}
\end{figure}
The verb \emph{errang} `won' licensed by the lexical rule requires an empty head. This empty head contains the representation of the
syntactic and semantic properties of the original verb inside its \dslv -- importantly also its semantic content $erringen(x,y)$,
whereby $x$ is linked to the subject and $y$ to the object. This means that by assigning its arguments, $x$ refers to \emph{Clark}
(abbreviated to $c$), while $y$ refers to \emph{die Weltmeisterschaft} (abbreivated to $w$).
Since the \locv of the extraction trace is identical to the \locw of the filler and therefore its \dsl is located inside its
\textsc{loc}, the \dsl value of the extraction trace is also identical to the \dsl value of the filler. Since \dsl is a head
feature, it is present at all nodes inside of the verbal projection in the prefield and on the verb trace in the prefield. Inside the
verb trace, the \contv under \dsl is identified with the \contv of the trace itself. The computation and projection of the semantic 
content inside of the complex constituent in the prefield then follows via the normal principles of HPSG:
The combination of the trace with its complement \emph{die Weltmeisterschaft} results in the projection of the \contv of the head
($erringen(c,w)$). When then combined with the adjunct \emph{zum zweiten Mal} `for the second time', the semantic content of the adjunct (2*$erringen(c,w)$)
is projected.
The semantic representation of the filler is identical to the semantic representation of the extraction trace. Through specification in
our lexical rule, the semantic content of the verb is associated with the semantic content of the selected (projection of the) verb trace
\iboxb{5}, \ie the trace that stands for \emph{errang} adopts the semantic representation of the extraction trace (2*$erringen(c,w)$).
This meaning is then projected along the head chain up to the verb in initial position and from there it is projected to the entire clause.


As was shown by examples (\ref{bsp-gezielt-mitglieder}) and (\ref{bsp-kurz-die-bestzeit}) on page~\pageref{bsp-gezielt-mitglieder}
as well as the examples in (\ref{bsp-idioms-nicht-adjazent}) on page~\pageref{bsp-idioms-nicht-adjazent}, the elements in the prefield
do not have to be adjacent to the verb on which they are dependent. A modal or auxiliary verb can occupy the initial position. The verb which
selects the elements in the prefield is then located in the right verbal bracket.
Figure~\vref{anal4-vlast} shows how the example in (\mex{1}) (which conforms to this pattern) should be analyzed.
\ea
\gll Zum zweiten Mal die Weltmeisterschaft hat Clark 1965 errungen.\\
	 to.the second time the world.championship has Clark 1965 won\\
\glt `Clark won the world championship in 1965 for the second time.'
\z

\begin{figure}
\resizebox{!}{\textheight-3\baselineskip}{%
\begin{sideways}
\begin{forest}
sm edges
[V{[\comps \eliste]}
	[V \ibox{5}\,{[\dsl \ibox{7}, \comps \sliste{ \ibox{1} }]}
		[PP
			[zum zweiten Mal;to.the second time, roof]]
		[V{[\dsl \ibox{7}, \comps \sliste{ \ibox{1} }]}
			[\ibox{2} NP{[\textit{acc}]}
				[die Weltmeisterschaft; the world championship, roof]]
			[V\feattab{\dsl \ibox{7},\\
                                   \comps \sliste{ \ibox{1}, \ibox{2} }}
				[\trace]]]]
	[V\feattab{\comps \eliste,\\
                   \textsc{slash} \sliste{ \ibox{5} }}
		[V{[\comps \sliste{ \ibox{6}}]}
			[V{[\comps \ibox{3} $\oplus$ \sliste{ \ibox{8} }]}, tier=np,edge label={node[midway,right]{V1-LR}}
				[hat;has]]]
		[\ibox{6}\,V\feattab{\comps \eliste,\\ 
                                     \textsc{slash} \sliste{ \ibox{5} }}
			[\ibox{1} NP{[\textit{nom}]}, tier=np
				[Clark;Clark]]
			[V\feattab{\comps \ibox{3},\\
                                   \textsc{slash} \sliste{ \ibox{5} }}
				[\ibox{8}\,\feattab{\comps \ibox{3},\\
                                                    \textsc{slash} \sliste{ \ibox{5} }}
					[\ibox{4}\,{[\begin{tabular}[t]{@{}l@{}}
                                                    \textsc{loc} \ibox{5}\,{[\begin{tabular}[t]{@{}l@{}}
                                                                          \dsl \ibox{7},\\
                                                                          \comps \ibox{3}\,],
                                                                         \end{tabular}}\\
                                                    \textsc{slash} \sliste{ \ibox{5} }]
                                                    \end{tabular}}
						[\trace]]
					[V{[\comps \ibox{3}\, \sliste{ \ibox{1} } $\oplus$ \sliste{ \ibox{4} }]}, l sep+=2ex
						[V{[\comps \sliste{ \ibox{1}, \ibox{2} }]},edge label={node[midway,right]{MF-LR}}
							[errungen;won]]]]
				[V{[\comps \ibox{3}\,\sliste{ \ibox{1} } $\oplus$ \sliste{ \ibox{8} }]}
					[\trace]]]]]]
\end{forest}
\end{sideways}
}
\caption{\label{anal4-vlast}Analysis of \emph{Zum zweiten Mal die Weltmeisterschaft hat Clark 1965 errungen.} `Clark has won the world championship in 1965 for the second time.'}
\end{figure}



In contrast to the analysis discussed here, the lexical rule for putative multiple fronting is not applied
to the finite verb (which was present in initial position), but rather to the non-finite verb in final position.
The verb which is the output of the lexical rule requires a verbal complex (something that is \lex +) and attracts its previously non-realised
arguments \iboxb{3}. This verbal complex is realised as the extraction trace. The combination of the extraction trace
and \emph{errungen} `won' forms a verbal complex, which becomes the complement of the verb trace,
which corresponds to \emph{hat} `has' in initial position. The complex consisting of extraction trace, \emph{errungen} and verb trace is then combined with the
arguments, i.e. \emph{Clark}, not realised in the prefield. The percolation of the \textsc{slash} and \dsl values
proceeds parallel to the previously discussed example.


It still remains to be seen how we can rule out the following structure:
\ea[*]{
\gll dass Clark 1965 zum zweiten Mal die Weltmeisterschaft [\_\sub{V} hat]\\
     that Clark 1965 for.the second time the world.championship {}    has\\
}
\z
Without further restrictions, the silent head could be combined with the auxiliary \emph{hat} and take the
place of \emph{errungen}. This structure can however be ruled out under the assumption that all verbs directly specified
in the lexicon which are able to select other verbs require that the embedded verb should have \type{none} as its \dsl value.
In this way, it is ensured that the trace cannot be combined with the normal verb-final \emph{hat} `has', but rather only
with lexical items licensed by the lexical rule in (\ref{lr-mult-vf-zwei}).

The other data discussed in Section~\ref{sec-phenomenon-mult-front} can be analyzed entirely parallel to the examples discussed here:
adjuncts/arguments are linked to an empty verbal head, just as would be the case for their ordering in the middle-field and
their single fronting. The complex projection in the prefield enters a long-distance dependency with the extraction trace in the
verbal complex. If there is any motivation for analysing the data discussed in Section~\ref{sec-analyse-mfvorschlaege} as instances where
a non-verbal constituent precedes the finite verb, this would still be compatible with the analysis presented here. These examples would
have to be explained using the mechanisms presented in Chapter~\ref{chap-german-sentence-structure}, i.e. as standard fronting with a basic extraction trace.
My claim in Section~\ref{sec-analyse-mfvorschlaege} that these analyses cannot be applied to all the data presented in Section~\ref{sec-phenomenon-mult-front}
remains valid.

Finally, I would like to clarify one more point about the status of the lexical rule for multiple fronting. This rule is
entirely parallel to the verb movement rule, which is needed to derive the position of the finite verb. The verb-first rule
differs from the multiple fronting rule in that the verb-first rule mentions finiteness features and
the \textsc{initial} feature
relevant for its positioning. Furthermore, the \compsl of the embedded projection (\iboxt{4} in (\ref{lr-mult-vf-zwei})) is
instantiated as an empty list. This difference in the constraint of the \compsl corresponds to the difference between verbs
which form verbal complexes (the so-called coherent construction) and verbs which embed phrases (the
so-called incoherent construction).  

\begin{comment}
\section{Probleme}

Jacobs Negation: Neg CP

Ich kann Neg XP V nicht ausschließen und kriege immer Skopus über das am tiefsten eingebettete Verb. (darauf hat mich
Tibor hingewiesen).


Allerdings ist das folgende sicher nicht

Adv Neg CP sondern Adv Neg NP V

[Sicher] [nicht] [die letzte Aktion der BAW in diesem Zusammenhang] war am 30.\ Mai eine
      zweite Durchsuchung des Mehringhofes, bei der nochmals nach dem angeblichen Sprengstoffversteck
      gesucht wurde.


% Auch Marga Reis, 2003

Nur die Maria, die liebt jeder.

\end{comment}


\subsection{Left dislocation}
\label{sec-left-dislocation}

Marga Reis (p.\,c.\,2003) has pointed out that the following examples could pose a problem for the analysis
I have developed here:
\eal
\ex[]{
\gll Zum zweiten Mal die Weltmeisterschaft, die gewann Clark 1965.\\
	 to.the second time the world.championship.\fem{} that.\fem{} won Clark 1965\\
\glt `Clark won the world championship for the second time in 1965.'
}
\ex[*]{
\label{ex-zweiten-mal-weltmeisterschaft-das}
\gll Zum zweiten Mal die Weltmeisterschaft, das gewann Clark 1965.\\
	 to.the second time the world.championship.\fem{} that.\neu{} won Clark 1965\\
}
\zl
If a verb phrase is referred to in so-called left-dislocation structures, the pronoun 
\emph{das} (neuter) is obligatory:
\eal
\ex[]{
\gll Die Torte essen, das will Peter nicht.\\
	 the cake eat that.\neu{} wants Peter not\\
\glt `Peter doesn't want to eat the cake.'
}
\ex[*]{
\gll Die Torte essen, die will Peter nicht.\\
	 the cake eat that.\fem{} wants Peter not\\
}
\zl
If \emph{zum zweiten Mal} and \emph{die Weltmeisterschaft} were part of a verbal constituent, then -- just as
in (\mex{0}a) -- we would assume that \emph{das} is obligatory in left-dislocation. (\mex{-1}b) clearly shows
that this is not the case.



One could argue that this difference can be traced back to the fact that the pronoun  refers to an overt element.
The left-dislocated constituent could then be a verbal projecton, however, since this verbal projection does not
contain an overt verb and the closest overt phrase is the feminine NP \emph{die Weltmeisterschaft}, one has to use
the feminine demonstrative pronoun \emph{die}.\footnote{
		A reviewer from Linguistische Berichte pointed out the following kind of gapping data:
\ea
\gll Der Eva Buntstifte gekauft und der Rita Bauklötze, das hat Otto heute in der Stadt.\\
	 the Eva crayons bought and the rita building.blocks that.\neu{} has Otto today in the city\\
\glt `Otta went to town today and bought crayons for Eva and building blocks for Rita.'
\z
It is possible here to argue that the fronted constituent contains a verb. The verb is not in final position, but
still relevant for the anaphoric relation. Furthermore, the verb and pronoun do not have to be adjacent in cases of
extraposition: 
\ea
\gll Geschlafen in der Vorlesung, das hat sie nicht.\\
	 slept in the lecture.\fem{} that.\neu{} has she not\\
\glt `She didn't sleep during the lecture.'
\z
The overtly realised verb is however still anaphorically accessible.
}



Unfortunately, instances of multiple fronting do not show uniform behaviour when used in left-dislocation
constructions (as pointed out by a reviewer from \emph{Linguistische Berichte}). Examples like (\ref{bsp-dauerhaft-mehr-arbeitsplaetze}) %and (\ref{bsp-praedikativen-charakter})
optionally allow \emph{das}, whereas this is the only possibility with example (\ref{bsp-zeitgeist-zwei}):
\eal
\ex[]{
\gll {}Dauerhaft mehr Arbeitsplätze, das gebe es erst, wenn sich eine Wachstumsrate von  mindestens 2,5 Prozent über einen Zeitraum von drei oder vier Jahren halten lasse.\\
       constantly more jobs that.\neu{} gives it first when \textsc{refl} a growth.rate of  at.least 2.5 percent over a time.period of three or four years hold lets\\
\glt `In the long run, there will only be more jobs available when a growth rate of at least 2.5 percent 
can be maintained over a period of three or four years.'	
}
\ex[]{\label{ex-duaerhaft-arbeitsplaetze-die}
\gll {}Dauerhaft mehr Arbeitsplätze, die gebe es erst, wenn sich eine Wachstumsrate von  mindestens 2,5 Prozent über einen Zeitraum von drei oder vier Jahren halten lasse.\\
       constantly more jobs that.PL gives it first, when \textsc{refl} a growth.rate of  at.least 2.5 percent over a time.period of three or four years hold lets\\
}
%% \ex[?]{
%% \gll {}Noch entschiedener prädikativen Charakter, das hat das Adj., wenn [\ldots]\\
%% 	   still more.deciding predicative character.\mas{} that.\neu{} has the adj. if\\
%% \glt  `Even more decidingly, the adjective has a predicative character, if \ldots'
%% }
%% \ex[?]{
%% \gll {}Noch entschiedener prädikativen Charakter, den hat das Adj., wenn [\ldots]\\
%% 	   still more.deciding predicative character.\mas{} that.\mas{} has the adj. if\\
%% }
\ex[?]{\label{ex-zeitgeist-rechnung-das}
\gll {}Dem Zeitgeist Rechnung, das tragen im unterfränkischen Raum die privaten, städtischen und kommunalen Musikschulen.\\
      the Zeitgeist attention that.\neu{} carry in.the lower.Franconian area the private urban and communal music.schools\\
\glt `The private urban and communal music schools in the lower Franconian area account for the Zeitgeist.'
		}
\ex[*]{\label{ex-zeitgeist-rechnung-die}
\gll {}Dem Zeitgeist Rechnung, die tragen im unterfränkischen Raum die privaten, städtischen und kommunalen Musikschulen.\\
       the Zeitgeist attention.\fem{} that.\fem{} carry in.the lower.Franconian area the private urban and communal music.schools\\
}
\zl

The considerable deviance of (\ref{ex-zeitgeist-rechnung-die}) could be down to the fact that we are dealing with an idiomatic construction here and
that referring to individual parts of an idiom often results in ungrammaticality. As for why there is more than one possibility 
for the other examples, this will have to be shown by future research.

So if we would take the existence of clauses with \emph{das} as a criterion, the data in (\mex{0})
would support the analysis that treats the complex \vf as a unit since (\mex{0}a,c) show that reference with \emph{das} is indeed possible. The alternative realization of
\emph{die} in (\ref{ex-duaerhaft-arbeitsplaetze-die}) can be explained as a proximity effect where a
meaning corresponding to (\mex{1}) is taken up by the demonstrative pronoun.
\ea
\gll mehr dauerhafte Arbeitsplätze\\
     more constantly jobs\\
\z

\noindent
Note, however, that we are dealing with special cases of left dislocation anyway. According to the analysis
suggested here, the meaning of the empty verb in the fronted constituent corresponds to the meaning
of the overt verb in the remainder of the clause. For (\ref{ex-zweiten-mal-weltmeisterschaft-das}),
we would have \emph{zum zweiten Mal die Weltmeisterschaft} `for the second time the world
championship' meaning \emph{zum zweiten Mal die Weltmeisterschaft gewonnen} `for the second time the
world championship won'. This meaning is then referred to by \emph{das}. But such a meaning of
\emph{das} would be incompatible with \emph{gewann Clark 1965} `won Clark in 1965' since \emph{win}
selects for a competition and not an event of winning a competition (This was pointed out by
Joachim Jacobs in personal communication to Julia Winkler, see \citew[\page 39]{Winkler2014a}). Of course the
same argument applies to (\ref{ex-zeitgeist-rechnung-das}): in principle, this example should be
excluded as well. I guess what is happening here is that we are dealing with very marked structures
that cannot be processed according to usual grammar rules. So instead of a reading that would
correspond to (\mex{1}), \emph{dauerhaft mehr Arbeits\-plätze} may be perceived as a complex situation of a certain duration
in which there are more jobs and \emph{das} refers to this situation.
\ea[*]{
\gll Dauerhaft  mehr Arbeitsplätze geben, das gebe es erst, wenn \ldots\\
     constantly more jobs          give   this gives it first if\\
}
\z
% Freies Thema muss im Nominativ stehen. Altmann81a: 50
%
% Dauerhaft den Zuschlag, den gebe es erst

\subsection{Extraposition inside the complex prefield}
\label{sec-extraposition}

Tibor Kiss (p.\,c.\ 2002) has pointed out that the analysis with a verb trace allows sentences such as (\mex{1}):
\ea[*]{
\label{bsp-dem-mann-etwas-rs-hat}
\gll Dem Mann etwas \_\sub{V}, der dort steht, hat sie zugeflüstert.\\
	 the man something {} that there stands has she whispered\\
\glt `She whisphered something to the man standing over there.'
}
\z
In (\mex{0}), the silent verb head forms the right verbal bracket and the relative clause belonging to \emph{Mann}
is in the postfield of the verbal projection. These examples should be grammatical in the same way (\mex{1}) is:
\ea[]{
\label{bsp-dem-mann-etwas-zugefluestert}
\gll Dem Mann etwas zugeflüstert, der dort steht, hat sie.\\
     the.\dat{} man something whispered that there stands has she\\
\glt `She whispered something to the man standing over there.'
}
\z

This argument against the analysis with a verbal head in the \vf can be rejected right away since
there are examples like (\ref{ex-los-damit}) -- repeated here as (\mex{1}) -- that clearly show that extraposition in the complex \vf is possible:
\ea 
\gll {}[Los] [damit] geht es schon am 15. April.\footnotemark\\
	\spacebr{}off \spacebr{}there.with goes it \particle{} on 15. April\\
\footnotetext{
        taz, 01.03.2002, p.\,8.
    }
\glt `The whole thing starts on the 15th April.'
\z 
The particle \emph{los} marks the right sentence bracket and \emph{damit} is located inside the \nf
in the complex \vf.

Nevertheless, there remains the question why (\ref{bsp-dem-mann-etwas-rs-hat}) is impossible. First,
multiple fronting with indefinite pronouns like \emph{etwas} seems to be impossible. (\ref{bsp-dem-mann-etwas-rs-hat}) is ungrammatical even without extraposition of the relative clause:
\ea[*]{
\gll Dem Mann etwas hat sie zugeflüstert.\\
     the.\dat{} man something has she whispered\\
}
\z
If one modifies the preceding example so that one has two full noun phrases with a contrastive interpretation,
one observes an improvement in acceptability (and -- as noted in the Section~\ref{sec-dat-acc-vf} -- there are attested examples of this pattern):
\ea[?]{
\gll Dem Mann die Nachricht hat sie zugeflüstert.\\
	 the.\dat{} man the.\acc{} message has she whispered\\
\glt `She whispered the message to the man.'
}
\z
If we add a relative clause to one of the noun phrases, we see that the already marginally acceptable example becomes even
worse:
\ea[?*]{
\gll Dem Mann, der dort steht, die Nachricht hat sie zugeflüstert.\\
	 the.\dat{} man that there stands the message has she whispered\\
\glt `She whispered the message to the man standing there.'
}
\z
Our example becomes completely ungrammatical if we then try and extrapose the relative clause:
\ea[*]{
\gll Dem Mann die Nachricht, der dort steht, hat sie zugeflüstert.\\
     the.\dat{} man the.\acc{} message that there stands has she whispered\\
}
\z
%
Example (\ref{bsp-dem-mann-etwas-zugefluestert}) differs from (\mex{0}) in that \emph{dem Mann} `the
man' is stressed in (\ref{bsp-dem-mann-etwas-zugefluestert}), whereas \emph{etwas} is unstressed. Following the generalization proposed
by \ao, the elements involved in multiple fronting have to bear the same communicative importance, which is not the case
for (\ref{bsp-dem-mann-etwas-rs-hat}) and (\mex{0}). 

While further work is needed for the formalization of the respective constraints, it is clear that
extraposition inside of complex \vf{}s is possible and hence the assumptions of structures like the
one that is assumed in the current analysis is legitimate.


\subsection{Traces in undesired positions}
\label{sec-unwanted-traces}

The analysis in (\mex{1}) is ruled out by the fact that the second lexical item for \emph{errang} (licensed
by the rule in (\ref{lr-mult-vf-zwei})) selects a \textsc{lex}+ element.
\ea
\label{bsp-trace-mit-argumenten-im-mf}
\gll dass Clark 1965 [[zum zweiten Mal die Weltmeisterschaft \_\sub{V}] errang]\\
     that Clark 1965 \hspaceThis{[[}to.the second time the world.championship {} won\\
\glt `that Clark won the world championship for the second time in 1965'
\z
This structure is ruled out for the same reason as embedding of verbal projections in obligatorily coherent 
constructions.

There is however still the analysis in (\mex{1}), which is entirely parallel to verbal complex formation and therefore
cannot be ruled out by a \textsc{lex} feature.
\ea
\gll dass Clark 1965 zum zweiten Mal die Weltmeisterschaft     [[\_\sub{V} errungen] hat].\\
     that Clark 1965 to.the second time the world.championship {} won has\\
\z
Furthermore, we have not yet encountered anything that would rule out the possibility of a verbal trace in the prefield as a filler
for a long-distance dependency.
\ea
\gll \_$_V$ hat Clark 1965 zum zweiten Mal die Weltmeisterschaft [~\_$_i$ errungen]\\
     {}     has Clark 1965 to.the second time the world.championship {} won\\
\z
\citet[\page 100]{Fanselow87a} discussed cases with one fronted PP and noticed that such sentences
are ambiguous since they could be analyzed as structures in which a single constituent is fronted or
as structures in which a complex constituent containing one element is fronted.

As has already been suggested, there are various conditions for cases of supposed multiple fronting that rely on the thematic status
of the constituents preceding the finite verb. If we require that there be certain relations between such constituents, then the corresponding
constrains would prohibit any case where there are no constituents in the prefield, \ie where the verb trace does not project. (\mex{0}) and
also examples with a verb trace and a single constituent are also ruled out by general constraints
on putative cases of multiple fronting.\todostefan{Does the IS stuff restrict this? Should I include
the implicational constraint that enforces extraction?}


\section{Alternatives}
\label{sec-analyse-mfvorschlaege}

The problem posed by the present data for all theories assuming verb-second order cannot simply be solved
by marking problematic examples with `*' as \citet[\page 37]{Bungarten73a} does for examples like
(\ref{bsp-zum-zweiten-mal-die-Weltmeisterschaft}). There are just too many attested examples and for
this reason this data may not be ignored. There have been several proposals in the 1980ies and
1990ies and I will discuss each in turn.

\subsection{Movement of parts of the \mf and the verbal complex}

\citet*{Loetscher85a}\ia{Lötscher} has sketched the beginnings of a theory, which -- under certain conditions -- would allow
for an unlimited amount of constituents to be fronted.\footnote{
  Also see \citew[\page 412--413]{Eisenberg94a} for suggestion of a similar analysis.%
}
His proposal makes use of several rules, which have to be applied in a set order. These kinds of analyses are by their very nature
incompatible with theories based in a HPSG framework, since the principles of HPSG are unordered and hold equally for all structures.
Lötscher assumes that any chain in the left edge of the verbal complex can be fronted. These chains can contain verbs, which
would explain the fronting of partial projections. The adjacency of elements of the chain to the verbal complex could have come
about by movement operations in the middle-field. \citet[\page 92]{Duerscheid89a} has criticised Fanselow's \citeyearpar{Fanselow87a} approach,
and this criticism can also be applied to Lötscher's proposal: if fronting were in fact movement of any continuous chain from the left
periphery of a verbal complex into the initial position of a sentence, then (\mex{1}c) would be the underlying structure for the
fronting operation in (\mex{1}b).



\eal
\ex 
\gll dass ein Professor seinen Schüler nicht prüfen muss\\
     that a professor his student not test must\\
\glt `that a professor does not have to test his student'
\ex
\gll Seinen Schüler prüfen muss ein Professor nicht.\\
     his student test must a professor not\\
\ex 
\gll dass ein Professor nicht seinen Schüler prüfen muss\\
     that a   professor not   his student test must\\
\zl
The sentential negation precedes the verbal complex in example (\mex{0}a). In (\mex{0}c), the negation
has scope over \emph{seinen Schüler} `his student' and therefore does not correspond to the expected
base order for (\mex{0}b). According to \citet[\page 103]{Duerscheid89a}, a similar argumentation goes back to \citep{Thiersch86a}.

% Thiersch86a: 42-43

%
% \eal
% \ex{\iw{hinterlassen}
% [Der Nachwelt hinterlassen] hat sie [eine aufgeschlagene {\it Hör zu\/} und einen kurzen Abschiedsbrief]:\footnote{
%         taz, 18.11.1998, p.\,20.%
% }\label{bsp-der-nachwelt-hinterlassen-pred-compl}
% }
% \ex{
% weil sie der Nachwelt eine aufgeschlagene Hör zu und einen kurzen Abschiedsbrief hinterlassen hat.
% }
% \ex{
% weil sie eine aufgeschlagene Hör zu und einen kurzen Abschiedsbrief der Nachwelt hinterlassen hat.
% }
% \zl
%
% Sätze wie (\ref{bsp-vom-leutnant}) stellen keine Verletzung der Generalisierung dar,
% wenn man annimmt, daß es sich bei \emph{vom Leutnant zum Hauptmann}
% um eine Phrase handelt. Indizien für diese Annahme liefern Sätze wie
% (\mex{1}), in denen die Lokalangaben auch als eine Konstituente betrachtet
% werden können.
% \eal
% \ex[]{
% Wir befinden uns in Berlin.\iw{befinden}
% }
% \ex[]{
% Wir befinden uns in Berlin am Flußufer.
% }
% \ex[]{
% Wir befinden uns in Berlin am Flußufer unter der Brücke.
% }
% \ex[*]{
% Wir befinden uns.
% }
% \zl
% %Allerdings könnte man bei solche einem Ansatz nicht erklären, wieso die beiden Phrasen
% %verschiedene Rollen des Verbs füllen.
% %Allerdings können \emph{vom Leutnant} und \emph{zum Hauptmann} auch
% %getrennt im Mittelfeld auf"|treten.
% %\ea
% %Karl hat Peter vom Leutnant gestern zum Hauptmann befördert und nicht vorgestern.
% %} 

\subsection{Complex PPs formed from several PPs}

\citet*[\page 79]{Wunderlich84} suggested treating the fronted phrases in (\mex{1}) as a single constituent,
more specifically, a prepositional phrase.
\eal
\ex 
\gll {}[\sub{PP} [\sub{PP} Zu ihren Eltern] [\sub{PP} nach Stuttgart]] ist sie gefahren.\\
	{}       {}        to her parents   {}        to Stuttgart is she driven\\
\glt `She drove to Stuttgart to her parents.'
\ex\label{bsp-von-muenchen-nach}
\gll {}[\sub{PP} [\sub{PP} Von München]     [\sub{PP} nach Hamburg]]   sind es 900 km.\\
       {}        {} from Munich  {} to Hamburg are it 900 km.\\
\glt `It is 900 km from Munich to Hamburg.'	
\ex 
\gll {}[\sub{PP} [\sub{PP} Durch den Park]  [\sub{PP} zum Bahnhof]]    sind sie gefahren.\\
	{}       {}        through the park {}        to.the train.station are they driven\\
\glt `They drove through the park to the train station.'
\zl
Wunderlich assumes that the second PP in (\mex{0}) always modifies the first. This is possible when both
 PPs bear the same semantic role.\footnote{
        See \citew[\page 107--109]{Duerscheid89a} for a similar suggestion.%
}
In (\mex{0}a), both prepositional phrases denote the destination of some movement. Wunderlich admits that the
thematic roles in (\mex{0}b) and (\mex{0}c) are different (source, route or destination of movement) and tries to
subsume them under the broader heading of `localization of movement'.
This approach is not satisfactory, however, as it would be difficult for a HPSG grammar to reconstruct the individual roles 
related to each verb from the broader `localization of movement'. The examples in (\mex{0}) and also examples such as 
(\ref{bsp-vom-leutnant}) can only be analyzed in the way Wunderlich does if each prepositional
phrase is analyzed as modifier, that is, if they do not receive a semantic role from some verb.
\ea
\label{bsp-vom-leutnant}%
\gll [Vom Leutnant] [zum Hauptmann] wird Karl befördert.\iw{befördern}\\
	 \spacebr{}from lieutenant \spacebr{}to.the captain becomes Karl promoted\\
\glt `Karl is getting promoted from lieutenant to captain.'
\z
This is, in my opinion, not an adequate explanation.
%Eine genau Analyze solcher Konstruktionen steht also noch aus.

%Duerscheid89a:87
% zitiert allerdings Engel und nimmt die Beispiele aus



\citet[\page 62]{Riemsdijk78a} discusses data from Dutch, which are parallel to 
(\ref{bsp-von-muenchen-nach}). He suggests analysing the first PP as the specifier of the second.
The specifier analysis also runs into problems when both prepositional phrases are complements and
are independently associated with a verb.

\citet[\page 217--218]{Dowty79a} discusses (\mex{1}) in a different context:
\ea
John drives a car from Boston to Detroit.
\z
He suggests that \emph{Boston} as well as \emph{to Detroit} are complements of \emph{from}. This analysis
would not however be able to shed light on (\ref{bsp-vom-leutnant}). Furthermore, it is not compatible with
other cases of multiple fronting.

\subsection{Fronting and LF correspondence restrictions}

\ia{Haider|(}
\citet*[\page 17]{Haider82} formulated a condition similar to that of Wunderlich. According to Haider, the LF"=projection\is{Logical Form (LF)}
of the prefield has to correspond to a single LF"=constituent. LF stands for `Logical Form' in Government and Binding theory. Haider's condition
allows for the simultaneous fronting of adverbs and fronting of certain non-maximal verbal projections.

Haider discusses the contrast between the following examples in (\mex{1}):
\eal
\ex[]{
\gll Wann und wo hat sie sich mit ihm getroffen?\\
	 when and where has she \textsc{refl} with him met\\
\glt `When and where did she meet him?'
}
\ex[*]{
\gll Wann und wer hat sich mit ihm getroffen?\\
     when and who has \textsc{refl} with him met\\
}
\zl
He explains the difference by claiming that the \emph{wh}-words together bind a single empty adverbial position. This is not
possible for (\mex{0}b). He offers a similar explanation for (\mex{1}).
\ea
\label{bsp-gestern-am-strand-haider}
\gll Gestern am Strand hat sie sich mit ihm getroffen.\\
	 yesterday on.the beach has she \textsc{refl} with him met\\
\glt `She met him yesterday on the beach.'
\z
It is plausible to assume, as Haider does, that temporal and spatial adjuncts form a single constituent. In this case, instances of fronting such as
(\mex{0}) would be unproblematic. Nevertheless, we have seen in Section~\ref{sec-phenomenon-mult-front} that complements can be fronted along with adjuncts.
If we compare examples (\ref{bsp-nichts-mit-derartigen}) and (\ref{bsp-zum-zweiten-mal-die-Weltmeisterschaft}) with the previous examples, it is clear
that the coordination test does not really tell us much:



\eal
\ex[]{
\gll {}[Nichts] [mit derartigen Entstehungstheorien] hat es natürlich zu tun, \ldots\\
	    \spacebr{}nothing \spacebr{}with those.kind origin.theories has it of.ocurse to do\\
\glt `It of course has nothing to do with those kinds of theories of origin.'
}
\ex[]{
\gll Was hat das mit derartigen Entstehungstheorien zu tun?\\
	 what has that with those.kind origin.theories to do\\
\glt `What has that got to do with those kinds of theories of origin?'
}
\ex[]{
\gll Womit hat das nichts zu tun?\\
	 with.what has that nothing to do\\
\glt `What has that got nothing to do with?'
}
\ex[*]{
\gll Was und womit hat das zu tun?\\
	 what and with.what has that to do\\
}
\zl
\eal
\ex[]{
\gll {}[Zum zweiten Mal] [die Weltmeisterschaft] errang\iw{erringen} Clark 1965 \ldots\\
	    \spacebr{}to.the second time  \spacebr{}the world.championship won Clark 1965\\
\glt `Clark won the world championship for the second time in 1965.'
}
\ex[]{
\gll Zum wievielten Mal errang Clark 1965 die Weltmeisterschaft?\\
	 to.the how.many time won Clark 1965 the world.championship\\
\glt `How many times was it that Clark had won the world championship in 1965?'
}
\ex[]{
\gll Was errang Clark 1965 zum zweiten Mal?\\
	 what won Clark 1965 to.the second time\\
\glt `What did Clark win for the second time in 1965?'
}
\ex[*]{
\gll Was und zum wievielten Mal errang Clark 1965?\\
	 what and to.the how.many time won Clark 1965\\
	 }
\zl

There are also other combinations of adjuncts in the prefield, \eg (\ref{bsp-instrument}),
where assuming a single constituent of the Haider type is somewhat questionable. 
%% An anonymous reviewer from \emph{Linguistische
%% Berichte} pointed out that the question of why (\mex{1}) is worse than (\ref{bsp-gestern-am-strand-haider})
%% also remains unanswered.
%% \ea
%% \gll Gestern und am Strand hat sie sich mit ihm getroffen.\\
%% 	 yesterday and on.the beach has she \textsc{refl} with him met\\
%% \glt `She met him yesterday at the beach.'
%% \z



Furthermore, Haider's constraint exludes fronting of non-maximal projections which consist of a verb
and a dative object \citet*[\page 17]{Haider82}. Haider offers the following example, which he classes as ungrammatical:
\ea
\gll Seiner Tochter erzählen konnte er ein Märchen mit ruhiger Stimme.\\
	 his daughter.\dat{} tell could he.\nom{} a fairy.tale\acc{} with quiet voice\\
\glt `He could tell his daughter a fairy tale in a quiet voice.'
\z
The unacceptability of the sentence has nothing to do with its syntactic structure, but is rather to do with the information
structural requirements which must be fulfilled for a verbal projection to be fronted. If we change the lexical material in (\mex{0}),
the result is a perfectly acceptable sentence:
\ea
\gll Den Wählern erzählen sollte man so was nicht.\\
	 to the voters.\dat{} tell should one.\nom{} such a.thing.\acc{} not\\
\glt `One shouldn't tell the voters something like that.'
\z
Examples in (\ref{bsp-besonders-einsteigern}) and (\ref{bsp-der-nachwelt-hinterlassen}) are further cases of fronting a verb with its dative object:\footnote{
	The data in (\ref{bsp-pvp}) can also be found in \citep[\page 353--354]{Mueller99a}.
	 \citet[\page 91]{Thiersch82a}, \citet[\page 429]{Sternefeld85a},
%\citet[\page 103]{Scherpenisse86a}, 
\citet[\page 159]{Uszkoreit87a},
\citet[\page 459]{SS88a},
\citet[Chapter~1.5.3.3.1]{Oppenrieder91a}, \citet[\page 1301]{Grewendorf93} and G.\ \citet[\page 5]{GMueller98a}
offer their own examples of a dative complement being fronted together with its verb.%
}
Haider's constraint can therefore be rejected as being too restrictive.

As with Wunderlich's analysis, Haider's approach also struggles to explain (\ref{bsp-vom-leutnant}). 
\ia{Haider|)}

%Bei Sätzen wie (\ref{pvp-fvg-idioms}) handelt es sich um 
%Funktionsverbgefüge\is{Funktionsverbgefüge}\footnote{
%        Zur Analyze von Funktionsverbgefügen siehe
%\citep*{KE94}\iaf{Krenn}\iaf{Erbach} und \citep*{Kuhn95}\iaf{Kuhn}.
%} bzw.\ idiomatische Wendungen, \dh, es liegen
%komplexe Prädikate vor: {\it etwas in etwas bringen\/},\iw{bringen!etwas in etwas $\sim$} 
%{\it zum Opfer fallen\/}.\iw{fallen!zum Opfer $\sim$}
%Die Voranstellungen in (\ref{pvp-fvg-idioms}) haben das Muster der
%Voranstellung von Phrasenteilen, die in Chapter~\ref{pvp} beschrieben
%wird. Eine genaue Analyze für diese Sätze steht ebenfalls noch aus.

\subsection{Apparent multiple frontings as multiple frontings}
\label{sec-mueller-2000}\label{sec-speyer2008-syntax}


In an earlier proposal I assumed that multiple frontings are just multiple extractions
\citep{Mueller2000d}. The respective analysis is sketched in Figure~\vref{fig-mult-extraction}.
\begin{figure}
\centering
\begin{forest}
[{S[\slasch \eliste]}
  [C$_1$] 
  [{S[\slasch \sliste{ C$_1$ }]}
     [C$_2$] 
     [{S[\slasch \sliste{ C$_1$, C$_2$ }]}]
  ]]
\end{forest}
\caption{Multiple frontings as multiple extractions according to \citet{Mueller2000d}}\label{fig-mult-extraction}
\end{figure}%
A sentence with two gaps (C$_1$, C$_2$) is combined with appropriate fillers in two steps.

Similarly, \citet{Speyer2008a} suggests a Rizzi-style analysis of German (\citealp{Rizzi97a-u};
Grewendorf \citeyear[\page 85, 240]{Grewendorf2002a}; \citeyear{Grewendorf2009a}) in which he
assumes several functional projections for topic and focus before the finite verb. For instance,
Figure~\vref{fig-clause-structure-speyer} shows the analysis of (\mex{1}):
\ea
\gll Briefe hat Uller geschrieben.\\
     letters has Uller written\\
\glt `Uller has written letters.'
\z
\begin{figure}
\centerfit{%
\begin{forest}
[ ForceP
   [ SpecForceP ]
   [ Force$'$
     [Force ]
     [Top1P
        [SpecTop1P]
        [Top1$'$
           [Top]
           [FocP
             [SpecFocP\\
              \emph{Briefe}\sub{[foc] 2}]
             [Foc$'$
               [Foc\\
                {$\varnothing$,[foc]}]
               [Top2P
                 [SpecTop2P]
                 [Top2$'$
                   [Top2]
                   [FinP
                     [SpecFinP\\
                      t$_2$]
                     [Fin$'$
                       [Fin\\
                        \emph{hat}$_1$]
                       [IP
                        [Uller t$_2$ geschrieben t$_1$,roof]]]]]]]]]]]]
\end{forest}
}
\caption{Analysis of \emph{Briefe hat Uller geschrieben} `Uller has written letters.' according to
  \citet[\page 471]{Speyer2008a}}\label{fig-clause-structure-speyer}
\end{figure}
The subject and main verb and finite auxiliary are generated as part of the IP and then the finite
verb is moved to the head position of the Fin head. The object of \emph{geschrieben} `written' moves
to the specifier position of the Fin head and leaves a trace there when moving on to the specifier
position of an empty Focus head. In sentences in which a topic fills the \vf it is assumed that the
fronted element moves on from the SpecFinP position into the specifier position of a topic head. The
Topic and Focus projections are assumed to be present in the structure even if no focus or topic element is present
in the clause. The Force head is assumed to host features that are relevant for determining the
clause type.

In such approaches, a sentence like our example in
(\ref{bsp-zum-zweiten-mal-die-Weltmeisterschaft}) -- repeated here as (\mex{1}) -- has an analysis
in which there are two extraction traces in the \mf: one for \emph{zum zweiten Mal} and one for
\emph{die Weltmeisterschaft}.

\ea
\gll {}[Zum zweiten Mal]$_i$ [die Weltmeisterschaft]$_j$ errang \_$_i$ \_$_j$ Clark 1965 \ldots\\
	   \spacebr{}to.the second time \spacebr{}the world.championship won {} {} Clark 1965 {}\\%
\footnote{
        \citep*[\page 162]{Benes71}
      }\label{bsp-zum-zweiten-mal-die-Weltmeisterschaft-anal}%
\glt `Clark won the world championship for the second time in 1965.'
\z

\subsubsection{Same verb constraint}

This proposal has various problems: first, it cannot be explained why the elements in the \vf have
to depend on the same verb (see Section~\ref{sec-clause-mates}). The following example from
\citet[\page 57]{Fanselow87a} shows that more than one extraction can go on in German sentences.
\ea
\label{ex-radios-weiss-ich-nicht}
\gll Radios weiß ich nicht, wer repariert.\\
     radios know I not who.\nom{} repairs\\
\glt `I do not know who repairs radios.'
\z
The interrogative pronoun is in initial position of the interrogative clause, which is usually
analyzed as extraction since the interrogative phrase may depend on a deeply embedded
head. \emph{Radios} is the object or \emph{repariert} `repairs' and hence extracted from the
interrogative clause \emph{wer repariert} `who repairs'.

Now, the question is: why are sentences like Fanselow's sentences in
(\ref{ex-mult-front-same-verb}b,d) on page~\pageref{ex-mult-front-same-verb} impossible? The first
two of the sentences in (\ref{ex-mult-front-same-verb}) are repeated below for convenience:
\eal
\ex[]{
\gll Ich glaube  dem Linguisten nicht, einen Nobelpreis  gewonnen zu haben.\\
     I   believe the linguist   not    a     Nobel.prize won      to have\\
\glt  `I don't believe the linguist's claim that he won a Nobel prize.'
}
\ex[*]{
\gll Dem Linguisten$_i$ einen Nobelpreis$_j$  glaube \_$_i$ ich nicht [ \_$_j$ gewonnen zu haben].\\
     the linguist   a     Nobel.price believe {}    I   not   {} {}   won      to have\\
}
\zl
In the analysis presented in the previous section it is clear that sentences like (\mex{0}b) are
ruled out since the fronted material has to depend on the same verb. There is no such explanation
for the multiple extraction approach.

\subsubsection{Elements that cannot be extracted (idiom parts)}

Furthermore, as already explained in Section~\ref{sec-idiom-parts-mf}, idioms pose a challenge for the
multiple-extraction approach.

\eal
\ex[]{\label{ex-oel-ins-feuer}
\gll {}[Öl] [ins Feuer] goß gestern das Rote-Khmer-Radio\footnotemark\\
	 \spacebr{}oil \spacebr{}in.the fire poured yesterday the Rote-Khmer-Radio \\
\footnotetext{
        taz, 18.06.1997, p.\,8.
}
\glt `Rote-Khmer-Radio fanned the flames yesterday'
}
\ex[*]{\label{ex-ins-feuer-goss}
\gll [Ins Feuer] goß gestern das Rote-Khmer-Radio Öl.\\
     \spacebr{}in.the fire poured yesterday the Rote-Khmer-Radio oil\\
}
\zl 

\noindent
If both \emph{Öl} `oil' and \emph{ins Feuer} `in.the fire' are extracted in (\mex{0}a), it is difficult to see how
(\mex{0}b) can be ruled out. In the approach with an empty verbal head in the \vf, neither \emph{Öl} `oil'
nor \emph{ins Feuer} `in.the fire' is extracted but both phrases are just combined with an empty verbal head as
they are in sentences like (\mex{1}):
\ea
\gll Das Rote-Khmer-Radio goß gestern Öl ins Feuer.\\
     the Rote-Khmer-Radio poured yesterday oil in.the fire\\
\glt `Rote-Khmer-Radio fanned the flames yesterday`.'
\z
(\mex{1}) shows that \emph{Öl} `oil' can be extracted, but \emph{ins Feuer} `in.the fire' cannot be extracted as
(\ref{ex-ins-feuer-goss}) shows.
\ea
\gll Öl goss auch Lord O’Donnel ins Feuer.\footnotemark\\
     oil poured also Lord O'Donnel in.the fire\\
\glt `Lord O'Donnel also fanned the flames.'
\footnotetext{
\url{http://www.swp.de/ulm/nachrichten/politik/Brexit-ja-aber-nicht-so-fix;art1222886,3985964}, 26.09.2016.
}
\z
So, if (\ref{ex-oel-ins-feuer}) is analyzed as double extraction, one has to find ways to say that
\emph{ins Feuer} `in.the fire' can be extracted only if \emph{Öl} `oil' is extracted as well. It may be possible to do
this but it is highly likely that the system of constraints that is needed to pin that down formally
is highly complex.

%% Zusätzlich Öl ins Feuer goss, dass sich just in dem Moment in Indien ein todbringendes Erdbeben
%% ereignete, als die Statue enthüllt wurde.
%% http://derwaechter.net/superzelle-oder-uebernatuerliches-phaenomen-unglaublicher-sturm-wuetet-ueber-cern

%% Kräftig Öl ins Feuer goss dagegen heute Morgen der türkische Präsident Recep Tayyip Erdogan.
%% http://www.t-online.de/nachrichten/ausland/krisen/id_77442016/krieg-um-berg-karabach-aserbaidschan-macht-rueckzieher.html

\subsubsection{Order of elements in the \vf}

Finally, approaches that assume that individual items are extracted from the \mf and fronted
independently have to explain why the fronted material has to appear in the same order as it appears
in the unmarked order in the \mf. This is automatically explained if one assumes that the fronted
material is part of a verbal projection since then of course one would have all the verbal fields
available: \mf, right sentence bracket, and \nf. As the discussion above showed we need all these
topological fields: particles of particle verbs may fill the right sentence bracket inside a complex
\vf and pronominal adverbs may be extraposed in the complex \vf (\ref{ex-los-damit}), which is
evidence for a \nf. If the fronted material is part of a complex \vf that is the projection of a
verbal head, all facts are explained immediately.

One could try and derive the constraints on the order in the \vf from more general constraints that
are usually assumed in the literature. For instance, one could assume that there must not be any
crossing dependencies. However, there are sentences like (\ref{ex-radios-weiss-ich-nicht}) -- repeated here as (\mex{1}) -- in which
the object is realized before the subject although both subject and object are moved.
\ea
\label{ex-radios-weiss-ich-nicht-zwei}
\gll Radios weiß ich nicht, wer repariert.\\
     radios know I not who.\nom{} repairs\\
\glt `I do not know who repairs radios.'
\z
Note furthermore that such a general ban on extraction structures would be in conflict with Speyer's
original motivation for a Rizzi-style analysis. He agreed that my proposal for the analysis of
modern German is basically on the right track  but criticized the fact that it was not applicable to Early New High
German\il{Early New High German} \citep[\page 461]{Speyer2008a}. The interesting fact about earlier stages of German is that the constraint that the elements
in the \vf have to be in the same order as they would appear in the \mf in unmarked order does not
hold for Middle Low German\il{Middle Low German} \citep[Section~5.2]{Petrova2012a}. The following sentences from \citew[\page
  174--175]{Petrova2012a} illustrate:
\eal
%%\ex
%% hier sieht man keine Verbklammer
%% \gll [Dre sone] [he] leet\\
%%    \spacebr{}three sons \spacebr{}he left\\
%% \glt `He left behind three sons.’
\ex
\gll [Eine warheit] [ich] wille dir sagen\\
     \spacebr{}one truth \spacebr{}I want you-Dat tell\\
\glt `I want to tell you a certain truth.’
\ex
\gll [Sea] [en thegan] habda Joseph gimahlit\\
     \spacebr{}she.\acc{} \spacebr{}a man had Joseph married\\
\glt `A man [called] Joseph had married her.'
\zl
In both sentences the direct object is realized before the subject.

Petrova also assumes a Rizzi-style analysis of
Early New High German. So, since there are languages that allow the order of fronted elements to
differ from the normal order in the \mf, the restrictions that we observe in modern German cannot be
explained by reference to general constraints like the Shortest Move Constraint or the Minimal Link
Condition. It follows that the \vf-\mf correspondence would have to be stipulated for modern German
in Speyer's model, while it is entirely expected in any model with an empty verbal head.


\subsubsection{Assignment to Topic and Focus positions}
\label{sec-assignment-to-top-foc}

While the proposal in \citew{Mueller2000d} does not make any claims about information structure, the
proposal by \citet{Speyer2008a} assumes Rizzi-style functional projections. According to
\citet[\page 482]{Speyer2008a} the \vf consists of elements that are moved to the specifier position
of a SceneP, a FocP, and a TopP that are ordered in this way. He assumes that the upper topic phrase
in the analysis of \citet{Rizzi97a-u} and \citet{Grewendorf2002a} is specialized to contain
frame-setting elements. So, the TopP, FocP, TopP sequence of the former models -- which is also
depicted in Figure~\ref{fig-clause-structure-speyer} -- is more constrained in Speyer's model. Speyer claims that the historical development from Early High German to
modern German resulted in this specialization. Speyer's proposal as presented in his paper
predicts that there can be at most three constituents in the \vf: one scene element, one focus
element, one topic. So if complex frontings with more than three elements are possible, Speyer's
theory is falsified. In Section~\ref{sec-fronting-more-than-two} I showed that frontings with three elements can be found and
provided Arne Zeschel's example (\ref{bsp-ihnen-für-heute-zwei}) with a complex \vf containing four
elements. The example is repeated here as (\mex{1}) for convenience:
\ea\label{bsp-ihnen-für-heute-drei}
\gll {}[Ihnen] [für heute] [noch] [einen schönen Tag] wünscht Claudia Perez.\footnotemark\\
  \spacebr{}you.\dat{} \spacebr{}for today \spacebr{}still \spacebr{}a.\acc{} nice day wishes Claudia Perez\\
\footnotetext{
  Claudia Perez, Länderreport, Deutschlandradio.
}%
\glt `Claudia Perez wishes you a nice day.'
\z

\noindent
Note that Rizzi and Grewendorf assume that the
Topic projections are recursive. So in principle there could be as many topic positions as needed,
followed by one optional focus position, followed by arbitrarily many topic positions. This could
account for multiple elements in the \vf provided some of them are topics. Note, however, that none
of the fronted constituents in (\mex{0}) are topics. The radio speaker announced the program for the
next week and said good buy to the hearers. The reason for the fronting was to emphasize the name of
the speaker. The fronted material is not put in the \vf because it has a certain information
structural function like topic or contrastive focus, it is moved out of the way to make other material more
prominent. \citet{BC2010a} discussed a similar construction, which they called Presentational Multiple Fronting. It
will be discussed in more detail in Section~\ref{sec-presentational-MF}. An example of this
construction is (\mex{1}b), with (\mex{1}a) and (\mex{1}c) providing some context:
\eal
\ex Spannung pur herrschte auch bei den Trapez-Künstlern. [\ldots]  Musika\-lisch begleitet wurden die einzelnen Nummern vom Orchester des Zir\-kus Busch [\ldots]\\
    `It was tension pure with the trapeze artists. [\ldots] Each act was musically accompanied by Circus Busch's own orchestra.'
\ex
\gll [Stets] [einen Lacher] [auf ihrer Seite] hatte \textit{die} \textit{Bubi} \textit{Ernesto} \textit{Family}\lindex{i}.\\
 \hspaceThis{[}always  \hspaceThis{[}a laugh  \hspaceThis{[}on their side had the Bubi Ernesto Family\\
\glt `Always good for a laugh was the Bubi Ernesto Family.'
\ex Die Instrumental-Clowns\lindex{i} zeigten ausgefeilte Gags und Sketche [\ldots]\\
    `These instrumental clowns presented sophisticated jokes and sketches.'\\\sigle{M05/DEZ.00214} 
\zl
This example will be discussed in more detail on page~\pageref{pres}. What is important here is that
the material in the \vf is moved out of the way in order to present the NP in the \mf, which is then
the topic of the following clause. Speyer's analysis fails on examples like this and on the
Claudia Perez example in (\ref{bsp-ihnen-für-heute-drei}). In general, feature driven accounts that
assume that movement is triggered by features that have to be checked\is{feature checking} \citep{Chomsky95a-u}
fail on this data since the movement that is required here is altruistic movement\is{movement!altruistic}, that is, movement
that takes place for the benefit of some other element. See also \citet{Fanselow2003b} on other
cases of altruistic movement.

Speyer's proposal assumed Rizzi/Grewendorf structures and this aspect was criticized in this
section. Speyer works in the framework of Stochastic Optimality Theory in order to explain the
markedness and rareness of the phenomenon in Modern Standard German and in order to explain the
historical development from Early High German. I will turn to the discussion of the OT aspects in Section~\ref{sec-speyer-stochastic-OT}.


\subsection{V3 as adverb + clause}

For examples with sentences adverbs similar to (\ref{bsp-vermutlich}) -- repeated here as (\mex{1})
for convenience --, \citet[\page 111]{Jacobs86a} proposed a rule which combines
a verbal projection with an adverb. 
\ea
\label{bsp-vermutlich-zwei}
\gll {}[Vermutlich] [Brandstiftung] war die Ursache für ein Feuer in einem Waschraum in der Heidelberger Straße.\footnotemark\\
	   \spacebr{}supposedly \spacebr{}arson was the cause for a fire in a washroom in the Heidelberger Street\\
\footnotetext{
Mannheimer Morgen, 04.08.1989, Lokales; Pflanzendieb.
}
\z
Jacobs' rule also licenses the combination of a V2-clause with a sentence adverb and hence can be
used for the analysis of sentences like (\mex{0}). However, this approach encounters problems with
similar examples where the sentence adverb follows a preposed constituent:\footnote{ 
        The following examples are taken from \citet[\page 228]{Engel88}.%
}
\eal
\ex 
\gll Damit freilich muß er allein fertig werden.\\
     with.that simply must he alone finished become\\
\glt `He will simply have to come to terms with it himself.'
\ex 
\gll Ein paar Wochen immerhin ist noch Zeit.\\
     a few weeks nevertheless is still time\\
\glt `Well, we've still got a few weeks.'
\zl
\citet[\page 26]{Duerscheid89a} argues that these kinds of examples should also be analyzed as instances of multiple fronting, since the
sentence adverb refers to the entire sentence and not just to the fronted constituent.
In order to explain examples such as (\mex{0}), Jacobs would have to allow prepositional phrases or pronominal adverbs such as \emph{damit}
and NPs such as \emph{ein paar Wochen} to be combined with V2-clauses. This analysis is very similar
to the one discussed in Subsection~\ref{sec-mueller-2000} and therefore shares the previously discussed drawbacks of this analysis.


%% \subsection{Stochastic OT}
%% \label{sec-speyer-stochastic-OT}

%% \citew{Speyer2008a} developed an Optimality Theory account of multiple frontings. As outlined in
%% Section~\ref{sec-speyer2008-syntax}, he assumes that multiple fronting is a general option in the syntax of German. There
%% is a Scene, a Focus and a Topic position that can be filled by one element. Multiple fronting exists
%% as a phenomenon but it is rare in general. Optimality Theory assumes a component that generates
%% various sentences. These sentences are then compared and the optimal candidate is chosen. In
%% addition to general constraints on the well-formedness of the generated structures (which are often
%% not specified in OT papers) there are constraints that are used to rank the alternatives. Speyer
%% assumes a version of \xbart that is compatible with Rizzi/Grewendorf analyses of clause structure
%% and additionally specifies OT constraints. If these constraints were strict constraints including a
%% V2 constraint, the V2 constraint should dominated all other constraints and the result would be that
%% V3 sentences would always be filtered out since the competition model of OT would not allow for
%% orders that are less successful in terms of constraint rankings.

%% Due to this situation Speyer adopted the framework of Stochastic OT that works with probabilities:
%% even though a constraint is higher ranked than others there is certain variability in the rankings
%% and hence even lower ranked constraints may win from time to time \citep[\page 463]{Speyer2008a}. This allows for occasional V3 structures to be generated.

%% Speyer carefully examined Böll's book \emph{Ansichten eines Clowns} and states that it contains only
%% one example of multiple fronting, namely the sentence in (\ref{ex-inzuepfners-box-der-mercedes}) -- repeated here for convenience:
%% \ea
%% \label{ex-inzuepfners-box-der-mercedes-zwei}
%% \gll {}[[In Züpfners Box] [der Mercedes]] bewies, dass Züpfner zu Fuß gegangen war.\footnotemark\\
%%        \hspaceThis{[[}in Züpfners box \spacebr{}the Mercedes proofed that Züpfner by foot went was\\
%% \footnotetext{
%% Böll, Heinrich (1963): \emph{Ansichten eines Clowns}. Köln: Kiepenheuer \& Witsch. Quoted from
%% \citew[\page 456]{Speyer2008a}.
%% }
%% \glt `The Mercedes in Züpfners box was proof of Züpfner's walking.'
%% \z
%% Speyer states that this sentence is one out of approximately 6.000, which corresponds to a
%% percentage of 0,016\,\%. According to Speyer this rareness is reflected in the stochastic part of
%% his approach: in general the V2 constraint (1-VF: Exactly one constituent is placed in the \vf) is
%% ranked higher than other constraints but due to the stochastic nature, constraints like
%% Scene-Setting-VF (a frame-setting element is in the \vf) and Contrast-VF (a contrastive element is
%% in the \vf) may be ranked higher. Such deviant rankings allow the derivation of sentences like
%% (\mex{1}):
%% \ea
%% \gll Gestern Briefe hat Uller geschrieben.\\
%%      yesterday letters has Uller written\\
%% \glt `Uller wrote letters yesterday.'
%% \z
%% \citet[\page 466]{Speyer2008a} assumes that such a paradoxical ranking happened when Böll wrote the
%% sentence in (\ref{ex-inzuepfners-box-der-mercedes-zwei}).\footnote{
%% ``Nehmen wir an, dass die PP \emph{In Züpfners Box} ähnlich behandelt wird wie ein klassisches rahmenbildendes
%% Element (von dem es sich, wie angedeutet, nur durch den Skopus, also
%% ein strukturelles Merkmal, aber nicht intrinsisch unterscheidet). Dann kann man
%% annehmen, dass genau solch ein paradoxes Ranking vorgefallen sein muss, als
%% Böll diesen Satz generiert hat.''
%% }
%% I find this approach rather unappealing. It would entail that Nobel prices for literature are due to
%% chance. If it would really be a matter of chance whether certain candidates win in a competition or
%% not one would expect that authors who check their manuscript before submission and the page proofs
%% after acceptance would weed out highly unlikely (and therefore marked) sentences, since the chance
%% that the same constraint ranking happens again in the very moment they read their sentence again is
%% minimal to the extreme. In addition there is an editor who checks manuscripts and the constraint
%% ranking has to happen for him/her at exactly the moment when he/she reads the unlikely/marked sentence.

%% So rather than attributing all multiple frontings to paradoxical constraint violations I would like
%% to claim that there are certain reasons to deviate from the canonical German V2 structure. Writers
%% who master their language are capable to form sentences that are appropriate for certain contexts
%% and know when the respective structure is appropriate. We try to describe some of these situations
%% and provide the respective constraints on information structure properties of the sentences in Chapter~\ref{chap-is}.

%% Apart from these considerations I think that the example from Böll is not an instance of multiple
%% constituents in the \vf but an instance of NP-internal fronting (see Section~\ref{sec-np-internal-frontings}).



\subsection{Remnant movement}


The analyses which come closest to the one I will develop in the following section are those of
\citet{Fanselow93a} and G.\ \citet[Chapter~5.3]{GMueller98a}. Both authors assume that a sentence such as
(\ref{bsp-zum-zweiten-mal-die-Weltmeisterschaft}) has a structural representation as in (\mex{1}) (although
both authors make different assumptions about the nature of the verb trace in the prefield).
\ea
\gll {}[\sub{VP} [Zum zweiten Mal] [die Weltmeisterschaft] \_\sub{V} ]$_i$ errang$_j$ Clark 1965 \_$_i$ \_$_j$.\\
     {}          \spacebr{}to.the second time  \spacebr{}the world.championship {} {} won Clark 1965\\
\z
Fanselow claims that \_\sub{V} is a verb trace, similar to the one which plays a role in gapping. G. Müller,
on the other hand, assumes that \_\sub{V} is a normal verb trace and that cases such as (\mex{0}) should be analyzed
as (\emph{remnant movement}).\footnote{
        Analyses using remnant movement have a long tradition. They started with the work of
		Gert Webelhuth und Hans den Besten \citeyearpar{WdB87a}
        and Craig \citet{Thiersch86a}, which was sadly unpublished.%
}
\citet[Abschnitt~7]{Fanselow2002a} follows G. Müller's remnant movement analysis for cases of multiple fronting.

\citet[\page 281]{Haider93a}, \citet{deKuthy2002a}, \citet{dKM2001a} and
\citet{Fanselow2002a} have however shown that remnant movement analyses of discontinuous NPs and the fronting of incomplete
verbal and adjectival projections run into empirical problems. G.\ \citet{GMueller2014a-u} discusses the scrambling of indefinite problems, but ignores the other problems pointed out by the authors just cited. Therefore, I will pursue an analysis in which 
putative cases of multiple fronting are explained via argument attraction (this corresponds to reanalysis approaches
in the Principles and Parameters Framework). 




\section{Conclusion}
\label{sec-zusammenfassung}

In this chapter, I have presented data which had previously been neglected in many other works. Upon
closer inspection, however, it becomes clear that multiple fronting is in fact not that unusual and
that it is possible to identify clear patterns. This chapter was an attempt to integrate multiple
fronting into the current syntax of German. This chapter provides the analysis of the syntax of
apparent multiple frontings and explains how the interface to semantics works. Of course further
constraints on prosody and information structure are needed for a better understanding of the
phenomenon. I will turn to information structure in Chapter~\ref{chap-is} after having discussed clause types
in the following chapter.



%% \subsection{Multiple NPs preceding the finite verb}

%% The reason for the general unacceptability of (\mex{1}) should also be a focus of future research:
%% \ea[?*]{
%% \label{zwei-np-in-vf}
%% \gll Maria Peter stellt Max vor.\\
%%      Maria Peter introduces Max PRT\\
%% \glt `Max introduces Maria to Peter.'
%% }
%% \z
%% It is not possible to rule out these examples by implementing a general ban on two objects in the prefield
%% (as Haider does \citeyearpar[\page 15]{Haider82}), since these kinds of sentences do seem possible
%% as in the examples in (\mex{1}) and the example in (\mex{2}), which was presented to me by Anette Frank (p.\,c.\ 2002).
%% \eal
%% \ex 
%% \gll Der Maria einen Ring glaube ich nicht, dass er je schenken wird.\footnotemark\\
%% 	 the Maria a ring believes I not that he ever give will\\
%% \footnotetext{
%% \citew[\page 67]{Fanselow93a}.
%% }
%% \glt `I dont think that he would ever give Maria a ring.'
%% \ex 
%% \gll Ihm den Stern hat Irene gezeigt.\footnotemark\\
%% 	 him the star has Irene shown\\
%% \footnotetext{
%%   \citew[\page 412]{Eisenberg94a}.%
%% }
%% \glt `Irene showed him the star.'
%% \ex 
%% \gll (Ich glaube) Kindern Bonbons gibt man besser nicht.\footnotemark\\
%%      \hspaceThis{(}I think children candy gives one better not\\
%% \footnotetext{
%%         G.\ \citew[\page 260]{GMueller98a}.
%% }
%% \glt `I think it's better not to give candy to children.'
%% \zl
%% \ea 
%% \gll Studenten einem Lesetest unterzieht er des öfteren.\\
%%      students a reading.test subjects.to he the often\\
%% \glt `He often makes his students do a reading comprehension test.'	
%% \z
%% Attested examples with two noun phrases in the prefield were provided in
%% (\ref{ex-dem-saft-eine-kraeftige-farbe}), (\ref{bsp-zeitgeist}) and (\ref{bsp-weiterhin-derjugend}).


%% It might be possible to rule out sentences such as (\ref{zwei-np-in-vf}) by additional constraints, which take the
%% information structure of an utterance into account. It is known from research into the fronting of partial
%% verb phrases that factors such as the definiteness of noun phrases affects the acceptability of
%% respective frontings (\citealp[\page 45--46]{Kratzer84a}\ia{Kratzer};
%% \citealp{Haider90a}).

%% With regard to \pref{die-kinder-nach-stuttgart} and \pref{mit-den-huehnern-ins-bett}, \citet[\page 81]{Engel70a} notes
%% that multiple fronting is also used for means of constrastive focus. Multiple fronting is then often used to
%% focus several elements simultaneously. The examples discussed in Section~\ref{sec-phenomenon-mult-front} und in \citew{Mueller2003b}
%% show, however, that constrastive interpretation cannot be the only reason for multiple fronting.

%% \newcommand{\ao}{Avgustinova und Oliva\xspace}%
%% Avgustinova and Oliva \citeyearpar{AO95a,AO97a} examined constituents which can occur in front of clitics in Czech. Normally, there
%% is only exactly one constituent in this position. \ao discuss exceptions to the position of clitics as second in the clause and claim
%% that constituent groupings which may precede clitics also occur before the finite verb in V2"=languages such as German, Dutch and Swedish.
%% They investigated path topicalization, PP-iteration and fronting of various adverbials.
%% %
%% They reach the generalization that syntactic constituents with the same communicative weight assignment (the first "'considerably communicative"'
%% segment) can occur in fronting of the designated second position. In a similar vein, \citet[\page 1639]{Hoberg97a} appeals to a "'specific kind of
%%  minimal communicative entity"'.
 
%% This generalization is itself not sufficient to rule out cases of ungrammatical fronting such
%% as (\mex{1}) as an answer to the question \emph{Who ordered what?}:
%% \ea[*]{
%% \gll Ich das Wienerschnitzel habe bestellt.\footnotemark\\
%%      I   the wiener.schnitzel have ordered\\
%% \footnotetext{
%%         \citew[\page 316]{Lenerz86a}. Also see \citet[\page 32]{Duerscheid89a}.
%% }	 
%% \glt `I ordered the Wiener schnitzel.'
%% }
%% \z










\nocite{HN94a,Kiss95a}
%\bibliography{bib-abbr,biblio}
%\bibliographystyle{natbib.myfullname}


\if 0 



\fi


% Local variables:
% mode: lazy-lock
% End:


%      <!-- Local IspellDict: en_US-w_accents -->
