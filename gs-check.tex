%%%%%%%%%%%%%%%%%%%%%%%%%%%%%%%%%%%%%%%%%%%%%%%%%%%%%%%%%
%%   $RCSfile: grammatiktheorie.tex,v $
%%  $Revision: 1.3 $
%%      $Date: 2010/01/18 14:55:27 $
%%     Author: Stefan Mueller (CL Uni-Bremen)
%%    Purpose: 
%%   Language: LaTeX
%%%%%%%%%%%%%%%%%%%%%%%%%%%%%%%%%%%%%%%%%%%%%%%%%%%%%%%%%

\documentclass[ number=??
                ,series=eotms,
                ,isbn=xxx-x-xxxxxx-xx-x,
%                ,url=http://langsci-press.org/catalog/book/14,
	        ,output=long    % long|short|inprep              
	        %,blackandwhite
	        %,smallfont
                ,bibtex
                ,newtxmath
	        ,draftmode  
		  ]{langsci/langscibook}                          


%% \documentclass[headings=small,bibliography=totoc,index=totoc,%
%% %oneside,% Stauffenburg Style
%% 10pt,%    Stauffenburg Style 
%% %cleardoublempty% unused
%% %draft
%% ]{scrbook}



\usepackage{etoolbox}
\newtoggle{draft}\newtoggle{finished}

%\toggletrue{draft}\togglefalse{finished}
\togglefalse{draft}\togglefalse{finished}


\newcommand{\NOTE}[1]{}
%\newcommand{\NOTE}[1]{\marginpar{#1}}

\newcommand{\LATER}[1]{}


%\includeonly{gs-introduction}
%\includeonly{gs-sentence-structure}
%\includeonly{gs-mult-fronting}
%\includeonly{gs-clause-types}
%\includeonly{gs-informationstructure}
\includeonly{gs-alternatives}
%\includeonly{gs-empty-elements}


\input german-sentence-include
