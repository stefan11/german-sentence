%% -*- coding:utf-8 -*-
%%%%%%%%%%%%%%%%%%%%%%%%%%%%%%%%%%%%%%%%%%%%%%%%%%%%%%%%%
%%   $RCSfile: grammatiktheorie-include.tex,v $
%%  $Revision: 1.13 $
%%      $Date: 2010/11/16 08:40:32 $
%%     Author: Stefan Mueller (CL Uni-Bremen)
%%    Purpose: 
%%   Language: LaTeX
%%%%%%%%%%%%%%%%%%%%%%%%%%%%%%%%%%%%%%%%%%%%%%%%%%%%%%%%%


\chapter{German clause structure}
\label{chap-german-sentence-structure}

This chapter deals with the basic sentence structure of German. Section~\ref{sec-german-order}
introduces the phenomena that have to be covered. As Brigitta Haftka formulated it in the title of
her paper, German is a verb second language with verb last order and free constituent order
\citep{Haftka96a}. This first sounds contradictory, but as will be shown in the following section,
these three properties are indeed independent. I first motivate the categorization of German as
an SOV language in Section~\ref{sec-german-sov}, then I discuss the free constituent order (Section~\ref{sec-free-order-phen}) and the V2
property (Section~\ref{sec-v2-phen}). Verbal complexes interact with free constituent order and are discussed in
Section~\ref{sec-vc-phen}. Frontings of parts of the verbal complex and non-verbal arguments are discussed in
Section~\ref{sec-pvp-phen}.

Section~\ref{sec-analysis-v1-v2} provides the analysis of these phenomena.


\section{The phenomenon}
\label{sec-german-order}



(\mex{1}) provides examples of the main clause types in German: (\mex{1}a) is a verb last (VL) sentence,
(\mex{1}b) is a verb first (V1) sentence, and (\mex{1}c) a verb second (V2) sentence:
\eal
\ex 
\gll dass Peter Maria ein Buch gibt\\
     that Peter Maria a book gives\\
\glt `that Peter gives a book to Maria'
\ex
\gll Gibt Peter Maria ein Buch?\\
     gives Peter Maria a book\\
\glt `Does Peter give a book to Maria?'
\ex
\gll Peter gibt Maria ein Buch.\\
     Peter gives Maria a book\\
\glt `Peter gives a book to Maria.'
\zl
The following subsections deal with all these sentences types and address the question whether one
of them is basic.


\subsection{German as a SOV language}
\label{sec-german-sov}

It is assumed by many researchers that German is an SOV language, although this order is only
visible in embedded clauses like (\mex{0}a) and not in yes/no questions like (\mex{0}b) and
declarative main clauses like (\mex{0}c). The reason for this assumption is that German patterns
with many SOV languages and differs from SVO languages (for example Scandinavian languages). The
analysis of German as an SOV language is almost as old as Transformational Grammar: it was first
suggested by \citet*[\page34]{Bierwisch63a}.
	Bierwisch attributes the assumption of an underlying verb"=final order to \citet{Fourquet57a}. A German translation of the
	French manuscript cited by Bierwisch can be found in \citew[\page117--135]{Fourquet70a}. For other proposals, see \citew{Bach62a},
\citew{Reis74a}, \citew{Koster75a}, and \citew[Chapter~1]{Thiersch78a}. 
%, \citet{denBesten83a} wohl angeblich schon 77 am MIT
Analyses which assume that
German has an underlying SOV pattern were also suggested in \gpsg \citep[\page110]{Jacobs86a}, 
LFG \citep[Section~2.1.4]{Berman96a-u} and HPSG   (\citealp*[Section~4.7]{KW91a}; \citealp{Oliva92a}; \citealp*{Netter92};   
\citealp*{Kiss93}; \citealp*{Frank94}; \citealp*{Kiss95a}; \citealp{Feldhaus97},
\citealp{Meurers2000b}; \citealp{Mueller2005c}). 

The assumption of verb"=final order\label{page-verbletzt} as the base order is motivated by the following observations:\footnote{%
	For points 1 and 2, see \citew[\page34--36]{Bierwisch63a}. For point~\ref{SOV-Skopus} see \citew[Section~2.3]{Netter92}.%
}


\begin{enumerate}
\item Verb particles form a close unit with the verb.
\eal
\ex 
\gll weil er morgen an-fängt\\
	 beause he tomorrow \textsc{prt}-starts\\
\glt `because he is starting tomorrow'
\ex 
\gll Er fängt morgen an.\\
	 he starts tomorrow \textsc{prt}\\
\glt `He is starting tomorrow.'
\zl
This unit can only be seen in verb"=final structures, which speaks for the fact that this structure reflects the base order.

\item Verbs formed by backformation often cannot be separated.

Verbs which are derived from a noun by back-formation\is{back-formation} (\eg \emph{uraufführen} 
`to perform something for the first time', can often not be divided into their component parts and
V2 clauses are therefore ruled out (This was first mentioned by \citet{Hoehle91b} in unpublished
work. The first published source is \citew[\page 62]{Haider93a}):
\eal
\ex[]{
\gll weil sie das Stück heute urauf-führen\\
	 because they the play today \textsc{prt}-lead\\
\glt `because they are performing the play for the first time today'
}
\ex[*]{
\gll Sie urauf"|führen heute das Stück.\\
	 they \textsc{prt}-lead today the play\\
}
\ex[*]{
\gll Sie führen heute das Stück urauf.\\
	 they lead today the play \textsc{prt}\\
}
\zl
The examples show that there is only one possible position for the verb. This order is the one that
is assumed to be the base order.

\item Some constructions allow SOV order only.

Similarly, it is sometimes impossible to realize the verb in initial position when elements like
\emph{mehr als} `more than' are present in the clause \citep{Haider97c,Meinunger2001a}: 
\eal
\ex[]{
\gll dass Hans seinen Profit letztes Jahr mehr als verdreifachte\\
     that Hans his         profit last       year more than tripled\\
\glt `that Hans increased his profit last year by a factor greater than three'
}
\ex[]{
\gll Hans hat seinen Profit letztes Jahr mehr als verdreifacht.\\
     Hans has his    profit last    year more than tripled\\
\glt `Hans increased his profit last year by a factor greater than three.'
}
\ex[*]{
\gll Hans verdreifachte seinen Profit letztes Jahr mehr als.\\
     Hans tripled       his    profit last year more than\\
}
\zl
So, it is possible to realize the adjunct together with the verb in final position, but there are
constraints regarding the placement of the finite verb in initial position.


\item Verbs in non"=finite clauses and in finite subordinate clauses with a conjunction are
always in final position (I am ignoring the possibility of extraposing constituents):
\eal
\ex 
\gll Der Clown versucht, Kurt-Martin die Ware zu geben.\\
     the clown tries Kurt-Martin the goods to give\\
\glt `The clown is trying to give Kurt-Martin the goods.'
\ex 
\gll dass der Clown Kurt-Martin die Ware gibt\\
	 that the clown Kurt-Martin the goods gives\\
\glt `that the clown gives Kurt-Martin the goods'
\zl
The English translation shows that English has VO order where German has an OV order.

\item If one compares the position of the verb in German to Danish\is{Danish} (Danish is an SVO language
like English), then one can clearly see that the verbs in German form a cluster at the end of the sentence,
whereas they occur before any objects in Danish \citep{Oersnes2009b}:
\eal
\label{ex-VO-OV}
\ex 
\gll dass er ihn gesehen$_3$ haben$_2$ muss$_1$\\
	 that he him seen have must\\
\ex 
\gll at han må$_1$ have$_2$ set$_3$ ham\\
     that he must have seen him\\
\glt `that he must have seen him'
\zl


\item\label{SOV-Skopus}\is{scope|(} The scope relations of the adverbs in (\ref{bsp-absichtlich-nicht-anal}) depend on their order:
the left"=most adverb has scope over the two following elements.\footnote{%
At this point, it should be mentioned that there seem to be exceptions from the rule that modifiers to the left take scope over those to
their right. \citet*[\page47]{Kasper94a} discusses examples such as (i), which go back to \citet*[\page137]{BV72}.
\eal
\label{bsp-peter-liest-gut-wegen}
\ex 
\gll Peter liest gut wegen der Nachhilfestunden.\\
	 Peter reads well because.of the tutoring\\
\glt `Peter can read well thanks to the tutoring.'
\ex 
\gll Peter liest wegen der Nachhilfestunden gut.\\
	 Peter reads because.of the tutoring well\\
\zl
% Kiss95b:212
	As \citet[Section~6]{Koster75a} and \citet*[\page67]{Reis80a} have shown, these are not particularly convincing counter"=examples
	as the right sentence bracket is not filled in these examples and it must therefore not necessarily constitute normal reordering inside
	of the middle field, but could instead be a case of extraposition\is{extraposition}.
	As noted by Koster and Reis, these examples become ungrammatical if one fills the right bracket and does not extrapose the causal adjunct:
\eal
\ex[*]{
\gll Hans hat gut  wegen      der Nachhilfestunden gelesen.\\
     Hans has well because.of the tutoring read\\
}
\ex[]{
\gll Hans hat gut gelesen wegen der Nachhilfestunden.\\
	 Hans has well read because.of the tutoring\\
\glt `Hans has been reading well because of the tutoring.'
}
\zl
However, the following example from \citet[\page 383]{Crysmann2004a} shows that, even with the right bracket occupied, one can still have an
order where an adjunct to the right has scope over one to the left:
\ea
\gll Da muß es schon erhebliche Probleme mit der Ausrüstung gegeben haben, da wegen schlechten
  Wetters ein Reinhold Messmer niemals aufgäbe.\\
  there must it already serious problems with the equipment given have since because.of bad weather a Reinhold Messmer never
  would.give.up\\
 \glt `There really must have been some serious problems with the equipment because someone like Reinhold Messmer would never give
  up just because of some bad weather.'
%\ex Stefan  ist wohl deshalb krank geworden, weil er äußerst hart wegen der Konferenz in Bremen gearbeitet hat.
\z
Nevertheless, this does not change anything regarding the fact that the corresponding cases in (\ref{bsp-absichtlich-nicht-anal}) 
and (\ref{bsp-absichtlich-nicht-anal-v1}) have the same meaning regardless of the position of the verb. The general means of semantic
composition may well have to be implemented in the way suggested by Crysmann.

Another word of caution is in order here: There are SVO languages like French that also have a left
to right scoping of adjuncts \citep[\page 156--161]{BGK2004a-u}. So, the argumentation above should not be seen as the only
fact supporting the SOV status of German. In any case the analyses of German that were
worked out in various frameworks can explain the facts nicely.
}
This was explained with the following structure:
\eal
\label{bsp-absichtlich-nicht-anal}
\ex 
\gll weil er [absichtlich [nicht lacht]]\\
	 because he \spacebr{}intentionally \spacebr{}not laughs\\
\glt `because he is intentionally not laughing'
\ex 
\gll weil er [nicht [absichtlich lacht]]\\
     because he \spacebr{}not \spacebr{}intentionally laughs\\
\glt `because he is not laughing intentionally'
\zl
If one compares (\mex{0}) and (\mex{1}) one can see that scope relations are not affected by verb
position. If one assumes that sentences with verb"=second order have the underlying structure in
(\mex{0}), then this fact requires no further explanation. (\mex{1}) shows the structure for (\mex{0}):
\eal
\label{bsp-absichtlich-nicht-anal-v1}
\ex 
\gll Er lacht$_i$ [absichtlich [nicht \_$_i$]].\\
     he laughs \spacebr{}intentionally \spacebr{}not\\
\glt `He is intentionally not laughing.'
\ex 
\gll Er lacht$_i$  [nicht [absichtlich \_$_i$]].\\
     he laughs \spacebr{}not \spacebr{}intentionally\\
\glt `He is not laughing intentionally.'
\zl\is{scope}
%\item Verum-Fokus
\nocite{Hoehle88a,Hoehle97a}
\end{enumerate}\is{verb final language}\is{scope|)}

These properties have been taken as evidence for an underlying SOV order of German. That is, V1 and
V2 sentences are assumed to be derived from or to be somehow related to SOV sentences. It is
possible though to represent the clause types on their own right without relating them. Respective
proposals will be discussed in Chapter~\ref{chap-alternatives}. I assumed such an analysis for ten
years and I think the basic sentence structures can be explained quite well. However, the apparent
multiple frontings, which will be discussed in the next chapter, do not integrate nicely into the
alternative analyses. This caused me to drop my analysis and to revise my grammar in a way that is
inspired by early transformational analyses.

\subsection{German as a language with free constituent order}
\label{sec-free-order-phen}

As was already mentioned in the introduction, German is a language with rather free constituent
order. For example, a verb with three arguments allows for six different orders of the
arguments. This is exemplified with the ditransitive verb \emph{geben} in (\mex{1}):
\eal
\label{ex-free-order}
\ex 
\gll {}[weil] der Mann der Frau das Buch gibt\\
     {}\spacebr{}because the.\nom{} man the.\dat{} woman the.\acc{} book gives\\
\glt `because the man gives the book to the woman'
\ex 
\gll {}[weil] der Mann das Buch der Frau gibt\\
     {}\spacebr{}because the.\nom{} man the.\acc{} book the.\dat{} woman gives\\
\ex 
\gll {}[weil] das Buch der Mann der Frau gibt\\
{}\spacebr{}because the.\acc{} book the.\nom{} man the.\dat{} woman gives\\
\ex 
\gll {}[weil] das Buch der Frau der Mann gibt\\
{}\spacebr{}because the.\acc{} book the.\dat{} woman the.\nom{} man gives\\
\ex 
\gll {}[weil] der Frau der Mann das Buch gibt\\
{}\spacebr{}because the.\dat{} woman the.\nom{} man the.\acc{} book gives\\
\ex 
\gll {}[weil] der Frau das Buch der Mann gibt\\
{}\spacebr{}because the.\dat{} woman the.\acc{} book the.\nom{} man gives\\
\zl

Adjuncts can be placed anywhere between the arguments as the examples in (\mex{1}) show.
\eal
\ex
\gll {}[weil] jetzt der Mann der Frau das Buch gibt\\
     {}\spacebr{}because now the.\nom{} man the.\dat{} woman the.\acc{} book gives\\
\glt `because the man gives the book to the woman now'
\ex
\gll {}[weil] der Mann jetzt der Frau das Buch gibt\\
     {}\spacebr{}because the.\nom{} man now the.\dat{} woman the.\acc{} book gives\\
\glt `because the man gives the book to the woman now'
\ex
\gll {}[weil] der Mann der Frau jetzt das Buch gibt\\
     {}\spacebr{}because the.\nom{} man the.\dat{} woman now the.\acc{} book gives\\
\glt `because the man gives the book to the woman now'
\ex
\gll {}[weil] der Mann der Frau das Buch jetzt gibt\\
     {}\spacebr{}because the.\nom{} man the.\dat{} woman the.\acc{} book now gives\\
\glt `because the man gives the book to the woman now'
\zl
(\mex{0}) is the result of inserting the adverb \emph{jetzt} `now' into every possible position in
(\mex{-1}a). Of course adverbs can be inserted into each of the other sentences in (\mex{-1}) in the
same way and it is also possible to have several adjuncts per clause in all the positions. (\mex{1})
is an example by \citet[\page 145]{Uszkoreit87a} that illustrates this point:
\ea
\gll \emph{Gestern} hatte \emph{in} \emph{der} \emph{Mittagspause} der Vorarbeiter \emph{in}
     \emph{der} \emph{Werkzeugkammer} dem Lehrling \emph{aus Boshaftigkeit} \emph{langsam} zehn
     schmierige Gußeisenscheiben \emph{unbemerkt} in die Hosentasche gesteckt. \\ 
     yesterday had during the lunch.break the foreman in the tool.shop the apprentice maliciously slowly ten
     greasy cast.iron.disks unnoticed in the pocket put\\
\glt `Yesterday during lunch break, the foreman maliciously and
unnoticed, put ten greasy cast iron disks slowly into the
apprentice's pocket.'
\z

In transformational theories it is sometimes assumed that there is a base configuration from which
all other orders are derived. For instance, there could be a VP including the verb and the two
objects and this VP is combined with the subject to form a complete sentence. For all orders in
which one of the objects preceedes the subject it is assumed that there is a movement process that
takes the object out of the VP and attaches it to the left of the sentence.

An argument that has often been used to support this analysis is the fact that scope ambiguities
exist in sentences with reorderings which are not present in the base order. The explanation of such
ambiguities comes from the assumption that the scope of quantifiers
can be derived from their position before movement as well as their position after movement. When
there has not been any movement, then there is only one reading possible. If movement has taken
place, however, then there are two possible readings \citep[\page ]{Frey93a}:
\eal
\ex 
\gll Es ist nicht der Fall, daß er mindestens einem Verleger fast jedes Gedicht anbot.\\
     it is not the case that he at.least one publisher almost every poem offered\\
\glt `It is not the case that he offered at least one publisher almost every poem.'
\ex 
\gll Es ist nicht der Fall, daß er fast jedes Gedicht$_i$ mindestens einem Verleger \_$_i$ anbot.\\
	 it is not the case that he almost every poem at.least one publisher {} offered\\
\glt `It is not the case that he offered almost every poem to at least one publisher.'
\zl
The position from which the NP \emph{jedes Gedicht} `every poem' is supposed to be moved is marked
by a trace (\_$_i$) in the example above. 

It turns out that approaches assuming traces run into problems as they predict certain readings for sentences with multiple traces, which
do not exist (see \citealp[\page 146]{Kiss2001a} and \citealp[Section~2.6]{Fanselow2001a}). 
For instance in an example such as (\mex{1}), it should be possible to interpret \emph{mindestens einem Verleger} `at least one publisher' at
the position of \_$_i$, which would lead to a reading where \emph{fast jedes Gedicht} `almost every poem' has scope over \emph{mindestens einem Verleger} 
`at least one publisher'.
\ea
\gll Ich glaube, dass mindestens einem Verleger$_i$ fast jedes Gedicht$_j$ nur dieser Dichter \_$_i$ \_$_j$ angeboten hat.\\
	 I believe that at.least one publisher almost every poem only this poet {} {} offered has\\
\glt `I think that only this poet offered almost every poem to at least one publisher.'
\z
This reading does not exist, however.
\is{scope|)}

The alternative to a movement analysis is called \emph{base generation}\is{base generation} in
transformational frameworks. The possible orders are not derived by movement but are licensed by
grammar rules directly. Such a base-generation analysis, that is the direct licensing of orders without
any additional mechanisms, is the most common analysis in non-transformational frameworks like HPSG \citep{Pollard90a},
LFG \citep{Berman2003a}, Construction Grammar \citep{Micelli2012a} and Dependency Grammar \citep{Eroms2000a,GO2009a} and I provide such an analysis in Section~\ref{sec-scrambling-analysis}.
\is{constituent order|)}

\subsection{German as a verb second language}
\label{sec-v2-phen}

German is a verb second (V2) language (\citealp[Chapter~2.4]{Erdmann1886a};
\citealp[\page 69, \page 77]{Paul1919a}), that is, (almost) any constituent (an adjunct, subject or
complement) can be placed infront of the finite verb. (\mex{1}) shows some prototypical examples again involving the ditransitive verb
\emph{geben} `to give':
\eal
\ex[]{
\gll Der Mann gibt der Frau das Buch.\\
    the man gives the woman the book\\
\glt `The man gives the woman the book.'
}
\ex[]{
\gll Der Frau gibt der Mann das Buch.\\
    the woman gives the man the book\\
\glt `The man gives the woman the book.'
}
\ex[]{
\gll Das Buch gibt der Mann der Frau.\\
    the book gives the man the woman\\
\glt `The man gives the woman the book.'
}
\ex[]{
\gll Jetzt gibt der Mann der Frau das Buch.\\
    now gives the man the woman the book\\
\glt `The man gives the woman the book now.'
}
\zl
If this is compared with English, one sees that English has XP SVO order, that is the basic SVO
order stays intact and one constituent is placed infront of the sentence into which it belongs:
\eal
\ex The woman, the man gives the book.
\ex The book, the man gives the woman.
\ex Now, the man gives the woman the book.
\zl
Languages like Danish on the other hand are V2 languages like German but nevertheless SVO languages
(see the discussion of (\ref{ex-VO-OV}) on page~\pageref{ex-VO-OV}). Although the verb in embedded
sentences like (\ref{ex-VO-OV}) precedes the object and follows the subject, the finite verb appears
initially and one of the constituents is fronted. The resulting orders are identical to the ones we
see in German. 



Examples such as (\mex{1}) show that occupation of the prefield cannot simply be explained as an ordering variant of an element dependent on 
the finite verb (in analogy to reorderings in the middle field):

\ea
\gll{}[Um zwei Millionen Mark]$_i$ soll er versucht haben, [eine Versicherung \_$_i$ zu betrügen].\footnotemark\\
      \spacebr{}around two million Deutschmarks should he tried have \spacebr{}an insurance.company
              {} to defraud\\
\footnotetext{%
         taz, 04.05.2001, p.\,20.
}
\glt `He supposedly tried to defraud an insurance company of two million Deutschmarks.'
\z
%
The head that governs the PP (\emph{betrügen} `defraud') is located inside of the infinitive clause. The PP as such is not directly dependent on the finite
verb and can therefore not have reached the prefield by means of a simple local reordering operation. This
shows that the dependency between \emph{betrügen} and \emph{um zwei Millionen} `around two million
Deutschmarks' is a long distance dependency: an element belonging to a deeply embedded head has been fronted over several phrasal borders.

Such long distance dependencies are often modeled by devices that assume that there is a position in
the local domain where one would expect the fronted constituent. This is indicated by the \_$_i$,
which is called a gap or a trace. The gap is related to the filler. The alternative to assuming such a gap is
to establish some dependency between the filler and the head on which the filler is dependent. This
is done in Dependency Grammar \citep{Hudson2000a} and in traceless approaches in HPSG
\citep*{BMS2001a} and LFG \citep{KZ89a}. The question is whether
it is reasonable to assume that even simple V2 sentences, that is sentences in which the filler does
not belong to a deeply embedded head, also involve a filler-gap dependency. Approaches that assume
that sentences like (\mex{1}a) are just a possible linearization variant of the verb and its
dependents will have problems in explaining the ambiguity of this sentence. (\mex{1}a) has two
readings, which correspond to the readings of (\mex{1}b) and (\mex{1}c):\todostefan{If reordering in
  the MF can have this effect, this argument is void. Vorfeldbesetzung then would be just a formal
  reordering. The only argument would then be uniformity.}
\eal
\ex\label{ex-oft-liest-er-das-buch-nicht} 
\gll Oft liest er das Buch nicht.\\
     often reads he the book not\\
\glt `It is often that he does not read the book.' or `It is not the case that he reads the book
often.'
\ex
\gll dass er das Buch nicht oft liest\\
     that he the book not often reads\\
\glt `It is not the case that he reads the book often.'
\ex
\gll dass er das Buch oft nicht liest\\
     that he the book often not reads\\
\glt `It is often that he does not read the book.'
\zl
If one assumes that there is a filler-gap dependency in (\mex{0}a), one can assume that the
dependency can be introduced before the negation is combined with the verb or after the
combination. This would immedeatly explain the two readings that exist for (\mex{0}a). Approaches
that assume that the order is a simple ordering variant of the involved constituents would predict
that (\mex{0}a) has the reading of (\mex{0}c) since (\mex{0}a) and (\mex{0}c) have the same order of
\emph{oft} `often' and \emph{nicht} `not' and the order is important for scope determination in German.

\subsection{Distribution of complementizer and finite verb}


\subsection{Verbal complexes}
\label{sec-vc-phen}

It is common to assume that verb and objects form a phrase in VO languages like English. However,
for languages like German, it seems more appropriate to assume that verbs in the right sentence
bracket form a verbal complex and that this verbal complex acts like one complex predicate when it
is combined with the nonverbal arguments. The following examples support this view. If one would
assume a structure like the one in (\mex{1}a), it is difficult to explain the ordering of
(\ref{ex-of}) because the auxiliary \emph{wird} `will' is located between two elements
of the verb phrase.

\eal
\ex
\label{ex-uf}
\gll dass Karl [[das Buch lesen] können] wird]\\
     that Karl \hspaceThis{[[}the book read can will\\
\glt `that Karl will be able to read the book'
\ex
\label{ex-of}
\gll dass Karl das Buch wird lesen können\\
     that Karl the book will read can\\
\glt `that Karl will be able to the read the book.'
\zl

Furthermore, the sentences in (\mex{1}) are not ruled out by such an analysis since \emph{das Buch
  lesen} `the book read' forms a phrase which would be predicted to be able to scramble left in the
middle-field as in (\mex{1}a) or appear in a so-called pied-piping construction with a relative
clause as in (\mex{1}b).
\eal
\ex[*]{
\gll dass [das Buch lesen] Karl wird\\
     that \spacebr{}the book read Karl will\\
}
\ex[*]{
\gll das Buch, [das lesen] Karl wird\\
	 the book \spacebr{}that read Karl will\\
}
\zl
%
\citet*{HN94a}\ia{Hinrichs}\ia{Nakazawa} therefore suggest that (certain) verbal complements are
saturated before non-verbal ones. This means that, in the analysis of (\ref{ex-uf}) and
(\ref{ex-of}), \emph{lesen} `to read' is first combined with \emph{können} `can' and the resulting
verbal complex is then combined with \emph{wird} `will':
\ea
\gll dass Karl das Buch [[lesen können] wird]\\
     that Karl the book \hspaceThis{[[}read can will\\
\z
\emph{wird} `will' can be placed to the right of the embedded verbal complex (as in (\mex{0})), or indeed to the left as
in \pref{ex-of}. After the construction of the verbal complex \emph{lesen können wird}, it is then combined with the 
arguments of the involved verbs, that is with \emph{Karl} and \emph{das Buch} `the book'.\footnote{%
  This kind of structure has already been suggested by \citet*{Johnson86a} in connection with an
  analysis of partial verb phrase fronting.
}

There are also coordination data, such as the example in (\mex{1}), which support this kind of approach.
\ea
\gll Ich liebte ihn, und ich fühlte, daß er mich auch geliebt hat oder doch, daß er mich hätte lieben wollen oder lieben müssen.\footnotemark\\
     I   loved  him and I felt that he me also loved had or PRT that he me   would.have love want or love must.\\
\footnotetext{%
        \citep*[\page 36]{Hoberg81a}
}
\iw{müssen}
\glt `I loved him and felt that he loved me too, or at least he would have wanted to love me or would have had to.'
\z
If one assumes that modal verbs form a verbal complex, \emph{lieben wollen} `love want' and
\emph{lieben müssen} `love must' are constituents and as such they can be coordinated in a symmetric
coordination. The result of the coordination can then function as the argument of \emph{hätte}
`had'.

Arguments of the verbs that are part of a verbal complex may be scrambled as the following example
from \citet[\page 128]{Haider90b} shows:
\ea\label{ex-weil-es-ihr-jemand-zu-lesen-versprochen-hat}
\gll weil es ihr jemand zu lesen versprochen hat\\
     because it.\acc{} her.\dat{} somebody.\nom{} to read promised has\\
\glt `because somebody promised her to read it'
\z
\emph{jemand} `somebody' depends on \emph{hat} `has', \emph{ihm} `him' depends on \emph{versprochen}
`promised' and \emph{es} `it' depends on \emph{zu lesen} `to read'. In principle all six
permutations of these arguments are possible again and hence the verbal complex acts like a simplex
ditransitive verb.

\subsection{Partial verb phrase fronting}
\label{sec-pvp-phen}

The left peripheral elements of this verbal complex can (in some cases together with the adjacent material
from the middle field) be moved into the prefield:
\eal
\ex
\gll Gegeben hat er der Frau das Buch.\\
     given has he the woman the book\\
\glt `He gave the woman the book.'
\ex
\gll Das Buch gegeben hat er der Frau.\\
     the book given   has he the woman\\
\ex
\gll Der Frau gegeben hat er das Buch.\\
     the woman given  has he the book\\
\ex
\gll Der Frau  das Buch gegeben hat er.\\
     the woman the book given   has he\\
\zl
Since the verbal projections in (\mex{1}a--c) are partial, such frontings are called \emph{partial verb
phrase frontings}.



\section{The analysis}
\label{sec-analysis-v1-v2}

The following analysis uses Head-driven Phrase Structure Grammar (HPSG) as its main framework \citep{ps2}. It is, of course,
not possible to provide a comprehensive introduction to HPSG here, so a certain acquaintance with the general assumptions and mechanisms is
assumed for the following argumentation. The interested reader may refer to
\citew{MuellerLehrbuch3,MuellerHPSGHandbook} for introductions that are
compatible with what is presented here. In Section~\ref{sec-annahmen}, I will go over some basic assumptions to aid the understanding of the analysis, and
will also show how the relatively free ordering of constituents in the German \emph{Mittelfeld} can be analyzed. In Section~\ref{sec-v1}, I will recapitulate a 
verb-movement analysis for verb-first word orderings and in Section~\ref{sec-v2} I discuss the analysis of verb-second sentences. Section~\ref{sec-pred-compl}
will deal with the analysis of predicate complexes and the fronting of partial projections.  


\subsection{Background assumptions}
\label{sec-annahmen}
\label{sec-scrambling-analysis}

Every modern linguistic theory makes use of features in order to describe linguistic objects. In HPSG grammars, features are
systematically organized into `bundles'. These bundles correspond to certain characteristics of a linguistic object: for example, syntactic features
form one feature bundle, and semantic features form another. HPSG is a theory about linguistic signs in the sense of Saussure \citeyearpar{Saussure16a}\ia{Saussure}.
The modelled linguistic signs are pairs of form and meaning.

(\mex{1}) shows the feature geometry of signs that I will assume in the
following:
\ea
\ms[sign]
{ phonology   & \type{list~of~phoneme~strings}\\
  synsem & \onems[synsem]{ local \ms[local]{ category & \ms{ head   & \type{head} \\% maj & maj\/ \\} \\
                                                             spr    & \type{list of synsem-objects} \\ 
                                                             comps  & \type{list of synsem-objects} \\ 
                                                             arg-st & \type{list of synsem-objects} \\ 
                                                           } \\
                                             content & \type{cont} \\
                                           } \\
                            nonlocal  \type{nonloc} \\ 
                            lex \type{boolean}\\
             } \\
}
\z
The value of \textsc{phonology}\is{phonology} is a list of phonological forms.
Usually, the orthographic form is used to improve readability.

\textsc{synsem}\is{feature!synsem@\textsc{synsem}} contains syntactic and semantic information.
The feature \textsc{local}\is{feature!loc@\textsc{loc}} (\textsc{loc}) is called as such because syntactic and semantic
information in this path are those which are relevant in local contexts. In contrast,
there is, of course, also non-local information.
Such information is contained in the path \textsc{synsem$|$\-nonloc}\is{feature!nonloc@\textsc{nonloc}}. I will expand
on this in Section~\ref{sec-v2}.
%
Information about the syntactic category of a sign (\textsc{category})\is{feature!cat@\textsc{cat}} and information about its
semantic content (\textsc{content})\is{feature!cont@\textsc{cont}} are `local information'.
\textsc{head}\is{feature!head@\textsc{head}}, \spr, \comps,\isfeat{comps} and \argst\isfeat{arg-st} belong to the features which are
included in the path \textsc{synsem$|$\-loc$|$\-cat} in the feature description.
%
The value of \textsc{head} is a feature structure which specifies the syntactic characteristics that a certain lexical sign 
shares with its projections, that is, with phrasal signs whose head is the corresponding lexical sign.
%
The \textsc{arg-st} feature provides information about the argument structure\is{valence} of a particular sign. Its value is a list which 
includes the elements (possibly only partially specified) with which the sign has to be combined to produce a grammatically
complete phrase. The elements are mapped to valence features like \spr and \comps. I follow
\citet{Pollard90a} in assuming that finite verbs have all their arguments on the \compsl, that is,
there is no difference between subjects and complements as far as finite verbs are concerned. In SVO
languages like English and Danish, the subject is represented under \spr and all other arguments under \comps.

The \lexv has the value + with lexical signs and predicate complexes and $-$ with phrasal projections.\footnote{%
		\citet{Muysken82a} suggests a \textsc{min} feature in \xbart which corresponds to the
		\textsc{lex} feature. A \textsc{max} feature, in the way that Muysken uses it, is not needed since the
		maximality of a projection can be ascertained by the number of elements in its valence list: maximal 
		projections are completely saturated and therefore have empty valence lists.%
}
The lexical item in (\mex{1}) is an example of the finite form of the verb \emph{kennen} `to know'.

%\begin{figure}[htb]
\eas
Lexical item for \emph{kennt} `knows':\\
\label{le-kennt}
\ms[word]{
phon & \phonliste{ kennt }\\
synsem & \onems{ loc \ms{ cat & \ms{ head  & \ms[verb]{ vform & fin \\} \\
                                     spr   & \eliste\\
                                     comps & \sliste{ NP[\type{nom}]\ind{1}, NP[\type{acc}]\ind{2}  } \\
                                   } \\
                          cont & \onems{ hook|index \ibox{3}\\
                                         rels \liste{ \ms[kennen]{
                                                       arg0 & \ibox{3}\\
                                                       arg1 & \ibox{1}\\
                                                       arg2 & \ibox{2}\\
                                                       } }\\
                                    }\\
                        }\\
                 nonlocal \ms{ inherited$|$slash & \eliste \\
                               to-bind$|$slash & \eliste \\ 
                           } \\ 
                 lex $+$\\
             } \\
 }
\zs
%\vspace{-\baselineskip}\end{figure}
%
\emph{kennen} `to know' requires a subject (NP[\type{nom}]) and an accusative object (NP[\type{acc}]).
NP[\type{nom}] and NP[\type{acc}] are abbreviations for feature descriptions which are similar to
(\mex{0}). This requirement is represented on the \argstl, but since this list is identical to the
\compsl for finite verbs, it is not given here. It is in the lexical entry that the syntactic
information is linked to the semantic information. The subscript box on the NPs indicates the
referential index of that particular NP. This is identified with an argument role of the
\emph{kennen} relation. The semantic contribution of signs consist of an index and a list of
relations that are contributed by the sign. The index corresponds to a referential variable for
nouns and for an event variable for verbs. The referential index of a sign is usually linked to its
\textsc{arg0}. I assume Minimal Recursion Semantics (MRS; \citealp*{CFPS2005a}) as the format of the
representation of semantic information. This choice is not important for the analysis of the syntax
of the German clause that is discussed in this chapter and for the analysis of apparent multiple
frontings that is discussed in the following chapter. So the semantic representations are
abbreviated in the following. However, the semantic representation is important when it comes to the
representation of information structure and hence there will be a brief introduction to MRS in
Section~\ref{sec-intro-MRS}. 

Heads are combined with their required elements by means of a very general rule, which (when applied to the
conventions for writing phrase structure rules) can be represented as follows:
\ea
\label{h-c-regel}
H[\comps \ibox{1} $\oplus$ \ibox{3}] $\to$ H[\comps \ibox{1} $\oplus$ \sliste{ \ibox{2} } $\oplus$ \ibox{3} ]~~~ \ibox{2}
\z
The rule in (\mex{0}) combines an element \iboxb{2} from the \compsl of a head with the head itself.
The \compsl of the head is split into three lists using the relation append ($\oplus$), which splits
a list in two parts (or combines two lists into a new one). The first list is \ibox{1}, the second
list is the list containing \ibox{2} and the third list is \ibox{3}. If the \compsl of the head
contains just one element, \ibox{1} and \ibox{3} will be the empty list and since the \compsl of the mother is the concatenation of \ibox{1} and \ibox{3}, the \compsl of the mother node will be the empty list. The H in the rule stands for `Head'. Depending on which syntactic category a rule is instantiated by, the 
H can stand for noun, adjective, verb, preposition or another syntactic category. 
Figure~\vref{abb-weil-er-das-buch-kennt} is an example analysis for the sentence in (\mex{1}).\footnote{%
		In the following figures, H stands for `head', C for `complement', A for `adjunct', F for `filler'
		and CL for `cluster'.
}

\ea
\gll weil er das Buch kennt\\
	 because he the book knows\\
\glt `because he knows the book'
\z

\begin{figure}
\centering
\begin{forest}
sm edges
[V{[\textit{fin}, \comps \eliste]}
	[\ibox{1} NP{[\textit{nom}]}
		[er;he]]
	[V{[\textit{fin}, \comps \sliste{ \ibox{1} }]}
		[\ibox{2} NP{[\textit{acc}]}
			[das Buch;the book ,roof]]
		[V{[\textit{fin}, \comps \sliste{ \ibox{1}, \ibox{2} }]}
			[kennt;knows]]]]
\end{forest}
\caption{Analysis of \emph{weil er das Buch kennt} `because he knows the book'}\label{abb-weil-er-das-buch-kennt}
\end{figure}
Grammatical rules in HPSG are also described using feature descriptions. The rule in (\ref{h-c-regel})
corresponds to Schema~\ref{schema-bin}:

\begin{samepage}
\begin{schema}[Head-Complement Schema]
%~\\*
\label{schema-bin}
\type{head-complement-phrase} \impl\\
\ms{
      synsem & \onems{ loc$|$cat$|$comps \ibox{1} $\oplus$ \ibox{3}\\
                       lex $-$\\
                     }\\
      head-dtr & \onems{ synsem$|$loc$|$cat$|$comps \ibox{1} $\oplus$ \sliste{ \ibox{2} } $\oplus$ \ibox{3} \\
                       }\\
      non-head-dtrs & \liste{\onems{ synsem \ibox{2}  \\ }}\\[2mm]
}
\end{schema}
\end{samepage}
%
In this schema, the head daughter as well as the non-head daughters are represented as values of features  
(as value of \textsc{head-dtr} and as element in the list under \textsc{non-head-dtrs}). Since there are
also rules with more than one non-head daughters in HPSG grammars, the value of \textsc{non-head-dtrs} is a
list. The surface ordering of the daughters in signs licensed by these kinds of 
schemata is not in any sense determined by the schemata themselves. Special linearization rules, which
are factored out from the dominance schemata, ensure the correct serialization of
constituents. Therefore, Schema~\ref{schema-bin} allows both head-complement as well as complement-head orderings. 
The sequence in which the arguments are combined with their head is not specified by the schema. The
splitting of the lists with append allows the combination of any element of the \compsl with the head. The only condition for the possibility of combining a head and an complement is the adjacency of the
respective constituents. It is possible then to analyze (\mex{1}) using Schema~\ref{schema-bin}.   
\ea
\gll	weil das Buch jeder kennt\\
	because the book everyone knows\\
\glt	'because everyone knows the book'
\z
This is shown in Figure~\vref{abb-weil-das-buch-jeder-kennt}.
\begin{figure}
\centering
\begin{forest}
sm edges
[V{[\textit{fin}, \comps \eliste]}
	[\ibox{2} NP{[\textit{acc}]}
		[das Buch;the book, roof]]
	[V{[\textit{fin}, \comps \sliste{ \ibox{2} }]}
		[\ibox{1} NP{[\textit{nom}]}
			[jeder;everyone]]
		[V{[\textit{fin}, \comps \sliste{ \ibox{1}, \ibox{2} }]}
			[kennt;knows]]]]
\end{forest}
\caption{Analysis of \emph{weil das Buch jeder kennt} `because everybody knows the book'}\label{abb-weil-das-buch-jeder-kennt}
\end{figure}
\iboxt{1} and \ibox{3} can be lists containing elements or they can be the empty list. For languages
that do not allow for scrambling either \ibox{1} or \ibox{3} will always be the empty list. For
instance English and Danish combine the head with the complements in the order the elements are
given in the \compsl. Since \ibox{1} is assumed to be the empty list for such languages, Schema~\ref{schema-bin} delivers the right result. The nice effect of this analysis is that
languages that do not allow for scrambling have more constraints in their grammar (namely the additional
constraint that \ibox{1} = \eliste), while languages with less constrained constituent order have
fewer constraints in their grammar. This should be compared with movment"=based analyses where less
restrictive constituent order results in more complex analyses.

This analysis resembles Gunji's analysis for Japanese \citeyearpar{Gunji86a}. Gunji suggests the use of a set-valued
valence feature, which also results in a variable order of argument saturation. For a similar analysis in the terms
of the Minimalist Program, see \citew{Fanselow2001a}.  \citet[Section~3.1]{Hoffmann95a-u} and
\citet{SB2006a-u} suggest respective Categorial Grammar analyses.

In the lexical item for \emph{kennt} `knows' in (\ref{le-kennt}), the meaning of \emph{kennt} is represented as the
value of \textsc{cont}. The Semantics Principle \citep[\page 56]{ps2} ensures that, in Head"=Complement structures, the semantic contribution
of the head is identified with the semantic contribution of the mother. In this way, it is ensured that the meaning of \emph{er das Buch kennt}
is present on the highest node in Figure~\vref{abb-weil-er-das-buch-kennt-semp}. The association with the various arguments is already ensured by
the corresponding co-indexation in the lexical entry of the verb.\footnote{%
		The formula $kennen(er, buch)$ is a radical simplification. It is not possible to go into
		the semantic contribution of definite NPs or the analysis of quantifiers here. See
                \citew*{CFPS2005a} for an analysis of scope phenomena in Minimal Recursion Semantics.%
}


\begin{figure}
\centering
\begin{forest}
sm edges
[V{[\textsc{cont} \ibox{1}]}
	[NP{[\textit{nom}]}
		[er;he]]
	[V{[\textsc{cont} \ibox{1}]}
		[NP{[\textit{acc}]}
			[das Buch;the book, roof]]
		[V{[\textsc{cont}\,\ibox{1}\,kennen{(er, buch)}]}
			[kennt;knows]]]]
\end{forest}
\caption{Analysis of \emph{weil er das Buch kennt} `becaue he knows the book'}\label{abb-weil-er-das-buch-kennt-semp}
\end{figure}


After considering the syntactic and semantic analysis of Head"=Complement structures, I now turn to adjunct
structures.
%
Modifiers are treated as functors in HPSG. They select the head that they modify via the feature \textsc{mod}. The
adjunct can therefore determine the syntactic characteristics of the head that it modifies. Furthermore, it can access
the semantic content of the head and embed this under its own. The analysis of adjuncts will be made clearer by examining
the following example (\mex{1}):

\ea
\gll weil er das Buch nicht kennt\\
	 because he the book not knows\\
\glt `because he doesn't know the book'
\z
%
\emph{nicht} `not' modifies \emph{kennt} `knows' and embeds the relation $kennen(er, buch)$ under the
negation. The semantic contribution of \emph{nicht kennt} `not knows' is therefore $\neg kennen(er, buch)$.
The lexical entry for \emph{nicht} is shown in (\mex{1}).
\ea
Lexical entry for \emph{nicht} `not':\\
\ms{
cat & \ms{ head & \ms[adv]{ mod & \onems{ loc \onems{ cat$|$head \type{verb}\\
                                                      cont \ibox{1}\\
                                                    }\\
                                     }\\
                          }\\
           spr   & \eliste\\
           comps & \eliste\\
         }\\
cont & $\neg$ \ibox{1}\\
}
\z
This entry can modify a verb in head-adjunct structures which are licensed by Schema~\ref{ha-schema}.

\begin{samepage}
\begin{schema}[Head-Adjunct Schema]
\label{ha-schema}
\type{head-adjunct-phrase} \impl\\
\ms{
head-dtr      & \ms{ synsem & \ibox{2}\\
                   }\\[2mm]
non-head-dtrs & \liste{ \ms{ synsem$|$loc & \ms{ cat & \ms{ head$|$mod & \ibox{2}\\
                                                            spr & \eliste\\
                                                            comps   & \eliste\\
                                                          }\\
                                               }\\
                           }}\\
}
\end{schema}
\end{samepage}


Pollard and Sag's Semantics Principle ensures that the semantic content in head-adjunct structures 
is contributed by the adjunct daughter. Figure~\vref{abb-weil-er-das-buch-nicht-kennt-semp} shows
this analysis in detail.\todostefan{Add tree labels to all figures}
\begin{figure}
\centering
\begin{forest}
sm edges
[V{[\textsc{cont} \ibox{1}]}
	[NP{[\textit{nom}]}
		[er;he]]
	[V{[\textsc{cont} \ibox{1}]}
		[NP{[\textit{acc}]}, fit=band
			[das Buch;the book, roof]]
		[V{[\textsc{cont} \ibox{1}]}
			[Adv\feattab{\textsc{mod} \ibox{2} \textsc{cont} \ibox{3},\\
                                     \textsc{cont} \ibox{1} $\neg$ \ibox{3}}
				[nicht;not]]
			[\ibox{2} V{[\textsc{cont} \ibox{3} kennen{(er, buch)}]}
				[kennt;know]]]]]
\end{forest}
\caption{Analysis of \emph{weil er das Buch nicht kennt} `because he does not know the book'}\label{abb-weil-er-das-buch-nicht-kennt-semp}
\end{figure}

The \textsc{mod} value of the adjunct and the \synsemv of the verb are co-indexed by the Head"=Adjunct Schema \iboxb{2}.
Inside the lexical entry for \emph{nicht}, the \contv of the modified verb (\iboxt{3} in Figure~\ref{abb-weil-er-das-buch-nicht-kennt-semp})
is co-indexed with the argument of $\neg$. The semantic content of \emph{nicht} (\ibox{1} $\neg kennen(er, buch)$) becomes the semantic
content of the entire Head"=Adjunct structure and is passed along the head path until it reaches the highest node.

After this recapitulation of some basic assumptions, the following section will present a verb-movement analysis for 
verb-initial word order in German.

\subsection{V1}
\label{sec-v1}


As is common practice in Transformational Grammar and its successive models (\citealp*[\page 34]{Bierwisch63a}; \citealp{Bach62a}; \citealp{Reis74a}; 
\citealp[Chapter~1]{Thiersch78a}), I will assume that verb-first sentences have a structure that is
parallel to the one of verb-final sentences and that an empty element fills the position occupied by
the verb in verb-last sentences.\footnote{%
The alternative is that they are flat structures, which allow the verb to be positioned in both initial
and final position \citep{Uszkoreit87a,Pollard90a}, or linearization analyses 
\citep{Reape92a,Reape94a,Mueller99a,Mueller2002b,Kathol95a,Kathol2000a}. In linearization analyses, the domain
in which constituents can be permuatated is expanded so that, despite being a binary branching structure, verb-first
and verb-final orderings can be derived. 
The differing possibilities will be discussed further in Chapter~\ref{chap-alternatives}.%
} 

A radically simplified variant of the transformational analysis of (\mex{1}b) is presented in Figure~\vref{fig-verb-movement-gb}.
\eal
\ex 
\gll dass er das Buch kennt\\
     that he the book knows\\
\glt `that he knows the book'\label{bsp-dass-er-das-buch-kennt}
\ex 
\gll Kennt$_i$ er das Buch \_$_i$?\\
     knows{} he the book\\
\glt `Does he know the book?'\label{bsp-kennt-er-das-buch}
\zl
\begin{figure}
\centering
\begin{forest}
sm edges
[CP
	[\cnull
		[kennt;knows]]
	[VP
		[NP
			[er;he]]
		[VP
			[NP
				[das Buch;the book, roof]]
			[\vnull
				[\trace]]]]]
\end{forest}
\caption{\label{fig-verb-movement-gb}Analysis of \emph{Kennt er das Buch?} `Does he know the book?' with Move-$\alpha$}
\end{figure}


The verb is moved from verb-final position to C$^0$.\footnote{%
  In more recent analyses the verb is adjoined to C$^0$. While V-to-c-movement analyses work well for German and
  Dutch\il{Dutch} they fail for other V2 languages that allow for the combination of complementizers
  with V2 clauses \citep{Fanselow2009b}. This will be discussed in more detail in
  Subsection~\ref{sec-v2-germanic}.
} This movement can be viewed as creating a new tree structure out
of another, i.e. as a derivation. In the analysis of (\mex{0}b), two trees enter a relation with each other $-$ the
tree with verb-final ordering and the tree where the verb was moved into first position. One can alternatively assume a 
representational model where the original positions of elements are marked by traces (see %\citew{McCawley68a}; Pullum2007a:3 sagt, dass das nicht MTS ist
Koster \citeyear[\page ]{Koster78b-u}; \citeyear[\page 235]{Koster87a-u}; 
%\citealp[\page 66, Fußnote~4]{Bierwisch83a}; 
\citealp{KT91a}; \citealp[Section~1.4]{Haider93a}; 
\citealp[\page 14]{Frey93a}; \citealp[\page 87--88, 177--178]{Lohnstein93a-u}; \citealp[\page 38]{FC94a}; \citealp[\page 58]{Veenstra98a}, for example). This kind of representational view 
is also assumed in HPSG. In HPSG analyses, verb-movement is modeled by a verb-trace in final position coupled with the percolation of
the properties of the verb trace in the syntactic tree. 

In what follows, I discuss another option for modeling verb-movement.
The C-head in Figure~\ref{fig-verb-movement-gb} has different syntactic characteristics from
V$^0$ in verb-final orders: the valence of the verb in final position does not correspond to
the valence of the element in C. The functional head in C is combined with a VP (an IP in several
works), whereas the verb in final structures requires a subject and an object. 
In HPSG, the connection between the element in V1-position and the actual verb can be captured by an analysis which
assumes that there is a verb trace in verb-initial structures that has the same valence properties and the same semantic
contribution as an overt finite verb in final position and is also present in the same
position.\footnote{%
  In the grammar developed in this book, it is impossible to say that a head follows or precedes its
  dependents if the head is empty. The reason is that the head daughter and the non-head daughters
  are the values of different features: the head daughter is the value of \textsc{head-dtr} and the
  non-head daughters are members of the \textsc{non-head-dtrs} list. It is only the \phonvs of the
  daughters that are serialized \citep{Hoehle94a}. So in a structure like [NP$_1$ [NP$_2$ t]] one cannot tell whether
  NP$_2$ precedes t or follows it since in the AVM these two objects are just presented on top of each
  other and the phonology does not show any reflex of t that would help us to infer the order. Note
  however that t has the \initialv `$-$' and hence the phonology of t is appended to the end of the
  phonology of NP$_2$. It does not matter whether we append the empty string at the end or at the
  beginning of a list, but the \initialv of the head matters when NP$_1$ is combined with [NP$_2$ t]:
  the complex phrase [NP$_2$ t] has to be serialized to the right of NP$_1$. If both NP$_1$ and
  NP$_2$ contain phonological material, the material contributed by NP$_1$ will precede the material
  from NP$_2$. So, we will always know that the trace is in a unit that contains other material and
  this unit is serialized as if there would be a visible head in it. This means that despite Höhle's
  claim to the contrary traces can (roughly) be localized in structures.

  Note that \citet{GSag2000a-u} represent both head and non-head daughters in the same list. If one
  assumes that this list is ordered according to the surface order of the constituents, traces are
  linearized.

  Traces will be shown in final position in the tree visualizations throughout this book.
}
The element in intial-position
is licensed by a lexical rule, which licenses a verb that takes the initial position and selects for
a projection of the verb trace. To make this clearer, we will take a closer look at the sentence in (\ref{bsp-kennt-er-das-buch}):
the syntactic aspects of the analysis of (\ref{bsp-kennt-er-das-buch}) are shown in Figure~\vref{fig-verb-movement-syn}.
\begin{figure}
\centering
\begin{forest}
sm edges
[V{[\comps \eliste]}
	[V{[\comps \sliste{ \ibox{1} }]}
		[V{[\textsc{comps \ibox{2}}]}, tier=np,edge label={node[midway,right]{V1-LR}}
			[kennt;knows]]]
	[\ibox{1} V\feattab{\textsc{dsl}|\textsc{cat}|\comps \ibox{2},\\
                            \comps \eliste}
		[\ibox{3} NP{[\textit{nom}]}, tier=np
			[er;he]]
		[V\feattab{\textsc{dsl}|\textsc{cat}|\comps \ibox{2},\\
                           \comps \sliste{ \ibox{3} }}
			[\ibox{4} NP{[\textit{acc}]}
				[das Buch;the book, roof]]
			[V\feattab{\textsc{dsl}|\textsc{cat}|\comps \ibox{2},\\
                                   \comps \ibox{2} \sliste{ \ibox{3}, \ibox{4} }}
				[\trace]]]]]
\end{forest}
\caption{\label{fig-verb-movement-syn}Analysis of \emph{Kennt er das Buch?} `Does he know the book?'}
\end{figure}
%
In the verb trace, the \compsv of the trace is co-indexed with the value of the \compsf under
\textsc{dsl}. The feature \dsl was introduced by \citet*{Jacobson87} with the aim of describing head movement in inversion
structures in English. \dsl stands for \emph{double slash} and is sometimes abbreviated as `//' in figures.\footnote{%
  In addition to \dsl, there is a \slasch feature that is used for the analysis of nonlocal
  dependencies. This will be explained in Section~\ref{sec-v2}.
} \citet{Borsley89} adopted Jacobson's idea and translated it into HPSG terms thereby 
showing how head movement in a HPSG variant of the CP/IP system can be modeled using
\textsc{dsl}. The introduction of such a feature to HPSG in order to describe movement operations is
motivated by the fact that this kind of movement is local, unlike the long-distance dependencies discussed in Section~\ref{sec-v2}.

The verb trace in (\mex{1}) takes on the role of the finite verb in the analysis of
(\ref{bsp-dass-er-das-buch-kennt}).\footnote{%
  The \spr feature is ignored here. As will become clear later, the \sprv of the trace and the
  \dslf are also shared. The \sprv of finite verbs is always the empty list in German and hence the
  \sprv of the trace is the empty list as well.
}
\eas
Verb trace (valence information):\\
\label{le-verbspur}
\onems{
phon \eliste\\
synsem$|$loc$|$cat \ms{ head & \ms[verb]{
                                           dsl$|$cat$|$comps & \ibox{1}\\
                                           }\\
                                    comps & \ibox{1}
                                  }\\
}
\zs


Since \textsc{dsl} is a head feature, it is passed on towards the top of the tree so that information
about the valence of the verb trace is present at each projection.
A special version of the finite verb takes the projection of the verb trace (\iboxb{1} in Figure~\vref{fig-verb-movement-syn}) as its argument. As they are combined,
it is checked whether the valence of the original verb \iboxb{2} matches the valence of the verb
trace (\textsc{dsl$|$cat$|$comps} \ibox{2}\,).

The special lexical item for V1-ordering is licensed by the following lexical rule:\footnote{\label{fn-koord-vm}%
		I am adopting a view which integrates lexical rules into the formalism of HPSG and treats them as
		unary rules \citep{Meurers2001a}.
		Lexical rules are applied to stems or entire words \citep{Mueller2002b}.
		Verb-movement will -- as in previous publications about verb-movement in HPSG -- be described using 
		lexical rules. The following data suggests, however, that it is appropriate to speak of unary syntactic rules rather than lexical rules:
		 \ea
        \gll Karl kennt und schätzt diesen Mann.\\
             Karl knows and values this man\\
	\glt `Karl knows and values this man.'
        \z
		(i) cannot be analyzed applying the verb-movement rule to each verb individually and then coordinating the
		result, since \emph{kennen} `to know' and \emph{schätzen} `to value' have different \contvs. The \contv
		of the verb trace is determined by the \contv of the verb in initial position. The coordination of two products of
		a lexical rule for verb-movement would not be allowed as the standard coordination theory of \citet[\page 202]{ps2} states
		the valence requirements of both conjuncts be the same. Such a problem does not arise, however, if we apply a unary syntactic 
		rule (parallel to (\ref{lr-verb-movement2})) to the result of the coordination.%
}


\ea
\begin{tabular}[t]{@{}l@{}}
\label{lr-verb-movement}
Lexical rule for verb in initial position (valence information):\\
\ms{
synsem$|$loc & \ibox{1} \ms{ cat$|$head & \ms[verb]{ vform & fin\\
                                                     initial & $-$\\
                                             }\\
                  }\\
} $\mapsto$\\
\ms{
synsem$|$loc$|$cat & \ms{ head & \ms[verb]{vform & fin\\
                                           initial & $+$\\
                                             }\\
                           spr & \eliste\\
                           comps & \sliste{ \ms{ loc$|$cat \ms{ head  \ms[verb]{
                                                               dsl & \ibox{1}\\
                                                               }\\
%                                                         spr   \eliste\\
                                                                                  %SPR ist leer,
                                                                                  %weil die finiten
                                                                                  %Verben immer
                                                                                  %leere SPR-Werte
                                                                                  %haben, deshalb
                                                                                  %kann auch über
                                                                                  %den Trace nie
                                                                                  %etwas anderes
                                                                                  %projiziert werden
                                                                                  %und wir müssen
                                                                                  %hier nichts
                                                                                  %sagen. Oder? 08.06.2015
                                                         comps \eliste\\
                                                       }\\
                                              }}\\
                         }\\
}
\end{tabular}
\z

The verb licensed by this lexical rule selects the maximal projection of the verb trace which has
the same local properties as the input verb.\footnote{%
  In principle one would have to specify the \sprv of the selected argument to be the empty
  list. However, since the \sprv of the trace is identical to the \sprv of the fronted verb and
  since fronted verbs are always finite and since finite verbs have the empty list as the \sprv, the
  \sprv of the complement may be left unspecified in the lexical rule. This is different for Danish
  and in the Danish equivalent of the lexical rule the \sprv has to be specified.
}
This is achieved by co-indexing the \localv of the
input verb and the \dslv of the selected verbal projection. Only finite verbs in final position (\textsc{initial}$-$) 
can function as an input for this rule. The output is a verb in initial position (\textsc{initial}+).
Linearization rules make reference to the \textsc{initial} feature and ensure the correct ordering of heads in 
local trees.


Nothing has been said about semantics so far. It is assumed that the verb trace also shares the
semantic properties of the verb in initial position and that verb-initial clauses are interpreted
like their verb-final counterparts (see the discussion of (\ref{bsp-absichtlich-nicht-anal-v1}) on
page~\pageref{bsp-absichtlich-nicht-anal-v1}). This can be modeled by threading the semantic
contribution in parallel with the valence properties through the tree. (\ref{le-verbspur2-prelim})
shows the verb trace enriched with semantic information: 

\eas
\label{le-verbspur2-prelim}
Verb trace (Valence information and semantic content):\\
\onems{
phon \eliste\\
synsem$|$loc \ms{ cat & \ms{ head & \ms[verb]{
                                           dsl & \ms{ cat & \ms{ comps & \ibox{1}\\
                                                               }\\
                                                      cont & \ibox{2}\\
                                                   }\\
                                           }\\
                                    comps & \ibox{1}\\
                            }\\
                  cont & \ibox{2}\\
                }\\
}
\zs


By co-indexing the \contvs, the trace behaves semantically just like the original verb, which is
now in initial position.

If one allows cyclic feature structures, (\mex{0}) can be represented in a more compact manner
as in (\mex{1}) \citep[\page 207]{Meurers2000b}:  
\eas
\label{le-verbspur2}
Verb trace according to \citew[\page 207]{Meurers2000b}:\\
\onems{
phon \eliste\\
synsem$|$loc \ibox{1} \onems{ cat$|$head$|$dsl \ibox{1} }\\
}
\zs
The fact that all \local properties of a verb trace are represented under \textsc{dsl} is captured much
more directly here. It is no longer necessary to have separate structural sharings or explicitly mention individual types and features
under \textsc{head} (as in (\ref{le-verbspur2-prelim})).

The Semantics Principle ensures that the \contv is passed along the head projection during the combination of arguments towards the top of the tree. 
In the last step of the projection in Figure~\ref{fig-verb-movement-syn}, the verb in initial position is the head and therefore the semantic
content of this verb will be projected. In the lexical rule (\mex{1}) for the verb in initial position, 
the semantic content of the projection of the trace in final position \iboxb{2} is identified with the 
\contv of the verb in initial position.


\ea
\begin{tabular}[t]{@{}l@{}}
Lexical rule for verbs in initial position (valence und semantic\\
contribution):\\
\ms{
synsem$|$loc & \ibox{1} \ms{ cat$|$head & \ms[verb]{ vform & fin\\
                                                     initial & $-$\\
                                             }\\
                  }\\
} $\mapsto$\\*
\onems{
synsem$|$loc \onems{ cat  \ms{ head & \ms[verb]{ vform & fin\\
                                                     initial & $+$\\
                                             }\\
                               spr & \eliste\\
                           comps & \sliste{ \onems{ loc \onems{ cat \ms{ head & \ms[verb]{
                                                                                     dsl & \ibox{1}\\
                                                                                 }\\
                                                                         spr & \eliste\\
                                                                         comps & \eliste\\
                                                                    }\\
                                                           cont \ibox{2}\\
                                                         }\\
                                              }}\\
                         }\\
                   cont \ibox{2}\\
             }\\
}
\end{tabular}
\label{lr-verb-movement2}
\z


Due to this combination, the semantic content of the verb trace projection is then taken over by the verb 
in initial position and, as per the Semantics Principle, becomes the semantic contribution of the entire 
construction. Figure~\vref{fig-verb-movement-sem} shows the semantic aspects of the verb-movement analysis with the
trace in (\ref{le-verbspur2}) and the lexical rule in (\ref{lr-verb-movement2}).

\begin{figure}
\centering
\begin{forest}
sm edges
[V{[\textsc{cont} \ibox{2}\,]}
	[V{[\textsc{cont} \ibox{2}\,]}
		[V{[\textsc{cont} \ibox{1} kennen(er, buch)]}, tier=np,edge label={node[midway,right]{V1-LR}}
			[kennt;knows]]]
	[V\feattab{\textsc{dsl}|\textsc{cont} \ibox{1},\\
                   \textsc{cont} \ibox{2}\,}
		[NP{[\textit{nom}]}, tier=np
			[er;he]]
		[V\feattab{\textsc{dsl}|\textsc{cont} \ibox{1},\\
                           \textsc{cont} \ibox{2}\,}
			[NP{[\textit{acc}]}
				[das Buch;the book, roof]]
			[V\feattab{\textsc{dsl}|\textsc{cont} \ibox{1},\\
                                   \textsc{cont} \ibox{2}$=$\ibox{1}\,}
				[\trace]]]]]
\end{forest}
\caption{\label{fig-verb-movement-sem}Analysis of \emph{Kennt er das Buch?} `Does he know the book?'}
\end{figure}


Technically speaking, \iboxt{1} and \iboxt{2} in Figure~\ref{fig-verb-movement-sem} are identical. To make aid 
representation, they have been represented by different numbers. The identification of \ibox{1} and
\ibox{2} is enforced by the identification of the information under \textsc{local} and \textsc{dsl} in the
lexical entry of the trace (\ref{le-verbspur2}),
as \cont is a \textsc{local} feature.


The analysis in Figure~\ref{fig-verb-movement-sem} may seem somewhat complicated, since semantic information is passed on both
via the \dsl from the verb in initial position to the trace \iboxb{1} and by the verb trace to the verb in
initial position \iboxb{2}. However, once we consider examples with adjuncts, it will become clear that this seemingly complicated treatment is justified. The analysis of (\mex{1}) is given in 
Figure~\vref{fig-verb-movement-adjunkt-sem}.
\ea
\gll Kennt er das Buch nicht?\\
	 knows he the book not\\
\glt `Doesn't he know the book?'
\z
\begin{figure}
\oneline{%
\begin{forest}
sm edges
[V{[\textsc{cont} \ibox{1}\,]}
	[V{[\textsc{cont} \ibox{1}\,]}
		[V{[\textsc{cont} \ibox{2} kennen(er, buch)]}, tier=np,edge label={node[midway,right]{V1-LR}}
			[kennt;knows]]]
	[V\feattab{\textsc{dsl}|\textsc{cont} \ibox{2},\\
                   \textsc{cont} \ibox{1}\,}
		[NP{[\textit{nom}]}, tier=np
			[er;he]]
		[V\feattab{\textsc{dsl}|\textsc{cont} \ibox{2},\\ 
                           \textsc{cont} \ibox{1}\,}
			[NP{[\textit{acc}]}, fit=band
				[das Buch;the book, roof]]
			[V\feattab{\textsc{dsl}|\textsc{cont} \ibox{2},\\
                                   \textsc{cont} \ibox{1}\,}
				[Adv\feattab{\textsc{mod} \ibox{3} {[\textsc{loc}|\textsc{cont} \ibox{2}\,]},\\
                                             \textsc{cont} \ibox{1} $\neg$ \ibox{2}}
					[nicht;not]]
				[\ibox{3} V\feattab{\textsc{dsl}|\textsc{cont} \ibox{2},\\
                                                    \textsc{cont} \ibox{2}\,}
					[\trace]]]]]]
\end{forest}
}
\caption{Analysis of \emph{Kennt er das Buch nicht?} `Doesn't he
  know the book?'}\label{fig-verb-movement-adjunkt-sem}
\end{figure}


The initial verb \emph{kennt}, which is licensed by a lexical rule, requires a verbal projection
with a \textsc{dsl$|$cont} value of
$kennen(x, y)$, where the $x$ and $y$ in the lexicon entries for \emph{kennt} are already linked to arguments, which will later be
filled by \emph{er} and \emph{das Buch}. The \textsc{dsl$|$cont} value of the verbal projection is -- due to \dsl being a head feature -- also
restricted at the trace. At the trace, the \contv{} is co-indexed with \textsc{dsl$|$cont} value so that the trace has the same semantic representation
as the verb \emph{kennt}, which was the input for the verb-first lexical rule. The verb trace is then modified by the adjunct \emph{nicht} 
and the meaning of the head"=adjunct structure is passed up to the mother node \iboxb{1}. During the combination with its arguments, the 
meaning is then transmitted up to the maximal projection of the verb trace in Figure~\ref{fig-verb-movement-adjunkt-sem}. The \contv of this
projection is identical to the \contv of the initial verb due to the structure sharing in the lexical item for this verb, which is licensed by the lexical 
rule (\ref{lr-verb-movement2}). Because the verb in first position is the head of the entire structure and it is a head"=argument structure, the semantic
content of the structure is identical to that of the V1-verb, i.e. \iboxt{1} in Figure~\ref{fig-verb-movement-adjunkt-sem}.

Finally, sentences such as (\mex{1}) must be somehow ruled out:
\ea[*]{
\gll Kennt er das Buch kennt.\\
     knows he the book know\\
}
\z


(\mex{0}) could be analyzed in such a way that the first occurrence of \emph{kennt} is the ouput of a verb-movement rule
and the \textsc{dsl} value of the second \emph{kennt} is unrestricted, so that the second \emph{kennt} can take over the 
same role as the verb trace in our analysis. Generally speaking, it is not possible for all overtly realised verbs to demand that
their  \textsc{dsl} value be \emph{none} since these verbs represent the input for the lexical rule for verb movement and
the \textsc{local} value of the input verb is identified with the \textsc{dsl} value of the verb trace selected by the output verb.
If all overt verbs had the \textsc{dsl} value \emph{none}, it would lead to a contradiction during the combination with the verb
trace since the trace has a specified \textsc{dsl} value (the trace is cyclic, therefore the value of \textsc{loc$|$cat$|$\-head$|$\-dsl$|$\-cat$|$\-head$|$\-dsl}
is not compatible with \emph{none}).
(\mex{0}) is excluded by a restriction which states that a verb has to have the \textsc{dsl} value \emph{none} when it is
overtly realised and enters a syntactic structure. The desired result is achieved by the implication in (\mex{1}):

\ea
\ms{ head-dtr & \ms[word]{ phon & non-empty-list \\
                         }
} \impl \onems{ synsem$|$l$|$cat$|$head$|$dsl \type{none}\\
            }
\z
%
This restriction differs from that of \citet[\page 207]{Meurers2000b} and others in that the \textsc{head-daughter} in the antedecent must be of the type \emph{word}. Without this restriction, the
constraint could be applied to projections of the verb trace and thereby exclude well-formed sentences.

Following the discussion of the analysis of verb-first sentences, the next section focuses on the analysis of verb-second
sentences.


\subsection{V2}
\label{sec-v2}


Verb-second sentences such as (\mex{1}b) are, as we have already mentioned, related to verb-first sentences such as (\mex{1}a) 
in most German grammars.\footnote{%
		 \citet[Chapter~6.3]{Kathol95a}, \citet{GO2009a}, and \citet{Wetta2011a} are exceptions. These authors analyze
		 short fronting as in (\mex{1}b) as an alternative ordering for 
		 the constituents in (\mex{1}a). They do, however, assume long-distance dependencies 
		 for sentences such as (\ref{bsp-um-zwei-millionen}). \citet{Kathol2001a} revised
                 his treatment and now assumes a uniform analysis of V2 phenomena in German. 

                 Approaches that treat local frontings different are discussed in more detail in Section~\ref{sec-local-frontings-alternatives}.
}

\eal
\ex
\gll Kennt er das Buch?\\
	 knows he the book\\
\glt `Does he know the book?'
\ex
\gll Das Buch$_i$ kennt \_$_i$ er.\label{ex-das-buch-kennt}\\
	 the book knows {}     he\\
\glt `He knows the book'	 
\zl\todostefan{position trace / Frey/Fanselow}

In the second example, \emph{das Buch} is situated in the prefield. The position in the middle field, where
the object could also occur, is empty. This position is most often represented by `\_'. (\mex{1}) shows that
elements  which are dependent on an embedded head can occur in the pre-field:

\ea
\label{bsp-um-zwei-millionen}
\gll{}[Um zwei Millionen Mark]$_i$ soll er versucht haben, [eine Versicherung \_$_i$ zu betrügen].\\
    \spacebr{}of two million Deutschmarks{} should he tried have \spacebr{}an insurance.company {} to defraud\\\footnote{%
         taz, 04.05.2001, \page 20
}
\glt `He supposedly tried to defraud an insurance company of two million Deutschmarks'
\z

Therefore, occupying the pre-field (fronting) creates a long-distance dependency. 
To deal with long-distance dependencies, \citet[Chapter\,4]{ps2}\ia{Pollard}\ia{Sag} suggest 
a silent element which introduces a non-local dependency: \footnote{%
		In Chapter~9, \citet{ps2} introduce a lexical rule for complement extraction.
		It is possible to describe long-distance dependencies with this rule and avoid using
		a phonologically null element. A further alternative would be unary projections, as I suggest in
		\citep*[Chapter~9, 10, 18]{Mueller99a}. A discussion of the alternatives can be found in 
		\citep[Chapter~6.2.5.1]{Mueller2002b} and in Chapter~\ref{chap-empty} of this book. In more recent works in HPSG, relational
                argument realization principles and lexical analyses are assumed for extraction
                \citep*{BMS2001a}. See \citew{LH2006a} for a discussion of such relational approaches.
		
		For phenomena such as relative and interrogative clauses, one needs the features \textsc{rel}
		and \textsc{que} in addition to \slasch. These features are omitted in what follows.
}

\eas
\label{trace}
Trace for the description of long-distance dependencies:\\
%\ms[lexical-sign]{
\ms{
 phon & \phonliste{} \\
 synsem & \ms{ local & \ibox{1} \\
               nonlocal & \ms{ inherited & \ms{ slash & \sliste{\ibox{1}} \\
                                               } \\
                               to-bind   & \ms{ slash & \eliste \\ 
                                              } \\ 
                              } \\ 
             } \\
}
\zs


This kind of trace can stand for a complement or an adjunct depending on the context.
The characteristics of the object, which are represented under \textsc{synsem$|$"-local}, are
entered into the \textsc{slash} list under \textsc{synsem$|$"-nonlocal$|$"-inherited$|$"-slash}.
The \textsc{nonloc} Principle ensures the percolation of non-local features from the daughter nodes
of complex signs to their mother nodes.

\begin{principle}[Nonlocal Feature Principle] 
\label{nonloc-prinzip}
$ \\ $
The \textsc{non\-loc$|$\-inherited} value of a phrasal sign is the union of the
\textsc{non\-loc$|$\-in\-her\-ited} values of its daughters minus the \textsc{non\-loc$|$\-to-bind} value of the
daughter of the head.
\end{principle}%

A \textsc{slash} element can be bound off by the Filler"=Head Schema.

\begin{samepage}
\begin{schema}[Filler-Head Schema (for German)]
\label{hf-schema}
~\\
\type{head-filler-phrase} \impl\\\ms{ 
head-dtr$|$synsem & \onems{ local \ms{ cat & \ms{ head & \ms[verb]{vform & fin\\
                                                                                     initial & $+$\\
                                                                                    }\\
                                                  spr   & \eliste{} \\
                                                  comps & \eliste\\
                                                }\\
                                     }\\
                            nonloc \ms{ inher$|$slash   & \sliste{ \ibox{1} }\\[2mm]
                                        to-bind$|$slash & \sliste{ \ibox{1} }\\
                                      }\\
                          }\\
non-head-dtrs & \sliste{ \ms{ synsem & \onems{ local \ibox{1}\\
                                               nonloc$|$inher$|$slash \eliste \\
                                             } \\
                                 }}\\
   }
\end{schema}\is{schema!head filler}
\end{samepage}
%
This schema describes structures in which finite clauses with the verb in initial position (\textsc{initial}+)
and with an element in \textsc{inher$|$slash} (\tbox{1}) are combined with a phrase with matching \textsc{local} properties.
In example (\ref{ex-das-buch-kennt}), \emph{kennt er} `knows he' is the finite clause with the corresponding element in \textsc{slash}
and \emph{das Buch} `the book' is the filler. Figure~\vref{abb-das-buch-kennt} shows the analysis for (\ref{ex-das-buch-kennt}).
\begin{figure}
\oneline{%
\begin{forest}
sm edges
[V\feattab{\comps \eliste,\\ 
           \textsc{inh}|\textsc{slash} \eliste}
	[NP \ibox{1} {[\textit{acc}]}, fit=band
		[das Buch;the book, roof]]
	[V\feattab{\comps \eliste,\\
                   \textsc{inh}|\textsc{slash} \sliste{ \ibox{1} },\\
                   \textsc{to-bind}|\textsc{slash} \sliste{ \ibox{1} }}
		[V{[\comps \sliste{ \ibox{2} }]}
			[V{[\comps \sliste{ \ibox{3}, \ibox{4} }]}, tier=trace,edge label={node[midway,right]{V1-LR}}
				[kennt;knows]]]
		[\ibox{2} V\feattab{\comps \eliste,\\
                                    \textsc{inh}|\textsc{slash} \sliste{ \ibox{1} },\\
                                    \textsc{to-bind}|\textsc{slash} \eliste}
			[\ibox{4} \feattab{\textsc{local} \ibox{1},\\
                                                  \textsc{inh}|\textsc{slash} \sliste{ \ibox{1} }},tier=trace
					[\trace]]
			[V\feattab{\comps \sliste{ \ibox{3} },\\
                                   \textsc{inh}|\textsc{slash} \sliste{ \ibox{1} },\\
                                   \textsc{to-bind}|\textsc{slash} \eliste}
				[\ibox{3} NP{[\textit{nom}]}
				[er;he]]
				[V{[\comps \sliste{ \ibox{3}, \ibox{4} }]}
					[\trace]]]]]]
\end{forest}
}
\caption{Analysis of \emph{Das Buch kennt er.} `He knew the book.'}\label{abb-das-buch-kennt}
\end{figure}


The verb movement trace for \emph{kennt} `knows' is combined with an extraction trace. 
The extraction trace in the example is the accusative object. The accusative object 
is described in the \compsl of the verb and the information about the properties
of the required NP are at the same time present in the extraction trace under \textsc{loc}
and \textsc{inher$|$slash}. The \textsc{slash} information is passed up the tree until it reaches
the point where the projection is combined with a filler (F). The Head-Filler Schema instantiates
the \textsc{to-bind$|$slash} value of the head daughter. The Nonlocal Feature Principle then comes into
play to cause the binding off of the \slashv, which percolated from the extraction trace, that is, the \slashv is no longer passed
along up the tree. The Head"=Filler Schema then makes sure that the filler daughter (the non-head daughter
in the schema) has exactly the same \textsc{loc} value as the extraction trace. It is only the accusative nominal
phrase which is a possible candidate for a filler in our example.


It is worth noting that Schema~\ref{hf-schema} does not say anything about the valence of the filler
daughter. The form of the filler daughter is only restricted by the specification of the properties of complements of lexical 
heads. Therefore, non-maximal projections are also licensed as fillers in long-distance dependencies by our schema. The theory
presented here does not correspond to the rules of \xbart \citep*{Jackendoff77a}. This is however not necessarily a negative
point, since \xbart does not restrict the generative capacity of grammars in any way as soon as
empty elements are permitted \citep{Pullum85a,KP90a}. The fact that non"=maximal projections are possible in sentence-initial position plays a central role
for the analysis of partial verb phrase fronting presented in the following section and also for the analysis of putative
multiple fronting, which are discussed in Chapter~\ref{chapter-mult-front}. 

Note also that the \textsc{inher$|$slash} value of the non-head daughter is specified to be the
empty list. As \citet[\page 96]{Mueller99a} pointed out, this excludes the analyses in (\mex{1}):
\eal
\ex[*]{
\gll [Von wem]$_i$ [\sub{S} [ein Bild \_$_i$]$_j$ hast du \_$_j$ gemalt]?\\
     \spacebr{}of who {} \spacebr{}a picture {} have you {} painted\\
}
\ex[*]{
\gll [Von wem]$_i$ denkst du, [S [ein Bild \_$_i$]$_j$ habe ich \_$_j$ gemalt]?\\
     \spacebr{}of who think you {} \spacebr{}a picture {} have I {} painted\\
}
\zl
The impossibility to extract from constituents that are extracted themselves is sometimes referred to
as \emph{freezing}\is{freezing}.\footnote{%
  Note that extraposition is also a nonlocal dependency phenomenon \citep{Mueller2004d}. As I showed
  in \citew[\page 217]{Mueller99a}, extraposition out of fronted constituents is possible. 
\ea
\gll {}[Eine Geschichte erzählen] wird uns der Mann jetzt gleich, die viele von Ihnen erstaunen wird.\\
     \spacebr{}a story tell       will us  the man  now   immediately which many of you surprise
     will\\
\glt `The man will tell us a story now which will surprise many of you.' \citep[\page 217]{Mueller99a}
\z
This is accounted for
  by assuming that extraposition is analyzed with the feature \textsc{extra} (\citealp{Keller94,Keller95b}; \citealp[Section~13.2]{Mueller99a}). Information in
  \textsc{extra} can be passed up independently of specifications of \slasch values. See
  Section~\ref{sec-freezing} for more on freezing.
}

In the following, I will present the analysis proposed by \citew{HN94a} for predicate complexes as well as the analysis of fronting of partial
constituents based on \citew{Mueller97c,Mueller99a,Mueller2002b, Meurers99a}. These analyses have become established within the
HPSG paradigm and alternative HPSG analyses will not be discussed here.
For such a discussion, the reader is referred to \citew[Chapter~18.3]{Mueller99a} and \citew[Chapter~2.3]{Mueller2002b}.



% \citet{HN89} haben auf der Grundlage von Voranstellungs- (\mex{1}) und Oberfeldumstellungsdaten (\mex{2}) dafür
% argumentiert, daß Hilfs- und Modalverben im Deutschen mit dem Hauptverb einen Verbalkomplex bilden.
% \ea
% Geholfen haben wird er dem Mann.
% \z
% % mehr
% %Since German is assumed to be a verb second language, i.e., a language with exactly one constituent before the
% %finite verb, examples like (\mex{0}) are evidence for the existence of the constituent \emph{geholfen haben}.
% \eal
% \ex
% daß er dem Mann helfen müssen wird.
% \ex
% daß er dem Mann wird helfen müssen.
% \zl
% Die Beispiele in (\mex{0}) kann man leicht erklären, wenn man annimmt, daß
% \emph{helfen} und \emph{müssen} einen Komplex bilden, der dann unter \emph{wird} eingebettet wird.
% \emph{wird} kann entweder links oder rechts des eingebetteten Verbalkomplexes stehen.
%
% In Hinrich und Nakazawas analyze bilden auch \emph{geholfen} und das Hilfsverb
% \emph{hat} in (\mex{1}) einen Verbalkomplex:
% \ea
% \label{ex-er-geholfen-hat}
% daß er dem Mann [geholfen hat].
% \z



\subsection{On the verbal complex and partial verb phrase fronting}
\label{sec-pred-compl}

In various works (for instance \citealp{Uszkoreit87a}\ia{Uszkoreit}), it is assumed that an
auxiliary verb takes a verb phrase as its complement.
\ea
\label{ex-uf-two}
\gll dass Karl [[das Buch lesen] können] wird]\\
	 that Karl the book read can will\\
\glt `that Karl will be able to read the book'
\z
However, if one asumes such structures, it is difficult to explain the ordering of (\ref{ex-of-two}) because the auxiliary \emph{wird} is located between two elements
of the verb phrase.
\ea
\label{ex-of-two}
\gll dass Karl das Buch wird lesen können\\
     that Karl the book will read can\\
\glt `that Karl will be able to the read the book.'
\z

Furthermore, the sentences in (\mex{1}) are not ruled out by such an analysis since \emph{das Buch lesen} 
forms a phrase which can be moved left in the middle-field or appear in a so-called pied-piping construction 
with a relative clause.
\eal
\ex[*]{
\gll dass das Buch lesen Karl wird\\
	 that the book read Karl will\\
}
\ex[*]{
\gll das Buch, das lesen Karl wird\\
	 the book, that read Karl will\\
}
\zl
%
\citet*{HN94a}\ia{Hinrichs}\ia{Nakazawa} therefore suggest using a special dominance schema which ensures that
(certain) verbal complements are saturated before non-verbal ones. This means that, in the analysis of (\ref{ex-uf}) and (\ref{ex-of}),
\emph{lesen} is first combined with \emph{können} and the resulting verbal complex is then combined with \emph{wird}:
\ea
\gll daß Karl das Buch [[lesen können] wird].\\
     that Karl the book \hspaceThis{[[}read can will\\
\z
\emph{wird} can be placed to the right of the embedded verbal complex (as in \pref{ex-uf}), or indeed to the left as
in \pref{ex-of}. After the construction of the verbal complex \emph{lesen können wird}, it is then combined with the 
arguments of the involved verbs, that is with \emph{Karl} and \emph{das Buch}.\footnote{%
		This kind of structure has already been suggested 
		by \citet*{Johnson86a} in connection with an analysis of partial verb phrase fronting.
}

There are also coordination data, such as the example in (\mex{1}), which support this kind of approach.
\ea
\gll Ich liebte ihn, und ich fühlte, daß er mich auch geliebt hat oder doch, daß er mich hätte lieben wollen oder lieben müssen.\footnotemark\\
     I   loved  him and I felt that he me also loved had or PRT that he me   would.have love want or love must.\\
\footnotetext{%
        \citep*[\page 36]{Hoberg81a}
}
\iw{müssen}
\glt `I loved him and felt that he loved me too, or at least he would have wanted to love me or would have had to.'
\z

The following schema, which is derived from the one suggested by Hinrichs and Nakazawa, licenses
predicate complexes:

\begin{samepage}
\begin{schema}[Schema for predicate complexes]
\label{schema-vk}
~\\
\type{head-cluster-phrase} \impl\\
\ms{
 synsem & \onems{ loc$|$cat$|$comps \ibox{1} \\ 
                } \\
 head-dtr & \onems{ synsem$|$loc$|$cat$|$comps \ibox{1} $\oplus$ \sliste{ \ibox{2} }\\ } \\
 nonhead-dtrs & \sliste{ \onems{ synsem  \ibox{2} \\}
                      } \\
}
\end{schema}
\end{samepage}
%

I will assume the representation in (\mex{1}) for the auxiliary verb \emph{werden}:\footnote{\label{subj-fn}%
	 \citet{Pollard90a} and \citet*{Kiss92} have suggested that the subject of non-finite verbs is better represented
	 as an element in a separate list (\subj) rather than in the \comps list of the verb. For reasons of simplicity,
	 I have placed the subjects of both finite and non-finite verbs in the \comps list. The separate representation of infinite
	 subjects predicts that subjects cannot occur in projections of non-finite verbs, unless one formulates special rules which would
	 license such combinations.%
%% Für die Analyze der scheinbar mehrfachen Vorfeldbesetzung würde ebenfalls vorausgesagt, daß
%% Subjekte nicht zusammen mit anderen Konstituenten im Vorfeld stehen können, was empirisch korrekt
%% zu sein scheint. Zu einer Analyze von Prädikatskomplexen, die das Subjekt infiniter Verben separat repräsentiert,
%% siehe \citew{Mueller99a,Mueller2002b}.%
%Nimmt man die in \citew[Chapter~3]{Mueller2002b}
%vorgeschlagene Passivanalyze an, so sind die Beispiele in (\ref{bsp-mehrfach-vf-subjekt}),
%in denen ein Oberflächensubjekt in Passivkonstruktionen zusammen mit einer anderen Konstruktion
%vorangestellt wurde, auch mit einem separat repräsentierten Subjekt problemlos analysierbar.%
}
\eas
\label{le-wird}
\emph{werden} (Future auxiliary):\\
\ms[cat]{
 head   & \type{verb}\\*
 comps & \ibox{1} $\oplus$ \sliste{ \textsc{V[\textsc{lex}+, \type{bse}, comps~\ibox{1} ]}} \\
}
\zs
\emph{werden} selects a verb in its \type{bse} form, that is an infinitive without \emph{zu} `to'.

In example (\mex{1}), \emph{wird} takes over the partial specification of the arguments \emph{Karl} and
\emph{mir} `me' from \emph{helfen} `to help'.
\ea
\gll dass Karl mir helfen wird\\
	 that Karl me help will\\
\glt `that Karl will help me'
\z
This argument attraction is made possible by the structural sharing expressed by the box \iboxt{1}
in (\ref{le-wird}). The \comps list of \emph{wird helfen} `will help' therefore is identical
with  the \comps list for \emph{hilft} `helps'. The combination of \emph{helfen} `help' and
 \emph{wird} `will' is shown in Figure~\vref{abb-helfen-wird}. 

\begin{figure}
\centering
\begin{forest}
sm edges
[\ms{ head   & \ibox{1}\\
      comps & \ibox{2}\\
    }
[\iboxt{3}~\onems{ loc \onems{ head  \ms[verb]{ vform & bse \\
                                             }\\
                                comps~\ibox{2} \sliste{ NP[\type{nom}], NP[\type{dat}] }\\[2mm]
                               }\\
                }
[helfen;help]]
[\ms{ head & \ibox{1} \ms[verb]{ vform & fin \\
                               }\\
      comps & \ibox{2} $\oplus$\sliste{ \ibox{3} }\\[2mm]
                    }
[wird;will]]]
\end{forest}
\caption{\label{abb-helfen-wird}%
Analysis of \emph{helfen wird} `will help'}
\end{figure}

Auxiliaries are like raising verbs: They do not assign semantic roles to either subjects or
complements. For this reason, it is not surprising that \iboxt{1} in (\ref{le-wird}) can be 
instantiated by the empty list:
\ea
\gll Morgen wird getanzt werden.\\
	 tomorrow will danced become\\
\glt `There will be dancing tomorrow.'
\z
In (\mex{0}), subjectless construction created by passivization (\emph{getanzt werden}) has been
embedded under a future auxiliary.


% \footnote{
%       Note that this is the only purpose \textsc{lex} has in my grammar.
%       \textsc{lex} has the value $-$ if a head has been combined with a complement and +
%       otherwise. So if an unsaturated verb is combined with an adjunct its \textsc{lex} value
%       is still $+$. This is not the way \textsc{lex} is seen in the standard framework, and
%       therefore it might be reasonable to choose a different feature name. However, I decided
%         to stick with the name \textsc{lex} for historical reasons.
% } 

Spurious ambiguities are ruled out by the specification of the \textsc{lex} value of the embedded 
verbal complex in (\ref{le-wird}). Without such a specification all three structures in (\mex{1})
would be admitted:
\eal
\ex 
\gll er seiner Tochter  ein Märchen [erzählen wird]\\
     he his daughter a fairy.tale   \spacebr{}tell will\\
\glt `he will tell his daughter a fairy tale'
\ex er seiner Tochter [[ein Märchen erzählen] wird]]\label{pvp-ein-maerchen-erzaehlen}
\ex er [[seiner Tochter ein Märchen erzählen] wird]]
\zl




The \textsc{lex} feature ensures that \emph{erzählen} is combined with \emph{wird} before
\emph{erzählen} is combined with its arguments. Since the mother node in head"=complement
structures is specified as \textsc{lex}$-$, the projections of \emph{erzählen} in (\mex{0}b--c)
cannot be combined with \emph{wird}.


% \footnote{
% Of course, phrases like \emph{ein Märchen erzählen} and \emph{seiner Tochter ein Märchen erzählen} are needed in the analysis
% of German sentences like those in (i) and (ii):
% \ea
% \gll weil er seiner Tochter hätte ein Märchen erzählen sollen.\\
%      because he his daughter had  a   fairytale tell should\\
% \glt `because he should have told his daughter a fairytale.'
% \z
% \eal
% \ex 
% \gll Ein Märchen erzählen wird er seiner Tochter.\\
%      a   fairytale tell   will he his daughter\\
% \ex 
% \gll Seiner Tochter ein Märchen erzählen wird er sicher.\\
%      his daughter   a   fairytale tell will he surely\\
% \glt `He surely will tell his daughter a   fairytale.'
% \zl
% The example in (i) is an instance of the so"=called \emph{Oberfeldumstellung} (\citealp{Bech55a};
% \citealp[\page 723]{Haftka81}) and the examples in (ii) are examples of (partial) verb phrase fronting 
% \citep[\page 720--721]{Haftka81}. For an analysis of Oberfledumstellung see \citep{HN94a} and \cite[Ch.~14]{Mueller99a-unlinked}
% and for an analysis of the partial verb phrase fronting examples see \citep{Mueller97c} and \citep{Meurers99a-unlinked}.
% In the HPSG analysis of these phenomena the constraint that all arguments of embedded verbal complexes have to be raised
% to the higher predicate is relaxed for fronting or in cases like (ii) where no spurious ambiguities can
% arise.
% }


The \textsc{lex} value of the mother in predicate complex structures  -- unlike in head"=argument structures
(see Schema~\ref{schema-bin} on page~\pageref{schema-bin})-- is not quite as restricted since predicate complexes can be
embedded under different verbs and subsequently form a predicate complex with these, as shown by (\ref{ex-er-geholfen-haben-wird}).

\ea
\label{ex-er-geholfen-haben-wird}
\gll dass er dem Mann [[geholfen haben] wird]\\
     that he the man \hspaceThis{[[}helped have will\\
\glt `that he will have helped the man'
\z

If we want to rule out spurious ambiguities, we have to make sure that sentences such as (\mex{0}) can only be analyzed
as shown in (\mex{0}) and that an analysis such as (\mex{1}) is not possible.
\ea
\label{bsp-non-complex-forming}
\gll dass er dem Mann [geholfen [haben wird]]\\
     that he the man  \spacebr{}helped \spacebr{}have will\\

\z
%
In the analysis of (\mex{0}), the verbal argument of \emph{haben} `have' is raised to the argument
of the complex \emph{haben wird} `have will'.
The complex \emph{haben wird} `have will' is then combined with \emph{geholfen} `helped' via the Head"=Argument Schema. The analysis in (\mex{0})
can be ruled out if one restricts the kind of elements which can be raised in the lexicon entries for raising predicates.
Furthermore, we need an additional condition for (\ref{le-wird}), namely that \iboxt{1} only
contains fully saturated, non-predicative elements with the \textsc{lex} value $-$. In formal terms, this can be expressed as a restriction on \iboxt{1}:\footnote{%
		\citet{BvN98a} formulate an equivalent restriction. They differentiate between an
		\emph{Inner Zone} and \emph{Outer Zone} in a sentence. The \emph{Inner Zone} 
		is the predicate complex. Elements which are marked as belonging to the \emph{Inner Zone}
		by the governing head may not be raised.

In light of this restriction for raised elements, my criticism \citep[\page 351--352]{Mueller99a} of Kiss' treatment of
		obligatory coherence as a subcase of optional coherence \citep[\page 183]{Kiss95a} is rendered obsolete: One lexical
		item suffices for optionally coherent verbs in the present analysis.
}
\ea
\label{constr-non-complex-forming}
list\_of\_non\_c\_forming\_synsems(\eliste).\\
list\_of\_non\_c\_forming\_synsems(\liste{ \onems{ loc$|$cat \onems{ head$|$prd $-$\\
                                                          comps \eliste\\
                                                        }\\
                                                lex $-$\\
                                                } $|$ \ibox{2} }) :=\\
\flushright        list\_of\_non\_c\_forming\_synsems( \ibox{2} ).
\z
A list consists of elements which do not form a predicate complex when the list is empty (first
clause), or when the list starts with an element that has an empty \compsl, a \textsc{lex} value and
\textsc{prd} value of `$-$' and when the rest of the list \iboxb{2} is itself a list\_of\_non\_c\_forming\_synsems.\footnote{%
		It is not possible to avoid mentioning the \textsc{lex} value, as embedded intransitive verbs have
		an empty valence list since the subject of non-finite verbs is represented separately. The \textsc{lex} value of
		intransitive verbs is not specified in the lexicon. They can therefore occur in positions, 
		where only phrases are permitted (in so-called incoherent constructions \citep{Bech55a}) as
		well as in positions in which only lexical elements are allowed (in coherent constructions). 
		This is also the reason for the fact that the \textsc{lex} value of the mother in predicate complex
		structures is not specified as \textsc{lex}+ (as is the case in the analyses of \citealp{HN94a,dKM2001a}) since combinations
		of verbs which embed an intransitive verb may be fully saturated. Such fully
                saturated verbal complexes may form an incoherent construction with a matrix
                verb. The \textsc{lex} value of verbal complexes is therefore only constrained by the superordinate verb.%
}
The \textsc{prd} feature was introduced by \citet[\page 64--67]{ps} for means of differentiating predicative and non"=predicative
elements.

At a later point, I will explain why this restriction not only plays a role for excluding spurious
ambiguities, but also for the exclusion of certain impossible frontings.
Figure~\vref{abb-kombin1} shows in detail how the analysis of (\ref{ex-er-geholfen-haben-wird}) works.

\begin{figure}
\begin{sideways}
%\oneline{%
\begin{forest}
sm edges
[{\ms[cat]{ head   & \ibox{1} \\
           comps & \ibox{2}\\
            }}
[\iboxt{4}~\onems{ loc \ms{ head & \ibox{3} \\
                                   comps & \ibox{2} \\
                                 }
}
[\iboxt{5}~\onems{ loc \onems{ head  \ms[verb]{ vform & ppp  \\
                                                }  \\
                                comps~\ibox{2} \sliste{ NP[\type{nom}], NP[\type{dat}] } \\
                              }\\
                }
[geholfen;helped]]
[{\onems[cat]{ head~\ibox{3} \ms[verb]{ vform & bse \\
                            } \\
            comps ~ \ibox{2} $\oplus$ \sliste{ \ibox{5} } \\
          }}
[haben;have]]]
[
                 {\ms[cat]{ head & \ibox{1} \ms[verb]{ vform & fin \\
                                                    } \\
                           comps & \ibox{2} $\oplus$ \sliste{ \ibox{4} } \\
                    }}
[wird;will]]]
\end{forest}
%}
\end{sideways}
\caption{Analysis of the verbal complex in \emph{dass Karl dem Mann geholfen haben wird} `that Karl
  will have helped the man'}\label{abb-kombin1}%
\end{figure}

The perfect auxiliary \emph{haben} embeds the past participle \emph{geholfen} (a verb with \textsc{vform} \type{ppp}).
It adopts the arguments of this verb \iboxb{2} as its own. The resulting verbal complex has the same valence as 
\emph{geholfen}. This complex is embedded under \emph{wird}. \emph{wird} also attracts the arguments of the embedded
complex so that the entire complex \emph{geholfen haben wird} requires the same arguments as \emph{geholfen}.

At first glance, it may seem problematic that we need phrases such as \emph{ein Märchen erzählen} `to tell a fairy tale'
for sentences in which this group of words appears in first position. While we want to exclude this phrase as a complement
in \pref{pvp-ein-maerchen-erzaehlen}, it needs to act as a binder for the long-distance dependency of fronting in (\mex{1}):
\ea
\gll Ein Märchen erzählen wird er ihr müssen.\\
     a fairy.tale tell will he her must\\
\glt `He will have to read her a fairy tale'
\z
Sentences such as (\mex{0}) are unproblematic if \textsc{lex} is represented under \textsc{synsem}, i.e. outside of \textsc{local}, 
unlike \cite{ps} where \textsc{lex} was represented under \textsc{cat} -- that is, inside \textsc{local} \citep{Hoehle94a,Mueller97c,Mueller99a,Mueller2002b,Meurers99a}.

Due to the fact that a filler in a long-distance dependency only shares the features of the trace which are under 
\textsc{local}, a verb can require an embedded trace to have the \textsc{lex} value $+$. The \textsc{lex} value of the trace
does not have to be identical to the \textsc{lex} value of the constituent in initial position. This means that word
groupings with a \textsc{lex} value of $-$ are possible fillers as well.\footnote{%
			This means that it is not wise to formulate a structure preserving principle for grammars of HPSG, which
			states that a moved constituent has to be identical to its trace. (See \eg \citew{Emonds76a-u} for
			his formulation of this kind of principle for transformations). This kind of structure preserving
			principle does not make sense for HPSG"=grammars, as overt realizations mostly differ from their traces
			in that the overt realizations have daughters, whereas traces do not. In HPSG grammars, only information
			under \textsc{local} is normally separated. Traces and fillers can have different values with respect to everything 
			else (\textsc{phon}, \textsc{head-dtr}, \textsc{non-head-dtr},\textsc{synsem$|$nonlocal},\textsc{synsem$|$lex}, \ldots). In order to prevent 
			overgeneration, there are general conditions on extraction which make reference to local contexts.%
}		
Figure~\vref{abb-seiner-tochter-erzaehlen} shows the analysis of (\mex{1}).
\ea
\gll Seiner Tochter erzählen wird er das Märchen.\\
	 his daughter tell will he the fairy.tale\\
\glt `He will read his daughter the fairy tale'
\z	

\begin{figure}
\resizebox{!}{\textheight-3\baselineskip}{%
\begin{sideways}
\begin{forest}
sm edges
[V\feattab{\textit{fin},\\
           \comps \eliste,\\
           \textsc{slash} \eliste}
	[V{[\begin{tabular}[t]{@{}l@{}}
                   \textsc{lex} $-$,\\
                   \textsc{loc} \ibox{1}\,[\begin{tabular}[t]{@{}l@{}}
                                        \textit{bse},\\
                                        \comps \ibox{2}\,\sliste{ \ibox{3}, \ibox{4} }] ]
                                        \end{tabular}
           \end{tabular}}
		[\ibox{5} NP{[\textit{dat}]}
			[seiner Tochter;his daughter, roof]]
		[V\feattab{\textit{bse},\\
                           \comps \sliste{ \ibox{3}, \ibox{4}, \ibox{5} }}
			[erzählen;tell]]]
	[V\feattab{\textit{fin},\\
                   \comps \eliste,\\
                   \textsc{slash} \sliste{ \ibox{1}}}
		[V\feattab{\textit{fin},\\ 
                          \comps \sliste{ \ibox{7} }}
			[V{[\comps \ibox{2} $\oplus$ \ibox{6}\,]}, tier=np,edge label={node[midway,right]{V1-LR}}
				[wird;will]]]
		[\ibox{7} V\feattab{\textit{fin},\\
                                    \comps \eliste,\\
                                    \textsc{slash} \sliste{ \ibox{1} }}
			[\ibox{3} NP{[\textit{nom}]}, tier=np
				[er;he]]
			[V\feattab{\textit{fin},\\
                                   \comps \sliste{ \ibox{3} },\\
                                   \textsc{slash} \sliste{ \ibox{1} }}
				[\ibox{4} NP{[\textit{acc}]}, fit=band
					[das Märchen;the fairy tale, roof]]
				[V\feattab{\textit{fin},\\
                                           \comps \ibox{2} \sliste{ \ibox{3}, \ibox{4} },\\
                                           \textsc{slash} \sliste{ \ibox{1} }}
					[\ibox{6} V\feattab{\textsc{lex} $+$,\\
                                                            \textsc{loc} \ibox{1},\\
                                                            \textsc{slash} \sliste{ \ibox{1} }}
						[\trace]]
					[V\feattab{\textit{fin},\\
                                                   \comps \ibox{2} \sliste{ \ibox{3}, \ibox{4} } $\oplus$ \sliste{ \ibox{6} }}
						[\trace]]]]]]]
\end{forest}
\end{sideways}
}
\caption{\label{abb-seiner-tochter-erzaehlen}%
Analysis of \emph{Seiner Tochter erzählen wird er das Märchen.} `He will tell his daughter the fairy
tale.'}
\end{figure}


Ungrammatical sentences such as (\mex{1}) are ruled out by the condition in (\ref{constr-non-complex-forming}).
\ea[*]{
\gll Müssen wird er ihr ein Märchen erzählen.\\
     must will he her a fairy.tale tell\\
}
\z
\emph{Wird} requires an infintive in the \emph{bse} form and then attracts its arguments. The
attracted elements must be \textsc{lex}$-$. Since \emph {müssen} selects \emph{erzählen} and requires it to be \lex+, it cannot be attracted. This explains
why a structure such as (\mex{1}) is ruled out:
\ea[*]{
\gll Müssen$_i$ wird$_j$ er ihr ein Märchen [erzählen [\_$_i$ \_$_j$]].\\
     must       will     he her a fairy.tale \spacebr{}tell\\
}
\z
For more on this, see the discussion of (\ref{bsp-non-complex-forming}) on page~\pageref{bsp-non-complex-forming}.

The analysis in (\mex{1}) is ruled out by a general condition which bans extraction traces in head
positions.
\ea[*]{
\gll Müssen$_i$ wird$_j$ er ihr ein Märchen  [[erzählen \_$_i$] \_$_j$].\\
     must       will     he her a fairy.tale \hspaceThis{[[}tell\\
}
\z
The contrast in (\mex{1}) can be explained by the fact that in (\mex{1}a) a predicative PP has to
be attracted, which is not the case in (\mex{1}b).
\eal
\ex[\#]{
\gll Halten wird er ihn für den Präsidenten.\\
	 hold will he him for the president\\
\glt `He will think he is the president'
}
\ex[]{
\gll Interessieren wird er sich für den Präsidenten.\\
	 be.interested will he \textsc{refl} for the president\\
\glt `He will be interested in the president'
}
\zl

The analysis presented is most certainly compatible with the analysis presented in \citew[Chapter~2]{Mueller2002b} of 
constructions such as \emph{halten für} as complex predicates.

In the Principles and Parameters Framework, fronting of incomplete projections is often analyzed as remnant movement
(see G.\,\citealp{GMueller96a,GMueller98a,GMueller2014a-u}). \citet{deKuthy2002a}, \citet{dKM2001a} and \citet{Fanselow2002a} have
shown however that remnant movement analyses face empirical problems which argument composition
approaches as the one suggested here do not.\todostefan{Add discussion of \citew{GMueller2014a-u}}

\subsection{Verb movement and extraction in other Germanic languages}
\label{sec-v2-germanic}

\subsubsection{Verb movement}

The Subsections~\ref{sec-v1} and~\ref{sec-v2} provide an analysis of the verb position in German. It is in some sense
similar to the GB analysis of \citew{Reis74a}, \citew{Koster75a}, \citep[Chapter~1]{Thiersch78a} and \citep{denBesten} where it is assumed that the
finite verb moves into the C position. See also Figure~\vref{fig-verb-movement-gb}. The V-to-C
movement analysis of verb initial sentences in German and Dutch was motivated by the observation
that the finite verb and the complementizer are in complementary distribution: if the complementizer
is present the verb may not be fronted. So it was assumed that the verb moves into the
complementizer position, provided it is empty. The drawbacks of this proposal will be discussed in
Section~\ref{sec-v-to-c-movement} in more detail. This section deals with one aspect: there are
other V2 languages that have complementizers that appear together with V2 sentences
(\citealt{Vikner95a}; \citealt{Bhatt99a}; \citealt[\page 87]{Fanselow2009b}). Analyses that
assume that a finite verb moves into the position of a complementizer do not extend to such
languages. This shows that the V-to-C analysis does not capture the verb placement phenomenon in its
whole breadth. In Section~\ref I claimed that the HPSG analysis is similar to the GB analysis but
the similarity does not extend to the problematic aspects. The HPSG analysis captures the similarity
between complementizers and finite verbs in German by assigning verbs in initial position a valence
frame that is almost identical to the one of a complementizer. Both complementizer and initial
finite verb select a verb final finite clause. The only difference between complementizer and
initial finite verb is that the former requires that the finite verb is realized within the selected
clause (\dsl \type{none}) while the latter requires the verb to be missing (\dsl is an object of
type \type{local}).

Now, the analysis suggested here is different from the V-to-C analysis in that it is compatible with
languages like Yiddish in which a complementizer is combined with a V2 sentence. (\mex{1}) shows a
Yiddish example:
\ea
\settowidth\jamwidth{(Yiddish)}
\gll Ikh meyn  az   haynt hot Max geleyent dos bukh.\footnotemark\\
     I   think that today has Max read the book\\\jambox{(Yiddish)}
\footnotetext{\citew[\page 58]{Diesing90a}.}
\glt `I think that Max has read the book today.'
\z
The analysis of the CP in the example is shown in Figure~\vref{fig-analysis-cp-yiddish}.
\begin{figure}
\centering
\begin{forest}
sm edges
[CP
  [{C[ \sliste{ S } ]} [az;that]]
  [S 
    [Adv [haynt;today]]
    [S/Adv
      [{V[ \sliste{ S//V } ]} 
        [V [hot;has]]]
        [S//V/Adv
          [NP [Max;Max]]
          [VP//V/Adv 
            [V//V [\trace]]
            [VP/Adv
              [VP [V [geleyent;read]]
                  [NP [dos bukh;the book, roof]]]
              [Adv/Adv [\trace]]]
]]]]]
\end{forest}
\caption{Analysis of the Yiddish sentence \emph{az   haynt hot Max geleyent dos bukh} `that Max has read the book today'}\label{fig-analysis-cp-yiddish}
\end{figure}%
I assume that adverbs attach to VPs in SOV languages like English and Danish. The adverb is
extracted in (\mex{0}), so Figure~\ref{fig-analysis-cp-yiddish} shows a trace in the adverb position following the
VP. The information about the adverb gap is passed up in the tree until it is bound off by the
adverb in front of the finite verb. The perfect auxiliary is realized adjacent to the VP it embeds
but since the sentence in (\mex{0}) is a verb second sentence, the verb appears in initial position,
that is, to the left of the subject. The normal position of the verb is taken by the verb
trace. Information about the missing verb is projected from the verb trace to the VP and the S
level. The finite verb in initial position takes a clause from which itself is missing. The result
of the combination is S/Adv, that is, a sentence with an adverb gap. The adverb is combined with the
S/Adv and binds off the gap information. The result of the combination is an S. This S is the
argument of the complementizer and the result of the combination of complementizer and S is a CP.

The difference between the German and the Yiddish complementizer is that the German complementizer
selects a finite clause with the verb in final position, while the Yiddish complementizer selects a
V2 clause. As is clear from Figure~\ref{fig-analysis-cp-yiddish} an analysis that assumes that the
finite verb moves to C would run into trouble unless one assumes that \emph{az} `that' embeds a
CP. The analysis developed here does not have this problem and extend easily to other V2
languages. I discuss the analysis of other Germanic languages in more detail in
\citew{MuellerGermanic}.\todostefan{Vikner95a zeigt, dass CP-Analyse schwierig ist.}

\subsubsection{Extraction}

\citet[]{Fanselow2009b} compares English with German and notes that a full sentence like \emph{John
  came} in (\mex{1}a) can be combined with the adverbial \emph{yesterday}, while the same is
impossible in German, as (\mex{1}b) shows:
\eal
\ex[]{
Yesterday, John came.\jambox{(English)}
}
\ex[*]{
\gll Gestern John kam.\\
     yesterday John came\\
}
\zl
Fanselow, working in a Minimalist setting, argues that the difference in (\mex{0}) is due to the
fact that \emph{John came} is a TPs while \emph{John kam} is a CP in German. In the analysis
suggested here, \emph{kam} `came' is a sentence with fully saturated valence requirements and one
element in \slasch. This is bound off by \emph{John} to form the V2 clause \emph{John kam}. Since
the Head-Filler Schema allows for exactly one element in \slasch and binds off this element, the
\slasch list of the mother is the empty list and hence there is no way to combine \emph{gestern} as
a filler with \emph{John kam}. Since adverbials modify verbs in final position (\initial$-$) and
since \emph{John kam} is head-initial (\initial$+$) a combination via Head-Adjunct Schema is also
ruled out.

\citet{ps2} suggested analyzing \emph{John came} as a fully saturated verbal projection. It may
contain a gap and if it does it is possible to combine \emph{John came} with the filler
\emph{yesterday}. The Filler-Head Schema is rather similar for English and German. The difference
between the two languages lies in the way of building verb initial projections that can be used in Filler-Head structures:
while German and other Germanic languages involve verb movement, English is SVO and the combination of the head with its objects
and subjects is licensed directly. For German it is sufficient to require the verbal projection to
be \initial$+$. Since only verbs that underwent verb fronting are \initial$+$ this captures the data
correctly. For SVO+V2 languages this would not be sufficient since verbs are classified as \initial$+$
in VO languages anyway. Here an additional distinction \textsc{inverted}$+/-$ is needed. In V2
Filler-Head structures the head daughter must be a verbal projection with a fronted verb, that is,
\textsc{inverted}$+$.

While the modification of \emph{John kam} by \emph{gestern} is excluded in German due to the fact
that adjuncts modify \initial$-$ verbal projections, this sequence is excluded for SVO languages
since adjuncts modify VPs rather than complete sentences in these languages. 

\section{Alternatives}

\subsection{V to (I to) C movement}
\label{sec-v-to-c-movement}

The preceding subsections provide an analysis for constituent order in German. It is in some sense
similar to the GB analysis of \citep[Chapter~1]{Thiersch78a} and \citep{denBesten} where it is assumed that the
finite verb moves into the C position. See also Figure~\vref{fig-verb-movement-gb}. While this
somehow captures the idea that complementizers and finite verbs in initial position share certain
properties \citep{Hoehle97a} the V-to-C analysis has several problematic aspects, as
\citet{Fanselow2009a} points out.



\subsection{Squeezing in}

\citet[\page34]{Bierwisch63a} suggested that V2 sentences are accounted for by assuming that the verb
is ``squeezed in'' into the sentence after the first constituent. As \citet{Fanselow2009a} notes,
this nicely explains the observation by Haider, Frey and Fanselow that the
element in the \vf basically has the same information structural status as it could have in the
left-most \mf position. However, as also noted by \citet{Fanselow2009b} there is a problem with
examples like (\mex{1}). While certain elements may take the first position in the \mf, they are
excluded from appearing in the \vf.
\eal
\ex
\zl
In a fronting proposal like the one suggested here, one can exclude certain elements from entering
nonlocal dependencies. This is impossible in a squeezing-in approach since the \mf constituent would
not move. It would stay in its original position and it would just be the finite verb that would be
inserted between the first and the second element in the \mf.

This is an empirical argument against Bierwisch's analysis. There is also a technical argument. If
it could be shown that the squeezing in analysis is the only sensible analysis of the phenomenon is
the squeezing in analysis, non"=transformational frameworks would be in trouble since they usually do
not assume that there are certain structures that can be broken up by other material that is
inserted in the middle in later steps of an analysis.

Furthermore, \citet[Section~2.2]{Reis80a} noted that certain elements can be fronted without being able to occur
in the left-most position in the \mf:
\eal
\ex[]{ 
\gll Verehrt hat er ihn.\\
     adored  has he him\\
\glt `He has adored him.'
}
\ex[*]{
\gll [dass] verehrt er ihn hat\\
     \spacebr{}that adored he him has\\
}
\ex[]{
\gll Das alles erwähnte der Autor. Nicht hat er hingegen berücksichtigt, dass \ldots \\
     this all mentioned the author not   has he however  taken.into.account that\\
\glt `The author mentioned all this, but he did not take into account that \ldots'
}
\ex[*]{
\gll [dass] nicht er hingegen berücksichtigt hat, dass \ldots\\
     \spacebr{}that not   he however   taken.into.account has that\\
}
\ex[]{
\gll [dass] er hingegen nicht berücksichtigt hat, dass \ldots\\
     \spacebr{}that he however not  taken.into.account has that\\
}
\zl
In a squezing in approach the order \emph{verehrt er ihn hat} `adored he him has' would have to be derived and then the
auxiliary would move between \emph{verehrt} and \emph{er}. Similarly, the sequence \emph{nicht er
  hingegen berücksichtigt hat} `not he however taken.into.account has' would be the basis for
squezing the auxiliary between \emph{nicht} and \emph{er}. This sequence is inappropriate with the
intended reading. The only reading that is acceptable for (\mex{0}d) is the constituent negation of
\emph{er} `he'. The correct order with the verb in final position is (\mex{0}e).


\subsection{Reprojection}

As \citet[\page 93]{Fanselow2009b} and others noted, the adjunction to C analysis that
was suggested in GB is excluded in Minimalist accounts for theory internal technical reasons.

What is suggested instead is something that is called \emph{reprojection}\is{reprojection} or \emph{Remerge}\is{Remerge} \citep{Suranyi2005a,Fanselow2009b}. It is assumed that a head
is realized at a different location, leaving a trace at the original position. In the new position
the head selects the projection from which it was moved. One such analysis is provided in
Figure~\vref{fig-reprojection}.
\begin{figure}
\begin{forest}
[HP
  [H,name=H oben]
  [K
    [~~(H)~~,roof,name=H unten]]]
%\draw (H unten.south) |-  ([yshift=-5mm]H unten.south |- H oben) -| (H oben);
\draw[->] (H unten) to[out=245,in=south] (H oben);
\end{forest}
\caption{Head movement as reprojection according to \citet{Suranyi2005a}}\label{fig-reprojection}
\end{figure}%
The proposals are never worked out in detail. For instance it is unclear why the fronted head
selects for the phrase it is missing from or if it is not selection what else would license the
combination of fronted element and projection of the trace of the head. It is not explained why the
head is governing in another direction once fronted. \citet[\page 105]{Fanselow2009b} suggests that
``The verb possesses the checking feature and feature to be checked at the same time (the probe and
the goal are identical).''. But it is unclear what ``at the same time'' means. The two instances of
\emph{aime} in Figure~\vref{fig-fanselow-reprojection} cannot be identical. If they were both had to have a checking feature
and a feature to be checked. This would result in a situation in which half of the features of the
lower instance and half of the features of the upper instance could not be used in the derivation.
\begin{figure}
\begin{forest}
[{[Tense, \st{V}]}
    [subject]
    [ {[Tense, \st{V}]}
        [ {[Tense, \st{V}]} [aime\\love]]
        [ {[Tense, V]}
          [ {[Tense, V]} [\st{aime}\\ love]]
          [object]]]]
\end{forest}
\caption{Fanselow's analysis of head-movement as reprojection \citeyearpar[\page 105]{Fanselow2009b}}\label{fig-fanselow-reprojection}
\end{figure}%
So there have to be two different instances of \emph{aime}, in fact of all verbs that undergo head
movement and of course any account should capture the fact that these instances are somehow related.

\citet[\page 14]{Suranyi2005a} suggests that inflectional affixes attach to stems directly ``prior to the point where the fully inflected stem merges with another (independent) element''.
So he assumes the representation in Figure~\ref{fig-verb-suranyi} for fully inflected
verbs.\footnote{%
  Interestingly this is very similar to what is assumed in HPSG: HPSG does not assume a
  decomposition in terms of verb shells. Instead decomposition is done lexically. A verb contains
  the information contributed by V and by \emph{v} in Minimalist approaches. Inflection is also done
  presyntactically in HPSG. HPSG assumes lexical rules for inflection. They are equivalent to the
  V-T combination in Figure~\ref{fig-verb-suranyi}.%
}
\begin{figure}
\begin{forest}
[V
  [V
    [V] [\emph{v}]]
  [T]]
\end{forest}
\caption{Verbal stem plus affixes according to \citet[\page 15]{Suranyi2005a}}\label{fig-verb-suranyi}
\end{figure}%
This structure is combined with the object DP as shown in Figure~\vref{fig-verb-object-suranyi}.
\begin{figure}
\begin{forest}
[V(P)
  [V
    [V
      [V] [\emph{v}]]
    [T]]
  [Obj]]
\end{forest}
\caption{Combination of verb and object according to \citet[\page 15]{Suranyi2005a}}\label{fig-verb-object-suranyi}
\end{figure}%
Surányi assumes that the verb tree in Figure~\ref{fig-verb-object-suranyi} moves to the left. Since
the features of V are checked already it is not the functor in the verbal tree any longer. Therefore
the labels in the fronted tree are not determined by V but by \emph{v}. The result of the head
reprojection is provided in Figure~\vref{fig-verb-movement-result-suranyi}.
\begin{figure}
\begin{forest}
[\emph{v}(P)
  [\emph{v}
    [\emph{v}
      [V] [\emph{v}]]
    [T]]
  [V(P)
    [t]
    [Obj]]]
\end{forest}
\caption{Combination of verb and object according to \citet[\page 15]{Suranyi2005a}}\label{fig-verb-movement-result-suranyi}
\end{figure}%
The left subtree of \emph{v}(P) is the moved verb from Figure~\ref{fig-verb-suranyi}. \emph{v}
provides the label of this subtree. It selects a V(P) and the result of the combination is a
\emph{v}(P), which may be combined with a subject in a later step.

As with Fanselow's analysis one has to say that the details are not worked out. What does V select?
An incomplete projection of \emph{v} as seems to be needed to justify trees like the one in
Figure~\ref{fig-verb-suranyi}? If this is the case, why is \emph{v} the functor in
Figure~\ref{fig-verb-movement-result-suranyi} taking a V to its left, a T to its right and a V(P)?
%Why is V linearized to the left of \emph{v}? If \emph{v} is the functor 

%% Note also that models in which linguistic objects change their categories during a derivation are
%% highly implausible from a psycholinguistic point of view since 
%% they presuppose a certain order of
%% processing. This is incompatible with the view that we use the same linguistic knowledge for both
%% parsing and generation and that human processing is quite robust, that is, we are capable of
%% processing fragments. We are able to work with verbs in 

All of this is provided by the proposal presented in this book: there is a lexical rule/unary schema that maps
the verb in final position to a verb in initial position that acts as a head-initial head that
selects for a projection in which a respective verb is missing. I think that the Minimalist
reprojection approaches are notational variants of the HPSG analyses, which were developed several
years earlier (\citealp*[Section~4.7]{KW91a}; \citealp*{Kiss93}; \citealp*{Frank94}; \citealp*{Kiss95a}; \citealp{Feldhaus97},
\citealp{Meurers2000b}) but while the Minimalist proposals remain on the level of sketches like the
one in Figure~\ref{fig-reprojection}, the HPSG analyses are worked out in detail.


%This tradition is grounded in a proposal Holmberg (1991) made for the analysis of VP-shells:
%substitutions into empty heads should be replaced by adjunc- tions of heads to maximal projections
%followed by a re-projection of the moved head.


\section{Summary}

In this chapter, I have presented a model of German sentence structure which can explain the relatively
free ordering of constituents in the Mittelfeld, the position of the finite verb, the predicate complex,
and fronting. I have argued against alternative analyses with variable linearization/variable branching.
The analysis put forward in this chapter forms the basis for the explanation of the previously discussed cases 
of supposed multiple fronting that is discussed in the next chapter. 


% Fanselow 2009b 92
%%
%% Some languages such as Venetian (see van Craenenbroek 2006; Zwart 2006) employ ordering requirements
%% such as topic > focus, focus > che, che > topic, which cannot be translated into a cartographic
%% system because of the obvious transitivity problems that arise when such or- dering statements are
%% translated into hierarchical relations between heads. Samek-Lodovici (2006) argues convincingly that
%% the le peripheral focus of Italian is in fact situated at the right edge of the clause, being
%% followed by right-dislocated material.  is seriously under- mines the line of reasoning in Rizzi
%% (1997) in favor of a Focus-phrase in Italian.



% 


%      <!-- Local IspellDict: en_US-w_accents -->


% Scherpenisse84a-u 219  gestern am Strand ist eine AdvP
