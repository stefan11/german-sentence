%% -*- coding:utf-8 -*-
\chapter{Conclusion}


The discussion in Chapter~\ref{chap-alternatives} showed that approaches that rely on surface patterns only have
problems with accounting for the data in German. First there are elliptical sentences (Topic Drop),
in which the \vf is not filled and which are main declarative clauses nevertheless. Second there is the
big problem of apparently multiple frontings which runs afoul the V2 property of German. I suggested
using an empty head that is related to a verb in the remainder of the sentence. This captures the
fact that the elments in the \vf have to depend on the same head and preserves the generalization
that German is a V2 language. I showed in Chapter~\ref{chap-empty} that grammars with empty elements
may be more compact and capture the insights more directly than grammars without empty elements.

The theory that is represented in this book is implemented in the TRALE system
\citep*{MPR2002a-u,Penn2004a-u}. The grammar was developed in 2003 and is now part of the grammar
that is maintained in the CoreGram project \citep{MuellerCoreGramBrief,MuellerCoreGram}. It can
be downloaded at \url{http://hpsg.fu-berlin.de/Fragments/Berligram/} and is also distributed with the Grammix Virtual Machine \citep{MuellerGrammix}. For a list of positive and negative example sentences see Appendix~\ref{chap-tsdb-examples}.

%The fragment developed in \citew{Mueller2005c} is available in electronic form 
%at \url{http://hpsg.fu-berlin.de/Fragments/mehr-vf.html}. The version that is presented in this book is part of the grammar developed in the CoreGram project \citep{MuellerCoreGramBrief,MuellerCoreGram} .



%% I have shown that a theory that requires positions to be filled for certain clause types
%% is problematic. It cannot cope with elliptic patterns where no finite verb is present or
%% where an element in the \textit{Vorfeld} is omitted. The only possibility to get the data described in
%% such models is to stipulate several constructions that correspond to the observable patterns.
%% The number of constructions that had to be stipulated in a construction-based approach
%% would be higher than the number of empty heads that are needed in more traditional approaches
%% and generalizations regarding combinations of syntactic material would be missed.

%% As an alternative, I suggested that clause types are determined with reference to features that get
%% instantiated in immediate dominance schemata. Furthermore I provided an HPSG analysis for Topic Drop in German. 

%% The discussion showed that an entirely surface-based syntax cannot capture regularities that can be
%% observed in the data in an insightful way. I therefore suggest returning to more traditional
%% approaches to German clausal syntax.
