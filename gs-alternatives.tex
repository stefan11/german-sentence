%% -*- coding:utf-8 -*-
\chapter{Alternatives}
\label{chap-alternatives}%

\label{sec-local-frontings-alternatives}% Wetta, GO2009, Kathol95a

This chapter discusses alternative proposals of German sentence structure. The phenomena that have
to be explained by all proposals are the placement of the (finite) verb in initial or final
position, the possibility of scrambling of arguments, the fact that German is a V2 language that
allows to front an arbitrary constituent even if the constituent is dependent on a deeply embedded
head and the fact that sometimes there seem to be more than one constituent in the position before
the finite verb.\todostefan{add \citew{Kasper94a}}

Existing approaches can be classified along the following dimensions:
\begin{itemize}
\item phrase structure"=based vs.\ dependency-based
\item flat structures vs.\ binary branching structures
\item discontinuous vs.\ continuous constituents
\item linearization vs.\ ``head movement''
\item linearization-based approaches vs.\ ``movement''
\end{itemize} 
Movement and head movement are put in quotes since I include GPSG, HPSG, and Dependency Grammar
analyses among the movement analyses although technically, there is no movement in any of these
frameworks, but there are special relationships between fillers and gaps. 
In the following I will explore proposals from various frameworks (GPSG, HPSG, Dependency Grammar)
that differ along these dimensions.

The first proposal I want to look at is a GPSG proposal that does not assume a head-movement
mechanism.


\section{Flat structures and free linearization of the verb}
\label{sec-flat-free-linearization-of-verb-gpsg}

\citet{Uszkoreit87a} has developed a GPSG grammar for German which assumes that a verb is realized with
its arguments in a local tree. As the verb and its arguments are dominated by the same node, they can -- under
GPSG assumptions -- exhibit free ordering as long as certain theory-specific linearization
constraints are respected. For instance there is a rule for ditransitive verbs that states that a
sentence (V3) may consist of a verb (the head, abbreviated as H) and three NPs:
\ea
V3 $\to$ H[8], N2[\textsc{case} dat], N2[\textsc{case} acc], N2[\textsc{case} nom] 
\z
Each lexical item of a verb comes with a number which is associated with its valence and regulates
into which kind of phrase a verb can be inserted. The example in (\mex{0}) shows a rule for
ditransitive verbs. Since this rule does not restrict the order in which the elements at the right
hand side of the rule have to be realized, verb initial and verb final orderings are
possible. Furthermore all six permutations of the NPs can be derived.

\citet{Pollard90a} has adapted Uszkoreit's approach for his HPSG analysis of sentence structure in
German.

These kinds of analyses have the advantage of not needing empty heads to describe the position
of the verb. However, there does not seem to be any possibility of expressing the generalizations that are captured
in the analysis of apparent multiple frontings that was presented in the previous chapter in a flat
linearization model. In head-movement analyses it is possible to assume that the verb trace forms a
constituent with other nonverbal material, but this option is simply excluded in approaches like the
GPSG one for the simple reason that there is no empty verbal head.

Of course one could assume an empty element int the \vf as I did in
\citew{Mueller2002f,Mueller2002c,Mueller2005d}, but this empty element would be a special empty
element that would not be needed in any other part of the grammar and it would be stipulated with
the only purpose of getting an analysis of apparent multiple frontings.

GPSG is famous for its non-transformational treatment of non-local dependencies \citep{Gazdar81}
and the tools that were developed by Gazdar for extraction in English were used by
\citet{Uszkoreit87a} for the analysis of V2 sentences in German. However, some researchers assume
that such mechanisms are not necessary for simple sentences. They see the possible orderings as a
simple reordering of elements that depend on the same head. Such proposals are discussed in the
following section. 


\section{Flat structures and no extraction in simple sentences}

This section deals with approaches that assume that the constituent orders in (\mex{1}) are just
linearization variants of each other:
\eal
\ex
\gll Der Mann kennt die Frau.\\
     the.\nom{} man  knows the.\acc{} woman\\
\ex
\gll Die Frau kennt der Mann.\\
     the.\acc{} woman knows the.\nom{} man\\
\ex
\gll Kennt der Mann die Frau?\\
     knows the.\nom{} man the.\acc{} woman\\
\ex
\gll {}[dass] der Mann die Frau kennt\\
     \spacebr{}that the.\nom{} man the.\acc{} woman knows\\     
\zl
(\mex{0}) shows two V2 sentences and one V1 and one VL sentence. While most theories assume that
\emph{der Mann} in (\mex{0}a) and \emph{die Frau} in (\mex{0}b) are extracted, there are some
researchers that assume that these two sentences are just possible linearizations of the dependents
of \emph{kennt} `knows'. Such linearization proposals have been made in HPSG
(\citealp[Chapter~6.3]{Kathol95a}; \citealp{Wetta2011a,Wetta2014a-u})\footnote{
  \citet{Kathol2001a} revised his treatment and assumes a uniform analysis of V2 phenomena in German.}
and in Dependency Grammar. In what follows, I discuss the Dependency Grammar proposal in more detail.

One option in a Dependency Grammar analysis would be to allow for discontinuous constituents and
assume that dependents of deeply embedded heads can be serialized in the \vf even if the head is not
adjacent to the \vf. However, such radical approaches are difficult to constrain
\citep[]{MuellerGT-Eng1} and are hardly ever proposed in Dependency Grammar. Instead Dependency
Grammarians like \citet{Kunze68a-u}, \citet{Hudson97a,Hudson2000a}, \citet*{KNR98a},
and \citet{GO2009a}
%\todostefan{S: rather Hudson 2000, Khanae et al. 1998, \citew{DD2001a-u}, etc.} 
suggested analyses in which dependents of a head rise
to a dominating head for those cases in which a discontinuity would arise otherwise. The approach is
basically parallel to the treatment of non-local dependencies in GPSG, HPSG, and LFG, but the
difference is that it is only assumed for those cases in which discontinuity would arise
otherwise. However, there seems to be a reason to assume that fronting should be treated by special mechanisms even in cases
that allow for continuous serialization. In what follows I discuss three phenomena that provide
evidence for a uniform analysis of V2 sentences: scope of adjuncts, coordination of simple and
complex sentences, and apparent multiple frontings that cross clause boundaries.

\subsection{Scope of adjuncts}

The ambiguity or lack of ambiguity of the examples in (\ref{ex-oft-liest-er-das-buch-nicht}) from
page~\pageref{ex-oft-liest-er-das-buch-nicht}--repeated here as (\mex{1})--cannot be explained in a straightforward way:
\eal
\ex\label{ex-oft-liest-er-das-buch-nicht-zwei} 
\gll Oft liest er das Buch nicht.\\
     often reads he the book not\\
\glt `It is often that he does not read the book.' or `It is not the case that he reads the book
often.'
\ex
\gll dass er das Buch nicht oft liest\\
     that he the book not often reads\\
\glt `It is not the case that he reads the book often.'
\ex
\gll dass er das Buch oft nicht liest\\
     that he the book often not reads\\
\glt `It is often that he does not read the book.'
\zl
The point about the three examples is that only (\mex{0}a) is ambiguous. Even though (\mex{0}c) has
the same order as far as \emph{oft} `often' and \emph{nicht} `not' are concerned, the sentence is
not ambiguous. So it is the fronting of an adjunct that is the reason for the ambiguity. The
dependency graph for (\mex{0}a) is shown in Figure~\vref{fig-oft-liest-er-das-buch-nicht-dg}.
\begin{figure}
\centering
\begin{forest}
dg edges
[V
  [Adv, dg adjunct [oft;often] ] 
  [liest;reads] 
  [N [er;he] ]
  [N 
    [Det [das;the] ]
    [Buch;book] ]
  [Adv, dg adjunct [nicht;not]] ]
\end{forest}
\caption{\label{fig-oft-liest-er-das-buch-nicht-dg}Dependency graph for \emph{Oft liest er das Buch
    nicht.} `He does not read the book often.'}
\end{figure}%
Of course the dependencies for (\mex{0}b) and (\mex{0}c) do not differ, so the graphs would be the
same only differing in serialization. Therefore, differences in scope could not be derived from the
dependencies and complicated statements like (\mex{1}) would be necessary:
\ea
If a dependent is linearized in the \vf it can both scope over and under all other adjuncts of the
head it is a dependent of.
\z
\citet[\page 320]{Eroms85a} proposes an analysis of negation in which the negation is treated as the head,
that is, the sentence in (\mex{1}) has the structure in Figure~\vref{dg-adv-head}.\footnote{
But see \citew[Section~11.2.3]{Eroms2000a}.
}
\begin{figure}
\begin{forest}
dg edges
[Adv 
  [V [N [er;he]]
     [kommt;comes]]
  [nicht;not]] 
\end{forest}
\caption{\label{dg-adv-head}Analysis of negation according to \citet[\page 320]{Eroms85a}}
\end{figure}%
% S:
%this clearly not a syntactic structure. You must use different conventions. See for instance the semantic graph of MTT.
%Moreover it is strange to propose a direct interface between semantics and word order. In mots DGs, semantics is lnked to an unordered dependency tree and this tree to the linear order.
%The necessary seperation between the syntactic dependencies and the linear order is extensively discussed in the begining of TEsnière's book.
%
This analysis is equivalent to analyses in the Minimalist Program that assume a NegP\is{NegP} and it
has the same problem: The category of the whole object is Adv, but it should be V. This is a problem
since higher predicates may select for a V rather than an Adv. See for instance the analysis of
embedded sentences like (\ref{ex-dass-er-nicht-singen-darf}) below.

The same is true for constituent negation or other scope bearing elements. For example, the analysis of (\mex{1})
would have to be the one in Figure~\vref{dg-alledged-murderer}.
\ea
\gll der angebliche Mörder\\
     the alleged murderer\\
\z
\begin{figure}
\begin{forest}
dg edges
[Adj
    [Det, no edge, name=det, l+=3\baselineskip, [der;the] ]
  [angebliche;alleged]
  [N, name=n [Mörder;murderer]]]
\draw (n.south)--(det.north);
\end{forest}
\caption{\label{dg-alledged-murderer}Analysis that would result if one considered all scope-bearing adjuncts
  to be heads}
\end{figure}%
This structure would have the additional problem of being non-projective. Eroms does treat the determiner
differently from what is assumed here, so this type of non"=projectivity may not be a problem for
him. However, the head analysis of negation would result in non"=projectivity in so"=called coherent
constructions in German. The following sentence has two readings: in the first reading the negation
scopes over \emph{singen} `sing' and in the second one over \emph{singen darf} `sing may'.
\ea\label{ex-dass-er-nicht-singen-darf} 
\gll dass er nicht singen darf\\
     that he not sing may\\
\glt `that he is not allowed to sing' or `that he is allowed not to sing'
\z
The reading in which \emph{nicht} scopes over the whole verbal complex would result in the
non-projective structure that is given in Figure~\vref{dg-nicht-singen-darf}.
\begin{figure}
\begin{forest}
dg edges
[Subj
  [dass;that]
  [Adv
    [N, no edge, name=n, tier=n [er;he]]
    [nicht;not]
    [V,name=v 
      [V, tier=n [singen;sing]]
      [darf;may]]]]
\draw (v.south)--(n.north);
\end{forest}
\caption{\label{dg-nicht-singen-darf}Analysis that results if one assumes the negation to be a head}
\end{figure}%
Eroms also considers an analysis in which the negation is a word part (`Wortteiläquivalent'), but
this does not help here since first the negation and the verb are not adjacent in V2 contexts like
(\ref{ex-oft-liest-er-das-buch-nicht}) and even in verb final contexts like
(\ref{ex-dass-er-nicht-singen-darf}). Eroms would have to assume that the object to which the negation
attaches is the whole verbal complex \emph{singen darf}, that is, a complex object consisting of two
words.

So, this leaves us with the analysis provided in Figure~\ref{fig-oft-liest-er-das-buch-nicht-dg} and
hence with a problem since we have one structure with two possible adjunct realizations that
correspond to different readings, which is not predicted by an analysis that treats the two possible
linearizations simply as alternative orderings.
% Eroms 2000: 159 nicht is an adjunct, später dann Ergänzungskanten

Thomas Groß (p.\,c.\ 2013) suggested an analysis in which \emph{oft} does not depend on the verb but
on the negation. This corresponds to constituent negation in phrase structure approaches. The
dependency graph is shown at the left-hand side in Figure~\vref{fig-oft-liest-er-das-buch-nicht-dg-constituent-negation}.
\begin{figure}
\hfill
\begin{forest}
dg edges
[V
  [Adv, edge=dashed [oft;often] ] 
  [liest;reads] 
  [N [er;he] ]
  [N 
    [Det [das;the] ]
    [Buch;book] ]
  [Adv\sub{g}, dg adjunct [nicht;not]] ]
\end{forest}
\hfill
\begin{forest}
dg edges
[V, l sep+=5pt
  [N [er;he] ]
  [N 
    [Det [das;the] ]
    [Buch;book] ]
  [Adv, dg adjunct=4pt [nicht;not]
    [Adv [oft;often] ] ] 
  [liest;reads] ]
\end{forest}
\hfill\mbox{}
\caption{\label{fig-oft-liest-er-das-buch-nicht-dg-constituent-negation}Dependency graph for \emph{Oft liest er das Buch
    nicht.} `He does not read the book often.' according to Groß and verb-final variant}
\end{figure}%
The figure at the right-hand side shows the graph for the corresponding verb-final sentence. The
reading that corresponds to constituent negation can be illustrated with contrastive
expressions. While in (\mex{1}a) it is just the \emph{oft} `often' that is negated, it is \emph{oft
  gelesen} `often read' that is in the scope of negation in (\mex{1}b).
\eal
\ex 
\gll Er hat das Buch nicht oft gelesen, sondern selten.\\
     he has the book not often read     but seldom\\
\glt `He did not read the book often, but seldom.'
\ex
\gll Er hat das Buch nicht oft gelesen, sondern selten gekauft.\\
     he has the book not often read     but seldom bought\\
\glt `He did not read the book often but rather bought it seldom.'
\zl
These two readings correspond to the two phrase structure trees in Figure~\vref{fig-er-das-buch-nicht-oft-liest-psg}.
\begin{figure}
\hfill
\begin{forest}
sm edges
  [V
    [N [er;he] ]
    [V
      [NP 
        [Det [das;the] ]
        [N [Buch;book] ] ]
      [V 
        [Adv [nicht;not]] 
        [V [Adv [oft;often] ] 
           [V [liest;reads] ] ] ] ] ]
\end{forest}
\hfill
\begin{forest}
sm edges
  [V
    [N [er;he] ]
    [V
      [NP 
        [Det [das;the] ]
        [N [Buch;book] ] ] 
      [V 
        [Adv 
           [Adv [nicht;not] ]
           [Adv [oft;often] ] ]  
        [V [liest;reads] ] ] ] ]
\end{forest}
\hfill\mbox{}
\caption{\label{fig-er-das-buch-nicht-oft-liest-psg}Possible syntactic analyses for \emph{er das
    Buch  nicht oft liest} `He does not read the book often.'}
\end{figure}%
Note that in an HPSG analysis, the adverb \emph{oft} would be the head of the phrase \emph{nicht oft}
`not often'. This is different from the Dependency Grammar analysis suggested by Groß. Furthermore,
the Dependency Grammar analysis has two structures: a flat one with all adverbs depending on the
same verb and one in which \emph{oft} depends on the negation. The phrase structure"=based analysis
has three structures: one with the order \emph{oft} before \emph{nicht}, one with the order
\emph{nicht} before \emph{oft} and the one with direct combination of \emph{nicht} and
\emph{oft}. The point about the example in (\ref{ex-oft-liest-er-das-buch-nicht}) is that one of the
first two structures is missing in the Dependency Grammar representations. This probably does not make it
impossible to derive the semantics, but it is more difficult than it is in constituent"=based approaches.

\subsection{Coordination of simple and complex sentences}


A further argument against linearization approaches for simple sentences can be based on the following coordination
example:
\ea
\gll Wen$_i$ kennst \_$_i$ du  und glaubst du, dass \_$_i$ jeder kennt?\\
     who     knows  {} you and believe you that {} everybody  knows\\
\glt `Who do you know and do you believe that everybody knows?'
\z
The classical analysis of Across the Board Extraction\is{Across the Board Extraction} by
\citet{Gazdar81} assumes that two slashed clauses are coordinated. If one assumes that simple
clauses are analyzed via linearization of one element into the \vf while long-distance dependencies
are analyzed with a special mechanism (for instance the \slasch mechanism of GPSG/HPSG or special
dependencies \citew{Hudson2000a}), the two coordinated clauses in (\mex{0}) would differ
fundamentally in their structure and all coordination theories would fail. The conclusion is that
coordination forces us to treat the frontings in the sentences in (\mex{1}) in the same way:
\eal
\ex
\gll Wen kennst du?\\
     who     knows  you\\
\glt `Who do you know?'
\ex
\gll Wen glaubst du, dass jeder kennt?\\
     who     believe you that everybody knows\\
\glt `Who do you believe that everybody knows?'
\zl
Either both sentences are analyzed via linearization or both are analyzed using a special
mechanism for extraction. Since linearization analyses of (\mex{0}b) are either very complicated
(in HPSG) or open Pandora's box (in Dependency Grammar, see \citealp[Section~11.7.1]{MuellerGT-Eng1}), extraction-based
analyses with a special mechanism for both sentences should be preferred.


\subsection{Apparent multiple frontings}

Furthermore, note that models that directly relate dependency graphs to topological fields will not be able to
account for sentences like (\mex{1}).
\ea
\gll Dem Saft eine kräftige Farbe geben Blutorangen.\footnotemark\\
     the juice a   strong   color give blood.oranges\\
\footnotetext{
\citet{BC2010a} found this example in the \emph{Deutsches Referenzkorpus} (DeReKo), hosted at Institut
für Deutsche Sprache, Mannheim: \url{http://www.ids-mannheim.de/kl/projekte/korpora}
}
\glt `Blood oranges give the juice a strong color.'
\z
The dependency graph of this sentence is given in Figure~\vref{fig-dem-saft-eine-kraefitge-farbe-dg}.
\begin{figure}[htb]
\centerline{
\begin{forest}
dg edges
[V
  [N [Det,tier=eine [dem;the]]
   [Saft;juice] ]
  [N, l sep+=3pt 
      [Det,tier=eine [eine;a] ]
      [Adj, dg adjunct=4pt  [kräftige;strong]]
      [Farbe;color]]
  [geben;give] 
  [N [Blutorangen;blood.oranges] ] ]
\end{forest}
}
\caption{\label{fig-dem-saft-eine-kraefitge-farbe-dg}Dependency graph for \emph{Dem Saft eine kräftige Farbe geben Blutorangen.} `Blood oranges give the juice a strong color.'}
\end{figure}%

Such apparent multiple frontings are not restricted to NPs. As was shown in Section~\ref{sec-phenomenon-mult-front}, various types of dependents can be
placed in the \vf. Any theory that is based on dependencies alone and that does not allow for
empty elements is forced to give up the restriction that is commonly assumed in the analysis of V2 languages, namely that the verb is in second position. 
In comparison, analyses like \gb and those \hpsg variants that assume an empty verbal head can
assume that a projection of such a verbal head occupies the \vf. This explains why the material in
the \vf behaves like a verbal projection containing a visible verb: Such \emph{Vorfelds} are
internally structured topologically, they may have a filled \nf and even a particle that fills the
right sentence bracket (Examples with verbal particle and \mf or \nf are given in
(\ref{ex-adv-particle}) on page~\pageref{ex-adv-particle}). The equivalent of the analysis in Gross
\& Osborne's framework \citeyearpar{GO2009a} would be something like the graph that is shown 
in Figure~\vref{fig-dem-saft-eine-kraefitge-farbe-empty-dg}, but note that \citet[\page 73]{GO2009a}
explicitly reject empty elements and in any case an empty element that is stipulated just to get the
multiple fronting cases right would be entirely ad hoc.\footnote{
  I stipulated such an empty element in a linearization-based variant of HPSG allowing for
  discontinuous constituents \citep{Mueller2002c},
  but later modified this analysis so that only continuous constituents are allowed and verb
  position is treated as head-movement and multiple frontings involve the same empty verbal head as
  is used in the verb movement analysis. The revised theory is presented in this book.
}
\begin{figure}
\centerline{
\begin{forest}
dg edges
[V\sub{g}
  [V,edge=dashed [N [Det [dem;the]]
      [Saft;juice] ]
     [N [Det [eine;a] ]
        [Adj, dg adjunct=4pt [kräftige;strong]]
        [Farbe;color]]
     [ \trace ] ]
  [geben;give] 
  [N [Blutorangen;blood.oranges] ] ]
\end{forest}
}
\caption{\label{fig-dem-saft-eine-kraefitge-farbe-empty-dg}Dependency graph for \emph{Dem Saft eine
    kräftige Farbe geben Blutorangen.} `Blood oranges give the juice a strong color.' with an empty
  verbal head for the \vf}
\end{figure}%
It is important to note that the issue is not solved by simply dropping the V2 constraint and
allowing dependents of the finite verb to be realized to its left, since the fronted constituents do
not necessarily depend on the finite verb as the examples in (\ref{bsp-gezielt-mitglieder}) and
(\ref{bsp-kurz-die-bestzeit}) from page~\pageref{bsp-gezielt-mitglieder} -- repeated
here as (\mex{1}) -- show:
\eal
\ex
\gll [Gezielt] [Mitglieder] [im     Seniorenbereich]       wollen  die Kendoka allerdings nicht werben.\footnotemark\\
    \spacebr{}targeted \spacebr{}members     \spacebr{}in.the senior.citizens.sector want.to the Kendoka however    not   recruit\\
\glt `However, the Kendoka do not intend to target the senior citizens sector with their member recruitment strategy.'%
\label{bsp-gezielt-mitglieder-zwei}
\footnotetext{
        taz, 07.07.1999, p.\,18. Quoted from \citew{Mueller2002c}.
      }
\ex 
\gll {}[Kurz] [die Bestzeit] hatte der Berliner Andreas Klöden [\ldots] gehalten.\footnotemark\\
	 \spacebr{}briefly \spacebr{}the best.time had the Berliner Andreas Klöden {} held\\
\footnotetext{
        Märkische Oderzeitung, 28./29.07.2001, p.\,28.
}\label{bsp-kurz-die-bestzeit-zwei}     
\glt `Andreas Klöden from Berlin had briefly held the record time.'
\zl
And although the respective structures are marked, such multiple frontings can even cross clause boundaries:
\eal
\label{ex-der-maria-einen-ring}
\ex 
\gll Der Maria einen Ring glaube ich nicht, daß er je schenken wird.\footnotemark\\
	 the Maria a ring believes I not that he ever give will\\
\footnotetext{
\citew[\page 67]{Fanselow93a}.
}
\glt `I dont think that he would ever give Maria a ring.'
\ex 
\gll (Ich glaube) Kindern Bonbons gibt man besser nicht.\footnotemark\\
     \hspaceThis{(}I think children candy gives one better not\\
\footnotetext{
        G.\ \citew[\page 260]{GMueller98a}.
}
\glt `I think it's better not to give candy to children.'
\zl
If such dependencies are permitted it is really difficult to constrain them. As was discussed in
Section~\ref{sec-ausgeschlossen-mvf}, I started with an approach that admitted several elements in \slasch, but the
disadvantage was that it was difficult to explain why certain parts of idioms could not be
extracted. Furthermore, it would be difficult to represent the fact that the fronted elements have
to be clausemates (see also Section~\ref{sec-ausgeschlossen-mvf}).

So far I only discussed Dependency Grammar approaches but all the issues mentioned in this and the
previous subsections are also problematic for Wetta's approach
\citeyearpar{Wetta2011a,Wetta2014a-u}. Wetta, working in the framework of linearization"=based HPSG 
\citep{Reape94a,Kathol95a,Kathol2001a,Mueller99a,Mueller2002b}, assumes that sentences in which
the\todostefan{Rui has some further suggestions, check whether published}
fronted elements belong to a verb in the same clause are simply reordering variants of sentences
with the verb in initial or final position. For the analysis of apparent multiple frontings \citet{Wetta2011a}
assumes a relational constraint that takes arbitrarily many preverbal objects and forms a new complex
one:\footnote{
  I added a \textsc{dom} feature in the first domain element of the mother, which was missing in the
  original.
}
\begin{exe}
\ex Discourse prominence constructions for German according to \citet[\page 264]{Wetta2011a}:
\begin{xlist}
\ex doms$_\bigcirc$(\sliste{ [\textsc{dom} X$_1$], \ldots, [\textsc{dom} X$_n$] }) $\equiv$ X$_1$
$\bigcirc$ \ldots{} $\bigcirc$ X$_n$
\ex \type{prom-part-compact-dom-cxt} \impl\\
    \ms{
    mtr & \ms{ dom \sliste{ \ms{ prom & \normalfont +\\
                                 dom  & \normalfont doms$_\bigcirc$(L$_1$) } } $\bigcirc$
      \normalfont doms$_\bigcirc$(L$_2$) \\
                               }\\
    dtrs & \upshape L$_1$:\type{list}( [\textsc{prom}+] ) $\bigcirc$ L$_2$:\type{list}\\
    }
\zl
The list L$_1$ in (\mex{0}) is a list of elements that are marked as prominent (\textsc{prom}+). The
idea is that these elements are compacted into one element which is then the element that is placed
in the \vf. The daughters that are not marked as prominent are collected in L$_2$.%
% Könnte man machen, indem man Linearisierungsbeschränkungen sagen lässt, dass es nur ein
% PROM+-Element pro Domäne geben darf.
\footnote{
  Actually the fact that L$_2$ does contain \textsc{prom}$-$ elements is not specified in
  (\mex{0}b). It may follow from other constraints in the theory. They were not given in the paper though.
}
Since they are combined with $\bigcirc$ with the prominent element, they can appear in any order
provided no linearization constraint is violated.
Figure~\ref{fig-v3-wetta2011} shows what (\mex{0}b) does.\footnote{%
  Note that \citet[\page 259]{Wetta2011a} assumes an unusual definition of compaction and hence that the \domv of \emph{die Weltmeisterschaft} is a single object, while
  all other domain"=based approaches assume that \emph{die Weltmeisterschaft} has two \domain
  objects: \emph{die} and \emph{Weltmeisterschaft} \citep{KP95a,Kathol2001a,Babel,Mueller99a}. Without these rather unusal assumptions the call
  of doms$_\bigcirc$ would be unnecassiry and L$_1$ and L$_2$ could be used directly.
}
\begin{figure}
\oneline{%
\begin{forest}
sn edges
[{\ms{ d & \sliste{ \ms{ prom & \normalfont + \\
                          f    & \phonliste{ zum, zweiten, Mal, die Weltmeisterschaft } },
                     \ms{ f    & \phonliste{ errang } },
                     \ms{ f    & \phonliste{ Clark } } } } } 
 [{\ms{ d & \sliste{ \ms{ f  & \phonliste{ Clark } \\
                            s  & \normalfont NP[\type{nom}] } }\\ }}]
 [{\ms{ d & \sliste{ \ms{ f  & \phonliste{ die, Weltmeisterschaft } \\
                            s  & \normalfont NP[\type{nom}] } }\\ }}]
 [{\ms{ d & \sliste{ \ms{ f  & \phonliste{ zum, zweiten, Mal } \\
                            s  & \normalfont PP } }\\ }}]
 [{\ms{ d & \sliste{ \ms{ f  & \phonliste{ errang } \\
                            s  & \normalfont V[\type{fin}] } }\\ }}]
]
\end{forest}}
\caption{Vn with partial compaction according to \citet[\page 264]{Wetta2011a}}\label{fig-v3-wetta2011}
\end{figure}
The daughters of the complete construction are shown in the figure. \emph{zum zweiten Mal} and \emph{die Weltmeisterschaft} will
be marked \textsc{prom}+ and their \domvs will be combined via $\bigcirc$ and the result will be the
\domv of the first element in the \doml of the mother. This begs the question what the properties of this new object would be. This is not made explicit
in Wetta's paper but he assumes that domain objects are of type \type{sign}, so this object has to
have syntactic and semantic properties. Note that HPSG grammars are descriptions of
models. Descriptions are usually partial. Everything that is not specified can vary as long as no
appropriateness conditions on types are violated \citep{King99a-u}. For example, a theory of German could leave the
actual case value of the German noun \emph{Frau} `woman' unspecified since it is clear that the
type \type{case} has the four subtypes \type{nom}, \type{gen}, \type{dat}, \type{acc}. In a model, the
value can only vary in this limit, that is, the actual case has to be one of these four values \citep[Section~2.7]{MuellerLehrbuch1}. If
this is applied to Wetta's theory we get infinitely many models since the syntactic properties of
the first domain element are not specified. Since valence lists are part of syntactic descriptions
and since they may be arbitrarily long in principle, there are infinitely many models for Wetta's
structures. To fix this, he would have to specify the category of the element in the \vf. But what
could the category be? In other partial compaction theories the category of the created element is
the category of the head \citep{KP95a,Kathol2001a,Babel,Mueller99a}, but in Wetta's theory there is no common head for the compacted
objects. He could stipulate that the newly created object would be something like the object one gets in an analysis with an empty
verbal head. But then a relational constraint is used that creates structure out of nothing
instead of the assumption of an empty head that basically behaves like a normal visible verb. The
constraint introducing this verbal element would be something unseen elsewhere in the grammar. Nothing is gained by such an analysis.

\citet{Wetta2014a-u} drops the relational constraint that compacts several fronted elements into one
consituent and just states that arbitrarily many constituents can appear in front of the finite verb
in German. This has the advantage that no stipulation of the properties of the preverbal constituent
is needed but it leaves the fact unexplained that the material in front of the finite verb behaves
like a verbal projection with all topological fields in them (see below). 

Neither \citet{Wetta2011a} nor \citew{Wetta2014a-u} addresses data like (\ref{ex-der-maria-einen-ring}) and indeed it would be difficult to integrate such data into
his picture, since he does assume that nonlocal frontings are handled via the \slasch
mechanism. \citet{Wetta2018a} suggested a modification of the Filler-Head Schema that allows
multiple elements in \slasch and inserts the \slasch elements into the order domain of the
head.\footnote{
  I adapted the schema and put the C inside of the angle brackets.
}
\ea
\label{schema-f-g-wetta}
\type{filler-s-p-cxt} \impl\\*
\ms{
mtr & \ms{ syn \ms{ cat & y \\
                    val & \eliste\\
                    gap & \eliste\\
                  }\\
         }\\
dtrs & \sliste{ H } $\bigcirc$ L$_1$ $\bigcirc$ L$_2$ $\bigcirc$ \sliste{ C }\\
hd-dtr & \upshape H: \ms{ syn \ms{ cat & Y: \ms{ vf & fin\\
                                      }\\
                          val & L$_1$ $\oplus$ \sliste{ C: \ms[clause]{
                                                           syn & \ms{ gap & L$_2$\\
                                                                    } } }\\
                       }\\
}
} 
\z
H is the head of the matrix clause, the finite verb, C is the clausal complement of H, L$_1$ is the
list of other arguments of H and L$_2$ is the list of gaps coming up from the embedded clause. The
\dtrsl of the complete construction consists of the shuffeling of lists containing the head and the
complement clause and L$_1$ and L$_2$. Shuffling means that the elements of the involved lists can be
ordered in any order as long as the relative order of elements in the lists remains constant \citep{Reape94a}. That
is, elements from L$_1$ can be ordered before or after any elements from the other lists as long as
the order of elements in L$_1$ is not changed.\footnote{
  In fact this makes wrong predictions as far as the order of arguments in L$_1$ is
  concerned. German is a scrambling language and hence all orders of arguments of a head are allowed
  for in principle. For example both the orderings of the subject and object of \emph{gebeten} in
  (i) are possible.
\eal
\ex 
\gll [Über dieses Thema]$_i$ hat noch niemand den Mann gebeten,~~~~~~~~~~~~~ [[einen Vortrag \_$_i$ zu halten].\\
     \spacebr{}about this topic  has yet  nobody.\nom{}  the man.\acc{}  asked    \hspaceThis{[[}a
         talk {} to give\\
\glt `Nobody asked the man yet to give a talk about this topic.'
\ex 
\gll [Über dieses Thema]$_i$ hat den Mann noch niemand gebeten,~~~~~~~~~~~~~ [[einen Vortrag \_$_i$ zu halten].\\
     \spacebr{}about this topic  has the man.\acc{} yet  nobody.\nom{}    asked    \hspaceThis{[[}a
         talk {} to give\\    
\zl
  Since the schema in (\ref{schema-f-g-wetta}) can account for only one of these orders, it is not empirically
  adequate. The same problem applies to several schemata in \citew{Wetta2014a-u}, for example to the
  schema he gives on page 164: the elements of L can be scrambled but the schema does not account
  for this. The problem can be solved by assuming a special constraint that maps
  a list to all lists containing permutations of its elements. Rather than combining L$_1$ with the other lists in
  (\ref{schema-f-g-wetta}), one would then combine permutations(L$_1$) with the lists. Of course
  stipulating such a constraint that is not used anywhere else in Wetta's grammar adds to the complexity of his approach.
}

This makes it possible to account for sentences like (\ref{ex-der-maria-einen-ring}) but this analysis does not explain that the elements that
appear in front of the finite verb have to be clause mates. \citew[\page 171]{Wetta2014a-u} captures
the same-clause restriction by analyzing Vn orders as local reorderings. Since in the 2014 approach nonlocal dependencies
are not involved in Vn constructions, it follows that the elements have to be clause mates
(dependents of the verbs in the highest clause). In order to deal with examples like (\ref{ex-der-maria-einen-ring}), \citet{Wetta2018a} drops the constraint that relates Vn to local reorderings. But as soon as
nonlocal dependencies are allowed in Vn constructions, the problem of mixing material depending on
different heads creeps back in again. Rui Chaves (p.\,c.\,2018) suggested that
one could fix this problem by information-structural constraints on the fronted material. He
suggested that all fronted elements are marked as [\textsc{prominent}+] and that all \textsc{prom}+
elements have to depend on the same semantic predicate. Note that such a constraint would be violated
in sentences in which a V2 sentence is embedded into another V2 sentence:
\ea
\gll Peter denkt, Klaus kommt morgen.\\
     Peter thinks Klaus comes tomorrow\\
\glt `Peter thinks that Klaus is coming tomorrow.'
\z
\emph{Peter} is fronted and hence \textsc{prom}+ and \emph{Klaus} is also fronted and
\textsc{prom}+. Both depend on different verbs and both are fronted but within separate
clauses. The fronted elements do not share a common \vf. So, the constraint that is supposed to rule
out (\mex{1}) would also rule out (\mex{0}).
\ea[*]{
\gll Peter Klaus denkt, kommt morgen.\\
     Peter Klaus thinks comes tomorrow\\
}
\z
There may be ways to formalize the intuition behind the original proposal but this cannot be done
exclusively on the semantic/""information structural level. One would have to find ways to know which
\vf one is talking about, that is, whether the constituents are in the same \vf or in different
ones. The empty verbal head seems to be better suited to capture such constraints than any other
device one may think of.
%
% Rui 20.01.2018
% 2. A way to solve the problem you raise wrt to my
%% information-structural hypothesis and examples in (15) and (16) would
%% be to relativize it to each clause: for a given clause either all
%% fronted elements are linked to the same local verb, or all fronted
%% elements are linked to the same verb via SLASH.  Hence (15) would be
%% fine because the rule applies twice, for each clause.  Another way
%% would be to distinguish between scrambling prominence and extraction
%% prominence and require that fronted elements be of the same kind of
%% prominence (i.e. no need to refer to the verb).
%
% St: OK, may work. But how do you ensure that the SLASHes are coming from the same verb?

Note that information structural approaches that assume that several independent items are fronted
and that these fronted elements have certain special information structure functions also fail on examples
like the Claudia Perez sentence in (\ref{bsp-ihnen-für-heute-drei}) on page~\pageref{bsp-ihnen-für-heute-drei}, which was discussed in
Section~\ref{sec-assignment-to-top-foc}. Sentences like (\ref{ex-los-damit}) from page~\pageref{ex-los-damit}--repeated here as
(\mex{1})--are also problematic since it is not the case that the fronted elements have a special 
information structural status. It is not appropriate to just label \emph{los} and \emph{damit} as
\textsc{prom}+ and state that all \textsc{prom}+ elements are ordered before the finite verb.  It is the fronted phrase that has to be taken care of. Without the
assumption of a phrase there, there is no straight-forward way to do this.
\ea
\label{ex-los-damit-zwei} 
\gll \emph{Los} damit \emph{geht} es schon am 15. April.\footnotemark\\
	  off there.with goes it PRT on 15. April\\
\footnotetext{
        taz, 01.03.2002, p.\,8.
    }
\glt `The whole thing starts on the 15th April.'
\z
Note also that the order of the elements before the finite verb corresponds to the order that we
would see without fronting. \emph{los} is a right sentence bracket and \emph{damit} is
extraposed. Any theory that assumes that \emph{los} and \emph{damit} are in the same order domain as
the finite verb would run into deep trouble since it would have to assume a right sentence bracket
(\emph{los} `off') and a \nf (\emph{damit} `there.with') to the left of the left sentence bracket (\emph{geht} `goes').\footnote{
  There is a technical solution to the problem: one could set up linearization constraints that only apply to constituents with the same
  \textsc{prom} value. \textsc{prom}+ would be the \vf, \textsc{prom}$-$ the rest of the clause. In
  order to avoid the problems with information structural properties of the elements in the \vf, the
  \textsc{prom} feature could be renamed into \textsc{vorfeld}. 
\eal
\ex Mittelfeld [\textsc{vorfeld}  \ibox{1}] $<$ right bracket [\textsc{vorfeld}  \ibox{1}]
\ex right bracket [\textsc{vorfeld}  \ibox{1}] $<$ Nachfeld [\textsc{vorfeld}  \ibox{1}]
\zl         
See \citew*{KKP95} on linearization constraints with structure sharing.
}

Another problem that \slasch-based approaches have is also present again for Wetta's revised
proposal: it is difficult to restrict the fronting of idiom parts. It is possible to construct
nonlocal frontings that are parallel to the examples that were discussed in Section~\ref{sec-phraseolog}.
\ea 
\gll {}[Öl] [ins Feuer] behauptete der Vorsitzende, dass gestern das Rote-Khmer-Radio gegossen habe.\\
	 \spacebr{}oil \spacebr{}in.the fire claimed the chairman that yesterday the
         Rote-Khmer-Radio poured has \\
\glt `The chairman claimed that Rote-Khmer-Radio fanned the flames yesterday.'
\z
As was explained in Section~\ref{sec-idiom-parts-mf}, fronting of \emph{ins Feuer} `in.the fire'
without \emph{Öl} `oil' results into a literal reading. It is not obvious how this constraint can be formalized in an
extraction-based approach while the restriction that certain idiom parts want to stick together in a
verbal projection falls out immediately from an approach using an empty verbal head. I argued that the head
is the same head as is used in the verb movement analysis. So no new empty elements are needed to
get the data discussed in this section.

In what follows I have a look at other verb movement analyses that have been suggested in HPSG.


\section{Binary branching and linearization domains}

\citet{Kathol2000a} suggests an analysis with binary branching structures in which all arguments are
inserted into a linearization domain and can be serialized there in any order provided no LP rule is
violated. Normally one would have the elements of the \compsl in a fixed order, combine the head
with one element from the \compsl after the other, and let the freedom in the \doml be responsible
for the various attested orders. So both sentences in (\mex{1}) would have analyses in which the
verb \emph{erzählt} is combined with \emph{Geschichten} first and then \emph{Geschichten erzählt} is
combined with \emph{den Wählern}. Since the verb and all its arguments are in the same linearization
domain they can be ordered in any order including the two orders in (\mex{1}):
\eal
\label{ex-waehlern-geschichten-erzaehlt}
\ex weil er den Wählern Geschichten erzählt
\ex weil er Geschichten den Wählern erzählt
\zl
The problem with this approach is that examples like (\mex{1}) show that grammars have to account
for combinations of any of the objects to the exclusion of the other:
\eal
\label{ex-geschichten-erzaehlen}
\ex Geschichten erzählen sollte man den Wählern nicht.
\ex Den Wählern erzählen sollte man diese Geschichten nicht.
\zl
\citet{Kathol2000a} accounts for examples like (\mex{0}) by relaxing the order of the objects in the
valence list. He uses the shuffle operator in the valence representation:
\ea
\sliste{ NP[nom] } $\oplus$ \sliste{ NP[dat] } $\bigcirc$ \sliste{ NP[acc] }
\z
This solves the problem with examples like (\mex{-1}) but it introduces a new one: sentences like
(\ref{ex-waehlern-geschichten-erzaehlt}) now have two analyses each. One is the analysis we had before and another one is the one
in which \emph{den Wählern} is combined with \emph{erzählt} first and the result is then combined
with \emph{Geschichten}. Since both objects are inserted into the same linearization domain, both
orders can be derived. So we have too much freedom: freedom in linearization and freedom in the
order of combination. The proposal that I suggested has just the freedom in the order of combination
and hence can account for both (\ref{ex-waehlern-geschichten-erzaehlt}) and
(\ref{ex-geschichten-erzaehlen}) without spurious ambiguities.

\section{Binary branching in different directions}


\mbox{}\citet[\page 159]{Steedman2000a-u}, working in the framework of Categorial Grammar, proposed
an analysis with variable branching for Dutch\il{Dutch}, that is, there are two lexical entries for
\emph{at} `eat': an initial one with its arguments to the right, and another occupying final
position with its arguments to its left. 

\eal
\ex \emph{at} `eat' in verb-final position: (s\sub{+SUB}$\backslash$np)$\backslash$np
\ex \emph{at} `eat' in verb-initial position: (s\sub{$-$SUB}/np)/np
\zl
Steedman uses the feature \textsc{sub} to differentiate between subordinate and non-subordinate
sentences. Both lexical items are related via lexical rules.\is{lexical rule}

Such approaches were criticized by \citet{Netter92} since the branching in verb-initial sentences
is the mirror image of verb-final sentences. The scope facts in sentences like (\ref{bsp-absichtlich-nicht-anal-v1}) on
page~\pageref{bsp-absichtlich-nicht-anal-v1} cannot be explained easily, while they fall out
automatically in a verb movement approach as is shown in the examples in (\mex{1}):
\eal
\label{bsp-absichtlich-nicht-anal-v1-zwei}
\ex 
\gll Er lacht$_i$ [absichtlich [nicht \_$_i$]].\\
     he laughs \spacebr{}intentionally \spacebr{}not\\
\glt `He is intentionally not laughing.'
\ex 
\gll Er lacht$_i$  [nicht [absichtlich \_$_i$]].\\
     he laughs \spacebr{}not \spacebr{}intentionally\\
\glt `He is not laughing intentionally.'
\zl\is{scope}
Now, it has to be said that the scoping is the same in SVO languages like French\il{French} even though no
movement took place. So there may be a more general analysis of adjunct scope that covers both SVO
languages and the two verb placements that are possible in V2 languages with SOV order.

Independent of the scope question is the analysis of apparent multiple frontings: if there is no
empty head it is not obvious how the phenomenon that was discussed in
Chapter~\ref{chapter-mult-front} can be analyzed. The proposals with binary branching structures and
different branching directions are basically similar to the GPSG proposal with flat structures and
two alternative serializations of the finite verb. See Section~\ref{sec-flat-free-linearization-of-verb-gpsg}.


\section{Alternative verb-movement analyses}
\label{sec-alternative-verbbewegung}
\label{sec-kopfbewegung-vs-fernabhaengigkeit}

The rule for verb-first placement in German proposed here is similar to that of
\citet{KW91a}, \citet[Chapter~2.2.4.2]{Kiss95a} and \citet{Frank94}. However, there are differences
and these will be discussed in what follows.

\citet{Kiss95a} views \dsl not as a head feature (as I do here), but rather as a
\nonloc-feature. His head trace has the following form, which is parallel to the extraction trace:

\ea
Head trace \citep[\page 72]{Kiss95a}:\\*
\ms{
synsem & \onems{ loc \ibox{1}\\
                 nonloc$|$inher$|$dsl \menge{ \ibox{1} }\\
               }\\
}
\z
Kiss uses the same percolation mechanism for head movement as for extraction, namely percolation
via \textsc{nonloc|inher}.

The lexical rule which licenses the verb in initial position is represented as follows:\footnote{
		I have omitted a superfluous structure sharing between the \textsc{head} value in the input
		of the rule and the \textsc{head} value of the element in \subcat. The respective
		restrictions follow on from the specification of the trace.%
}

\eas
\label{kiss-dsl-lr}
\ms{ synsem$|$loc & \ibox{3} \onems{ cat$|$head \ibox{1} \ms{ vform & fin \\
                                                       } \\
                              } \\
   } $\mapsto$ \\
\ms{ synsem & \ms{ loc & \onems{ cat \ms{ head & \ibox{1} \\
                                         subcat & \liste{ \onems{ loc  \onems{ cat$|$subcat \liste{} \\
                                                                                    cont \ibox{2} \\
                                                                                  }  \\
                                                                          nl$|$inher$|$dsl  \{ \ibox{3} \} \\
                                                                        }
                                                        } \\
                                       } \\
                              cont \ibox{2} \\
                             } \\
                    nl & \ms{ to-bind$|$dsl & \textrm{\upshape \{ \ibox{3} \}} \\
                                } \\
                 } \\
}
\zs

\noindent
\citet{Frank94} has criticized Kiss' analysis as it does not predict the locality
restrictions of head movement.\footnote{
		For further discussion see \citew[\page 231--234]{Kiss95b}. Kiss proposes
		the exclusion of sentences such as (\mex{1}) by stating that complementizers
		always require that embedded sentences have an empty list as the value
		\textsc{nl$|$inher$|$dsl}. It is assumed in \citew{BFGKKN96a} that \dsl is  
		a \textsc{nonloc} feature, but that it is projected along a head path only.%
}
Without further assumptions, a sentence such as (\mex{1}a) 
would be predicted to be grammatical:

\eal
\ex[*]{
\gll Kennt$_i$ Peter glaubt,  dass Fritz Maria \_$_i$?\\
     knows     Peter believes that Fritz Maria\\
}
\ex[]{
\gll Glaubt$_i$ Peter \_$_i$,  dass Fritz Maria kennt?\\
     believes   Peter {}      that Fritz Maria knows\\
}
\zl
In the incorrect analysis of (\mex{0}a), the lexical rule (\ref{kiss-dsl-lr}) is applied
to \emph{kennt}. It is however not ensured that the element in \textsc{dsl} that is bound off by
\emph{kennt} (\,\ibox{3} in (\mex{-1})) is the head of the verbal projection that is selected by
\emph{kennen} `to know'. In the analysis of the well-formed (\mex{0}b) \emph{glaubt} is combined
with \emph{Peter \_$_i$,  dass Fritz Maria kennt} and the verb trace \_$_i$ is in the same local
domain as the verb \emph{glaubt}: \_$_i$ is the head of the clause that is combined with
\emph{glaubt}; it is the head of both \emph{Peter} and \emph{dass Fritz Maria kennt}. In the analysis of (\mex{0}a) the information about the verb trace crosses a clause
boundary. There is nothing that prevents the percolation of \dsl information from a more deeply
embedded clause. 

Frank has developed an analysis which creates a finer-grained distinction between functional
and lexical elements and suggests therefore the following solution for the locality issue:
the semantic content of the input for a lexical rule is identified with the semantic content
in the output of the lexical rule. When applied to Kiss' analysis, it would look like this:

\ea
\label{kiss-dsl-lr-2}
\begin{tabular}[t]{@{}l@{}}
\ms{ synsem$|$loc & \ibox{3} \onems{ cat$|$head  \ibox{1} \ms{ vform & fin \\
                                                       } \\
                                  cont  \ibox{2}\\
                              } \\
   } $\mapsto$ \\
\ms{ synsem & \ms{ loc & \onems{ cat \ms{ head & \ibox{1} \\
                                         subcat & \liste{ \onems{ loc  \onems{ cat$|$subcat \liste{} \\
                                                                                    cont \ibox{2} \\
                                                                                  }  \\
                                                                          nl$|$inher$|$dsl  \{ \ibox{3} \} \\
                                                                        }
                                                        } \\
                                       } \\
                              cont \ibox{2} \\
                             } \\
                    nl & \ms{ to-bind$|$dsl & \textrm{\upshape \{ \ibox{3} \}} \\
                                } \\
                 } \\
}
\end{tabular}
\z
This analysis fails, however, as soon as we have to deal with adjuncts. These are combined
with the verb trace and the \contv of the verb trace projection is therefore no longer
identical to the the \contv contained in \textsc{dsl}. See Figure~\vref{fig-verb-movement-adjunkt-sem}
for the exact representation of this.

The most simple solution to restrict verb movement to head domains is to make the corresponding
information a head feature, and for this reason only available along the head projection. \citet{Oliva92b}
and \citet{Frank94,Frank94b} have suggested representing valence information under \textsc{head} and
accessing this information inside the verb trace. As shown in Section~\ref{sec-v1}, valence information
alone is not enough to model verb movement correctly and we should therefore, as \citet{Kiss95a} suggests, 
assume that all local information , i.e. semantic content as well as syntactic information, percolates.


Furthermore, placing the head movement information inside of \textsc{local} features is
necessary for the analysis of cases of supposed multiple fronting as a verb trace is present
in initial position in such cases, \ie the verb trace is part of a filler in a long-distance dependency. A \textsc{dsl} value
which is percolated inside of \textsc{nonloc} in the constituent in initial position could not be checked at the extraction
site since only the features under \textsc{local} are shared by the extraction trace and filler.


\section{V1 via argument composition}

\citet{Jacobs91a}, working in Categorial Grammar, and \citet{Netter92}, working in HPSG, suggest an
analysis in which an empty head selects for the arguments of the verb and the verb itself. This
analysis is basically using the technique of argument composition that is also used for the analysis
of verbal complexes in German (see Section~\ref{sec-pred-compl}). The analysis of the example sentence in (\mex{1}) is shown in
Figure~\vref{fig-verb-head-argument-composition}.
\ea
\gll Isst er ihn?\\
     eats he him\\
\glt `Does he eat it/him.'
\z
\begin{figure}
\begin{forest}
sm edges
[{V[ \subcat \eliste ]}
  [ {\ibox{3} V[\subcat \sliste{ \ibox{1}, \ibox{2} } ]}
    [ \ibox{3} [isst;eats]] ]
  [ {V[ \subcat \sliste{ \ibox{3} } ]}
    [ \ibox{2} {NP[\type{nom}]} [ er;he ] ]
    [ {V[ \subcat \sliste{ \ibox{2}, \ibox{3} } ]}
      [ \ibox{1} {NP[\type{acc}]}[ ihn;him ] ]
      [ {V[ \subcat \sliste{ \ibox{1}, \ibox{2}, \ibox{3} } ]}
        [ \trace ]]]]]
\end{forest}
\caption{\label{fig-verb-head-argument-composition}Analysis of verb-initial sentences according to Jacobs and Netter}
\end{figure}
The trace is the head in the entire analysis: it is first combined with the accusative object and then with the subject. In a final step,
it is combined with the transitive verb in initial-position. A problem with this kind of analysis is that the verb \emph{isst} `eats', as well as \emph{er} `he' and
\emph{ihn} `him'/`it', are arguments of the verb trace in (\mex{1}).
\ea
\gll Morgen [isst [er [ihn \_]]]\\
	 tomorrow \spacebr{}eats \spacebr{}he \spacebr{}him\\
\glt `He will eat it/him tomorrow.'
\z
Since adjuncts can occur before, after or between arguments of the verb in German, one would expect that \emph{morgen} `tomorrow' can occur before the verb
\emph{isst}, since \emph{isst} is just a normal argument of the verbal trace in final position. As adjuncts do not change the categorial status of a projection, the phrase \emph{morgen isst er ihn} `tomorrow he eats him' should be able to
occur in the same positions as \emph{isst er ihn}. This is not the case, however. If we replace
\emph{isst er ihn} by \emph{morgen isst er ihn} in (\mex{1}a) the result is (\mex{1}b), which is ungrammatical.
\eal
\ex[]{
\gll Deshalb isst er ihn.\\
     therefore eats he him\\
\glt `Therefore he eats it/him.'
}
\ex[*]{
\gll Deshalb morgen isst er ihn.\\
	 therefore tomorrow eats he him\\
}
\zl
If one compares the analysis in Figure~\ref{fig-verb-head-argument-composition} with the one
suggested in this book it is clear how this problem can be avoided: in the analysis suggested in Section~\ref{sec-v1},
the verb in initial position is the head that selects for a projection of the empty verb in final
position. Since adjuncts attach to head-final verbs only, they cannot attach to \emph{isst er ihn}
`eats he him' in a normal head-adjunct structure. The only way for an adjunct to be combined with
\emph{isst er ihn} is as a filler in a V2 structure.


\section{V1 as underspecification}

\citet{Frank94} has suggested to eliminate the lexical rule for verb-placement and instead use
underspecification and model both order variants in the type system. The advantage of this would be
that one would not have to claim that one order is more basic and the other one is derived from
it. Frank's starting point is a version of the V1 lexical rule as it was developed by Tibor Kiss in
his dissertation \citep[\page 144]{Kiss93}. This version is given in (\mex{1}):
\eas
\onems{ loc  \ibox{3} \ms{ cat & \ms{ head \ibox{1} \ms{ vform & fin\\
                                                     }\\
                                  }\\
                         cont & \ibox{2} \\
                       }\\
   }
$\mapsto$\\
\onems{ loc  \ms{ cat & \ms{ head & \ibox{1}\\
                           subcat & \liste{ \ms{ loc$|$cat$|$head \ibox{1}\\
                                               nonloc$|$inher$|$dsl \{ \ibox{3} \}\\
                                             } }\\
                         }\\
                cont & \ibox{2}\\
              }\\
    nonloc$|$to-bind$|$dsl \{ \ibox{3} \}\\
}
\zs
Frank develops a type hierarchy in which there is a general type that both subsumes lexical verbs as
they are used in verb-final sentences and lexical verbs as they would be used in verb-initial
sentences. That is the result of the lexical rule application is encoded as a type. The lexical
entries for verbs would contain an underspecified description and since all feature structures in
actual models have to be maximal, it is ensured that actual instantiations of the lexical entries in
the lexicon are either verb-initial or verb-final verbs. (\mex{1}) shows the two AVMs that result if
information from the subtypes is filled in.
\eal
\ex \locv of the verb-final version of \emph{kennen} `to know':
\ms{ cat & \ms{ head   & \ms[verb]{ vform  & fin \\
                                    subcat & \sliste{ \ibox{1} NP[\type{nom}]\ind{2}, \ibox{3} NP[\type{acc}]\ind{4} }\\
                                  } \\
                subj   & \sliste{ \ibox{1} }\\
                comps  & \sliste{ \ibox{3} }\\
              } \\
    cont &  \ms[kennen]{
             arg1 & \ibox{2}\\
             arg2 & \ibox{4}\\
             }\\
}
\ex 
\begin{tabular}[t]{@{}l}
\locv of verb-initial version of \emph{kennen}:\\
\ms{ cat & \ms{ head & \ibox{3} \ms[verb]{ vform & fin \\
                                           subcat & \sliste{ \ibox{1} NP[\type{nom}]\ind{2}, \ibox{3} NP[\type{acc}]\ind{4} }\\
                                         } \\
                subj  & \eliste\\
                comps & \liste{ \ms{ loc$|$cat$|$head \ibox{1}\\
                                      nonloc$|$inher$|$dsl \{ \ldots{} \}\\
                                             } }}\\
         cont & \ms[kennen]{
             arg1 & \ibox{2}\\
             arg2 & \ibox{4}\\
             }\\
       }
\end{tabular}
\zl
The \dslv is not given in (\mex{0}b) since it is identical to (\mex{0}a). Frank assumes that there
is a separate head feature \subcat, which contains all arguments. Such a feature is also used in more
recent versions of HPSG, but it is called \argst and it is usually not a head feature.

Now, the problem with this approach, as with Kiss' original formalization of the lexical rule is
that the \contv that is contributed by the projection of the verb trace may differ from the
contribution of the verb (compare the analysis in Figure~\ref{fig-verb-movement-adjunkt-sem} on
page~\pageref{fig-verb-movement-adjunkt-sem}). This means that the semantics of the verb in initial
position has to be taken over from the element that is selected via \comps. This leaves us in the
rather unpleasant state that the argument-linking cannot be stated at a common supertype, since the
\contv of (\mex{0}a) is different from the \contv of (\mex{0}b).

It may be possible to rescue this analysis if one assumes a sort of default inheritance which allows overwriting information in subtypes \citep{LC99a}.
These kinds of defaults are however not compatible with all assumptions about the formal principles
of HPSG and in the case at hand, it would lead to a ``misuse'' anyway, as we want to express that there are
always two different \contvs, which means that we are not dealing with one general case which does not
hold true for certain exceptions.  
%% For the example discussed here, we would certainly need a special kind of
%% default treatment, which allows values to be overwritten in the syntax by another type. With the
%% default unification described in \citep{LC99a}, the necessary concessions for the \contvs of
%% subtypes cannot be made.

Another possibility to rescue the underspecification analysis comes in the form of the introduction of a feature
\textsc{cont2} for general types. The linking would be done with respect to the \textsc{cont2} value. The verb-final type would have a \contv identical to
\textsc{cont2}. The \contv of the verb-initial type would be independent of the \textsc{cont2}
value and hence conflicts would be avoided.

\citet{Frank94b} discusses the problem that adjuncts pose and notes that the adjunct problem is not shared by
approaches that assume an underspecified semantics and a modified Semantics Principle\is{Semantics Principle} which does
not project the meaning of the mother node from the daughter of the head, but rather combines lists
with the semantic contribution of all daughters (Frank uses Underspecified DRS \citep{FR95a-u}, but using MRS
as suggested in the previous Chapter would be an alternative option). The adjunct problem does not arise because the semantic
content of adjuncts is included in the VP, which is in turn combined with the verb in first
position. The verb in initial position contributes the meaning encoded in the lexicon. For this to
work, the actual relation that is contributed by the verb has to be represented outside of the
\contv that is shared with the projection of the verbal trace and it has to be ensured that only the
event variable is shared.

All these solutions fail however when one considers the coordination data discussed in
footnote~\vref{fn-koord-vm}, which is repeated here for convenience:
\ea
\gll Karl kennt und schätzt diesen Mann.\\
     Karl knows and values this man\\
\glt `Karl knows and values this man.'
\z
The example shows that it is not sufficient to develop accounts that explain the placement of single
verbs in initial position. To assume that (\mex{0}) is analyzed involving the coordination of V1 versions of lexical items like (\mex{-1}b) is not
appropriate, since the semantics of the initial verb has to be connected to the semantics of the
verb trace. In the original proposals the complete semantic representation of the verb was shared
with the trace, in approaches with underspecified semantics it would be an event variable that is
shared. If V1 versions of \emph{kennt} and \emph{schätzt} would be coordinated in the analysis of
(\mex{0}), the event variables of the two verbs would be wrongly identified. What is needed instead is an
event variable that refers to the conjoined event that includes both the \emph{kennen} and the
\emph{schätzen} event. This event variable is then present at the verb trace and adjuncts can refer
to it.

So either single verbs or arbitrarily complex coordinations of single verbs
can be placed in initial position. As was explained in the footnote referenced above, this can be
captured by a unary projection that relates single verbs or coordinations of single verbs to the
properties that are required for elements in initial position. If one uses a single underspecified
type for the description of lexical verbs that are supposed to be used either in initial or in final
position, this will never extend to complex coordinations as the one in (\mex{0}).


\section{A little bit of movement}
\label{crysmann}

In \citew[Chapter~11.5.2]{Mueller99a} and \citew{Mueller2004b}, I suggested that systematic bottom-up
processing is rather costly for grammars with empty verb heads due to the fact that any number of phrases
can be combined with empty verb heads. This follows from the fact that the valence and semantic content of
the verb trace remains unknown up to the point where its projection is combined with the verb in initial
position.
Berthold Crysmann took the grammar I developed as part of the \verbmobil-project \citep{MK2000a} and modified
it to improve it from a processing perspective \citep{Crysmann2003b}. Furthermore, he removed the unary-branching grammatical rules
which mimic the verb trace (see Chapter~\ref{chap-empty}) and -- rather than for an analysis with uniform right-branching -- opted for a left-branching 
analysis when the right verbal bracket is empty, and a right-branching one when the right bracket is occupied. The sentences in  
(\mex{1}) would have structures with different directions of branching:
\eal
\ex 
\gll {}[[[Gibt er] dem Mann] das Buch]?\\
      \hspaceThis{[[[}gives he the man the book\\
\glt `Is he going to give the man the book?'
\ex 
\gll {}[Hat [er [dem Mann [das Buch gegeben]]]]?\\
     \spacebr{}has \spacebr{}he \spacebr{}the man \spacebr{}the book given\\
\glt `Has he given the man the book?'
\zl
% Leider läßt sich die Linksversetzung wohl nicht auf einfache pronominale Referenz zurückführen.
% Das folgende Beispiel zeigt, daß Argumente, die zur linksversetzten Phrase gehören, im Satz mit
% dem Demonstrativpronomen realisiert werden können.
% \ea
% Artig finden, das kann Jan den Langweiler nicht.\footnote{
%         \citet[\page 33]{Neeleman94a}\iafdata{Neeleman} gibt ein analoges
%         niederländisches Beispiel.
% }
% \z
%}
In this sense, there is verb movement in Crysmann's analysis when there is a verbal complex in the sentence.
There is no verb movement, however, if the right verbal bracket is not filled. For similar suggestions, see
\citew*[\page 225]{KW91a} and \citew*{SRTD96a}. This avoids the processing problems that
an empty verb head brings with it, but then we are no longer able to explain the cases of supposed
multiple fronting by means of an empty verb head.

Instead of modifying the analysis of verb position, one should, for practical applications, turn to statistical components 
which predict the position of verb traces \citep{BFGKKN96a,FBCKS2003a}. If one processes the traces according
to their probability, one gets first readings quickly and dispreferred readings later. The structures which use traces
classified as `improbable' by the statistical component will be computed last.\footnote{
	Berthold Crysmann has pointed out that the changes to the grammar he proposed
	have reduced the running time by a factor of 14, whereas the techniques described
	in \citew{BFGKKN96a} only resulted in a reduction of 46\,\% (less than a factor of 2)
	for the grammar they were using.

        However, the grammar that was used for the experiments done by \citew{BFGKKN96a} had a
        smaller coverage than grammars like the one that was developed in Saarbrücken by me, Walter
        Kasper and Berthold Crysmann and BerliGram, which is used in the CoreGram project. Therefore
        the use of a statistical component that was described by Batliner and colleagues probably
        would result in an even higher factor in the reduction of the run time. However, this would
        have to be studied experimentally. It seems unlikely though that a factor of 14 will be reached.
	
	It could then be the case that one still gets an overall slower system despite the application of the processing
	methods above than if one had modified the grammar.
}

Crysmann argues that his analysis ``leads to a more general grammar, if the
formalism does not support empty categories.'' He reduced the number of grammar rules which were needed for the
implementation of the LKB"=system \citep{Copestake2002a-unlinked} for verb movement (see Section~\ref{sec-verb-movement-LKB}) from 24 to 6.
The 24 rules were needed in the grammar for the exact reasons that empty elements were not allowed. The decision
to outlaw empty elements is, in that sense, a conscious decision on the part of the developer of the system and
is not necessarily driven by linguistic or computational necessities. As the implementation of the
analysis that is described here
in the TRALE system \citep*{MPR2002a-u,Penn2004a-u} demonstrates,\footnote{
        The grammar is freely available at \url{http://hpsg.fu-berlin.de/Fragments/b-ger-gram.html}.
        } it is most certainly possible to use an empty
head in the implementation of the verb-movement analysis that was developed in \verbmobil.

In order to describe verb movement, one empty element is required and one lexical rule. This kind of
grammar is therefore more compact than that of Crysmann, who needs six rules to achieve
this. Processing is unproblematic as empty elements are automatically removed from the grammar
before parsing while still remaining transparent for the developer of the grammar. The result of the
compilation of the grammar is identical to what developers who use grammar development systems such
as the LKB system had to produce tediously by hand. For more on empty elements, see
Chapter~\ref{chap-empty}.



\section{Special valence features for arguments forming a complex}

A special valence feature (\textsc{gov}) has been suggested by \citet*{Chung93a} for Korean\il{Korean} and
\citet*{Rentier94} for Dutch\il{Dutch} which is used for the selection of elements which form a 
verbal complex with their head. This approach was adopted by \citet{Kathol98b,Kathol2000a} and \citet{Mueller97c,Mueller99a}
for German. In \citew{Mueller2002b}, I expanded my earlier analysis to include resultative
constructions and subject and object predicatives of the \emph{jemanden für etwas/jemanden halten} `consider somebody for somebody/something' kind. 
Embedded predicates are also seen as being selected by a special valence feature (\vcomp) in this analysis.


The theory suggested here does not require this kind of additional feature. This has the advantage
that optional coherence can be analyzed as a special case of coherence as suggested by \citet{Kiss95a}.
We only need one lexical entry for verbs such as \emph{versprechen} `to promise' rather than two, which would be
needed for both coherent and incoherent constructions. 

By reducing the number of valence features, it is possible to considerably simplify the
analysis of multiple fronting. In \citew{Mueller2005d}, I suggest a lexical rule for
sentences such as (\ref{bsp-smvfb}) on page~\pageref{bsp-smvfb}, which is parallel to the verb-movement
rule in (\ref{lr-verb-movement2}) on page~\pageref{lr-verb-movement2}. 
Previous multiple fronting analyses of mine \citep{Mueller2002f,Mueller2002c}
have made use of the special valence feature \vcomp and this was the reason why the parallels of both of these verb-movement rules
remained hidden. With the feature geometry used here, cases of putative multiple fronting can be understood as
an optional variant of simple verb movement, which forms a complex. The details of the analysis are
discussed in Chapter~\ref{chapter-mult-front}.







%      <!-- Local IspellDict: en_US-w_accents -->
