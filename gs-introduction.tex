%% -*- coding:utf-8 -*-
%%%%%%%%%%%%%%%%%%%%%%%%%%%%%%%%%%%%%%%%%%%%%%%%%%%%%%%%%
%%   $RCSfile: hpsg-satz-struktur-lb.tex,v $
%%  $Revision: 1.7 $
%%      $Date: 2009/03/27 16:28:44 $
%%     Author: Stefan Mueller (DFKI)
%%    Purpose: 
%%   Language: LaTeX
%%%%%%%%%%%%%%%%%%%%%%%%%%%%%%%%%%%%%%%%%%%%%%%%%%%%%%%%%



\chapter{Introduction}
\label{chapter-introduction}

The German sentence can be adequately described using the topological model of (\citealp*{Reis80a}; \citealp*{Hoehle86}; \citealp{Askedal86}).
In the sentence (\mex{1}), the verbs \emph{hat} `has' and \emph{gegeben} `given' form a `frame'
around the rest the of the sentence. The finite verb \emph{hat} `has' occupies the left sentence bracket and the
infinitive \emph{gegeben} `given' the right one. 
\ea
\gll Der Mann hat der Frau das Buch gegeben, das wir alle kennen.\\
     the man  has the woman the book given   that we all  know\\
\glt `The man gave the woman the book that we all know.'
\z
Situated between the sentence brackets is the so-called `middle-field' (\emph{Mittelfeld}). The prefield (\emph{Vorfeld}) precedes
the left bracket, and the postfield (\emph{Nachfeld}) follows the right bracket. 

In subordinate clauses introduced by a conjunction, the conjunction takes the left sentence bracket
and the finite verb is located with the rest of the non-finite verb forms in the right bracket:
\ea
\label{ex-gegeben-hat}
\gll dass der Mann der Frau das Buch gegegben hat\\
     that the man  the woman the book given has\\
\glt `that the man gave the woman the book'
\z
In this book I develop an analysis which -- like many analyses of German clause structure before -- establishes a link between verb-first and verb-final sentences.


Constituents in the middle-field exhibit a relatively free ordering:
\eal
\label{bsp-GPSG-anordnung}
\ex 
\gll {}[weil] der Mann der Frau das Buch gibt\\
     {}\spacebr{}because the.\nom{} man the.\dat{} woman the.\acc{} book gives\\
\glt `because the man gives the book to the woman'
\ex 
\gll {}[weil] der Mann das Buch der Frau gibt\\
     {}\spacebr{}because the man.\nom{} the book.\acc{} the woman.\dat{} gives\\
\ex 
\gll {}[weil]          das        Buch der        Mann der        Frau gibt\\
     \spacebr{}because the.\acc{} book the.\nom{} man  the.\dat{} woman gives\\
\ex 
\gll {}[weil] das Buch der Frau der Mann gibt\\
{}\spacebr{}because the.\acc{} book the.\dat{} woman the.\nom{} man gives\\
\ex 
\gll {}[weil] der Frau der Mann das Buch gibt\\
{}\spacebr{}because the.\dat{} woman the.\nom{} man the.\acc{} book gives\\
\ex 
\gll {}[weil] der Frau das Buch der Mann gibt\\
{}\spacebr{}because the.\dat{} woman the.\acc{} book the.\nom{} man gives\\
\zl
This is accounted for by assuming that a head may combine with its arguments in any order. Of course
there are restrictions, but these restrictions are represented independently of the general
combinatory mechanism.

The prefield can be occupied by one constituent (an adjunct, subject or complement), which is why German is
viewed as a verb-second language (\citealp[Chapter~2.4]{Erdmann1886a};
\citealp[\page 69, \page 77]{Paul1919a}). Examples such as (\mex{1}) show that occupation of the prefield cannot simply be explained as an ordering variety of an element dependent on 
the finite verb (in analogy to reorderings in the middle field):

\ea
\gll{}[Um zwei Millionen Mark]$_i$ soll er versucht haben, [eine Versicherung \_$_i$ zu betrügen].\footnotemark\\
      \spacebr{}around two million Deutschmarks should he tried have \spacebr{}an insurance.company
              {} to defraud\\
\footnotetext{
         taz, 04.05.2001, p.\,20.
}
\glt `He supposedly tried to defraud an insurance company of two million Deutschmarks.'

\z
%
The head that governs the PP (\emph{betrügen} `defraud') is located inside of the infinitive clause. The PP as such is not directly dependent on the finite
verb and can therefore not have reached the prefield by means of a simple local reordering operation. This
shows that the dependency between \emph{betrügen} and \emph{um zwei Millionen} `around two million
Deutschmarks' is a long distance dependency: an element belonging to a deeply embedded head has been fronted over several phrasal borders.

\citet{Thiersch78a}, \citet[\page 55]{denBesten83a}, \citet{Uszkoreit87a}\ia{Uszkoreit} and others have suggested 
a connection between verb-second and verb-first sentences, and that verb-second sentences should be
analyzed as verb-first sentences with an extracted constituent placed in the prefield.

\eal
\ex 
\gll Kennt er das Buch?\\
	knows he the book\\
\glt `Does he know the book?'
\ex 
\gll Das Buch kennt er.\\
	the book knows he\\
\glt `He knows the book.'
\zl 
This is also the approach that I assume in this book.


The elements in the right bracket form a complex. I assume that such complexes are formed first and
are then combined with the arguments that depend on the elements in the complex. For instance, \emph{gegeben} `given' and
\emph{hat} `has' in (\ref{ex-gegeben-hat}) form one unit, which is then combined with \emph{das
  Buch} `the book', \emph{der Frau} `the woman', and  \emph{der Mann} `the man' in later steps. 

The left peripheral elements of this verbal complex can (in some cases together with the adjacent material
from the middle field) be moved into the prefield:
\eal
\ex
\gll Gegeben hat er der Frau das Buch.\\
     given has he the woman the book\\
\glt `He gave the woman the book.'
\ex
\gll Das Buch gegeben hat er der Frau.\\
     the book given   has he the woman\\
\ex
\gll Der Frau gegeben hat er das Buch.\\
     the woman given  has he the book\\
\ex
\gll Der Frau  das Buch gegeben hat er.\\
     the woman the book given   has he\\
\zl
Since the fronted verbal projections in (\mex{0}a--c) are partial, such frontings are called \emph{partial verb
phrase frontings}.

While there is a broad consensus among reasearchers from various frameworks that German is a V2
language, some challenging examples can be found that seem to contradict the V2 characteristic
of German (see \citew{Mueller2003b,Bildhauer2011a} and the literature discussed there). Some examples are given in
(\mex{1}) and further examples are discussed in Section~\ref{sec-phenomenon-mult-front}.

\eal
\label{bsp-smvfb}
\ex 
\gll {}[Dauerhaft] [mehr Arbeitsplätze] gebe es erst, wenn sich eine Wachstumsrate von  mindestens 2,5 Prozent über einen Zeitraum von drei oder vier Jahren halten lasse.\footnotemark\\ 
       \spacebr{}constantly \spacebr{}more jobs give it first when REFL a growth.rate of  at.least 2.5 percent over a time.period of three or four years hold lets\\
\footnotetext{
        taz, 19.04.2000, p.\,5. %taz Nr. 6123 vom 19.4.2000 Seite 5
} 
\glt `In the long run, there will only be more jobs available, when a growth rate of at least 2.5 percent 
can be maintained over a period of three of four years.'	      
\ex 
\gll [Unverhohlen verärgert] [auf Kronewetters Vorwurf] reagierte Silke Fischer.\footnotemark\\
     \spacebr{}Blatantly annoyed \spacebr{}by Kronewetter's reproach reacted Silke Fischer\\
\footnotetext{
    taz berlin, 23.04.2004, p.\,21.
}
\glt `Blatantly annoyed, Silke Fischer reacted to Kronewetter's reproach.'
\ex 
\gll {}[Hart] [ins Gericht] ging Klug mit dem Studienkontenmodell der Landesregierung.\footnotemark\\
       \spacebr{}hard \spacebr{}in.the court went Klug with the tuition.account.model. of.the
       state.government\\
\footnotetext{
  taz nord, 19.02.2004, p.\,24.
  }
\glt `Klug roasted the state government's tuition account model.'
\zl
In Chapter~\ref{chapter-mult-front} I show how these examples can be analyzed using a special variant
of the lexical rule that is suggested for the analysis of verb-initial sentences in combination with verbal
complex formation and partial verb phrase fronting.




