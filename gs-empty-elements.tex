%% -*- coding:utf-8 -*-
%%%%%%%%%%%%%%%%%%%%%%%%%%%%%%%%%%%%%%%%%%%%%%%%%%%%%%%%%
%%   $RCSfile: grammatiktheorie-include.tex,v $
%%  $Revision: 1.13 $
%%      $Date: 2010/11/16 08:40:32 $
%%     Author: Stefan Mueller (CL Uni-Bremen)
%%    Purpose: 
%%   Language: LaTeX
%%%%%%%%%%%%%%%%%%%%%%%%%%%%%%%%%%%%%%%%%%%%%%%%%%%%%%%%%


\chapter{Empty elements}
\label{chap-empty}


%\citew[Chapter~6.2.5.1]{Mueller2002b}; \citew{Mueller2004e}



In some frameworks there is a dogma that empty elements should not be used in analyses. The argument
is that they are invisible and hence cannot be acquired from the input. I think this argumentation
is not correct in general since some empty elements correspond to visible entities and hence the
knowledge that is required to deal with such ommisions can be acquired. I distinguish between good
and bad empty elements: the good ones are the ones that correspond to visible entities in the
langauge under considerations and the bad ones are those that are semantically empty or that are
motivated by overt items in other languages. Empty expletives are suggested in GB and Minimalism and
empty functional heads like AgrO and other categories have been suggested for German on the basis of
evidence from Basque. I think for the latter examples the criticism by proponents of Construction
Grammar is fully legitimate, but I want to argue that this criticism went too far in throwing out the
good empty elements with the bath water.


In this chapter, I want to discuss the relation of grammars with empty elements to those
without empty elements. This will enable us to compare the solution with an empty verb head 
to solutions without empty elements.


\section{Empty elements in the German NP}

I want to start with a simple example and motivate the use of empty elements in the German noun
phrase. Consider the following nominal structures:

\eal
\ex 
\gll die Frauen\\
     the women\\
\ex 
\gll die klugen Frauen\\
     the smart  women\\
\ex 
\gll die klugen Frauen aus Greifswald\\
     the smart  women  from Greifswald\\
\ex 
\gll Frauen\\
     women\\
\ex\label{bsp-kluge-frauen} 
\gll kluge Frauen\\
     smart women\\
\ex 
\gll die klugen\\
     the smart\\
\glt `the smart ones'
\ex 
\gll die klugen aus Greifswald\\
     the smart  from Greifswald\\
\glt `the smart ones from Greifswald'
\ex 
\gll kluge aus Greifswald\\
     smart from Greifswald\\
\glt `smart ones from Greifswald'
\ex 
\gll kluge\\
     smart\\
\glt `smart ones'
\zl
As in English, the determiner may be omitted in the plural and with mass nouns. In addition, the
noun may be omitted in elliptical structures:
\eal
\ex 
\gll Ich kenne die klugen.\\
     I know the smart\\
\glt `I know the smart ones.'
\ex 
\gll Ich kenne kluge aus Greifswald.\\
     I know smart from Greifswald\\
\glt `I know smart ones from Greifswald.'
\ex 
\gll Ich kenne kluge.\\
     I know smart\\
\glt `I know smart ones.'
\zl
I think that the description I just gave, namely that the noun or the determiner or both may be
omitted is the most straightforward description of the phenomenon. This is what children have to
acquire. Of course the omitted elements do have a meaning. The noun may only be omitted, if the
whole nominal expression refers to somthing/somebody. If one uses the phrase \emph{kluge aus
  Greifswald} `smart from Greifswald', women, man or children or something else that can be smart
have to be mentioned in the preceeding discourse. Formally this can be represented in the following
small grammar:\footnote{
The grammar predicts that all bare determiners can function as full NPs, which is
not empirically correct:
\begin{exe}\exi{(i)}\begin{xlist}[iv.]
\ex[]{
\gll Ich helfe den Männern.\\
     I help the men\\
}
\ex[*]{
\gll Ich helfe den.\\
     I   help  the\\
}
\ex[]{
\gll Ich helfe denen.\\
     I help those\\
}
\zllast
}
\ea
\begin{tabular}[t]{@{}l@{ $\to$ }l}
NP    & Det \nbar\\
\nbar & Adj \nbar\\
\nbar & \nbar PP\\
\nbar & \trace\\
Det   & \trace\\
Det & die\\
Adj & klugen\\
\nbar & Frauen\\
\end{tabular}
\z
\nbar is an abbreviation for nouns that require a determiner and the rules \nbar $\to$
\trace{} and Det $\to$ \trace{} state that \nbar and Det may be omitted. The grammar is not
complete. Lexical entries and rules for the PP are missing. Furthermore, the grammar is not precise
enough since all inflectional information is left out. But it is sufficient for the discussion of
the advantages of empty elements. To facilitate discussion, let me introduce terminology: a rule
describes which symbols can be rewritten by other symbols. Some symbols are considered as special in
the sense that they are never rewritten by other symbols. They end the replacement process and are
therefore called \emph{terminal symbols}\is{symbol!terminal}. These are the words in the grammar in
(\mex{0}) and the empty element \trace. All other symbols are so-called \emph{non-terminal
  symbols}. The rules in (\mex{0}) that have a terminal symbol on the right"=hand side are basically
lexical entries: they specify the category symbol for a specific word. The other rules are grammar
rules that combine two non"=terminals. There are two grammar rules in (\mex{0}) and four lexical items.

The grammar licenses for instance the structures in Figure~\ref{abb-np}.
\begin{figure}
\oneline{\begin{forest}
sm edges
[NP
       [Det [die;the] ]
       [\nbar [Frauen;women] ] ]
\end{forest}
\hfill
\begin{forest}
sm edges
[NP
       [Det [die;the] ]
       [\nbar 
         [Adj [klugen;smart] ]
         [\nbar [Frauen;women] ] ] ]
\end{forest}
\hfill
\begin{forest}
sm edges
[NP
       [Det [die;the] ]
       [\nbar 
         [Adj [klugen;smart] ]
         [\nbar [\trace] ] ] ]
\end{forest}
\hfill
\begin{forest}
sm edges
[NP
       [Det [\trace] ]
       [\nbar 
         [Adj [kluge;smart] ]
         [\nbar [Frauen;women] ] ] ]
\end{forest}
\hfill
\begin{forest}
sm edges
[NP
       [Det [\trace] ]
       [\nbar 
         [Adj [kluge;smart] ]
         [\nbar [\trace] ] ] ]
\end{forest}}

\caption{\label{abb-np}Various nominal structures}
\end{figure}
\citet*[\page 153, Lemma~4.1]{BHPS61a} developed a procedure to transform grammars that use empty
elements into grammars without empty elements. To that end one has to compute all symbols from which
the empty word can be derived. These symbols can then be inserted into the right"=hand
sides of rules resulting in new rules. The grammar that one obtains by such replacements is a
grammar without empty elements. For the grammar in (\mex{0}) one gets:\footnote{
  In principle the grammar in (\mex{0}) allows for completely empty NPs. This has to be blocked by
  features in the grammar \citep[\page 81, Exercise~3]{MuellerGT-Eng1}.
}
\ea
\begin{tabular}[t]{@{}l@{ $\to$ }l}
NP    & Det \nbar\\
NP    & Det\\
NP    & \nbar\\
\nbar & Adj \nbar\\
\nbar & Adj\\
\nbar & \nbar PP\\
\nbar & PP\\
Det & die\\
Adj & klugen\\
\nbar & Frauen\\
\end{tabular}
\z
The rule \nbar $\to$ \trace from (\mex{-1}) can be used to derive the empty word. The \nbar was
inserted into the rule NP $\to$ Det \nbar in (\mex{-1}) and the rule NP $\to$ Det in (\mex{0}) resulted. 

The grammar in (\mex{0}) licenses among others the structures in Figure~\ref{abb-np2}.
\begin{figure}
%\oneline
{\begin{forest}
sm edges
[NP
       [Det [die;the] ]
       [\nbar [Frauen;women] ] ]
\end{forest}
\hfill
\begin{forest}
sm edges
[NP
       [Det [die;the] ]
       [\nbar 
         [Adj [klugen;smart] ]
         [\nbar [Frauen;women] ] ] ]
\end{forest}
\hfill
\begin{forest}
sm edges
[NP
       [Det [die;the] ]
       [\nbar 
         [Adj [klugen;women] ] ] ]
\end{forest}
\hfill
\begin{forest}
sm edges
[NP
       [\nbar 
         [Adj [kluge;smart] ]
         [\nbar [Frauen;women] ] ] ]
\end{forest}
\hfill
\begin{forest}
sm edges
[NP
       [\nbar 
         [Adj [kluge;smart] ] ] ]
\end{forest}}

\caption{\label{abb-np2}Verschiedene Nominalstrukturen ohne leere Elemente}
\end{figure}
The branches with empty elements were simply omitted. Comparing the two grammars it can be noted
that the grammar without empty elements contains more rules. It contains seven rules without terminals, whereas the one
with empty elements contains only three such rules.
Even if one includes the rules for lexical items in the counting and hence takes into account the lexical items for different empty elements, one gets a proportion of
nine to seven. In the end the grammar with empty elements is a more compact description of the
phenomenon and it covers directly what has to be acquired: the noun and the determiner can be
left unpronounced under certain circumstances.

Several attempts were made to account for noun phrases without empty elements. For inctance
\citet[\page 78]{Michaelis2006a} suggested a special lexical rule for nouns in the plural. The
plural items that are licensed by this lexical rule differ from other lexical items for nouns in not selecting for a
determiner. Thus one would have two lexical items for \emph{Frauen} `women': one of category \nbar
and one of category NP. The problem is that \emph{Frauen} `women' can be modified by \emph{kluge}
`smart' (\ref{bsp-kluge-frauen}) even when no determiner is present. If one admits adjectives to
modify NPs, phrases like (\mex{1}) cannot be excluded any longer:\footnote{
  See also \citew*[\page 265, Problem~2]{SWB2003a}.
}
\ea[*]{
\gll kluge die Frauen\\
     smart the women\\
}
\z





\section{Empty elements for verb movement}
\label{sec-verb-movement-LKB}


To demonstrate more clearly what the consequences of trace elemination are, I want to discuss
a transformation of the grammar that I suggest in this book for the German sentence structure: a grammar that uses a trace
for extraction and trace for verb movement.
\citet[p.\,92]{Kathol2000a} argues against head movement approaches for the verb position,
claiming that traceless accounts are not possible.
However, this is not correct as the following transformation of (\mex{1}) into (\mex{2}) shows:
\ea
\label{ex-grammar-eps-head}
\begin{tabular}[t]{@{}ll@{}}
\baro{v}  $\to$ \mbox{np}, v\\
v $\to$ $\epsilon$\\
\end{tabular}
\z
\ea
\label{ex-grammar-head}
\baro{v}   $\to$ \mbox{np}, v\\
\baro{v}   $\to$ \mbox{np}
\z
Instead of using a verb trace as in (\mex{0}) one can fold it into the rule. 
If we assume binary branching structures for head"=argument combination, head"=adjunct combination
and head"=cluster combination, such a trace elemination results in three new schemata in which no head daughter is present
since it was removed due to the elemination of the verbal trace.

Eliminating extraction traces from a phrase structure grammar works parallel to the
elemination of verb traces in (\mex{0}). For the grammar in (\mex{1}) we get (\mex{2}):
\ea
\begin{tabular}[t]{@{}ll@{}}
\baro{v}   $\to$ \mbox{np}, v\\
np $\to$ $\epsilon$\\
\end{tabular}
\z
\ea
\label{ex-grammar-trace-elim}
\baro{v}   $\to$ \mbox{np}, v\\
\baro{v}   $\to$ v
\z
In our HPSG grammar we get three new schemata
since arguments, adjuncts, and parts of the predicate complex can
be extracted. In the extraction case, the non"=head"=daughter is removed from the rule. The sentences
in (\mex{1}) are examples in the analysis of which these six rules will be needed:
\eal
\ex 
\gll Er$_i$ liest$_j$ t$_i$ ihn t$_j$.\\
     he     reads     {}    him {}\\
\glt `He reads it.'
\ex
\gll Oft$_i$ liest$_j$ er  ihn t$_i$ nicht t$_j$.\\
     often   reads     he  him {}    not   {}\\
\glt `He does not read it often.'
\ex
\gll Lesen$_i$ wird$_j$ er es t$_i$ müssen t$_j$.\\
     read      will     he it {}    must   {}\\      
\glt `He will have to read it.'
\zl
t$_j$ is the verb trace and t$_i$ is an extraction trace. In (\mex{0}a) the verb trace forms a constituent
with an argument, in (\mex{0}b) with an adjunct and in (\mex{0}c) with \emph{müssen}, which is a part of the predicate complex.
For these cases we need the first three rules. The second set of rules is needed for the combination with extraction
traces of respective types: In (\mex{0}a) the extracted element is an argument, in (\mex{0}b) it is an adjunct,
and in (\mex{0}c) it is a part of the predicate complex. 

If we look at grammars containg two traces we get the following situation:
\ea
\label{bsp-grammar-np-v-trace}
\begin{tabular}[t]{@{}ll@{}}
\baro{v}   $\to$ \mbox{np}, v\\
v $\to$ $\epsilon$\\
np $\to$ $\epsilon$\\
\end{tabular}
\z
The categories that can be rewritten as $\epsilon$ are v and np but also \baro{v} since both elements
on the right"=hand side of the first rule can be rewritten as $\epsilon$. If we omit all those
categories from right"=hand sides that can be rewritten to $\epsilon$, we get the following rules:
%% In order to facilitate eposition, I organize these rules in the set (\mex{1}a). Inserting the v
%% $\to$ $\epsilon$ into all places where v appears on the right"=hand side, we get (\mex{1}b).
%% \eal
%% \ex \{ \baro{v}   $\to$ \mbox{np}, v; v $\to$ $\epsilon$; np $\to$ $\epsilon$ \}
%% \ex \{ \baro{v}   $\to$ \mbox{np}, v; \baro{v} $\to$ \mbox{np}; np $\to$ $\epsilon$ \}
%% \zl
%Taking the rules from (\ref{ex-grammar-head}) and (\ref{ex-grammar-trace-elim}) we get:
\ea
\baro{v}   $\to$ \mbox{np}, v\\
\baro{v}   $\to$ \mbox{np}\\
\baro{v}   $\to$ v
\z
Due to the elimination of the extraction trace in (\mex{-1}), we get the rule \baro{v} $\to$ v and
the elimination of the verbal trace results in \baro{v}   $\to$ \mbox{np}.

For our HPSG grammar this means that we get nine new grammar rules: we have three new empty
elements that arise when a verb movement trace is directly combined with an extraction trace.
Since the extraction trace can be the non"=head daughter in the head"=argument structure (\mex{1}a),
head"=adjunct structure (\mex{1}b) or head"=cluster structure (\mex{1}c):
\eal
\ex 
\gll Er$_i$    [schläft$_j$ t$_i$ t$_j$].\\
     he        \spacebr{}sleeps\\
\glt `He sleeps.'
\ex 
\gll Jetzt$_i$ [schlaf$_j$ t$_i$ t$_j$]!\\
     now       \spacebr{}sleep\\
\glt `Sleep now!'
\ex 
\gll Geschlafen$_i$ [wird$_j$ t$_i$ t$_j$]!\\
     slept \spacebr{}is\\
\glt `Sleep!'
\zl
Due to these new three traces we need three aditional rules where each of the new traces is folded
into the rule instead of the argument daughter in the head"=argument schema.

For the examples in (\mex{1}) and (\mex{2}) we need six new rules, since the trace combinations
can function as heads in head"=argument structures (\mex{1}) and in head"=adjunct structures (\mex{2}):
\eal
\ex 
\gll Den Aufsatz$_i$ liest$_j$ [er t$_i$ t$_j$].\\
     the paper       reads     \spacebr{}he\\
\ex 
\gll Oft$_i$ liest$_j$ er [ihn t$_i$ t$_j$].\\
     often reads he \spacebr{}it\\
\glt `He reads it often.'
\ex 
\gll Lesen$_i$ wird$_j$ er [ihn t$_i$ t$_j$].\\
     read      will     he \spacebr{}it\\
\glt `He will read it.'
\zl
\eal
\ex 
\gll Den Aufsatz$_i$ liest$_j$ er [nicht t$_i$ t$_j$].\\
     the paper       reads     he \spacebr{}not\\
\glt `He does not read the paper.'
\ex 
\gll Oft$_i$ liest$_j$ er ihn [nicht t$_i$ t$_j$].\\
     often   reads     he it \spacebr{}not\\
\glt `He does not read it often.'
\ex 
\gll Lesen$_i$ wird$_j$ er ihn [nicht t$_i$ t$_j$].\\
     read      will    he it  \spacebr{}not\\
\zl


I applied this technique of epsilon elimination to the HPSG grammar that
was developed for the \verbmobil system \citep{MK2000a}, 
but there are processing systems, like Trale \citep*{MPR2002a-u},
that do such grammar conversion automatically \citep{Penn99b}.
The grammar in (\ref{bsp-grammar-np-v-trace}) and the corresponding
HPSG equivalent directly encode the claim that the np and v can be omitted, while
this information is only implicitly contained in the rules we get from specifying an
epsilon free grammar by hand. 
The same would be true for a grammar that accounts for copulaless sentences by
stipulating several constructions for questions and declarative sentences with
a missing finite verb.

Using grammar transformations to get epsilon"=free linguistic descriptions can yield rather
complicated rules that do not capture the facts in an insightful way. This is
especially true in cases where two or more empty elements are eliminated by
grammar transformation. While this is not a problem for computational algorithms
that deal with formally specified grammars, it is a problem for linguistic specifications.
For more discussion see Müller (\citeyear[Chapter~6.2.5.1]{Mueller2002b};
\citeyear{Mueller2005c}; \citeyear{Mueller2004e}; \citeyear[Chapter~19]{MuellerGT-Eng1}).



%% A reviewer remarked that empty elements are against the spirit of HPSG. The questions
%% to be asked is: ``What is the spirit of HPSG?'', ``Who defines it?'', ``Does it change?''.
%% %
%% \citet{Bender2000a} argued in her thesis for an empty copula in
%% African American Vernacular English. This empty copula is also discussed in
%% \citew*[p.\,464]{SWB2003a-unlinked}. Once we admit a single empty element in our grammars,
%% we cannot say that empty elements are ``not in the spirit of HPSG''.


\section{Conclusion}

This brief chapter showed that sometimes grammars that use empty elements can capture insights more
directly than grammars from which the empty elements were eliminated.


%      <!-- Local IspellDict: en_US-w_accents -->
