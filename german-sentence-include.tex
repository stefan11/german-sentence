%% -*- coding:utf-8 -*-
%%%%%%%%%%%%%%%%%%%%%%%%%%%%%%%%%%%%%%%%%%%%%%%%%%%%%%%%%
%%   $RCSfile: grammatiktheorie-include.tex,v $
%%  $Revision: 1.13 $
%%      $Date: 2010/11/16 08:40:32 $
%%     Author: Stefan Mueller (CL Uni-Bremen)
%%    Purpose: 
%%   Language: LaTeX
%%%%%%%%%%%%%%%%%%%%%%%%%%%%%%%%%%%%%%%%%%%%%%%%%%%%%%%%%

%% -*- coding:utf-8 -*-
\title{German clause structure: An analysis with special consideration of so-called multiple frontings}
\author{Stefan Müller\\with contributions by Felix Bildhauer and Philippa Cook}
\renewcommand{\lsSpineAuthor}{Stefan Müller}

\typesetter{Stefan Müller}
\translator{Andrew Murphy, Stefan Müller}
\proofreader{Viola Auermann}


\BackTitle{German clause structure: An analysis with special consideration of so-called multiple frontings}
\BackBody{This book argues for a head-movement analysis of German within the framework of HPSG. While many
surface-based analyses of German sentence structure are compatible with simple verb last, verb first
and verb second sentences, there are examples that seem to contradict the verb second property of
German in that more than one constituent is placed before the finite verb. I argue that
surface-based approaches do not capture the phenomenon adequately and that an analysis with an empty
verbal head is the only one that gets the facts right.

The book discusses alternative GPSG, HPSG, Dependency Grammar and Construction Grammar analyses.
It ends with a general discussion of empty elements.
}


\dedication{For Max}
%% \renewcommand{\lsISBNdigital}{978-3-944675-21-3}
%% \renewcommand{\lsISBNhardcoverOne}{978-3-946234-29-6}
%% \renewcommand{\lsISBNhardcoverTwo}{978-3-946234-40-1}
%% \renewcommand{\lsISBNsoftcoverOne}{978-3-946234-30-2}
%% \renewcommand{\lsISBNsoftcoverTwo}{978-3-946234-41-8}
%% \renewcommand{\lsISBNsoftcoverusOne}{978-1-530465-62-0}
%% \renewcommand{\lsISBNsoftcoverusTwo}{978-1-523743-82-7}
\renewcommand{\lsSeries}{eotms} % use lowercase acronym, e.g. sidl, eotms, tgdi
\renewcommand{\lsSeriesNumber}{???} %will be assigned when the book enters the proofreading stage
\renewcommand{\lsURL}{http://langsci-press.org/catalog/book/???} % contact the coordinator for the right number

\usepackage{csquotes}

% http://tex.stackexchange.com/questions/38607/no-room-for-a-new-dimen
\usepackage{etex}\reserveinserts{28}



\hypersetup{bookmarksopenlevel=0}

%% now loaded by the langsci class
%% \iftoggle{draft}{
%% \usepackage{todonotes}
%% }{
%% \usepackage[disable]{todonotes}
%% }

\iftoggle{draft}{}{
\presetkeys{todonotes}{disable}{}
}


\usepackage{metalogo} % xelatex

\usepackage{multicol}

\usepackage{bookmark}


% This has side effects on my-ccg commands do no know why
%\usepackage{./langsci/langsci-optional}

% used to be in this package
\providecommand{\citegen}{}
\renewcommand{\citegen}[2][]{\citeauthor{#2}'s (\citeyear*[#1]{#2})}
\providecommand{\lsptoprule}{}
\renewcommand{\lsptoprule}{\midrule\toprule}
\providecommand{\lspbottomrule}{}
\renewcommand{\lspbottomrule}{\bottomrule\midrule}
\providecommand{\largerpage}{}
\renewcommand{\largerpage}[1][1]{\enlargethispage{#1\baselineskip}}

\providecommand{\tablevspace}{}
% vertical space to structure tables
\renewcommand{\tablevspace}{\\[-.5em]}


\usepackage{./styles/oneline}

\usepackage{langsci-lgr}

% \newcommand{\nom}{\NOM}
% \newcommand{\gen}{\GEN}
% \newcommand{\dat}{\DAT}
% \newcommand{\acc}{\ACC}


\usepackage{graphicx}



\usepackage{lastpage,float,soul,tabularx}




\usepackage{./styles/mycommands}% \spacebr


\usepackage{langsci-gb4e}


% not needed as of texlive 2020
%\usepackage{jambox}

\usepackage{subfig}

%\renewcommand{\xbar}{X̅\xspace}


% for reasons I do not understand this cannot be moved further down and 
% the loading of forest further down cannot be removed. St. Mü. 26.01.2017
% It breaks the dependency grammar trees in forest.
\usepackage{langsci-forest-setup}


% has to be loaded after forest-setup because of incompatibilities with the dg-style.
% sorry folks, this has to be loaded. All this babel-shorthands stuff is way too complicated and
% does not work for my abbreviations. It does not work in bibtex items since it is language dependent.
\usepackage{german}\selectlanguage{USenglish}

\usepackage{styles/merkmalstruktur,styles/abbrev,styles/makros.2020,styles/my-xspace,styles/article-ex,styles/additional-langsci-index-shortcuts,
styles/eng-date,styles/my-theorems}

% loaded in macros.2e \usepackage[english]{varioref}
% do not stop and warn! This will be tested in the final version
%\vrefwarning
\let\vref\ref


\setcounter{secnumdepth}{4}

% draw a grid for getting the coordinates
\usepackage{styles/tikz-grid}

% for offsets in trees
\newlength{\offset}
\newlength{\offsetup}

\ifxetex
\usepackage{styles/eng-hyp-utf8}
\else
\usepackage{styles/eng-hyp}
\fi

\usepackage{appendix}


% adds lines to both the odd and even page.
% bloddy hell! This is really an alpha package! Do not use the draft option! 07.03.2016
\usepackage{addlines}

%% \let\addlinesold=\addlines
%% % there is one optional argument. Second element in brackets is the default 
%% \renewcommand{\addlines}[1][1]{
%% \todosatz{addlines}
%% \addlinesold[#1]
%% }


% do nothing now
%\let\addlinesold=\addlines
%\renewcommand{\addlines}[1][1]{}

% for addlines to work
\strictpagecheck



% http://tex.stackexchange.com/questions/3223/subscripts-for-primed-variables
%
% to get 
% {}[ af   [~]\sub{V} ]\sub{V$'$}
%
% typeset properly. Thanks, Sebastian.
%
\usepackage{subdepth}


%\usepackage{caption}










%% -*- coding:utf-8 -*-



%\usepackage{ogonek}        % For Ewa Dabrowska


%\newcommand{\tag}{TAG\indextag} % has to be here, conflict with latexbeamer

% mit der Index-Version geht die Silbentrennung nicht
\renewcommand{\word}[1]{\emph{#1}}

\newcommand{\dom}{\textsc{dom}\xspace}

\renewcommand{\relation}[1]{\textit{#1}}


%% \newcommand{\iw}[1]{}
%% \newcommand{\iaf}[1]{}
%% \newcommand{\iadata}[1]{}
%% \newcommand{\iawrong}[1]{}


\iftoggle{draft}{
\proofmodetrue
}



\newcommand{\page}{}



\let\mc=\multicolumn



\setcounter{secnumdepth}{4}

\provideboolean{hpsg-buch}
\setboolean{hpsg-buch}{false}




\newcommand{\sigle}[1]{#1}
\newcommand{\lindex}[1]{$_{#1}$}



\bibliography{bib-abbr,biblio}

\begin{document}


\frontmatter

\maketitle

\tableofcontents


\chapter*{Preface}

This book motivates an analysis of the German clause in which the verb in initial position (verb
first or verb second) is related to a trace in final position. Such analyses involving so"=called
verb movement are standard in Mainstream Generative Grammar but are frowned upon by all those
researchers that want to avoid empty elements. Working in the framework of Head-Driven Phrase
Structure Grammar I followed a linearization"=based approach \citep{Reape90a,Reape94a} from
1993--2003. In the year 2000 I noticed data that looked as if more than one constituent was fronted,
which is problematic for theories of German, since German is regarded as a verb second language,
that is, there should be exactly one constituent in front of the finite verb in declarative clauses
(leaving aside eliptical sentences). I developed analyses in the linearization"=based framework I
was working in, but for reasons that will be explained in this book, they were not satisfying. In
the end I changed my mind and completely revised my theories and computer implementations and
adopted a verb-movement analysis that is similar in spirit to the GB analysis. This analysis --
which was developed by \citet{Meurers99a}, based on work of \citet{KW91a,Kiss95a} -- is compatible
with the analysis of multiple frontings that is developed in this book. 

The present book is based on two articles that appeared in German in the journal \emph{Linguistische
  Berichte} in 2005 \citep{Mueller2005c,Mueller2005d}. Since these two papers belong to those of my papers
that are cited most often, I decided that it might be a good idea to make them availible to a wider
audience. The chapter~\ref{chap-german-sentence-structure} on German clause structure and the
Chapter~\ref{chapter-mult-front} on apparent multiple frontings and parts of
Chapter~\ref{chap-alternatives} on alternative analyses of the German clause are based on these
papers.  
The book ends with a chapter on empty elements, which is adapted from \citew{Mueller2004e}. This
chapter is meant to be a general discussion that shows what the cost is of alternative approaches
that try to avoid empty elements. Finally, there is an Appendix continaing a list of example
sentences that are used as a test suite for testing the computer-processable grammar that covers the
phenomena described in this book.



\section*{Acknowledgements}

I would like to thank the following people for their helpful comments and insights:
Bettina Braun,
Veronika Ehrich,
Gisbert Fanselow,
Peter Gallmann,
Rosemarie Lühr,
Detmar Meurers,
Susan Olsen,
Marga Reis, 
Christine Römer and 
Jan-Philipp Söhn 
as well as the anonymous reviewers from Formal Grammar, Konvens, CSLI Publications and \emph{Linguistische Berichte}.
I would also like to thank Anette Frank, Hans-Martin Gärtner, Tibor Kiss and Karel Oliva for their helpful discussion.

I have presented the analysis developed in this book at Formal Grammar 2002 in Trento, Konvens
2002 in Saarbrücken, at an invited talk for the SFB 441 at the University of Tübingen in 2002, at
the HPSG"=Workshop for Germanic languages 2003 at the University of Bremen, 
%
the Workshop Deutsche Syntax: Empirie und Theorie 2004 in Gothenburg,
%
at the Institute for Linguistics of the University Leipzig in 2004,
%
at the Formal Grammar conference 2004 in Nancy,
the 	Zentrum für Allgemeine Sprachwissenschaft, Typologie und
Universalienforschung (ZAS) in Berlin 2006,
%
at the Institute for German Language in Wuppertal in 2011,
%
at the colloquium
\emph{Der Satzanfang im Deutschen: syntaktische, semantisch-​pragmatische und
  informationsstrukturelle Integration vs. Desintegration} 2011 in Paris,
%
and at the workshop \emph{Satztypen und Konstruktionen im Deutschen: Satztypen:
  lexikalisch oder konfigurational?} at the Johannes Gutenberg-​Universität Mainz in
2013.
%
Many thanks to those present at these events for the subsequent discussion and I would especially like to thank the respective people/institutions for inviting me. 

The data discussed in this book come, in the most part, from my own collection of material I have
read over the years. Examples from the Mannheimer Morgen, Frankfurter Rundschau, St.\ Galler
Tagblatt, Tiroler Tageszeitung and the Züricher Tagesanzeiger were retrieved from the German Reference Corpus (\citealp{DeReKo}, \url{http://ids-mannheim.de/DeReKo}). Additional examples can be found at
\url{http://hpsg.fu-berlin.de/~stefan/Pub/mehr-vf-ds.html} and in the database of annotated  examples that was constructed from DeReKo data by Felix \citet{Bildhauer2011a} in the project A6 of the Collaborative Research Center/SFB 632 (accessible at \url{http://hpsg.fu-berlin.de/Resources/MVB/}).%

I would like to thank Andrew Murphy for translating \citew{Mueller2005c} and \citew{Mueller2005d},
which are the core of the Chapters~\ref{chapter-introduction}--\ref{chapter-mult-front} and for
proofreading these parts.


%~\bigskip

~\medskip

\noindent
Berlin, \today\hfill Stefan Müller



\mainmatter

%% -*- coding:utf-8 -*-
%%%%%%%%%%%%%%%%%%%%%%%%%%%%%%%%%%%%%%%%%%%%%%%%%%%%%%%%%
%%   $RCSfile: hpsg-satz-struktur-lb.tex,v $
%%  $Revision: 1.7 $
%%      $Date: 2009/03/27 16:28:44 $
%%     Author: Stefan Mueller (DFKI)
%%    Purpose: 
%%   Language: LaTeX
%%%%%%%%%%%%%%%%%%%%%%%%%%%%%%%%%%%%%%%%%%%%%%%%%%%%%%%%%



\chapter{Introduction}
\label{chapter-introduction}

The German sentence can be adequately described using the topological model of (\citealp*{Reis80a}; \citealp*{Hoehle86}; \citealp{Askedal86}).
In the sentence (\mex{1}), the verbs \emph{hat} `has' and \emph{gegeben} `given' form a `frame'
around the rest the of the sentence. The finite verb \emph{hat} `has' occupies the left sentence bracket and the
infinitive \emph{gegeben} `given' the right one. 
\ea
\gll Der Mann hat der Frau das Buch gegeben, das wir alle kennen.\\
     the man  has the woman the book given   that we all  know\\
\glt `The man gave the woman the book that we all know.'
\z
Situated between the sentence brackets is the so-called `middle-field' (\emph{Mittelfeld}). The prefield (\emph{Vorfeld}) precedes
the left bracket, and the postfield (\emph{Nachfeld}) follows the right bracket. 

In subordinate clauses introduced by a conjunction, the conjunction takes the left sentence bracket
and the finite verb is located with the rest of the non-finite verb forms in the right bracket:
\ea
\label{ex-gegeben-hat}
\gll dass der Mann der Frau das Buch gegegben hat\\
     that the man  the woman the book given has\\
\glt `that the man gave the woman the book'
\z
In this book I develop an analysis which -- like many analyses of German clause structure before -- establishes a link between verb-first and verb-final sentences.


Constituents in the middle-field exhibit a relatively free ordering:
\eal
\label{bsp-GPSG-anordnung}
\ex 
\gll {}[weil] der Mann der Frau das Buch gibt\\
     {}\spacebr{}because the.\nom{} man the.\dat{} woman the.\acc{} book gives\\
\glt `because the man gives the book to the woman'
\ex 
\gll {}[weil] der Mann das Buch der Frau gibt\\
     {}\spacebr{}because the man.\nom{} the book.\acc{} the woman.\dat{} gives\\
\ex 
\gll {}[weil]          das        Buch der        Mann der        Frau gibt\\
     \spacebr{}because the.\acc{} book the.\nom{} man  the.\dat{} woman gives\\
\ex 
\gll {}[weil] das Buch der Frau der Mann gibt\\
{}\spacebr{}because the.\acc{} book the.\dat{} woman the.\nom{} man gives\\
\ex 
\gll {}[weil] der Frau der Mann das Buch gibt\\
{}\spacebr{}because the.\dat{} woman the.\nom{} man the.\acc{} book gives\\
\ex 
\gll {}[weil] der Frau das Buch der Mann gibt\\
{}\spacebr{}because the.\dat{} woman the.\acc{} book the.\nom{} man gives\\
\zl
This is accounted for by assuming that a head may combine with its arguments in any order. Of course
there are restrictions, but these restrictions are represented independently of the general
combinatory mechanism.

The prefield can be occupied by one constituent (an adjunct, subject or complement), which is why German is
viewed as a verb-second language (\citealp[Chapter~2.4]{Erdmann1886a};
\citealp[\page 69, \page 77]{Paul1919a}). Examples such as (\mex{1}) show that occupation of the prefield cannot simply be explained as an ordering variety of an element dependent on 
the finite verb (in analogy to reorderings in the middle field):

\ea
\gll{}[Um zwei Millionen Mark]$_i$ soll er versucht haben, [eine Versicherung \_$_i$ zu betrügen].\footnotemark\\
      \spacebr{}around two million Deutschmarks should he tried have \spacebr{}an insurance.company
              {} to defraud\\
\footnotetext{
         taz, 04.05.2001, p.\,20.
}
\glt `He supposedly tried to defraud an insurance company of two million Deutschmarks.'

\z
%
The head that governs the PP (\emph{betrügen} `defraud') is located inside of the infinitive clause. The PP as such is not directly dependent on the finite
verb and can therefore not have reached the prefield by means of a simple local reordering operation. This
shows that the dependency between \emph{betrügen} and \emph{um zwei Millionen} `around two million
Deutschmarks' is a long distance dependency: an element belonging to a deeply embedded head has been fronted over several phrasal borders.

\citet{Thiersch78a}, \citet[\page 55]{denBesten83a}, \citet{Uszkoreit87a}\ia{Uszkoreit} and others have suggested 
a connection between verb-second and verb-first sentences, and that verb-second sentences should be
analyzed as verb-first sentences with an extracted constituent placed in the prefield.

\eal
\ex 
\gll Kennt er das Buch?\\
	knows he the book\\
\glt `Does he know the book?'
\ex 
\gll Das Buch kennt er.\\
	the book knows he\\
\glt `He knows the book.'
\zl 
This is also the approach that I assume in this book.


The elements in the right bracket form a complex. I assume that such complexes are formed first and
are then combined with the arguments that depend on the elements in the complex. For instance, \emph{gegeben} `given' and
\emph{hat} `has' in (\ref{ex-gegeben-hat}) form one unit, which is then combined with \emph{das
  Buch} `the book', \emph{der Frau} `the woman', and  \emph{der Mann} `the man' in later steps. 

The left peripheral elements of this verbal complex can (in some cases together with the adjacent material
from the middle field) be moved into the prefield:
\eal
\ex
\gll Gegeben hat er der Frau das Buch.\\
     given has he the woman the book\\
\glt `He gave the woman the book.'
\ex
\gll Das Buch gegeben hat er der Frau.\\
     the book given   has he the woman\\
\ex
\gll Der Frau gegeben hat er das Buch.\\
     the woman given  has he the book\\
\ex
\gll Der Frau  das Buch gegeben hat er.\\
     the woman the book given   has he\\
\zl
Since the fronted verbal projections in (\mex{0}a--c) are partial, such frontings are called \emph{partial verb
phrase frontings}.

While there is a broad consensus among reasearchers from various frameworks that German is a V2
language, some challenging examples can be found that seem to contradict the V2 characteristic
of German (see \citew{Mueller2003b,Bildhauer2011a} and the literature discussed there). Some examples are given in
(\mex{1}) and further examples are discussed in Section~\ref{sec-phenomenon-mult-front}.

\eal
\label{bsp-smvfb}
\ex 
\gll {}[Dauerhaft] [mehr Arbeitsplätze] gebe es erst, wenn sich eine Wachstumsrate von  mindestens 2,5 Prozent über einen Zeitraum von drei oder vier Jahren halten lasse.\footnotemark\\ 
       \spacebr{}constantly \spacebr{}more jobs give it first when REFL a growth.rate of  at.least 2.5 percent over a time.period of three or four years hold lets\\
\footnotetext{
        taz, 19.04.2000, p.\,5. %taz Nr. 6123 vom 19.4.2000 Seite 5
} 
\glt `In the long run, there will only be more jobs available, when a growth rate of at least 2.5 percent 
can be maintained over a period of three of four years.'	      
\ex 
\gll [Unverhohlen verärgert] [auf Kronewetters Vorwurf] reagierte Silke Fischer.\footnotemark\\
     \spacebr{}Blatantly annoyed \spacebr{}by Kronewetter's reproach reacted Silke Fischer\\
\footnotetext{
    taz berlin, 23.04.2004, p.\,21.
}
\glt `Blatantly annoyed, Silke Fischer reacted to Kronewetter's reproach.'
\ex 
\gll {}[Hart] [ins Gericht] ging Klug mit dem Studienkontenmodell der Landesregierung.\footnotemark\\
       \spacebr{}hard \spacebr{}in.the court went Klug with the tuition.account.model. of.the
       state.government\\
\footnotetext{
  taz nord, 19.02.2004, p.\,24.
  }
\glt `Klug roasted the state government's tuition account model.'
\zl
In Chapter~\ref{chapter-mult-front} I show how these examples can be analyzed using a special variant
of the lexical rule that is suggested for the analysis of verb-initial sentences in combination with verbal
complex formation and partial verb phrase fronting.





%% -*- coding:utf-8 -*-
%%%%%%%%%%%%%%%%%%%%%%%%%%%%%%%%%%%%%%%%%%%%%%%%%%%%%%%%%
%%   $RCSfile: grammatiktheorie-include.tex,v $
%%  $Revision: 1.13 $
%%      $Date: 2010/11/16 08:40:32 $
%%     Author: Stefan Mueller (CL Uni-Bremen)
%%    Purpose: 
%%   Language: LaTeX
%%%%%%%%%%%%%%%%%%%%%%%%%%%%%%%%%%%%%%%%%%%%%%%%%%%%%%%%%


\chapter{German clause structure}
\label{chap-german-sentence-structure}

This chapter deals with the basic sentence structure of German. Section~\ref{sec-german-order}
introduces the phenomena that have to be covered. As Brigitta Haftka formulated it in the title of
her paper, German is a verb second language with verb last order and free constituent order
\citep{Haftka96a}. This first sounds contradictory, but as will be shown in the following section,
these three properties are indeed independent. I first motivate the categorization of German as
an SOV language in Section~\ref{sec-german-sov}, then I discuss the free constituent order (Section~\ref{sec-free-order-phen}) and the V2
property (Section~\ref{sec-v2-phen}). Verbal complexes interact with free constituent order and are discussed in
Section~\ref{sec-vc-phen}. Frontings of parts of the verbal complex and non-verbal arguments are discussed in
Section~\ref{sec-pvp-phen}.

Section~\ref{sec-analysis-v1-v2} provides the analysis of these phenomena.


\section{The phenomenon}
\label{sec-german-order}



(\mex{1}) provides examples of the main clause types in German: (\mex{1}a) is a verb last (VL) sentence,
(\mex{1}b) is a verb first (V1) sentence, and (\mex{1}c) a verb second (V2) sentence:
\eal
\ex 
\gll dass Peter Maria ein Buch gibt\\
     that Peter Maria a book gives\\
\glt `that Peter gives a book to Maria'
\ex
\gll Gibt Peter Maria ein Buch?\\
     gives Peter Maria a book\\
\glt `Does Peter give a book to Maria?'
\ex
\gll Peter gibt Maria ein Buch.\\
     Peter gives Maria a book\\
\glt `Peter gives a book to Maria.'
\zl
The following subsections deal with all these sentences types and address the question whether one
of them is basic.


\subsection{German as a SOV language}
\label{sec-german-sov}

It is assumed by many researchers that German is an SOV language, although this order is only
visible in embedded clauses like (\mex{0}a) and not in yes/no questions like (\mex{0}b) and
declarative main clauses like (\mex{0}c). The reason for this assumption is that German patterns
with many SOV languages and differs from SVO languages (for example Scandinavian languages). The
analysis of German as an SOV language is almost as old as Transformational Grammar: it was first
suggested by \citet*[\page34]{Bierwisch63}.
	Bierwisch attributes the assumption of an underlying verb"=final order to \citet{Fourquet57a}. A German translation of the
	French manuscript cited by Bierwisch can be found in \citew[\page117--135]{Fourquet70a}. For other proposals, see \citew{Bach62a},
\citew{Reis74a}, \citew{Koster75a}, and \citew[Chapter~1]{Thiersch78a}. 
%, \citet{denBesten83a} wohl angeblich schon 77 am MIT
Analyses which assume that
German has an underlying SOV pattern were also suggested in \gpsg \citep[\page110]{Jacobs86a}, 
LFG \citep[Section~2.1.4]{Berman96a-u} and HPSG   (\citealp*[Section~4.7]{KW91a}; \citealp{Oliva92a}; \citealp*{Netter92};   
\citealp*{Kiss93}; \citealp*{Frank94}; \citealp*{Kiss95a}; \citealp{Feldhaus97},
\citealp{Meurers2000b}; \citealp{Mueller2005c}). 

The assumption of verb"=final order\label{page-verbletzt} as the base order is motivated by the following observations:\footnote{%
	For points 1 and 2, see \citew[\page34--36]{Bierwisch63}. For point~\ref{SOV-Skopus} see \citew[Section~2.3]{Netter92}.%
}


\begin{enumerate}
\item Verb particles form a close unit with the verb.
\eal
\ex 
\gll weil er morgen an-fängt\\
	 beause he tomorrow \textsc{prt}-starts\\
\glt `because he is starting tomorrow'
\ex 
\gll Er fängt morgen an.\\
	 he starts tomorrow \textsc{prt}\\
\glt `He is starting tomorrow.'
\zl
This unit can only be seen in verb"=final structures, which speaks for the fact that this structure reflects the base order.

\item Verbs formed by backformation often cannot be separated.

Verbs which are derived from a noun by back-formation\is{back-formation} (\eg \emph{uraufführen} 
`to perform something for the first time', can often not be divided into their component parts and
V2 clauses are therefore ruled out (This was first mentioned by \citet{Hoehle91b} in unpublished
work. The first published source is \citew[\page 62]{Haider93a}):
\eal
\ex[]{
\gll weil sie das Stück heute urauf-führen\\
	 because they the play today \textsc{prt}-lead\\
\glt `because they are performing the play for the first time today'
}
\ex[*]{
\gll Sie urauf"|führen heute das Stück.\\
	 they \textsc{prt}-lead today the play\\
}
\ex[*]{
\gll Sie führen heute das Stück urauf.\\
	 they lead today the play \textsc{prt}\\
}
\zl
The examples show that there is only one possible position for the verb. This order is the one that
is assumed to be the base order.

\item Some constructions allow SOV order only.

Similarly, it is sometimes impossible to realize the verb in initial position when elements like
\emph{mehr als} `more than' are present in the clause \citep{Haider97c,Meinunger2001a}: 
\eal
\ex[]{
\gll dass Hans seinen Profit letztes Jahr mehr als verdreifachte\\
     that Hans his         profit last       year more than tripled\\
\glt `that Hans increased his profit last year by a factor greater than three'
}
\ex[]{
\gll Hans hat seinen Profit letztes Jahr mehr als verdreifacht.\\
     Hans has his    profit last    year more than tripled\\
\glt `Hans increased his profit last year by a factor greater than three.'
}
\ex[*]{
\gll Hans verdreifachte seinen Profit letztes Jahr mehr als.\\
     Hans tripled       his    profit last year more than\\
}
\zl
So, it is possible to realize the adjunct together with the verb in final position, but there are
constraints regarding the placement of the finite verb in initial position.


\item Verbs in non"=finite clauses and in finite subordinate clauses with a conjunction are
always in final position (I am ignoring the possibility of extraposing constituents):
\eal
\ex 
\gll Der Clown versucht, Kurt-Martin die Ware zu geben.\\
     the clown tries Kurt-Martin the goods to give\\
\glt `The clown is trying to give Kurt-Martin the goods.'
\ex 
\gll dass der Clown Kurt-Martin die Ware gibt\\
	 that the clown Kurt-Martin the goods gives\\
\glt `that the clown gives Kurt-Martin the goods'
\zl
The English translation shows that English has VO order where German has an OV order.

\item If one compares the position of the verb in German to Danish\is{Danish} (Danish is an SVO language
like English), then one can clearly see that the verbs in German form a cluster at the end of the sentence,
whereas they occur before any objects in Danish \citep{Oersnes2009b}:
\eal
\label{ex-VO-OV}
\ex 
\gll dass er ihn gesehen$_3$ haben$_2$ muss$_1$\\
	 that he him seen have must\\
\ex 
\gll at han må$_1$ have$_2$ set$_3$ ham\\
     that he must have seen him\\
\glt `that he must have seen him'
\zl


\item\label{SOV-Skopus}\is{scope|(} The scope relations of the adverbs in (\ref{bsp-absichtlich-nicht-anal}) depend on their order:
the left"=most adverb has scope over the two following elements.\footnote{%
At this point, it should be mentioned that there seem to be exceptions from the rule that modifiers to the left take scope over those to
their right. \citet*[\page47]{Kasper94a} discusses examples such as (i), which go back to \citet*[\page137]{BV72}.
\eal
\label{bsp-peter-liest-gut-wegen}
\ex 
\gll Peter liest gut wegen der Nachhilfestunden.\\
	 Peter reads well because.of the tutoring\\
\glt `Peter can read well thanks to the tutoring.'
\ex 
\gll Peter liest wegen der Nachhilfestunden gut.\\
	 Peter reads because.of the tutoring well\\
\zl
% Kiss95b:212
	As \citet[Section~6]{Koster75a} and \citet*[\page67]{Reis80a} have shown, these are not particularly convincing counter"=examples
	as the right sentence bracket is not filled in these examples and it must therefore not necessarily constitute normal reordering inside
	of the middle field, but could instead be a case of extraposition\is{extraposition}.
	As noted by Koster and Reis, these examples become ungrammatical if one fills the right bracket and does not extrapose the causal adjunct:
\eal
\ex[*]{
\gll Hans hat gut  wegen      der Nachhilfestunden gelesen.\\
     Hans has well because.of the tutoring read\\
}
\ex[]{
\gll Hans hat gut gelesen wegen der Nachhilfestunden.\\
	 Hans has well read because.of the tutoring\\
\glt `Hans has been reading well because of the tutoring.'
}
\zl
However, the following example from \citet[\page 383]{Crysmann2004a} shows that, even with the right bracket occupied, one can still have an
order where an adjunct to the right has scope over one to the left:
\ea
\gll Da muß es schon erhebliche Probleme mit der Ausrüstung gegeben haben, da wegen schlechten
  Wetters ein Reinhold Messmer niemals aufgäbe.\\
  there must it already serious problems with the equipment given have since because.of bad weather a Reinhold Messmer never
  would.give.up\\
 \glt `There really must have been some serious problems with the equipment because someone like Reinhold Messmer would never give
  up just because of some bad weather.'
%\ex Stefan  ist wohl deshalb krank geworden, weil er äußerst hart wegen der Konferenz in Bremen gearbeitet hat.
\z
Nevertheless, this does not change anything regarding the fact that the corresponding cases in (\ref{bsp-absichtlich-nicht-anal}) 
and (\ref{bsp-absichtlich-nicht-anal-v1}) have the same meaning regardless of the position of the verb. The general means of semantic
composition may well have to be implemented in the way suggested by Crysmann.

Another word of caution is in order here: There are SVO languages like French that also have a left
to right scoping of adjuncts \citep[\page 156--161]{BGK2004a-u}. So, the argumentation above should not be seen as the only
fact supporting the SOV status of German. In any case the analyses of German that were
worked out in various frameworks can explain the facts nicely.
}
This was explained with the following structure:
\eal
\label{bsp-absichtlich-nicht-anal}
\ex 
\gll weil er [absichtlich [nicht lacht]]\\
	 because he \spacebr{}intentionally \spacebr{}not laughs\\
\glt `because he is intentionally not laughing'
\ex 
\gll weil er [nicht [absichtlich lacht]]\\
     because he \spacebr{}not \spacebr{}intentionally laughs\\
\glt `because he is not laughing intentionally'
\zl
If one compares (\mex{0}) and (\mex{1}) one can see that scope relations are not affected by verb
position. If one assumes that sentences with verb"=second order have the underlying structure in
(\mex{0}), then this fact requires no further explanation. (\mex{1}) shows the structure for (\mex{0}):
\eal
\label{bsp-absichtlich-nicht-anal-v1}
\ex 
\gll Er lacht$_i$ [absichtlich [nicht \_$_i$]].\\
     he laughs \spacebr{}intentionally \spacebr{}not\\
\glt `He is intentionally not laughing.'
\ex 
\gll Er lacht$_i$  [nicht [absichtlich \_$_i$]].\\
     he laughs \spacebr{}not \spacebr{}intentionally\\
\glt `He is not laughing intentionally.'
\zl\is{scope}
%\item Verum-Fokus
\nocite{Hoehle88a,Hoehle97a}
\end{enumerate}\is{verb final language}\is{scope|)}

These properties have been taken as evidence for an underlying SOV order of German. That is, V1 and
V2 sentences are assumed to be derived from or to be somehow related to SOV sentences. It is
possible though to represent the clause types on their own right without relating them. Respective
proposals will be discussed in Chapter~\ref{chap-alternatives}. I assumed such an analysis for ten
years and I think the basic sentence structures can be explained quite well. However, the apparent
multiple frontings, which will be discussed in the next chapter, do not integrate nicely into the
alternative analyses. This caused me to drop my analysis and to revise my grammar in a way that is
inspired by early transformational analyses.

\subsection{German as a language with free constituent order}
\label{sec-free-order-phen}

As was already mentioned in the introduction, German is a language with rather free constituent
order. For example, a verb with three arguments allows for six different orders of the
arguments. This is exemplified with the ditransitive verb \emph{geben} in (\mex{1}):
\eal
\label{ex-free-order}
\ex 
\gll {}[weil] der Mann der Frau das Buch gibt\\
     {}\spacebr{}because the.\nom{} man the.\dat{} woman the.\acc{} book gives\\
\glt `because the man gives the book to the woman'
\ex 
\gll {}[weil] der Mann das Buch der Frau gibt\\
     {}\spacebr{}because the.\nom{} man the.\acc{} book the.\dat{} woman gives\\
\ex 
\gll {}[weil] das Buch der Mann der Frau gibt\\
{}\spacebr{}because the.\acc{} book the.\nom{} man the.\dat{} woman gives\\
\ex 
\gll {}[weil] das Buch der Frau der Mann gibt\\
{}\spacebr{}because the.\acc{} book the.\dat{} woman the.\nom{} man gives\\
\ex 
\gll {}[weil] der Frau der Mann das Buch gibt\\
{}\spacebr{}because the.\dat{} woman the.\nom{} man the.\acc{} book gives\\
\ex 
\gll {}[weil] der Frau das Buch der Mann gibt\\
{}\spacebr{}because the.\dat{} woman the.\acc{} book the.\nom{} man gives\\
\zl

Adjuncts can be placed anywhere between the arguments as the examples in (\mex{1}) show.
\eal
\ex
\gll {}[weil] jetzt der Mann der Frau das Buch gibt\\
     {}\spacebr{}because now the.\nom{} man the.\dat{} woman the.\acc{} book gives\\
\glt `because the man gives the book to the woman now'
\ex
\gll {}[weil] der Mann jetzt der Frau das Buch gibt\\
     {}\spacebr{}because the.\nom{} man now the.\dat{} woman the.\acc{} book gives\\
\glt `because the man gives the book to the woman now'
\ex
\gll {}[weil] der Mann der Frau jetzt das Buch gibt\\
     {}\spacebr{}because the.\nom{} man the.\dat{} woman now the.\acc{} book gives\\
\glt `because the man gives the book to the woman now'
\ex
\gll {}[weil] der Mann der Frau das Buch jetzt gibt\\
     {}\spacebr{}because the.\nom{} man the.\dat{} woman the.\acc{} book now gives\\
\glt `because the man gives the book to the woman now'
\zl
(\mex{0}) is the result of inserting the adverb \emph{jetzt} `now' into every possible position in
(\mex{-1}a). Of course adverbs can be inserted into each of the other sentences in (\mex{-1}) in the
same way and it is also possible to have several adjuncts per clause in all the positions. (\mex{1})
is an example by \citet[\page 145]{Uszkoreit87a} that illustrates this point:
\ea
\gll \emph{Gestern} hatte \emph{in} \emph{der} \emph{Mittagspause} der Vorarbeiter \emph{in} \emph{der} \emph{Werkzeugkammer} dem Lehrling \emph{aus
Boshaftigkeit} \emph{langsam} zehn schmierige Gußeisenscheiben \emph{unbemerkt} in die Hosentasche gesteckt. \\
yesterday had during the lunch.break the foreman in the tool.shop the apprentice maliciously slowly ten
greasy cast.iron.disks unnoticed in the pocket put\\
\glt `Yesterday during lunch break, the foreman maliciously and
unnoticed, put ten greasy cast iron disks slowly into the
apprentice's pocket.'
\z

In transformational theories it is sometimes assumed that there is a base configuration from which
all other orders are derived. For instance, there could be a VP including the verb and the two
objects and this VP is combined with the subject to form a complete sentence. For all orders in
which one of the objects preceedes the subject it is assumed that there is a movement process that
takes the object out of the VP and attaches it to the left of the sentence.

An argument that has often been used to support this analysis is the fact that scope ambiguities
exist in sentences with reorderings which are not present in the base order. The explanation of such
ambiguities comes from the assumption that the scope of quantifiers
can be derived from their position before movement as well as their position after movement. When
there has not been any movement, then there is only one reading possible. If movement has taken
place, however, then there are two possible readings \citep[\page ]{Frey93a}:
\eal
\ex 
\gll Es ist nicht der Fall, daß er mindestens einem Verleger fast jedes Gedicht anbot.\\
     it is not the case that he at.least one publisher almost every poem offered\\
\glt `It is not the case that he offered at least one publisher almost every poem.'
\ex 
\gll Es ist nicht der Fall, daß er fast jedes Gedicht$_i$ mindestens einem Verleger \_$_i$ anbot.\\
	 it is not the case that he almost every poem at.least one publisher {} offered\\
\glt `It is not the case that he offered almost every poem to at least one publisher.'
\zl
The position from which the NP \emph{jedes Gedicht} `every poem' is supposed to be moved is marked
by a trace (\_$_i$) in the example above. 

It turns out that approaches assuming traces run into problems as they predict certain readings for sentences with multiple traces, which
do not exist (see \citealp[\page 146]{Kiss2001a} and \citealp[Section~2.6]{Fanselow2001a}). 
For instance in an example such as (\mex{1}), it should be possible to interpret \emph{mindestens einem Verleger} `at least one publisher' at
the position of \_$_i$, which would lead to a reading where \emph{fast jedes Gedicht} `almost every poem' has scope over \emph{mindestens einem Verleger} 
`at least one publisher'.
\ea
\gll Ich glaube, dass mindestens einem Verleger$_i$ fast jedes Gedicht$_j$ nur dieser Dichter \_$_i$ \_$_j$ angeboten hat.\\
	 I believe that at.least one publisher almost every poem only this poet {} {} offered has\\
\glt `I think that only this poet offered almost every poem to at least one publisher.'
\z
This reading does not exist, however.
\is{scope|)}

The alternative to a movement analysis is called \emph{base generation}\is{base generation} in
transformational frameworks. The possible orders are not derived by movement but are licensed by
grammar rules directly. Such a base-generation analysis, that is the direct licensing of orders without
any additional mechanisms, is the most common analysis in non-transformational frameworks like HPSG \citep{Pollard90a},
LFG \citep{Berman2003a}, Construction Grammar \citep{Micelli2012a} and Dependency Grammar \citep{Eroms2000a,GO2009a} and I provide such an analysis in Section~\ref{sec-scrambling-analysis}.
\is{constituent order|)}

\subsection{German as a verb second language}
\label{sec-v2-phen}

German is a verb second (V2) language (\citealp[Chapter~2.4]{Erdmann1886a};
\citealp[\page 69, \page 77]{Paul1919a}), that is, (almost) any constituent (an adjunct, subject or
complement) can be placed infront of the finite verb. (\mex{1}) shows some prototypical examples again involving the ditransitive verb
\emph{geben} `to give':
\eal
\ex[]{
\gll Der Mann gibt der Frau das Buch.\\
    the man gives the woman the book\\
\glt `The man gives the woman the book.'
}
\ex[]{
\gll Der Frau gibt der Mann das Buch.\\
    the woman gives the man the book\\
\glt `The man gives the woman the book.'
}
\ex[]{
\gll Das Buch gibt der Mann der Frau.\\
    the book gives the man the woman\\
\glt `The man gives the woman the book.'
}
\ex[]{
\gll Jetzt gibt der Mann der Frau das Buch.\\
    now gives the man the woman the book\\
\glt `The man gives the woman the book now.'
}
\zl
If this is compared with English, one sees that English has XP SVO order, that is the basic SVO
order stays intact and one constituent is placed infront of the sentence into which it belongs:
\eal
\ex The woman, the man gives the book.
\ex The book, the man gives the woman.
\ex Now, the man gives the woman the book.
\zl
Languages like Danish on the other hand are V2 languages like German but nevertheless SVO languages
(see the discussion of (\ref{ex-VO-OV}) on page~\pageref{ex-VO-OV}). Although the verb in embedded
sentences like (\ref{ex-VO-OV}) precedes the object and follows the subject, the finite verb appears
initially and one of the constituents is fronted. The resulting orders are identical to the ones we
see in German. 



Examples such as (\mex{1}) show that occupation of the prefield cannot simply be explained as an ordering variant of an element dependent on 
the finite verb (in analogy to reorderings in the middle field):

\ea
\gll{}[Um zwei Millionen Mark]$_i$ soll er versucht haben, [eine Versicherung \_$_i$ zu betrügen].\footnotemark\\
      \spacebr{}around two million Deutschmarks should he tried have \spacebr{}an insurance.company
              {} to defraud\\
\footnotetext{%
         taz, 04.05.2001, p.\,20.
}
\glt `He supposedly tried to defraud an insurance company of two million Deutschmarks.'
\z
%
The head that governs the PP (\emph{betrügen} `defraud') is located inside of the infinitive clause. The PP as such is not directly dependent on the finite
verb and can therefore not have reached the prefield by means of a simple local reordering operation. This
shows that the dependency between \emph{betrügen} and \emph{um zwei Millionen} `around two million
Deutschmarks' is a long distance dependency: an element belonging to a deeply embedded head has been fronted over several phrasal borders.

Such long distance dependencies are often modeled by devices that assume that there is a position in
the local domain where one would expect the fronted constituent. This is indicated by the \_$_i$,
which is called a gap or a trace. The gap is related to the filler. The alternative to assuming such a gap is
to establish some dependency between the filler and the head on which the filler is dependent. This
is done in Dependency Grammar \citep{Hudson2000a} and in traceless approaches in HPSG
\citep*{BMS2001a} and LFG \citep{KZ89a}. The question is whether
it is reasonable to assume that even simple V2 sentences, that is sentences in which the filler does
not belong to a deeply embedded head, also involve a filler-gap dependency. Approaches that assume
that sentences like (\mex{1}a) are just a possible linearization variant of the verb and its
dependents will have problems in explaining the ambiguity of this sentence. (\mex{1}a) has two
readings, which correspond to the readings of (\mex{1}b) and (\mex{1}c):\todostefan{If reordering in
  the MF can have this effect, this argument is void. Vorfeldbesetzung then would be just a formal
  reordering. The only argument would then be uniformity.}
\eal
\ex\label{ex-oft-liest-er-das-buch-nicht} 
\gll Oft liest er das Buch nicht.\\
     often reads he the book not\\
\glt `It is often that he does not read the book.' or `It is not the case that he reads the book
often.'
\ex
\gll dass er das Buch nicht oft liest\\
     that he the book not often reads\\
\glt `It is not the case that he reads the book often.'
\ex
\gll dass er das Buch oft nicht liest\\
     that he the book often not reads\\
\glt `It is often that he does not read the book.'
\zl
If one assumes that there is a filler-gap dependency in (\mex{0}a), one can assume that the
dependency can be introduced before the negation is combined with the verb or after the
combination. This would immedeatly explain the two readings that exist for (\mex{0}a). Approaches
that assume that the order is a simple ordering variant of the involved constituents would predict
that (\mex{0}a) has the reading of (\mex{0}c) since (\mex{0}a) and (\mex{0}c) have the same order of
\emph{oft} `often' and \emph{nicht} `not' and the order is important for scope determination in German.

\subsection{Distribution of complementizer and finite verb}


\subsection{Verbal complexes}
\label{sec-vc-phen}

It is common to assume that verb and objects form a phrase in VO languages like English. However,
for languages like German, it seems more appropriate to assume that verbs in the right sentence
bracket form a verbal complex and that this verbal complex acts like one complex predicate when it
is combined with the nonverbal arguments. The following examples support this view. If one would
assume a structure like the one in (\mex{1}a), it is difficult to explain the ordering of
(\ref{ex-of}) because the auxiliary \emph{wird} `will' is located between two elements
of the verb phrase.

\eal
\ex
\label{ex-uf}
\gll dass Karl [[das Buch lesen] können] wird]\\
     that Karl \hspaceThis{[[}the book read can will\\
\glt `that Karl will be able to read the book'
\ex
\label{ex-of}
\gll dass Karl das Buch wird lesen können\\
     that Karl the book will read can\\
\glt `that Karl will be able to the read the book.'
\zl

Furthermore, the sentences in (\mex{1}) are not ruled out by such an analysis since \emph{das Buch
  lesen} `the book read' forms a phrase which would be predicted to be able to scramble left in the
middle-field as in (\mex{1}a) or appear in a so-called pied-piping construction with a relative
clause as in (\mex{1}b).
\eal
\ex[*]{
\gll dass [das Buch lesen] Karl wird\\
     that \spacebr{}the book read Karl will\\
}
\ex[*]{
\gll das Buch, [das lesen] Karl wird\\
	 the book \spacebr{}that read Karl will\\
}
\zl
%
\citet*{HN94a}\ia{Hinrichs}\ia{Nakazawa} therefore suggest that (certain) verbal complements are
saturated before non-verbal ones. This means that, in the analysis of (\ref{ex-uf}) and
(\ref{ex-of}), \emph{lesen} `to read' is first combined with \emph{können} `can' and the resulting
verbal complex is then combined with \emph{wird} `will':
\ea
\gll dass Karl das Buch [[lesen können] wird]\\
     that Karl the book \hspaceThis{[[}read can will\\
\z
\emph{wird} `will' can be placed to the right of the embedded verbal complex (as in {\mex{0})), or indeed to the left as
in \pref{ex-of}. After the construction of the verbal complex \emph{lesen können wird}, it is then combined with the 
arguments of the involved verbs, that is with \emph{Karl} and \emph{das Buch} `the book'.\footnote{%
		This kind of structure has already been suggested 
		by \citet*{Johnson86a} in connection with an analysis of partial verb phrase fronting.
}

There are also coordination data, such as the example in (\mex{1}), which support this kind of approach.
\ea
\gll Ich liebte ihn, und ich fühlte, daß er mich auch geliebt hat oder doch, daß er mich hätte lieben wollen oder lieben müssen.\footnotemark\\
     I   loved  him and I felt that he me also loved had or PRT that he me   would.have love want or love must.\\
\footnotetext{%
        \citep*[\page 36]{Hoberg81a}
}
\iw{müssen}
\glt `I loved him and felt that he loved me too, or at least he would have wanted to love me or would have had to.'
\z
If one assumes that modal verbs form a verbal complex, \emph{lieben wollen} `love want' and
\emph{lieben müssen} `love must' are constituents and as such they can be coordinated in a symmetric
coordination. The result of the coordination can then function as the argument of \emph{hätte}
`had'.

Arguments of the verbs that are part of a verbal complex may be scrambled as the following example
from \citet{Haider90b} shows:
\ea\label{ex-weil-es-ihm-jemand-zu-lesen-versprochen-hat}
\gll weil es ihm jemand zu lesen versprochen hat\\
     because it.\acc{} him.\dat{} somebody.\nom{} to read promised has\\
\glt `because somebody promised him to read it'
\z
\emph{jemand} `somebody' depends on \emph{hat} `has', \emph{ihm} `him' depends on \emph{versprochen}
`promised' and \emph{es} `it' depends on \emph{zu lesen} `to read'. In principle all six
permutations of these arguments are possible again and hence the verbal complex acts like a simplex
ditransitive verb.

\subsection{Partial verb phrase fronting}
\label{sec-pvp-phen}

The left peripheral elements of this verbal complex can (in some cases together with the adjacent material
from the middle field) be moved into the prefield:
\eal
\ex
\gll Gegeben hat er der Frau das Buch.\\
     given has he the woman the book\\
\glt `He gave the woman the book.'
\ex
\gll Das Buch gegeben hat er der Frau.\\
     the book given   has he the woman\\
\ex
\gll Der Frau gegeben hat er das Buch.\\
     the woman given  has he the book\\
\ex
\gll Der Frau  das Buch gegeben hat er.\\
     the woman the book given   has he\\
\zl
Since the verbal projections in (\mex{1}a--c) are partial, such frontings are called \emph{partial verb
phrase frontings}.



\section{The analysis}
\label{sec-analysis-v1-v2}

The following analysis uses Head-driven Phrase Structure Grammar (HPSG) as its main framework \citep{ps2}. It is, of course,
not possible to provide a comprehensive introduction to HPSG here, so a certain acquaintance with the general assumptions and mechanisms is
assumed for the following argumentation. The interested reader may refer to
\citew{MuellerLehrbuch3,MuellerHPSGHandbook} for introductions that are
compatible with what is presented here. In Section~\ref{sec-annahmen}, I will go over some basic assumptions to aid the understanding of the analysis, and
will also show how the relatively free ordering of constituents in the German \emph{Mittelfeld} can be analyzed. In Section~\ref{sec-v1}, I will recapitulate a 
verb-movement analysis for verb-first word orderings and in Section~\ref{sec-v2} I discuss the analysis of verb-second sentences. Section~\ref{sec-pred-compl}
will deal with the analysis of predicate complexes and the fronting of partial projections.  


\subsection{Background assumptions}
\label{sec-annahmen}
\label{sec-scrambling-analysis}

Every modern linguistic theory makes use of features in order to describe linguistic objects. In HPSG grammars, features are
systematically organized into `bundles'. These bundles correspond to certain characteristics of a linguistic object: for example, syntactic features
form one feature bundle, and semantic features form another. HPSG is a theory about linguistic signs in the sense of Saussure \citeyearpar{Saussure16a}\ia{Saussure}.
The modelled linguistic signs are pairs of form and meaning.

(\mex{1}) shows the feature geometry of signs that I will assume in the
following:
\ea
\ms[sign]
{ phonology   & \type{list~of~phoneme~strings}\\
  synsem & \onems[synsem]{ local \ms[local]{ category & \ms{ head   & \type{head} \\% maj & maj\/ \\} \\
                                                             spr    & \type{list of synsem-objects} \\ 
                                                             comps  & \type{list of synsem-objects} \\ 
                                                             arg-st & \type{list of synsem-objects} \\ 
                                                           } \\
                                             content & \type{cont} \\
                                           } \\
                            nonlocal  \type{nonloc} \\ 
                            lex \type{boolean}\\
             } \\
}
\z
The value of \textsc{phonology}\is{phonology} is a list of phonological forms.
Usually, the orthographic form is used to improve readability.

\textsc{synsem}\is{feature!synsem@\textsc{synsem}} contains syntactic and semantic information.
The feature \textsc{local}\is{feature!loc@\textsc{loc}} (\textsc{loc}) is called as such because syntactic and semantic
information in this path are those which are relevant in local contexts. In contrast,
there is, of course, also non-local information.
Such information is contained in the path \textsc{synsem$|$\-nonloc}\is{feature!nonloc@\textsc{nonloc}}. I will expand
on this in Section~\ref{sec-v2}.
%
Information about the syntactic category of a sign (\textsc{category})\is{feature!cat@\textsc{cat}} and information about its
semantic content (\textsc{content})\is{feature!cont@\textsc{cont}} are `local information'.
\textsc{head}\is{feature!head@\textsc{head}}, \spr, \comps,\isfeat{comps} and \argst\isfeat{arg-st} belong to the features which are
included in the path \textsc{synsem$|$\-loc$|$\-cat} in the feature description.
%
The value of \textsc{head} is a feature structure which specifies the syntactic characteristics that a certain lexical sign 
shares with its projections, that is, with phrasal signs whose head is the corresponding lexical sign.
%
The \textsc{arg-st} feature provides information about the argument structure\is{valence} of a particular sign. Its value is a list which 
includes the elements (possibly only partially specified) with which the sign has to be combined to produce a grammatically
complete phrase. The elements are mapped to valence features like \spr and \comps. I follow
\citet{Pollard90a} in assuming that finite verbs have all their arguments on the \compsl, that is,
there is no difference between subjects and complements as far as finite verbs are concerned. In SVO
languages like English and Danish, the subject is represented under \spr and all other arguments under \comps.

The \lexv has the value + with lexical signs and predicate complexes and $-$ with phrasal projections.\footnote{%
		\citet{Muysken82a} suggests a \textsc{min} feature in \xbart which corresponds to the
		\textsc{lex} feature. A \textsc{max} feature, in the way that Muysken uses it, is not needed since the
		maximality of a projection can be ascertained by the number of elements in its valence list: maximal 
		projections are completely saturated and therefore have empty valence lists.%
}
The lexical item in (\mex{1}) is an example of the finite form of the verb \emph{kennen} `to know'.

%\begin{figure}[htb]
\eas
Lexical item for \emph{kennt} `knows':\\
\label{le-kennt}
\ms[word]{
phon & \phonliste{ kennt }\\
synsem & \onems{ loc \ms{ cat & \ms{ head  & \ms[verb]{ vform & fin \\} \\
                                     spr   & \eliste\\
                                     comps & \sliste{ NP[\type{nom}]\ind{1}, NP[\type{acc}]\ind{2}  } \\
                                   } \\
                          cont & \onems{ hook|index \ibox{3}\\
                                         rels \liste{ \ms[kennen]{
                                                       arg0 & \ibox{3}\\
                                                       arg1 & \ibox{1}\\
                                                       arg2 & \ibox{2}\\
                                                       } }\\
                                    }\\
                        }\\
                 nonlocal \ms{ inherited$|$slash & \eliste \\
                               to-bind$|$slash & \eliste \\ 
                           } \\ 
                 lex $+$\\
             } \\
 }
\zs
%\vspace{-\baselineskip}\end{figure}
%
\emph{kennen} `to know' requires a subject (NP[\type{nom}]) and an accusative object (NP[\type{acc}]).
NP[\type{nom}] and NP[\type{acc}] are abbreviations for feature descriptions which are similar to
(\mex{0}). This requirement is represented on the \argstl, but since this list is identical to the
\compsl for finite verbs, it is not given here. It is in the lexical entry that the syntactic
information is linked to the semantic information. The subscript box on the NPs indicates the
referential index of that particular NP. This is identified with an argument role of the
\emph{kennen} relation. The semantic contribution of signs consist of an index and a list of
relations that are contributed by the sign. The index corresponds to a referential variable for
nouns and for an event variable for verbs. The referential index of a sign is usually linked to its
\textsc{arg0}. I assume Minimal Recursion Semantics (MRS; \citealp*{CFPS2005a}) as the format of the
representation of semantic information. This choice is not important for the analysis of the syntax
of the German clause that is discussed in this chapter and for the analysis of apparent multiple
frontings that is discussed in the following chapter. So the semantic representations are
abbreviated in the following. However, the semantic representation is important when it comes to the
representation of information structure and hence there will be a brief introduction to MRS in
Section~\ref{sec-intro-MRS}. 

Heads are combined with their required elements by means of a very general rule, which (when applied to the
conventions for writing phrase structure rules) can be represented as follows:
\ea
\label{h-c-regel}
H[\comps \ibox{1} $\oplus$ \ibox{3}] $\to$ H[\comps \ibox{1} $\oplus$ \sliste{ \ibox{2} } $\oplus$ \ibox{3} ]~~~ \ibox{2}
\z
The rule in (\mex{0}) combines an element \iboxb{2} from the \compsl of a head with the head itself.
The \compsl of the head is split into three lists using the relation append ($\oplus$), which splits
a list in two parts (or combines two lists into a new one). The first list is \ibox{1}, the second
list is the list containing \ibox{2} and the third list is \ibox{3}. If the \compsl of the head
contains just one element, \ibox{1} and \ibox{3} will be the empty list and since the \compsl of the mother is the concatenation of \ibox{1} and \ibox{3}, the \compsl of the mother node will be the empty list. The H in the rule stands for `Head'. Depending on which syntactic category a rule is instantiated by, the 
H can stand for noun, adjective, verb, preposition or another syntactic category. 
Figure~\vref{abb-weil-er-das-buch-kennt} is an example analysis for the sentence in (\mex{1}).\footnote{%
		In the following figures, H stands for `head', C for `complement', A for `adjunct', F for `filler'
		and CL for `cluster'.
}

\ea
\gll weil er das Buch kennt\\
	 because he the book knows\\
\glt `because he knows the book'
\z

\begin{figure}
\centering
\begin{forest}
sm edges
[V{[\textit{fin}, \comps \eliste]}
	[\ibox{1} NP{[\textit{nom}]}
		[er;he]]
	[V{[\textit{fin}, \comps \sliste{ \ibox{1} }]}
		[\ibox{2} NP{[\textit{acc}]}
			[das Buch;the book ,roof]]
		[V{[\textit{fin}, \comps \sliste{ \ibox{1}, \ibox{2} }]}
			[kennt;knows]]]]
\end{forest}
\caption{Analysis of \emph{weil er das Buch kennt} `because he knows the book'}\label{abb-weil-er-das-buch-kennt}
\end{figure}
Grammatical rules in HPSG are also described using feature descriptions. The rule in (\ref{h-c-regel})
corresponds to Schema~\ref{schema-bin}:

\begin{samepage}
\begin{schema}[Head-Complement Schema]
%~\\*
\label{schema-bin}
\type{head-complement-phrase} \impl\\
\ms{
      synsem & \onems{ loc$|$cat$|$comps \ibox{1} $\oplus$ \ibox{3}\\
                       lex $-$\\
                     }\\
      head-dtr & \onems{ synsem$|$loc$|$cat$|$comps \ibox{1} $\oplus$ \sliste{ \ibox{2} } $\oplus$ \ibox{3} \\
                       }\\
      non-head-dtrs & \liste{\onems{ synsem \ibox{2}  \\ }}\\[2mm]
}
\end{schema}
\end{samepage}
%
In this schema, the head daughter as well as the non-head daughters are represented as values of features  
(as value of \textsc{head-dtr} and as element in the list under \textsc{non-head-dtrs}). Since there are
also rules with more than one non-head daughters in HPSG grammars, the value of \textsc{non-head-dtrs} is a
list. The surface ordering of the daughters in signs licensed by these kinds of 
schemata is not in any sense determined by the schemata themselves. Special linearization rules, which
are factored out from the dominance schemata, ensure the correct serialization of
constituents. Therefore, Schema~\ref{schema-bin} allows both head-complement as well as complement-head orderings. 
The sequence in which the arguments are combined with their head is not specified by the schema. The
splitting of the lists with append allows the combination of any element of the \compsl with the head. The only condition for the possibility of combining a head and an complement is the adjacency of the
respective constituents. It is possible then to analyze (\mex{1}) using Schema~\ref{schema-bin}.   
\ea
\gll	weil das Buch jeder kennt\\
	because the book everyone knows\\
\glt	'because everyone knows the book'
\z
This is shown in Figure~\vref{abb-weil-das-buch-jeder-kennt}.
\begin{figure}
\centering
\begin{forest}
sm edges
[V{[\textit{fin}, \comps \eliste]}
	[\ibox{2} NP{[\textit{acc}]}
		[das Buch;the book, roof]]
	[V{[\textit{fin}, \comps \sliste{ \ibox{2} }]}
		[\ibox{1} NP{[\textit{nom}]}
			[jeder;everyone]]
		[V{[\textit{fin}, \comps \sliste{ \ibox{1}, \ibox{2} }]}
			[kennt;knows]]]]
\end{forest}
\caption{Analysis of \emph{weil das Buch jeder kennt} `because everybody knows the book'}\label{abb-weil-das-buch-jeder-kennt}
\end{figure}
\iboxt{1} and \ibox{3} can be lists containing elements or they can be the empty list. For languages
that do not allow for scrambling either \ibox{1} or \ibox{3} will always be the empty list. For
instance English and Danish combine the head with the complements in the order the elements are
given in the \compsl. Since \ibox{1} is assumed to be the empty list for such languages, Schema~\ref{schema-bin} delivers the right result. The nice effect of this analysis is that
languages that do not allow for scrambling have more constraints in their grammar (namely the additional
constraint that \ibox{1} = \eliste), while languages with less constrained constituent order have
fewer constraints in their grammar. This should be compared with movment"=based analyses where less
restrictive constituent order results in more complex analyses.

This analysis resembles Gunji's analysis for Japanese \citeyearpar{Gunji86a}. Gunji suggests the use of a set-valued
valence feature, which also results in a variable order of argument saturation. For a similar analysis in the terms
of the Minimalist Program, see \citew{Fanselow2001a}.  \citet[Section~3.1]{Hoffmann95a-u} and
\citet{SB2006a-u} suggest respective Categorial Grammar analyses.

In the lexical item for \emph{kennt} `knows' in (\ref{le-kennt}), the meaning of \emph{kennt} is represented as the
value of \textsc{cont}. The Semantics Principle \citep[\page 56]{ps2} ensures that, in Head"=Complement structures, the semantic contribution
of the head is identified with the semantic contribution of the mother. In this way, it is ensured that the meaning of \emph{er das Buch kennt}
is present on the highest node in Figure~\vref{abb-weil-er-das-buch-kennt-semp}. The association with the various arguments is already ensured by
the corresponding co-indexation in the lexical entry of the verb.\footnote{%
		The formula $kennen(er, buch)$ is a radical simplification. It is not possible to go into
		the semantic contribution of definite NPs or the analysis of quantifiers here. See
                \citew*{CFPS2005a} for an analysis of scope phenomena in Minimal Recursion Semantics.%
}


\begin{figure}
\centering
\begin{forest}
sm edges
[V{[\textsc{cont} \ibox{1}]}
	[NP{[\textit{nom}]}
		[er;he]]
	[V{[\textsc{cont} \ibox{1}]}
		[NP{[\textit{acc}]}
			[das Buch;the book, roof]]
		[V{[\textsc{cont}\,\ibox{1}\,kennen{(er, buch)}]}
			[kennt;knows]]]]
\end{forest}
\caption{Analysis of \emph{weil er das Buch kennt} `becaue he knows the book'}\label{abb-weil-er-das-buch-kennt-semp}
\end{figure}


After considering the syntactic and semantic analysis of Head"=Complement structures, I now turn to adjunct
structures.
%
Modifiers are treated as functors in HPSG. They select the head that they modify via the feature \textsc{mod}. The
adjunct can therefore determine the syntactic characteristics of the head that it modifies. Furthermore, it can access
the semantic content of the head and embed this under its own. The analysis of adjuncts will be made clearer by examining
the following example (\mex{1}):

\ea
\gll weil er das Buch nicht kennt\\
	 because he the book not knows\\
\glt `because he doesn't know the book'
\z
%
\emph{nicht} `not' modifies \emph{kennt} `knows' and embeds the relation $kennen(er, buch)$ under the
negation. The semantic contribution of \emph{nicht kennt} `not knows' is therefore $\neg kennen(er, buch)$.
The lexical entry for \emph{nicht} is shown in (\mex{1}).
\ea
Lexical entry for \emph{nicht} `not':\\
\ms{
cat & \ms{ head & \ms[adv]{ mod & \onems{ loc \onems{ cat$|$head \type{verb}\\
                                                      cont \ibox{1}\\
                                                    }\\
                                     }\\
                          }\\
           spr   & \eliste\\
           comps & \eliste\\
         }\\
cont & $\neg$ \ibox{1}\\
}
\z
This entry can modify a verb in head-adjunct structures which are licensed by Schema~\ref{ha-schema}.

\begin{samepage}
\begin{schema}[Head-Adjunct Schema]
\label{ha-schema}
\type{head-adjunct-phrase} \impl\\
\ms{
head-dtr      & \ms{ synsem & \ibox{2}\\
                   }\\[2mm]
non-head-dtrs & \liste{ \ms{ synsem$|$loc & \ms{ cat & \ms{ head$|$mod & \ibox{2}\\
                                                            spr & \eliste\\
                                                            comps   & \eliste\\
                                                          }\\
                                               }\\
                           }}\\
}
\end{schema}
\end{samepage}


Pollard and Sag's Semantics Principle ensures that the semantic content in head-adjunct structures 
is contributed by the adjunct daughter. Figure~\vref{abb-weil-er-das-buch-nicht-kennt-semp} shows
this analysis in detail.\todostefan{Add tree labels to all figures}
\begin{figure}
\centering
\begin{forest}
sm edges
[V{[\textsc{cont} \ibox{1}]}
	[NP{[\textit{nom}]}
		[er;he]]
	[V{[\textsc{cont} \ibox{1}]}
		[NP{[\textit{acc}]}, fit=band
			[das Buch;the book, roof]]
		[V{[\textsc{cont} \ibox{1}]}
			[Adv\feattab{\textsc{mod} \ibox{2} \textsc{cont} \ibox{3},\\
                                     \textsc{cont} \ibox{1} $\neg$ \ibox{3}}
				[nicht;not]]
			[\ibox{2} V{[\textsc{cont} \ibox{3} kennen{(er, buch)}]}
				[kennt;know]]]]]
\end{forest}
\caption{Analysis of \emph{weil er das Buch nicht kennt} `because he does not know the book'}\label{abb-weil-er-das-buch-nicht-kennt-semp}
\end{figure}

The \textsc{mod} value of the adjunct and the \synsemv of the verb are co-indexed by the Head"=Adjunct Schema \iboxb{2}.
Inside the lexical entry for \emph{nicht}, the \contv of the modified verb (\iboxt{3} in Figure~\ref{abb-weil-er-das-buch-nicht-kennt-semp})
is co-indexed with the argument of $\neg$. The semantic content of \emph{nicht} (\ibox{1} $\neg kennen(er, buch)$) becomes the semantic
content of the entire Head"=Adjunct structure and is passed along the head path until it reaches the highest node.

After this recapitulation of some basic assumptions, the following section will present a verb-movement analysis for 
verb-initial word order in German.

\subsection{V1}
\label{sec-v1}


As is common practice in Transformational Grammar and its successive models (\citealp*[\page 34]{Bierwisch63}; \citealp{Bach62a}; \citealp{Reis74a}; 
\citealp[Chapter~1]{Thiersch78a}), I will assume that verb-first sentences have a structure that is
parallel to the one of verb-final sentences and that an empty element fills the position occupied by
the verb in verb-last sentences.\footnote{%
The alternative is that they are flat structures, which allow the verb to be positioned in both initial
and final position \citep{Uszkoreit87a,Pollard90a}, or linearization analyses 
\citep{Reape92a,Reape94a,Mueller99a,Mueller2002b,Kathol95a,Kathol2000a}. In linearization analyses, the domain
in which constituents can be permuatated is expanded so that, despite being a binary branching structure, verb-first
and verb-final orderings can be derived. 
The differing possibilities will be discussed further in Chapter~\ref{chap-alternatives}.%
} 

A radically simplified variant of the transformational analysis of (\mex{1}b) is presented in Figure~\vref{fig-verb-movement-gb}.
\eal
\ex 
\gll dass er das Buch kennt\\
     that he the book knows\\
\glt `that he knows the book'\label{bsp-dass-er-das-buch-kennt}
\ex 
\gll Kennt$_i$ er das Buch \_$_i$?\\
     knows{} he the book\\
\glt `Does he know the book?'\label{bsp-kennt-er-das-buch}
\zl
\begin{figure}
\centering
\begin{forest}
sm edges
[CP
	[\cnull
		[kennt;knows]]
	[VP
		[NP
			[er;he]]
		[VP
			[NP
				[das Buch;the book, roof]]
			[\vnull
				[\trace]]]]]
\end{forest}
\caption{\label{fig-verb-movement-gb}Analysis of \emph{Kennt er das Buch?} `Does he know the book?' with Move-$\alpha$}
\end{figure}


The verb is moved from verb-final position to C$^0$.\footnote{%
  In more recent analyses the verb is adjoined to C$^0$. While V-to-c-movement analyses work well for German and
  Dutch\il{Dutch} they fail for other V2 languages that allow for the combination of complementizers
  with V2 clauses \citep{Fanselow2009b}. This will be discussed in more detail in
  Subsection~\ref{sec-v2-germanic}.
} This movement can be viewed as creating a new tree structure out
of another, i.e. as a derivation. In the analysis of (\mex{0}b), two trees enter a relation with each other $-$ the
tree with verb-final ordering and the tree where the verb was moved into first position. One can alternatively assume a 
representational model where the original positions of elements are marked by traces (see %\citew{McCawley68a}; Pullum2007a:3 sagt, dass das nicht MTS ist
Koster \citeyear[\page ]{Koster78b-u}; \citeyear[\page 235]{Koster87a-u}; 
%\citealp[\page 66, Fußnote~4]{Bierwisch83a}; 
\citealp{KT91a}; \citealp[Section~1.4]{Haider93a}; 
\citealp[\page 14]{Frey93a}; \citealp[\page 87--88, 177--178]{Lohnstein93a-u}; \citealp[\page 38]{FC94a}; \citealp[\page 58]{Veenstra98a}, for example). This kind of representational view 
is also assumed in HPSG. In HPSG analyses, verb-movement is modeled by a verb-trace in final position coupled with the percolation of
the properties of the verb trace in the syntactic tree. 

In what follows, I discuss another option for modeling verb-movement.
The C-head in Figure~\ref{fig-verb-movement-gb} has different syntactic characteristics from
V$^0$ in verb-final orders: the valence of the verb in final position does not correspond to
the valence of the element in C. The functional head in C is combined with a VP (an IP in several
works), whereas the verb in final structures requires a subject and an object. 
In HPSG, the connection between the element in V1-position and the actual verb can be captured by an analysis which
assumes that there is a verb trace in verb-initial structures that has the same valence properties and the same semantic
contribution as an overt finite verb in final position and is also present in the same
position.\footnote{%
  In the grammar developed in this book, it is impossible to say that a head follows or precedes its
  dependents if the head is empty. The reason is that the head daughter and the non-head daughters
  are the values of different features: the head daughter is the value of \textsc{head-dtr} and the
  non-head daughters are members of the \textsc{non-head-dtrs} list. It is only the \phonvs of the
  daughters that are serialized \citep{Hoehle94a}. So in a structure like [NP$_1$ [NP$_2$ t]] one cannot tell whether
  NP$_2$ precedes t or follows it since in the AVM these two objects are just presented on top of each
  other and the phonology does not show any reflex of t that would help us to infer the order. Note
  however that t has the \initialv `$-$' and hence the phonology of t is appended to the end of the
  phonology of NP$_2$. It does not matter whether we append the empty string at the end or at the
  beginning of a list, but the \initialv of the head matters when NP$_1$ is combined with [NP$_2$ t]:
  the complex phrase [NP$_2$ t] has to be serialized to the right of NP$_1$. If both NP$_1$ and
  NP$_2$ contain phonological material, the material contributed by NP$_1$ will precede the material
  from NP$_2$. So, we will always know that the trace is in a unit that contains other material and
  this unit is serialized as if there would be a visible head in it. This means that despite Höhle's
  claim to the contrary traces can (roughly) be localized in structures.

  Note that \citet{GSag2000a-u} represent both head and non-head daughters in the same list. If one
  assumes that this list is ordered according to the surface order of the constituents, traces are
  linearized.

  Traces will be shown in final position in the tree visualizations throughout this book.
}
The element in intial-position
is licensed by a lexical rule, which licenses a verb that takes the initial position and selects for
a projection of the verb trace. To make this clearer, we will take a closer look at the sentence in (\ref{bsp-kennt-er-das-buch}):
the syntactic aspects of the analysis of (\ref{bsp-kennt-er-das-buch}) are shown in Figure~\vref{fig-verb-movement-syn}.
\begin{figure}
\centering
\begin{forest}
sm edges
[V{[\comps \eliste]}
	[V{[\comps \sliste{ \ibox{1} }]}
		[V{[\textsc{comps \ibox{2}}]}, tier=np,edge label={node[midway,right]{V1-LR}}
			[kennt;knows]]]
	[\ibox{1} V\feattab{\textsc{dsl}|\textsc{cat}|\comps \ibox{2},\\
                            \comps \eliste}
		[\ibox{3} NP{[\textit{nom}]}, tier=np
			[er;he]]
		[V\feattab{\textsc{dsl}|\textsc{cat}|\comps \ibox{2},\\
                           \comps \sliste{ \ibox{3} }}
			[\ibox{4} NP{[\textit{acc}]}
				[das Buch;the book, roof]]
			[V\feattab{\textsc{dsl}|\textsc{cat}|\comps \ibox{2},\\
                                   \comps \ibox{2} \sliste{ \ibox{3}, \ibox{4} }}
				[\trace]]]]]
\end{forest}
\caption{\label{fig-verb-movement-syn}Analysis of \emph{Kennt er das Buch?} `Does he know the book?'}
\end{figure}
%
In the verb trace, the \compsv of the trace is co-indexed with the value of the \compsf under
\textsc{dsl}. The feature \dsl was introduced by \citet*{Jacobson87} with the aim of describing head movement in inversion
structures in English. \dsl stands for \emph{double slash} and is sometimes abbreviated as `//' in figures.\footnote{%
  In addition to \dsl, there is a \slasch feature that is used for the analysis of nonlocal
  dependencies. This will be explained in Section~\ref{sec-v2}.
} \citet{Borsley89} adopted Jacobson's idea and translated it into HPSG terms thereby 
showing how head movement in a HPSG variant of the CP/IP system can be modeled using
\textsc{dsl}. The introduction of such a feature to HPSG in order to describe movement operations is
motivated by the fact that this kind of movement is local, unlike the long-distance dependencies discussed in Section~\ref{sec-v2}.

The verb trace in (\mex{1}) takes on the role of the finite verb in the analysis of
(\ref{bsp-dass-er-das-buch-kennt}).\footnote{%
  The \spr feature is ignored here. As will become clear later, the \sprv of the trace and the
  \dslf are also shared. The \sprv of finite verbs is always the empty list in German and hence the
  \sprv of the trace is the empty list as well.
}
\eas
Verb trace (valence information):\\
\label{le-verbspur}
\onems{
phon \eliste\\
synsem$|$loc$|$cat \ms{ head & \ms[verb]{
                                           dsl$|$cat$|$comps & \ibox{1}\\
                                           }\\
                                    comps & \ibox{1}
                                  }\\
}
\zs


Since \textsc{dsl} is a head feature, it is passed on towards the top of the tree so that information
about the valence of the verb trace is present at each projection.
A special version of the finite verb takes the projection of the verb trace (\iboxb{1} in Figure~\vref{fig-verb-movement-syn}) as its argument. As they are combined,
it is checked whether the valence of the original verb \iboxb{2} matches the valence of the verb
trace ({\textsc{dsl$|$cat$|$comps} \ibox{2}\,).

The special lexical item for V1-ordering is licensed by the following lexical rule:\footnote{\label{fn-koord-vm}%
		I am adopting a view which integrates lexical rules into the formalism of HPSG and treats them as
		unary rules \citep{Meurers2001a}.
		Lexical rules are applied to stems or entire words \citep{Mueller2002b}.
		Verb-movement will -- as in previous publications about verb-movement in HPSG -- be described using 
		lexical rules. The following data suggests, however, that it is appropriate to speak of unary syntactic rules rather than lexical rules:
		 \ea
        \gll Karl kennt und schätzt diesen Mann.\\
             Karl knows and values this man\\
	\glt `Karl knows and values this man.'
        \z
		(i) cannot be analyzed applying the verb-movement rule to each verb individually and then coordinating the
		result, since \emph{kennen} `to know' and \emph{schätzen} `to value' have different \contvs. The \contv
		of the verb trace is determined by the \contv of the verb in initial position. The coordination of two products of
		a lexical rule for verb-movement would not be allowed as the standard coordination theory of \citet[\page 202]{ps2} states
		the valence requirements of both conjuncts be the same. Such a problem does not arise, however, if we apply a unary syntactic 
		rule (parallel to (\ref{lr-verb-movement2})) to the result of the coordination.%
}


\ea
\begin{tabular}[t]{@{}l@{}}
\label{lr-verb-movement}
Lexical rule for verb in initial position (valence information):\\
\ms{
synsem$|$loc & \ibox{1} \ms{ cat$|$head & \ms[verb]{ vform & fin\\
                                                     initial & $-$\\
                                             }\\
                  }\\
} $\mapsto$\\
\ms{
synsem$|$loc$|$cat & \ms{ head & \ms[verb]{vform & fin\\
                                           initial & $+$\\
                                             }\\
                           spr & \eliste\\
                           comps & \sliste{ \ms{ loc$|$cat \ms{ head  \ms[verb]{
                                                               dsl & \ibox{1}\\
                                                               }\\
%                                                         spr   \eliste\\
                                                                                  %SPR ist leer,
                                                                                  %weil die finiten
                                                                                  %Verben immer
                                                                                  %leere SPR-Werte
                                                                                  %haben, deshalb
                                                                                  %kann auch über
                                                                                  %den Trace nie
                                                                                  %etwas anderes
                                                                                  %projiziert werden
                                                                                  %und wir müssen
                                                                                  %hier nichts
                                                                                  %sagen. Oder? 08.06.2015
                                                         comps \eliste\\
                                                       }\\
                                              }}\\
                         }\\
}
\end{tabular}
\z

The verb licensed by this lexical rule selects the maximal projection of the verb trace which has
the same local properties as the input verb.\footnote{%
  In principle one would have to specify the \sprv of the selected argument to be the empty
  list. However, since the \sprv of the trace is identical to the \sprv of the fronted verb and
  since fronted verbs are always finite and since finite verbs have the empty list as the \sprv, the
  \sprv of the complement may be left unspecified in the lexical rule. This is different for Danish
  and in the Danish equivalent of the lexical rule the \sprv has to be specified.
}
This is achieved by co-indexing the \localv of the
input verb and the \dslv of the selected verbal projection. Only finite verbs in final position (\textsc{initial}$-$) 
can function as an input for this rule. The output is a verb in initial position (\textsc{initial}+).
Linearization rules make reference to the \textsc{initial} feature and ensure the correct ordering of heads in 
local trees.


Nothing has been said about semantics so far. It is assumed that the verb trace also shares the
semantic properties of the verb in initial position and that verb-initial clauses are interpreted
like their verb-final counterparts (see the discussion of (\ref{bsp-absichtlich-nicht-anal-v1}) on
page~\pageref{bsp-absichtlich-nicht-anal-v1}). This can be modeled by threading the semantic
contribution in parallel with the valence properties through the tree. (\ref{le-verbspur2-prelim})
shows the verb trace enriched with semantic information: 

\eas
\label{le-verbspur2-prelim}
Verb trace (Valence information and semantic content):\\
\onems{
phon \eliste\\
synsem$|$loc \ms{ cat & \ms{ head & \ms[verb]{
                                           dsl & \ms{ cat & \ms{ comps & \ibox{1}\\
                                                               }\\
                                                      cont & \ibox{2}\\
                                                   }\\
                                           }\\
                                    comps & \ibox{1}\\
                            }\\
                  cont & \ibox{2}\\
                }\\
}
\zs


By co-indexing the \contvs, the trace behaves semantically just like the original verb, which is
now in initial position.

If one allows cyclic feature structures, (\mex{0}) can be represented in a more compact manner
as in (\mex{1}) \citep[\page 207]{Meurers2000b}:  
\eas
\label{le-verbspur2}
Verb trace according to \citew[\page 207]{Meurers2000b}:\\
\onems{
phon \eliste\\
synsem$|$loc \ibox{1} \onems{ cat$|$head$|$dsl \ibox{1} }\\
}
\zs
The fact that all \local properties of a verb trace are represented under \textsc{dsl} is captured much
more directly here. It is no longer necessary to have separate structural sharings or explicitly mention individual types and features
under \textsc{head} (as in (\ref{le-verbspur2-prelim})).

The Semantics Principle ensures that the \contv is passed along the head projection during the combination of arguments towards the top of the tree. 
In the last step of the projection in Figure~\ref{fig-verb-movement-syn}, the verb in initial position is the head and therefore the semantic
content of this verb will be projected. In the lexical rule (\mex{1}) for the verb in initial position, 
the semantic content of the projection of the trace in final position \iboxb{2} is identified with the 
\contv of the verb in initial position.


\ea
\begin{tabular}[t]{@{}l@{}}
Lexical rule for verbs in initial position (valence und semantic\\
contribution):\\
\ms{
synsem$|$loc & \ibox{1} \ms{ cat$|$head & \ms[verb]{ vform & fin\\
                                                     initial & $-$\\
                                             }\\
                  }\\
} $\mapsto$\\*
\onems{
synsem$|$loc \onems{ cat  \ms{ head & \ms[verb]{ vform & fin\\
                                                     initial & $+$\\
                                             }\\
                               spr & \eliste\\
                           comps & \sliste{ \onems{ loc \onems{ cat \ms{ head & \ms[verb]{
                                                                                     dsl & \ibox{1}\\
                                                                                 }\\
                                                                         spr & \eliste\\
                                                                         comps & \eliste\\
                                                                    }\\
                                                           cont \ibox{2}\\
                                                         }\\
                                              }}\\
                         }\\
                   cont \ibox{2}\\
             }\\
}
\end{tabular}
\label{lr-verb-movement2}
\z


Due to this combination, the semantic content of the verb trace projection is then taken over by the verb 
in initial position and, as per the Semantics Principle, becomes the semantic contribution of the entire 
construction. Figure~\vref{fig-verb-movement-sem} shows the semantic aspects of the verb-movement analysis with the
trace in (\ref{le-verbspur2}) and the lexical rule in (\ref{lr-verb-movement2}).

\begin{figure}
\centering
\begin{forest}
sm edges
[V{[\textsc{cont} \ibox{2}\,]}
	[V{[\textsc{cont} \ibox{2}\,]}
		[V{[\textsc{cont} \ibox{1} kennen(er, buch)]}, tier=np,edge label={node[midway,right]{V1-LR}}
			[kennt;knows]]]
	[V\feattab{\textsc{dsl}|\textsc{cont} \ibox{1},\\
                   \textsc{cont} \ibox{2}\,}
		[NP{[\textit{nom}]}, tier=np
			[er;he]]
		[V\feattab{\textsc{dsl}|\textsc{cont} \ibox{1},\\
                           \textsc{cont} \ibox{2}\,}
			[NP{[\textit{acc}]}
				[das Buch;the book, roof]]
			[V\feattab{\textsc{dsl}|\textsc{cont} \ibox{1},\\
                                   \textsc{cont} \ibox{2}$=$\ibox{1}\,}
				[\trace]]]]]
\end{forest}
\caption{\label{fig-verb-movement-sem}Analysis of \emph{Kennt er das Buch?} `Does he know the book?'}
\end{figure}


Technically speaking, \iboxt{1} and \iboxt{2} in Figure~\ref{fig-verb-movement-sem} are identical. To make aid 
representation, they have been represented by different numbers. The identification of \ibox{1} and
\ibox{2} is enforced by the identification of the information under \textsc{local} and \textsc{dsl} in the
lexical entry of the trace (\ref{le-verbspur2}),
as \cont is a \textsc{local} feature.


The analysis in Figure~\ref{fig-verb-movement-sem} may seem somewhat complicated, since semantic information is passed on both
via the \dsl from the verb in initial position to the trace \iboxb{1} and by the verb trace to the verb in
initial position \iboxb{2}. However, once we consider examples with adjuncts, it will become clear that this seemingly complicated treatment is justified. The analysis of (\mex{1}) is given in 
Figure~\vref{fig-verb-movement-adjunkt-sem}.
\ea
\gll Kennt er das Buch nicht?\\
	 knows he the book not\\
\glt `Doesn't he know the book?'
\z
\begin{figure}
\oneline{%
\begin{forest}
sm edges
[V{[\textsc{cont} \ibox{1}\,]}
	[V{[\textsc{cont} \ibox{1}\,]}
		[V{[\textsc{cont} \ibox{2} kennen(er, buch)]}, tier=np,edge label={node[midway,right]{V1-LR}}
			[kennt;knows]]]
	[V\feattab{\textsc{dsl}|\textsc{cont} \ibox{2},\\
                   \textsc{cont} \ibox{1}\,}
		[NP{[\textit{nom}]}, tier=np
			[er;he]]
		[V\feattab{\textsc{dsl}|\textsc{cont} \ibox{2},\\ 
                           \textsc{cont} \ibox{1}\,}
			[NP{[\textit{acc}]}, fit=band
				[das Buch;the book, roof]]
			[V\feattab{\textsc{dsl}|\textsc{cont} \ibox{2},\\
                                   \textsc{cont} \ibox{1}\,}
				[Adv\feattab{\textsc{mod} \ibox{3} {[\textsc{loc}|\textsc{cont} \ibox{2}\,]},\\
                                             \textsc{cont} \ibox{1} $\neg$ \ibox{2}}
					[nicht;not]]
				[\ibox{3} V\feattab{\textsc{dsl}|\textsc{cont} \ibox{2},\\
                                                    \textsc{cont} \ibox{2}\,}
					[\trace]]]]]]
\end{forest}
}
\caption{Analysis of \emph{Kennt er das Buch nicht?} `Doesn't he
  know the book?'}\label{fig-verb-movement-adjunkt-sem}
\end{figure}


The initial verb \emph{kennt}, which is licensed by a lexical rule, requires a verbal projection
with a \textsc{dsl$|$cont} value of
$kennen(x, y)$, where the $x$ and $y$ in the lexicon entries for \emph{kennt} are already linked to arguments, which will later be
filled by \emph{er} and \emph{das Buch}. The \textsc{dsl$|$cont} value of the verbal projection is -- due to \dsl being a head feature -- also
restricted at the trace. At the trace, the \contv{} is co-indexed with \textsc{dsl$|$cont} value so that the trace has the same semantic representation
as the verb \emph{kennt}, which was the input for the verb-first lexical rule. The verb trace is then modified by the adjunct \emph{nicht} 
and the meaning of the head"=adjunct structure is passed up to the mother node \iboxb{1}. During the combination with its arguments, the 
meaning is then transmitted up to the maximal projection of the verb trace in Figure~\ref{fig-verb-movement-adjunkt-sem}. The \contv of this
projection is identical to the \contv of the initial verb due to the structure sharing in the lexical item for this verb, which is licensed by the lexical 
rule (\ref{lr-verb-movement2}). Because the verb in first position is the head of the entire structure and it is a head"=argument structure, the semantic
content of the structure is identical to that of the V1-verb, i.e. \iboxt{1} in Figure~\ref{fig-verb-movement-adjunkt-sem}.

Finally, sentences such as (\mex{1}) must be somehow ruled out:
\ea[*]{
\gll Kennt er das Buch kennt.\\
     knows he the book know\\
}
\z


(\mex{0}) could be analyzed in such a way that the first occurrence of \emph{kennt} is the ouput of a verb-movement rule
and the \textsc{dsl} value of the second \emph{kennt} is unrestricted, so that the second \emph{kennt} can take over the 
same role as the verb trace in our analysis. Generally speaking, it is not possible for all overtly realised verbs to demand that
their  \textsc{dsl} value be \emph{none} since these verbs represent the input for the lexical rule for verb movement and
the \textsc{local} value of the input verb is identified with the \textsc{dsl} value of the verb trace selected by the output verb.
If all overt verbs had the \textsc{dsl} value \emph{none}, it would lead to a contradiction during the combination with the verb
trace since the trace has a specified \textsc{dsl} value (the trace is cyclic, therefore the value of \textsc{loc$|$cat$|$\-head$|$\-dsl$|$\-cat$|$\-head$|$\-dsl}
is not compatible with \emph{none}).
(\mex{0}) is excluded by a restriction which states that a verb has to have the \textsc{dsl} value \emph{none} when it is
overtly realised and enters a syntactic structure. The desired result is achieved by the implication in (\mex{1}):

\ea
\ms{ head-dtr & \ms[word]{ phon & non-empty-list \\
                         }
} \impl \onems{ synsem$|$l$|$cat$|$head$|$dsl \type{none}\\
            }
\z
%
This restriction differs from that of \citet[\page 207]{Meurers2000b} and others in that the \textsc{head-daughter} in the antedecent must be of the type \emph{word}. Without this restriction, the
constraint could be applied to projections of the verb trace and thereby exclude well-formed sentences.

Following the discussion of the analysis of verb-first sentences, the next section focuses on the analysis of verb-second
sentences.


\subsection{V2}
\label{sec-v2}


Verb-second sentences such as (\mex{1}b) are, as we have already mentioned, related to verb-first sentences such as (\mex{1}a) 
in most German grammars.\footnote{%
		 \citet[Chapter~6.3]{Kathol95a}, \citet{GO2009a}, and \citet{Wetta2011a} are exceptions. These authors analyze
		 short fronting as in (\mex{1}b) as an alternative ordering for 
		 the constituents in (\mex{1}a). They do, however, assume long-distance dependencies 
		 for sentences such as (\ref{bsp-um-zwei-millionen}). \citet{Kathol2001a} revised
                 his treatment and now assumes a uniform analysis of V2 phenomena in German. 

                 Approaches that treat local frontings different are discussed in more detail in Section~\ref{sec-local-frontings-alternatives}.
}

\eal
\ex
\gll Kennt er das Buch?\\
	 knows he the book\\
\glt `Does he know the book?'
\ex
\gll Das Buch$_i$ kennt \_$_i$ er.\label{ex-das-buch-kennt}\\
	 the book knows {}     he\\
\glt `He knows the book'	 
\zl\todostefan{position trace / Frey/Fanselow}

In the second example, \emph{das Buch} is situated in the prefield. The position in the middle field, where
the object could also occur, is empty. This position is most often represented by `\_'. (\mex{1}) shows that
elements  which are dependent on an embedded head can occur in the pre-field:

\ea
\label{bsp-um-zwei-millionen}
\gll{}[Um zwei Millionen Mark]$_i$ soll er versucht haben, [eine Versicherung \_$_i$ zu betrügen].\\
    \spacebr{}of two million Deutschmarks{} should he tried have \spacebr{}an insurance.company {} to defraud\\\footnote{%
         taz, 04.05.2001, \page 20
}
\glt `He supposedly tried to defraud an insurance company of two million Deutschmarks'
\z

Therefore, occupying the pre-field (fronting) creates a long-distance dependency. 
To deal with long-distance dependencies, \citet[Chapter\,4]{ps2}\ia{Pollard}\ia{Sag} suggest 
a silent element which introduces a non-local dependency: \footnote{%
		In Chapter~9, \citet{ps2} introduce a lexical rule for complement extraction.
		It is possible to describe long-distance dependencies with this rule and avoid using
		a phonologically null element. A further alternative would be unary projections, as I suggest in
		\citep*[Chapter~9, 10, 18]{Mueller99a}. A discussion of the alternatives can be found in 
		\citep[Chapter~6.2.5.1]{Mueller2002b} and in Chapter~\ref{chap-empty} of this book. In more recent works in HPSG, relational
                argument realization principles and lexical analyses are assumed for extraction
                \citep*{BMS2001a}. See \citew{LH2006a} for a discussion of such relational approaches.
		
		For phenomena such as relative and interrogative clauses, one needs the features \textsc{rel}
		and \textsc{que} in addition to \slasch. These features are omitted in what follows.
}

\eas
\label{trace}
Trace for the description of long-distance dependencies:\\
%\ms[lexical-sign]{
\ms{
 phon & \phonliste{} \\
 synsem & \ms{ local & \ibox{1} \\
               nonlocal & \ms{ inherited & \ms{ slash & \sliste{\ibox{1}} \\
                                               } \\
                               to-bind   & \ms{ slash & \eliste \\ 
                                              } \\ 
                              } \\ 
             } \\
}
\zs


This kind of trace can stand for a complement or an adjunct depending on the context.
The characteristics of the object, which are represented under \textsc{synsem$|$"-local}, are
entered into the \textsc{slash} list under \textsc{synsem$|$"-nonlocal$|$"-inherited$|$"-slash}.
The \textsc{nonloc} Principle ensures the percolation of non-local features from the daughter nodes
of complex signs to their mother nodes.

\begin{principle}[Nonlocal Feature Principle] 
\label{nonloc-prinzip}
$ \\ $
The \textsc{non\-loc$|$\-inherited} value of a phrasal sign is the union of the
\textsc{non\-loc$|$\-in\-her\-ited} values of its daughters minus the \textsc{non\-loc$|$\-to-bind} value of the
daughter of the head.
\end{principle}%

A \textsc{slash} element can be bound off by the Filler"=Head Schema.

\begin{samepage}
\begin{schema}[Filler-Head Schema (for German)]
\label{hf-schema}
~\\
\type{head-filler-phrase} \impl\\\ms{ 
head-dtr$|$synsem & \onems{ local \ms{ cat & \ms{ head & \ms[verb]{vform & fin\\
                                                                                     initial & $+$\\
                                                                                    }\\
                                                  spr   & \eliste{} \\
                                                  comps & \eliste\\
                                                }\\
                                     }\\
                            nonloc \ms{ inher$|$slash   & \sliste{ \ibox{1} }\\[2mm]
                                        to-bind$|$slash & \sliste{ \ibox{1} }\\
                                      }\\
                          }\\
non-head-dtrs & \sliste{ \ms{ synsem & \onems{ local \ibox{1}\\
                                               nonloc$|$inher$|$slash \eliste \\
                                             } \\
                                 }}\\
   }
\end{schema}\is{schema!head filler}
\end{samepage}
%
This schema describes structures in which finite clauses with the verb in initial position (\textsc{initial}+)
and with an element in \textsc{inher$|$slash} (\tbox{1}) are combined with a phrase with matching \textsc{local} properties.
In example (\ref{ex-das-buch-kennt}), \emph{kennt er} `knows he' is the finite clause with the corresponding element in \textsc{slash}
and \emph{das Buch} `the book' is the filler. Figure~\vref{abb-das-buch-kennt} shows the analysis for (\ref{ex-das-buch-kennt}).
\begin{figure}
\oneline{%
\begin{forest}
sm edges
[V\feattab{\comps \eliste,\\ 
           \textsc{inh}|\textsc{slash} \eliste}
	[NP \ibox{1} {[\textit{acc}]}, fit=band
		[das Buch;the book, roof]]
	[V\feattab{\comps \eliste,\\
                   \textsc{inh}|\textsc{slash} \sliste{ \ibox{1} },\\
                   \textsc{to-bind}|\textsc{slash} \sliste{ \ibox{1} }}
		[V{[\comps \sliste{ \ibox{2} }]}
			[V{[\comps \sliste{ \ibox{3}, \ibox{4} }]}, tier=trace,edge label={node[midway,right]{V1-LR}}
				[kennt;knows]]]
		[\ibox{2} V\feattab{\comps \eliste,\\
                                    \textsc{inh}|\textsc{slash} \sliste{ \ibox{1} },\\
                                    \textsc{to-bind}|\textsc{slash} \eliste}
			[\ibox{4} \feattab{\textsc{local} \ibox{1},\\
                                                  \textsc{inh}|\textsc{slash} \sliste{ \ibox{1} }},tier=trace
					[\trace]]
			[V\feattab{\comps \sliste{ \ibox{3} },\\
                                   \textsc{inh}|\textsc{slash} \sliste{ \ibox{1} },\\
                                   \textsc{to-bind}|\textsc{slash} \eliste}
				[\ibox{3} NP{[\textit{nom}]}
				[er;he]]
				[V{[\comps \sliste{ \ibox{3}, \ibox{4} }]}
					[\trace]]]]]]
\end{forest}
}
\caption{Analysis of \emph{Das Buch kennt er.} `He knew the book.'}\label{abb-das-buch-kennt}
\end{figure}


The verb movement trace for \emph{kennt} `knows' is combined with an extraction trace. 
The extraction trace in the example is the accusative object. The accusative object 
is described in the \compsl of the verb and the information about the properties
of the required NP are at the same time present in the extraction trace under \textsc{loc}
and \textsc{inher$|$slash}. The \textsc{slash} information is passed up the tree until it reaches
the point where the projection is combined with a filler (F). The Head-Filler Schema instantiates
the \textsc{to-bind$|$slash} value of the head daughter. The Nonlocal Feature Principle then comes into
play to cause the binding off of the \slashv, which percolated from the extraction trace, that is, the \slashv is no longer passed
along up the tree. The Head"=Filler Schema then makes sure that the filler daughter (the non-head daughter
in the schema) has exactly the same \textsc{loc} value as the extraction trace. It is only the accusative nominal
phrase which is a possible candidate for a filler in our example.


It is worth noting that Schema~\ref{hf-schema} does not say anything about the valence of the filler
daughter. The form of the filler daughter is only restricted by the specification of the properties of complements of lexical 
heads. Therefore, non-maximal projections are also licensed as fillers in long-distance dependencies by our schema. The theory
presented here does not correspond to the rules of \xbart \citep*{Jackendoff77a}. This is however not necessarily a negative
point, since \xbart does not restrict the generative capacity of grammars in any way as soon as
empty elements are permitted \citep{Pullum85a,KP90a}. The fact that non"=maximal projections are possible in sentence-initial position plays a central role
for the analysis of partial verb phrase fronting presented in the following section and also for the analysis of putative
multiple fronting, which are discussed in Chapter~\ref{chapter-mult-front}. 


In the following, I will present the analysis proposed by \citew{HN94a} for predicate complexes as well as the analysis of fronting of partial
constituents based on \citew{Mueller97c,Mueller99a,Mueller2002b, Meurers99a}. These analyses have become established within the
HPSG paradigm and alternative HPSG analyses will not be discussed here.
For such a discussion, the reader is referred to \citew[Chapter~18.3]{Mueller99a} and \citew[Chapter~2.3]{Mueller2002b}.



% \citet{HN89} haben auf der Grundlage von Voranstellungs- (\mex{1}) und Oberfeldumstellungsdaten (\mex{2}) dafür
% argumentiert, daß Hilfs- und Modalverben im Deutschen mit dem Hauptverb einen Verbalkomplex bilden.
% \ea
% Geholfen haben wird er dem Mann.
% \z
% % mehr
% %Since German is assumed to be a verb second language, i.e., a language with exactly one constituent before the
% %finite verb, examples like (\mex{0}) are evidence for the existence of the constituent \emph{geholfen haben}.
% \eal
% \ex
% daß er dem Mann helfen müssen wird.
% \ex
% daß er dem Mann wird helfen müssen.
% \zl
% Die Beispiele in (\mex{0}) kann man leicht erklären, wenn man annimmt, daß
% \emph{helfen} und \emph{müssen} einen Komplex bilden, der dann unter \emph{wird} eingebettet wird.
% \emph{wird} kann entweder links oder rechts des eingebetteten Verbalkomplexes stehen.
%
% In Hinrich und Nakazawas analyze bilden auch \emph{geholfen} und das Hilfsverb
% \emph{hat} in (\mex{1}) einen Verbalkomplex:
% \ea
% \label{ex-er-geholfen-hat}
% daß er dem Mann [geholfen hat].
% \z



\subsection{On the verbal complex and partial verb phrase fronting}
\label{sec-pred-compl}

In various works (for instance \citealp{Uszkoreit87a}\ia{Uszkoreit}), it is assumed that an
auxiliary verb takes a verb phrase as its complement.
\ea
\label{ex-uf-two}
\gll dass Karl [[das Buch lesen] können] wird]\\
	 that Karl the book read can will\\
\glt `that Karl will be able to read the book'
\z
However, if one asumes such structures, it is difficult to explain the ordering of (\ref{ex-of-two}) because the auxiliary \emph{wird} is located between two elements
of the verb phrase.
\ea
\label{ex-of-two}
\gll dass Karl das Buch wird lesen können\\
     that Karl the book will read can\\
\glt `that Karl will be able to the read the book.'
\z

Furthermore, the sentences in (\mex{1}) are not ruled out by such an analysis since \emph{das Buch lesen} 
forms a phrase which can be moved left in the middle-field or appear in a so-called pied-piping construction 
with a relative clause.
\eal
\ex[*]{
\gll dass das Buch lesen Karl wird\\
	 that the book read Karl will\\
}
\ex[*]{
\gll das Buch, das lesen Karl wird\\
	 the book, that read Karl will\\
}
\zl
%
\citet*{HN94a}\ia{Hinrichs}\ia{Nakazawa} therefore suggest using a special dominance schema which ensures that
(certain) verbal complements are saturated before non-verbal ones. This means that, in the analysis of (\ref{ex-uf}) and (\ref{ex-of}),
\emph{lesen} is first combined with \emph{können} and the resulting verbal complex is then combined with \emph{wird}:
\ea
\gll daß Karl das Buch [[lesen können] wird].\\
     that Karl the book \hspaceThis{[[}read can will\\
\z
\emph{wird} can be placed to the right of the embedded verbal complex (as in \pref{ex-uf}), or indeed to the left as
in \pref{ex-of}. After the construction of the verbal complex \emph{lesen können wird}, it is then combined with the 
arguments of the involved verbs, that is with \emph{Karl} and \emph{das Buch}.\footnote{%
		This kind of structure has already been suggested 
		by \citet*{Johnson86a} in connection with an analysis of partial verb phrase fronting.
}

There are also coordination data, such as the example in (\mex{1}), which support this kind of approach.
\ea
\gll Ich liebte ihn, und ich fühlte, daß er mich auch geliebt hat oder doch, daß er mich hätte lieben wollen oder lieben müssen.\footnotemark\\
     I   loved  him and I felt that he me also loved had or PRT that he me   would.have love want or love must.\\
\footnotetext{%
        \citep*[\page 36]{Hoberg81a}
}
\iw{müssen}
\glt `I loved him and felt that he loved me too, or at least he would have wanted to love me or would have had to.'
\z

The following schema, which is derived from the one suggested by Hinrichs and Nakazawa, licenses
predicate complexes:

\begin{samepage}
\begin{schema}[Schema for predicate complexes]
\label{schema-vk}
~\\
\type{head-cluster-phrase} \impl\\
\ms{
 synsem & \onems{ loc$|$cat$|$comps \ibox{1} \\ 
                } \\
 head-dtr & \onems{ synsem$|$loc$|$cat$|$comps \ibox{1} $\oplus$ \sliste{ \ibox{2} }\\ } \\
 nonhead-dtrs & \sliste{ \onems{ synsem  \ibox{2} \\}
                      } \\
}
\end{schema}
\end{samepage}
%

I will assume the representation in (\mex{1}) for the auxiliary verb \emph{werden}:\footnote{\label{subj-fn}%
	 \citet{Pollard90a} and \citet*{Kiss92} have suggested that the subject of non-finite verbs is better represented
	 as an element in a separate list (\subj) rather than in the \comps list of the verb. For reasons of simplicity,
	 I have placed the subjects of both finite and non-finite verbs in the \comps list. The separate representation of infinite
	 subjects predicts that subjects cannot occur in projections of non-finite verbs, unless one formulates special rules which would
	 license such combinations.%
%% Für die Analyze der scheinbar mehrfachen Vorfeldbesetzung würde ebenfalls vorausgesagt, daß
%% Subjekte nicht zusammen mit anderen Konstituenten im Vorfeld stehen können, was empirisch korrekt
%% zu sein scheint. Zu einer Analyze von Prädikatskomplexen, die das Subjekt infiniter Verben separat repräsentiert,
%% siehe \citew{Mueller99a,Mueller2002b}.%
%Nimmt man die in \citew[Chapter~3]{Mueller2002b}
%vorgeschlagene Passivanalyze an, so sind die Beispiele in (\ref{bsp-mehrfach-vf-subjekt}),
%in denen ein Oberflächensubjekt in Passivkonstruktionen zusammen mit einer anderen Konstruktion
%vorangestellt wurde, auch mit einem separat repräsentierten Subjekt problemlos analysierbar.%
}
\eas
\label{le-wird}
\emph{werden} (Future auxiliary):\\
\ms[cat]{
 head   & \type{verb}\\*
 comps & \ibox{1} $\oplus$ \sliste{ \textsc{V[\textsc{lex}+, \type{bse}, comps~\ibox{1} ]}} \\
}
\zs
\emph{werden} selects a verb in its \type{bse} form, that is an infinitive without \emph{zu} `to'.

In example (\mex{1}), \emph{wird} takes over the partial specification of the arguments \emph{Karl} and
\emph{mir} `me' from \emph{helfen} `to help'.
\ea
\gll dass Karl mir helfen wird\\
	 that Karl me help will\\
\glt `that Karl will help me'
\z
This argument attraction is made possible by the structural sharing expressed by the box \iboxt{1}
in (\ref{le-wird}). The \comps list of \emph{wird helfen} `will help' therefore is identical
with  the \comps list for \emph{hilft} `helps'. The combination of \emph{helfen} `help' and
 \emph{wird} `will' is shown in Figure~\vref{abb-helfen-wird}. 

\begin{figure}
\centering
\begin{forest}
sm edges
[\ms{ head   & \ibox{1}\\
      comps & \ibox{2}\\
    }
[\iboxt{3}~\onems{ loc \onems{ head  \ms[verb]{ vform & bse \\
                                             }\\
                                comps~\ibox{2} \sliste{ NP[\type{nom}], NP[\type{dat}] }\\[2mm]
                               }\\
                }
[helfen;help]]
[\ms{ head & \ibox{1} \ms[verb]{ vform & fin \\
                               }\\
      comps & \ibox{2} $\oplus$\sliste{ \ibox{3} }\\[2mm]
                    }
[wird;will]]]
\end{forest}
\caption{\label{abb-helfen-wird}%
Analysis of \emph{helfen wird} `will help'}
\end{figure}

Auxiliaries are like raising verbs: They do not assign semantic roles to either subjects or
complements. For this reason, it is not surprising that \iboxt{1} in (\ref{le-wird}) can be 
instantiated by the empty list:
\ea
\gll Morgen wird getanzt werden.\\
	 tomorrow will danced become\\
\glt `There will be dancing tomorrow.'
\z
In (\mex{0}), subjectless construction created by passivization (\emph{getanzt werden}) has been
embedded under a future auxiliary.


% \footnote{
%       Note that this is the only purpose \textsc{lex} has in my grammar.
%       \textsc{lex} has the value $-$ if a head has been combined with a complement and +
%       otherwise. So if an unsaturated verb is combined with an adjunct its \textsc{lex} value
%       is still $+$. This is not the way \textsc{lex} is seen in the standard framework, and
%       therefore it might be reasonable to choose a different feature name. However, I decided
%         to stick with the name \textsc{lex} for historical reasons.
% } 

Spurious ambiguities are ruled out by the specification of the \textsc{lex} value of the embedded 
verbal complex in (\ref{le-wird}). Without such a specification all three structures in (\mex{1})
would be admitted:
\eal
\ex 
\gll er seiner Tochter  ein Märchen [erzählen wird]\\
     he his daughter a fairy.tale   \spacebr{}tell will\\
\glt `he will tell his daughter a fairy tale'
\ex er seiner Tochter [[ein Märchen erzählen] wird]]\label{pvp-ein-maerchen-erzaehlen}
\ex er [[seiner Tochter ein Märchen erzählen] wird]]
\zl




The \textsc{lex} feature ensures that \emph{erzählen} is combined with \emph{wird} before
\emph{erzählen} is combined with its arguments. Since the mother node in head"=complement
structures is specified as \textsc{lex}$-$, the projections of \emph{erzählen} in (\mex{0}b--c)
cannot be combined with \emph{wird}.


% \footnote{
% Of course, phrases like \emph{ein Märchen erzählen} and \emph{seiner Tochter ein Märchen erzählen} are needed in the analysis
% of German sentences like those in (i) and (ii):
% \ea
% \gll weil er seiner Tochter hätte ein Märchen erzählen sollen.\\
%      because he his daughter had  a   fairytale tell should\\
% \glt `because he should have told his daughter a fairytale.'
% \z
% \eal
% \ex 
% \gll Ein Märchen erzählen wird er seiner Tochter.\\
%      a   fairytale tell   will he his daughter\\
% \ex 
% \gll Seiner Tochter ein Märchen erzählen wird er sicher.\\
%      his daughter   a   fairytale tell will he surely\\
% \glt `He surely will tell his daughter a   fairytale.'
% \zl
% The example in (i) is an instance of the so"=called \emph{Oberfeldumstellung} (\citealp{Bech55a};
% \citealp[\page 723]{Haftka81}) and the examples in (ii) are examples of (partial) verb phrase fronting 
% \citep[\page 720--721]{Haftka81}. For an analysis of Oberfledumstellung see \citep{HN94a} and \cite[Ch.~14]{Mueller99a-unlinked}
% and for an analysis of the partial verb phrase fronting examples see \citep{Mueller97c} and \citep{Meurers99a-unlinked}.
% In the HPSG analysis of these phenomena the constraint that all arguments of embedded verbal complexes have to be raised
% to the higher predicate is relaxed for fronting or in cases like (ii) where no spurious ambiguities can
% arise.
% }


The \textsc{lex} value of the mother in predicate complex structures  -- unlike in head"=argument structures
(see Schema~\ref{schema-bin} on page~\pageref{schema-bin})-- is not quite as restricted since predicate complexes can be
embedded under different verbs and subsequently form a predicate complex with these, as shown by (\ref{ex-er-geholfen-haben-wird}).

\ea
\label{ex-er-geholfen-haben-wird}
\gll dass er dem Mann [[geholfen haben] wird]\\
     that he the man \hspaceThis{[[}helped have will\\
\glt `that he will have helped the man'
\z

If we want to rule out spurious ambiguities, we have to make sure that sentences such as (\mex{0}) can only be analyzed
as shown in (\mex{0}) and that an analysis such as (\mex{1}) is not possible.
\ea
\label{bsp-non-complex-forming}
\gll dass er dem Mann [geholfen [haben wird]]\\
     that he the man  \spacebr{}helped \spacebr{}have will\\

\z
%
In the analysis of (\mex{0}), the verbal argument of \emph{haben} `have' is raised to the argument
of the complex \emph{haben wird} `have will'.
The complex \emph{haben wird} `have will' is then combined with \emph{geholfen} `helped' via the Head"=Argument Schema. The analysis in (\mex{0})
can be ruled out if one restricts the kind of elements which can be raised in the lexicon entries for raising predicates.
Furthermore, we need an additional condition for (\ref{le-wird}), namely that \iboxt{1} only
contains fully saturated, non-predicative elements with the \textsc{lex} value $-$. In formal terms, this can be expressed as a restriction on \iboxt{1}:\footnote{%
		\citet{BvN98a} formulate an equivalent restriction. They differentiate between an
		\emph{Inner Zone} and \emph{Outer Zone} in a sentence. The \emph{Inner Zone} 
		is the predicate complex. Elements which are marked as belonging to the \emph{Inner Zone}
		by the governing head may not be raised.

In light of this restriction for raised elements, my criticism \citep[\page 351--352]{Mueller99a} of Kiss' treatment of
		obligatory coherence as a subcase of optional coherence \citep[\page 183]{Kiss95a} is rendered obsolete: One lexical
		item suffices for optionally coherent verbs in the present analysis.
}
\ea
\label{constr-non-complex-forming}
list\_of\_non\_c\_forming\_synsems(\eliste).\\
list\_of\_non\_c\_forming\_synsems(\liste{ \onems{ loc$|$cat \onems{ head$|$prd $-$\\
                                                          comps \eliste\\
                                                        }\\
                                                lex $-$\\
                                                } $|$ \ibox{2} }) :=\\
\flushright        list\_of\_non\_c\_forming\_synsems( \ibox{2} ).
\z
A list consists of elements which do not form a predicate complex when the list is empty (first
clause), or when the list starts with an element that has an empty \compsl, a \textsc{lex} value and
\textsc{prd} value of `$-$' and when the rest of the list \iboxb{2} is itself a list\_of\_non\_c\_forming\_synsems.\footnote{%
		It is not possible to avoid mentioning the \textsc{lex} value, as embedded intransitive verbs have
		an empty valence list since the subject of non-finite verbs is represented separately. The \textsc{lex} value of
		intransitive verbs is not specified in the lexicon. They can therefore occur in positions, 
		where only phrases are permitted (in so-called incoherent constructions \citep{Bech55a}) as
		well as in positions in which only lexical elements are allowed (in coherent constructions). 
		This is also the reason for the fact that the \textsc{lex} value of the mother in predicate complex
		structures is not specified as \textsc{lex}+ (as is the case in the analyses of \citealp{HN94a,dKM2001a}) since combinations
		of verbs which embed an intransitive verb may be fully saturated. Such fully
                saturated verbal complexes may form an incoherent construction with a matrix
                verb. The \textsc{lex} value of verbal complexes is therefore only constrained by the superordinate verb.%
}
The \textsc{prd} feature was introduced by \citet[\page 64--67]{ps} for means of differentiating predicative and non"=predicative
elements.

At a later point, I will explain why this restriction not only plays a role for excluding spurious
ambiguities, but also for the exclusion of certain impossible frontings.
Figure~\vref{abb-kombin1} shows in detail how the analysis of (\ref{ex-er-geholfen-haben-wird}) works.

\begin{figure}
\begin{sideways}
%\oneline{%
\begin{forest}
sm edges
[{\ms[cat]{ head   & \ibox{1} \\
           comps & \ibox{2}\\
            }}
[\iboxt{4}~\onems{ loc \ms{ head & \ibox{3} \\
                                   comps & \ibox{2} \\
                                 }
}
[\iboxt{5}~\onems{ loc \onems{ head  \ms[verb]{ vform & ppp  \\
                                                }  \\
                                comps~\ibox{2} \sliste{ NP[\type{nom}], NP[\type{dat}] } \\
                              }\\
                }
[geholfen;helped]]
[{\onems[cat]{ head~\ibox{3} \ms[verb]{ vform & bse \\
                            } \\
            comps ~ \ibox{2} $\oplus$ \sliste{ \ibox{5} } \\
          }}
[haben;have]]]
[
                 {\ms[cat]{ head & \ibox{1} \ms[verb]{ vform & fin \\
                                                    } \\
                           comps & \ibox{2} $\oplus$ \sliste{ \ibox{4} } \\
                    }}
[wird;will]]]
\end{forest}
%}
\end{sideways}
\caption{Analysis of the verbal complex in \emph{dass Karl dem Mann geholfen haben wird} `that Karl
  will have helped the man'}\label{abb-kombin1}%
\end{figure}

The perfect auxiliary \emph{haben} embeds the past participle \emph{geholfen} (a verb with \textsc{vform} \type{ppp}).
It adopts the arguments of this verb \iboxb{2} as its own. The resulting verbal complex has the same valence as 
\emph{geholfen}. This complex is embedded under \emph{wird}. \emph{wird} also attracts the arguments of the embedded
complex so that the entire complex \emph{geholfen haben wird} requires the same arguments as \emph{geholfen}.

At first glance, it may seem problematic that we need phrases such as \emph{ein Märchen erzählen} `to tell a fairy tale'
for sentences in which this group of words appears in first position. While we want to exclude this phrase as a complement
in \pref{pvp-ein-maerchen-erzaehlen}, it needs to act as a binder for the long-distance dependency of fronting in (\mex{1}):
\ea
\gll Ein Märchen erzählen wird er ihr müssen.\\
     a fairy.tale tell will he her must\\
\glt `He will have to read her a fairy tale'
\z
Sentences such as (\mex{0}) are unproblematic if \textsc{lex} is represented under \textsc{synsem}, i.e. outside of \textsc{local}, 
unlike \cite{ps} where \textsc{lex} was represented under \textsc{cat} -- that is, inside \textsc{local} \citep{Hoehle94a,Mueller97c,Mueller99a,Mueller2002b,Meurers99a}.

Due to the fact that a filler in a long-distance dependency only shares the features of the trace which are under 
\textsc{local}, a verb can require an embedded trace to have the \textsc{lex} value $+$. The \textsc{lex} value of the trace
does not have to be identical to the \textsc{lex} value of the constituent in initial position. This means that word
groupings with a \textsc{lex} value of $-$ are possible fillers as well.\footnote{%
			This means that it is not wise to formulate a structure preserving principle for grammars of HPSG, which
			states that a moved constituent has to be identical to its trace. (See \eg \citew{Emonds76a-u} for
			his formulation of this kind of principle for transformations). This kind of structure preserving
			principle does not make sense for HPSG"=grammars, as overt realizations mostly differ from their traces
			in that the overt realizations have daughters, whereas traces do not. In HPSG grammars, only information
			under \textsc{local} is normally separated. Traces and fillers can have different values with respect to everything 
			else (\textsc{phon}, \textsc{head-dtr}, \textsc{non-head-dtr},\textsc{synsem$|$nonlocal},\textsc{synsem$|$lex}, \ldots). In order to prevent 
			overgeneration, there are general conditions on extraction which make reference to local contexts.%
}		
Figure~\vref{abb-seiner-tochter-erzaehlen} shows the analysis of (\mex{1}).
\ea
\gll Seiner Tochter erzählen wird er das Märchen.\\
	 his daughter tell will he the fairy.tale\\
\glt `He will read his daughter the fairy tale'
\z	

\begin{figure}
\resizebox{!}{\textheight-3\baselineskip}{%
\begin{sideways}
\begin{forest}
sm edges
[V\feattab{\textit{fin},\\
           \comps \eliste,\\
           \textsc{slash} \eliste}
	[V{[\begin{tabular}[t]{@{}l@{}}
                   \textsc{lex} $-$,\\
                   \textsc{loc} \ibox{1}\,[\begin{tabular}[t]{@{}l@{}}
                                        \textit{bse},\\
                                        \comps \ibox{2}\,\sliste{ \ibox{3}, \ibox{4} }] ]
                                        \end{tabular}
           \end{tabular}}
		[\ibox{5} NP{[\textit{dat}]}
			[seiner Tochter;his daughter, roof]]
		[V\feattab{\textit{bse},\\
                           \comps \sliste{ \ibox{3}, \ibox{4}, \ibox{5} }}
			[erzählen;tell]]]
	[V\feattab{\textit{fin},\\
                   \comps \eliste,\\
                   \textsc{slash} \sliste{ \ibox{1}}}
		[V\feattab{\textit{fin},\\ 
                          \comps \sliste{ \ibox{7} }}
			[V{[\comps \ibox{2} $\oplus$ \ibox{6}\,]}, tier=np,edge label={node[midway,right]{V1-LR}}
				[wird;will]]]
		[\ibox{7} V\feattab{\textit{fin},\\
                                    \comps \eliste,\\
                                    \textsc{slash} \sliste{ \ibox{1} }}
			[\ibox{3} NP{[\textit{nom}]}, tier=np
				[er;he]]
			[V\feattab{\textit{fin},\\
                                   \comps \sliste{ \ibox{3} },\\
                                   \textsc{slash} \sliste{ \ibox{1} }}
				[\ibox{4} NP{[\textit{acc}]}, fit=band
					[das Märchen;the fairy tale, roof]]
				[V\feattab{\textit{fin},\\
                                           \comps \ibox{2} \sliste{ \ibox{3}, \ibox{4} },\\
                                           \textsc{slash} \sliste{ \ibox{1} }}
					[\ibox{6} V\feattab{\textsc{lex} $+$,\\
                                                            \textsc{loc} \ibox{1},\\
                                                            \textsc{slash} \sliste{ \ibox{1} }}
						[\trace]]
					[V\feattab{\textit{fin},\\
                                                   \comps \ibox{2} \sliste{ \ibox{3}, \ibox{4} } $\oplus$ \sliste{ \ibox{6} }}
						[\trace]]]]]]]
\end{forest}
\end{sideways}
}
\caption{\label{abb-seiner-tochter-erzaehlen}%
Analysis of \emph{Seiner Tochter erzählen wird er das Märchen.} `He will tell his daughter the fairy
tale.'}
\end{figure}


Ungrammatical sentences such as (\mex{1}) are ruled out by the condition in (\ref{constr-non-complex-forming}).
\ea[*]{
\gll Müssen wird er ihr ein Märchen erzählen.\\
     must will he her a fairy.tale tell\\
}
\z
\emph{Wird} requires an infintive in the \emph{bse} form and then attracts its arguments. The
attracted elements must be \textsc{lex}$-$. Since \emph {müssen} selects \emph{erzählen} and requires it to be \lex+, it cannot be attracted. This explains
why a structure such as (\mex{1}) is ruled out:
\ea[*]{
\gll Müssen$_i$ wird$_j$ er ihr ein Märchen [erzählen [\_$_i$ \_$_j$]].\\
     must       will     he her a fairy.tale \spacebr{}tell\\
}
\z
For more on this, see the discussion of (\ref{bsp-non-complex-forming}) on page~\pageref{bsp-non-complex-forming}.

The analysis in (\mex{1}) is ruled out by a general condition which bans extraction traces in head
positions.
\ea[*]{
\gll Müssen$_i$ wird$_j$ er ihr ein Märchen  [[erzählen \_$_i$] \_$_j$].\\
     must       will     he her a fairy.tale \hspaceThis{[[}tell\\
}
\z
The contrast in (\mex{1}) can be explained by the fact that in (\mex{1}a) a predicative PP has to
be attracted, which is not the case in (\mex{1}b).
\eal
\ex[\#]{
\gll Halten wird er ihn für den Präsidenten.\\
	 hold will he him for the president\\
\glt `He will think he is the president'
}
\ex[]{
\gll Interessieren wird er sich für den Präsidenten.\\
	 be.interested will he \textsc{refl} for the president\\
\glt `He will be interested in the president'
}
\zl

The analysis presented is most certainly compatible with the analysis presented in \citew[Chapter~2]{Mueller2002b} of 
constructions such as \emph{halten für} as complex predicates.

In the Principles and Parameters Framework, fronting of incomplete projections is often analyzed as remnant movement
(see G.\,\citealp{GMueller96a,GMueller98a,GMueller2014a-u}). \citet{deKuthy2002a}, \citet{dKM2001a} and \citet{Fanselow2002a} have
shown however that remnant movement analyses face empirical problems which argument composition
approaches as the one suggested here do not.\todostefan{Add discussion of \citew{GMueller2014a-u}}

\subsection{Verb movement and extraction in other Germanic languages}
\label{sec-v2-germanic}

\subsubsection{Verb movement}

The Subsections~\ref{sec-v1} and~\ref{sec-v2} provide an analysis of the verb position in German. It is in some sense
similar to the GB analysis of \citew{Reis74a}, \citew{Koster75a}, \citep[Chapter~1]{Thiersch78a} and \citep{denBesten} where it is assumed that the
finite verb moves into the C position. See also Figure~\vref{fig-verb-movement-gb}. The V-to-C
movement analysis of verb initial sentences in German and Dutch was motivated by the observation
that the finite verb and the complementizer are in complementary distribution: if the complementizer
is present the verb may not be fronted. So it was assumed that the verb moves into the
complementizer position, provided it is empty. The drawbacks of this proposal will be discussed in
Section~\ref{sec-v-to-c-movement} in more detail. This section deals with one aspect: there are
other V2 languages that have complementizers that appear together with V2 sentences
(\citealt{Vikner95a}; \citealt{Bhatt99a}; \citealt[\page 87]{Fanselow2009b}). Analyses that
assume that a finite verb moves into the position of a complementizer do not extend to such
languages. This shows that the V-to-C analysis does not capture the verb placement phenomenon in its
whole breadth. In Section~\ref I claimed that the HPSG analysis is similar to the GB analysis but
the similarity does not extend to the problematic aspects. The HPSG analysis captures the similarity
between complementizers and finite verbs in German by assigning verbs in initial position a valence
frame that is almost identical to the one of a complementizer. Both complementizer and initial
finite verb select a verb final finite clause. The only difference between complementizer and
initial finite verb is that the former requires that the finite verb is realized within the selected
clause (\dsl \type{none}) while the latter requires the verb to be missing (\dsl is an object of
type \type{local}).

Now, the analysis suggested here is different from the V-to-C analysis in that it is compatible with
languages like Yiddish in which a complementizer is combined with a V2 sentence. (\mex{1}) shows a
Yiddish example:
\ea
\settowidth\jamwidth{(Yiddish)}
\gll Ikh meyn  az   haynt hot Max geleyent dos bukh.\footnotemark\\
     I   think that today has Max read the book\\\jambox{(Yiddish)}
\footnotetext{\citew[\page 58]{Diesing90a}.}
\glt `I think that Max has read the book today.'
\z
The analysis of the CP in the example is shown in Figure~\vref{fig-analysis-cp-yiddish}.
\begin{figure}
\centering
\begin{forest}
sm edges
[CP
  [{C[ \sliste{ S } ]} [az;that]]
  [S 
    [Adv [haynt;today]]
    [S/Adv
      [{V[ \sliste{ S//V } ]} 
        [V [hot;has]]]
        [S//V/Adv
          [NP [Max;Max]]
          [VP//V/Adv 
            [V//V [\trace]]
            [VP/Adv
              [VP [V [geleyent;read]]
                  [NP [dos bukh;the book, roof]]]
              [Adv/Adv [\trace]]]
]]]]]
\end{forest}
\caption{Analysis of the Yiddish sentence \emph{az   haynt hot Max geleyent dos bukh} `that Max has read the book today'}\label{fig-analysis-cp-yiddish}
\end{figure}%
I assume that adverbs attach to VPs in SOV languages like English and Danish. The adverb is
extracted in (\mex{0}), so Figure~\ref{fig-analysis-cp-yiddish} shows a trace in the adverb position following the
VP. The information about the adverb gap is passed up in the tree until it is bound off by the
adverb in front of the finite verb. The perfect auxiliary is realized adjacent to the VP it embeds
but since the sentence in (\mex{0}) is a verb second sentence, the verb appears in initial position,
that is, to the left of the subject. The normal position of the verb is taken by the verb
trace. Information about the missing verb is projected from the verb trace to the VP and the S
level. The finite verb in initial position takes a clause from which itself is missing. The result
of the combination is S/Adv, that is, a sentence with an adverb gap. The adverb is combined with the
S/Adv and binds off the gap information. The result of the combination is an S. This S is the
argument of the complementizer and the result of the combination of complementizer and S is a CP.

The difference between the German and the Yiddish complementizer is that the German complementizer
selects a finite clause with the verb in final position, while the Yiddish complementizer selects a
V2 clause. As is clear from Figure~\ref{fig-analysis-cp-yiddish} an analysis that assumes that the
finite verb moves to C would run into trouble unless one assumes that \emph{az} `that' embeds a
CP. The analysis developed here does not have this problem and extend easily to other V2
languages. I discuss the analysis of other Germanic languages in more detail in
\citew{MuellerGermanic}.\todostefan{Vikner95a zeigt, dass CP-Analyse schwierig ist.}

\subsubsection{Extraction}

\citet[]{Fanselow2009b} compares English with German and notes that a full sentence like \emph{John
  came} in (\mex{1}a) can be combined with the adverbial \emph{yesterday}, while the same is
impossible in German, as (\mex{1}b) shows:
\eal
\ex[]{
Yesterday, John came.\jambox{(English)}
}
\ex[*]{
\gll Gestern John kam.\\
     yesterday John came\\
}
\zl
Fanselow, working in a Minimalist setting, argues that the difference in (\mex{0}) is due to the
fact that \emph{John came} is a TPs while \emph{John kam} is a CP in German. In the analysis
suggested here, \emph{kam} `came' is a sentence with fully saturated valence requirements and one
element in \slasch. This is bound off by \emph{John} to form the V2 clause \emph{John kam}. Since
the Head-Filler Schema allows for exactly one element in \slasch and binds off this element, the
\slasch list of the mother is the empty list and hence there is no way to combine \emph{gestern} as
a filler with \emph{John kam}. Since adverbials modify verbs in final position (\initial$-$) and
since \emph{John kam} is head-initial (\initial$+$) a combination via Head-Adjunct Schema is also
ruled out.

\citet{ps2} suggested analyzing \emph{John came} as a fully saturated verbal projection. It may
contain a gap and if it does it is possible to combine \emph{John came} with the filler
\emph{yesterday}. The Filler-Head Schema is rather similar for English and German. The difference
between the two languages lies in the way of building verb initial projections that can be used in Filler-Head structures:
while German and other Germanic languages involve verb movement, English is SVO and the combination of the head with its objects
and subjects is licensed directly. For German it is sufficient to require the verbal projection to
be \initial$+$. Since only verbs that underwent verb fronting are \initial$+$ this captures the data
correctly. For SVO+V2 languages this would not be sufficient since verbs are classified as \initial$+$
in VO languages anyway. Here an additional distinction \textsc{inverted}$+/-$ is needed. In V2
Filler-Head structures the head daughter must be a verbal projection with a fronted verb, that is,
\textsc{inverted}$+$.

While the modification of \emph{John kam} by \emph{gestern} is excluded in German due to the fact
that adjuncts modify \initial$-$ verbal projections, this sequence is excluded for SVO languages
since adjuncts modify VPs rather than complete sentences in these languages. 

\section{Alternatives}

\subsection{V to (I to) C movement}
\label{sec-v-to-c-movement}

The preceding subsections provide an analysis for constituent order in German. It is in some sense
similar to the GB analysis of \citep[Chapter~1]{Thiersch78a} and \citep{denBesten} where it is assumed that the
finite verb moves into the C position. See also Figure~\vref{fig-verb-movement-gb}. While this
somehow captures the idea that complementizers and finite verbs in initial position share certain
properties \citep{Hoehle97a} the V-to-C analysis has several problematic aspects, as
\citet{Fanselow2009a} points out.



\subsection{Squeezing in}

\citet[\page34]{Bierwisch63} suggested that V2 sentences are accounted for by assuming that the verb
is ``squeezed in'' into the sentence after the first constituent. As \citet{Fanselow2009a} notes,
this nicely explains the observation by Haider, Frey and Fanselow that the
element in the \vf basically has the same information structural status as it could have in the
left-most \mf position. However, as also noted by \citet{Fanselow2009b} there is a problem with
examples like (\mex{1}). While certain elements may take the first position in the \mf, they are
excluded from appearing in the \vf.
\eal
\ex
\zl
In a fronting proposal like the one suggested here, one can exclude certain elements from entering
nonlocal dependencies. This is impossible in a squeezing-in approach since the \mf constituent would
not move. It would stay in its original position and it would just be the finite verb that would be
inserted between the first and the second element in the \mf.

This is an empirical argument against Bierwisch's analysis. There is also a technical argument. If
it could be shown that the squeezing in analysis is the only sensible analysis of the phenomenon is
the squeezing in analysis, non"=transformational frameworks would be in trouble since they usually do
not assume that there are certain structures that can be broken up by other material that is
inserted in the middle in later steps of an analysis.

Furthermore, \citet[Section~2.2]{Reis80a} noted that certain elements can be fronted without being able to occur
in the left-most position in the \mf:
\eal
\ex[]{ 
\gll Verehrt hat er ihn.\\
     adored  has he him\\
\glt `He has adored him.'
}
\ex[*]{
\gll [dass] verehrt er ihn hat\\
     \spacebr{}that adored he him has\\
}
\ex[]{
\gll Das alles erwähnte der Autor. Nicht hat er hingegen berücksichtigt, dass \ldots \\
     this all mentioned the author not   has he however  taken.into.account that\\
\glt `The author mentioned all this, but he did not take into account that \ldots'
}
\ex[*]{
\gll [dass] nicht er hingegen berücksichtigt hat, dass \ldots\\
     \spacebr{}that not   he however   taken.into.account has that\\
}
\ex[]{
\gll [dass] er hingegen nicht berücksichtigt hat, dass \ldots\\
     \spacebr{}that he however not  taken.into.account has that\\
}
\zl
In a squezing in approach the order \emph{verehrt er ihn hat} `adored he him has' would have to be derived and then the
auxiliary would move between \emph{verehrt} and \emph{er}. Similarly, the sequence \emph{nicht er
  hingegen berücksichtigt hat} `not he however taken.into.account has' would be the basis for
squezing the auxiliary between \emph{nicht} and \emph{er}. This sequence is inappropriate with the
intended reading. The only reading that is acceptable for (\mex{0}d) is the constituent negation of
\emph{er} `he'. The correct order with the verb in final position is (\mex{0}e).


\subsection{Reprojection}

As \citet[\page 93]{Fanselow2009b} and others noted, the adjunction to C analysis that
was suggested in GB is excluded in Minimalist accounts for theory internal technical reasons.

What is suggested instead is something that is called \emph{reprojection}\is{reprojection} or \emph{Remerge}\is{Remerge} \citep{Suranyi2005a,Fanselow2009b}. It is assumed that a head
is realized at a different location, leaving a trace at the original position. In the new position
the head selects the projection from which it was moved. One such analysis is provided in
Figure~\vref{fig-reprojection}.
\begin{figure}
\begin{forest}
[HP
  [H,name=H oben]
  [K
    [~~(H)~~,roof,name=H unten]]]
%\draw (H unten.south) |-  ([yshift=-5mm]H unten.south |- H oben) -| (H oben);
\draw[->] (H unten) to[out=245,in=south] (H oben);
\end{forest}
\caption{Head movement as reprojection according to \citet{Suranyi2005a}}\label{fig-reprojection}
\end{figure}%
The proposals are never worked out in detail. For instance it is unclear why the fronted head
selects for the phrase it is missing from or if it is not selection what else would license the
combination of fronted element and projection of the trace of the head. It is not explained why the
head is governing in another direction once fronted. \citet[\page 105]{Fanselow2009b} suggests that
``The verb possesses the checking feature and feature to be checked at the same time (the probe and
the goal are identical).''. But it is unclear what ``at the same time'' means. The two instances of
\emph{aime} in Figure~\vref{fig-fanselow-reprojection} cannot be identical. If they were both had to have a checking feature
and a feature to be checked. This would result in a situation in which half of the features of the
lower instance and half of the features of the upper instance could not be used in the derivation.
\begin{figure}
\begin{forest}
[{[Tense, \st{V}]}
    [subject]
    [ {[Tense, \st{V}]}
        [ {[Tense, \st{V}]} [aime\\love]]
        [ {[Tense, V]}
          [ {[Tense, V]} [\st{aime}\\ love]]
          [object]]]]
\end{forest}
\caption{Fanselow's analysis of head-movement as reprojection \citeyearpar[\page 105]{Fanselow2009b}}\label{fig-fanselow-reprojection}
\end{figure}%
So there have to be two different instances of \emph{aime}, in fact of all verbs that undergo head
movement and of course any account should capture the fact that these instances are somehow related.

\citet[\page 14]{Suranyi2005a} suggests that inflectional affixes attach to stems directly ``prior to the point where the fully inflected stem merges with another (independent) element''.
So he assumes the representation in Figure~\ref{fig-verb-suranyi} for fully inflected
verbs.\footnote{%
  Interestingly this is very similar to what is assumed in HPSG: HPSG does not assume a
  decomposition in terms of verb shells. Instead decomposition is done lexically. A verb contains
  the information contributed by V and by \emph{v} in Minimalist approaches. Inflection is also done
  presyntactically in HPSG. HPSG assumes lexical rules for inflection. They are equivalent to the
  V-T combination in Figure~\ref{fig-verb-suranyi}.%
}
\begin{figure}
\begin{forest}
[V
  [V
    [V] [\emph{v}]]
  [T]]
\end{forest}
\caption{Verbal stem plus affixes according to \citet[\page 15]{Suranyi2005a}}\label{fig-verb-suranyi}
\end{figure}%
This structure is combined with the object DP as shown in Figure~\vref{fig-verb-object-suranyi}.
\begin{figure}
\begin{forest}
[V(P)
  [V
    [V
      [V] [\emph{v}]]
    [T]]
  [Obj]]
\end{forest}
\caption{Combination of verb and object according to \citet[\page 15]{Suranyi2005a}}\label{fig-verb-object-suranyi}
\end{figure}%
Surányi assumes that the verb tree in Figure~\ref{fig-verb-object-suranyi} moves to the left. Since
the features of V are checked already it is not the functor in the verbal tree any longer. Therefore
the labels in the fronted tree are not determined by V but by \emph{v}. The result of the head
reprojection is provided in Figure~\vref{fig-verb-movement-result-suranyi}.
\begin{figure}
\begin{forest}
[\emph{v}(P)
  [\emph{v}
    [\emph{v}
      [V] [\emph{v}]]
    [T]]
  [V(P)
    [t]
    [Obj]]]
\end{forest}
\caption{Combination of verb and object according to \citet[\page 15]{Suranyi2005a}}\label{fig-verb-movement-result-suranyi}
\end{figure}%
The left subtree of \emph{v}(P) is the moved verb from Figure~\ref{fig-verb-suranyi}. \emph{v}
provides the label of this subtree. It selects a V(P) and the result of the combination is a
\emph{v}(P), which may be combined with a subject in a later step.

As with Fanselow's analysis one has to say that the details are not worked out. What does V select?
An incomplete projection of \emph{v} as seems to be needed to justify trees like the one in
Figure~\ref{fig-verb-suranyi}? If this is the case, why is \emph{v} the functor in
Figure~\ref{fig-verb-movement-result-suranyi} taking a V to its left, a T to its right and a V(P)?
%Why is V linearized to the left of \emph{v}? If \emph{v} is the functor 

%% Note also that models in which linguistic objects change their categories during a derivation are
%% highly implausible from a psycholinguistic point of view since 
%% they presuppose a certain order of
%% processing. This is incompatible with the view that we use the same linguistic knowledge for both
%% parsing and generation and that human processing is quite robust, that is, we are capable of
%% processing fragments. We are able to work with verbs in 

All of this is provided by the proposal presented in this book: there is a lexical rule/unary schema that maps
the verb in final position to a verb in initial position that acts as a head-initial head that
selects for a projection in which a respective verb is missing. I think that the Minimalist
reprojection approaches are notational variants of the HPSG analyses, which were developed several
years earlier (\citealp*[Section~4.7]{KW91a}; \citealp*{Kiss93}; \citealp*{Frank94}; \citealp*{Kiss95a}; \citealp{Feldhaus97},
\citealp{Meurers2000b}) but while the Minimalist proposals remain on the level of sketches like the
one in Figure~\ref{fig-reprojection}, the HPSG analyses are worked out in detail.


%This tradition is grounded in a proposal Holmberg (1991) made for the analysis of VP-shells:
%substitutions into empty heads should be replaced by adjunc- tions of heads to maximal projections
%followed by a re-projection of the moved head.


\section{Summary}

In this chapter, I have presented a model of German sentence structure which can explain the relatively
free ordering of constituents in the Mittelfeld, the position of the finite verb, the predicate complex,
and fronting. I have argued against alternative analyses with variable linearization/variable branching.
The analysis put forward in this chapter forms the basis for the explanation of the previously discussed cases 
of supposed multiple fronting that is discussed in the next chapter. 


% Fanselow 2009b 92
%%
%% Some languages such as Venetian (see van Craenenbroek 2006; Zwart 2006) employ ordering requirements
%% such as topic > focus, focus > che, che > topic, which cannot be translated into a cartographic
%% system because of the obvious transitivity problems that arise when such or- dering statements are
%% translated into hierarchical relations between heads. Samek-Lodovici (2006) argues convincingly that
%% the le peripheral focus of Italian is in fact situated at the right edge of the clause, being
%% followed by right-dislocated material.  is seriously under- mines the line of reasoning in Rizzi
%% (1997) in favor of a Focus-phrase in Italian.



% 


%      <!-- Local IspellDict: en_US-w_accents -->


% Scherpenisse84a-u 219  gestern am Strand ist eine AdvP

%% -*- Coding:utf-8 -*-
%%%%%%%%%%%%%%%%%%%%%%%%%%%%%%%%%%%%%%%%%%%%%%%%%%%%%%%%%
%%   $RCSfile: mehr-vf-lb.tex,v $
%%  $Revision: 1.10 $
%%      $Date: 2009/03/27 16:28:44 $
%%     Author: Stefan Mueller (DFKI)
%%    Purpose: 
%%   Language: LaTeX
%%%%%%%%%%%%%%%%%%%%%%%%%%%%%%%%%%%%%%%%%%%%%%%%%%%%%%%%%
%% $Log: mehr-vf-lb.tex,v $
%% Revision 1.10  2009/03/27 16:28:44  stefan
%% *** empty log message ***
%%
%% Revision 1.9  2004/08/25 22:04:17  stefan
%% erste goeteborg-Version
%%
%% Revision 1.8  2004/07/25 16:27:18  stefan
%% removed Corpus-syntax now in corpus-syntax
%%
%% Revision 1.7  2004/07/20 13:08:58  stefan
%% *** empty log message ***
%%
%%
%%
%% Revision 1.1  2004/06/08 16:42:34  stefan
%% gesplittete Version, vor ersten Veränderungen
%%
%%%%%%%%%%%%%%%%%%%%%%%%%%%%%%%%%%%%%%%%%%%%%%%%%%%%%%%%%

\newcommand{\ao}{Avgustinova und Oliva\xspace}%


\chapter{Multiple fronting}
\label{chapter-mult-front}

\footnotetext{
This chapter is based on \citew{Mueller2005d}.
}

In the brief introductory Chapter~\ref{chapter-introduction}, I mentioned that German is a V2
language. This means that declarative sentences and certain interrogative sentences are formed by
placing a constituent in front of the finite verb.
\citet{Thiersch78a}, \citet[\page 55]{denBesten83a}, \citet{Uszkoreit87a}\ia{Uszkoreit}, among others, have suggested that
verb"=second sentences are in fact verb-initial sentences from which one constituent has been extracted and placed in the prefield. 
In the case of (\mex{1}b), it would be \emph{das Buch} which has been extracted from the verb-initial clause.
\eal
\ex 
\gll Kennt er das Buch?\\
	 knows he the book\\
\glt `Does he know the book?'
\ex 
\gll Das Buch kennt er.\\
	 the book knows he\\
\glt `He knows the book.'
\zl
This chapter deals with apparent exceptions to the V2 property of German of the type exemplified in (\mex{1}):
\ea
\gll {}[Trocken] [durch die Stadt] kommt man am Wochenende auch mit der BVG.\footnotemark\\
	 \spacebr{}dry \spacebr{}through the city comes one at.the weekend also with the BVG\\
\footnotetext{
        taz berlin, 10.07.1998, p.\,22.
      }
\glt `With the BVG, you can be sure to get around town dry at the weekend.'
\z
Neither \emph{trocken} `dry' depends on \emph{durch die Stadt} `through the city' nor the other way
round. Rather both constituents depend on \emph{kommt} `comes'.

Viewing fronting as the extraction of one element has become the most established analysis up to now.
Examples in which more than one constituent occupies the prefield have been discussed from time to time in the 
more theoretical literature. To account for these data, certain analyses have been developed where the 
constituents preceding the finite verb are viewed as a single constituent, i.e. it is assumed there is 
only a single constituent in the prefield (\citealp[\page 17]{Haider82}; \citealp[\page 79]{Wunderlich84}; 
Fanselow \citeyear[\page 99--100]{Fanselow87a}; \citeyear[Chapter~3]{Fanselow93a}; \citealp[\page
  1634]{Hoberg97a}; G.\ \citealp[Chapter~5.3]{GMueller98a}).

The exceptions to this are \citet{Grubacic65a}, \citet{Lee75a}, \citet{Loetscher85a}, \citet[\page
  412]{Eisenberg94a}, \citet{Jacobs86a}, \citet{BH2001a}, and \citet{Speyer2008a}. 
\citet{Jacobs86a} and \citet{BH2001a} argue that it is necessary to assume V3 order for sentences with focus particles such as \word{nur}, \word{auch} and \word{sogar} or rather
a special position for focus particles preceding verb-second clauses.
\ea
\label{ex-v3-particles-jacobs}
\gll Nur die Harten kommen in den Garten.\\
     only the hard come into the garden\\
\glt `Only the though ones make it into the garden.'
\z
For a discussion of these suggestions, see \citew{Reis2002a,Reis2005b} and \citew{Mueller2005e}.
Jacobs also assumes that several of the so-called `sentence adverbs' can occur in sentences with V3 constituent order.
He demonstrates this with \emph{leider} `unfortunately' und \emph{vermutlich} `probably' (p.\,107, p.\,112).
The examples which will be discussed in what follows are of a different kind. \citet{Grubacic65a} offers some examples which
I will view as cases of (apparent) multiple fronting. However, some of her examples are also of the same kind as discussed by \citet{Lee75a}.
\eal
\ex 
\gll Piachi, als ihm der Stab gebrochen war, verweigerte sich hartnäckig der Absolution.\footnotemark\\
     Piachi  when him the stick broken was   refused     \self{} persistent the absolution\\
\footnotetext{
Kleist, \emph{Der Findling}, p.\,214.
}
\glt `Piachi persistently refused the absolution, when the stick was broken over him.'
\ex  
\gll Der Junge, sobald er den Alten nur verstanden hatte, nickte und sprach: o ja, sehr gern.\footnotemark\\
     the boy    once   he the old only understood had    nodded and said    o yes very gladly\\
\footnotetext{
Kleist, \emph{Der Findling}, p.\,21 I.
}
\glt `As soon as the boy understood the old man, he nodded and said: O yes, I like to do this very much.'
\ex 
\gll Und damit, ehe ich noch recht begriffen, was sie sagt, auf dem Platz, vor Erstaunen sprachlos, läßt sie mich stehen.\footnotemark\\
     nad there.with before I yet right understood what she said on the place before asonishment
     speechless let she me stand\\
\footnotetext{
Kleist, \emph{Kohlhaas}, p.\,92.
}
\glt `After this she abandons me on the place speechless and before I fully understood what she was saying.'
\zl
I do not consider Lee's examples V3"=clauses in the sense that is relevant here. Some of the
examples are parenthetical insertions and some are of the type that is discussed in
Section~\ref{sec-analyse-mfvorschlaege}. For further discussion of Lee's data, see \citew[\page 33]{Mueller2003b}.



For expository purposes, I will discuss some data in the following section where it seems (at least on the surface) that more
than one constituent precedes the finite verb. Section~\ref{sec-analyse-mf} presents the analysis of
apparent multiple frontings. In Section~\ref{sec-analyse-mfvorschlaege}, I will show that many of the multiple fronting analyses suggested thus far make predictions
that are incompatible with the data in Section~\ref{sec-phenomenon-mult-front} and additional data from German.
In Section~\ref{sec-zusammenfassung}, I draw some conclusion.


\section{The phenomenon}
\label{sec-phenomenon-mult-front}

The assumption that only a single constituent can occur before the finite verb is well established and
descriptively correct for the vast majority of German sentences. In certain circumstances, however, several
constituents, that is, multiple phrases which are not syntactically dependent on each other, can occur there
together. The following sentences are examples of the occurence of different types of constituents in the prefield.
I have ordered the examples according to the type of the fronted elements. The division into constituents is shown
by the corresponding bracketing notation. In cases where multiple divisions are possible, I have omitted the brackets.

Many of the following examples were published in a descriptive paper that appeared in \emph{Deutsche
  Sprache} \citep{Mueller2003b}. I found most of these examples by careful reading. After the
publication of the paper in 2003 I continued to collect data and made it available to the community
on my webpage \citep{Mueller2013a}. A further resource that is also available online is a database
put together by Felix \citet{Bildhauer2011a} in the DFG project \emph{Theorie und Implementation
  einer Analyse der Informationsstruktur im Deutschen unter besonderer Berücksichtigung der linken
  Satzperipherie} (MU 2822/1-1 and SFB 632, A6). He collected 3.200 examples mainly from the corpora
that are available from the Institut für Deutsche Sprache in Mannheim at \url{https://clarin.ids-mannheim.de/SFB632/A6}.\footnote{
  \citet{Winkler2014a} uses almost exclusively data from
  \citew{Mueller2003b,Mueller2005d,Mueller2013a,MBC2012a,BC2010a,Bildhauer2011a} without proper
  acknowledgment of the source. Researchers who want to cite examples properly are urged to check
  the mentioned papers before attributing data to Winkler.
}
These examples are annotated with respect to part of speech, grammatical function and information structural status.

The following examples were discussed in many German publications but until now they were not
available with glossing and translation.





\subsection{Subject and adverb}
\label{sec-subj-mf}

In (\ref{bsp-richtig-geld}), an adjective used adverbially is present in the prefield with the subject of a passive 
clause. The same is true for the construction in (\ref{bsp-alle-traeume-gleichzeitig}):
The subject has been fronted along with an adjective.
\eal
\label{bsp-mehrfach-vf-subjekt}
\ex 
\gll {}[Richtig] [Geld] wird aber nur im Briefgeschäft verdient.\label{bsp-richtig-geld}\footnotemark\\
	 \spacebr{}right \spacebr{}money is PRT only in postal.services earned\\
\footnotetext{        
	taz, 28./29.10.2000, p.\,5.
}
\glt `It's only in postal services where you earn serious money.'
\ex 
\gll {}[Alle Träume] [gleichzeitig] lassen sich nur selten verwirklichen.\label{bsp-alle-traeume-gleichzeitig}\footnotemark\\
     \spacebr{}all dreams \spacebr{}simultaneously let \textsc{refl} only seldom realized\\
\footnotetext{        
		Broschüre der Berliner Sparkasse, 1/1999.
        }
\glt `All our dreams can only seldomly be realized at the same time.'
\zl
There are examples such as (\mex{1}) where one may be tempted to count the temporal adjunct
\emph{täglich} `daily' as part of the NP, but we are not
dealing with these kinds of constructions in \pref{bsp-alle-traeume-gleichzeitig} as the adverb obviously refers to \emph{verwirklichen} `to realize'.
\ea
\gll ein weiteres Großcenter [\ldots], das mit [20.000 Besuchern täglich] zu den beliebtesten gehört.\footnotemark\\
     a further big.centre    {} that with \spacebr{}20,000 vistors daily to the most.popular belongs\\
\footnotetext{
        taz berlin, 11.10.2002, p.\,13.
}
\glt `another large scale centre, which -- with its total of 20,000 visitors daily -- counts as one of the most popular.'
\z

Note that the fronted elements in (\ref{bsp-mehrfach-vf-subjekt}) are logical objects. The fronting
of logical subjects together with other constituents does -- if we ignore examples like
(\ref{ex-v3-particles-jacobs}) which are sometimes analyzed as V3 \citep{Jacobs86a} -- not seem to be possible
(see \citew[\page 413]{Eisenberg94a}).\todostefan{Was hat Eisenberg genau gesagt? Daten?}

As \citet[\page 316]{Lenerz86a}, \citet[\page 99]{Fanselow87a}, and \citet[\page 32]{Duerscheid89a} noted, examples like (\mex{1}) are absolutely unacceptable.
\eal
\ex[*]{
\gll Ich das Wienerschnitzel habe bestellt.\footnotemark\\
     I.\nom{}   the.\acc{} wiener.schnitzel have ordered\\
\footnotetext{
         \citew[\page 316]{Lenerz86a}.
}	 
\glt `I ordered the Wiener schnitzel.'
}
\ex[*]{
\gll Einen interessanten Vortrag der Sascha dürfte gehalten haben.\footnotemark\\
     an interesting      talk    the Sascha might  hold     have\\
\footnotetext{
\citep[\page 99]{Fanselow87a}.
}
\glt `Sascha probably gave an interesting talk.'
}
\zl

\noindent
However, examples like (\mex{1}) -- which are quoted from
\citew[\page 72]{BC2010a} and \citew[\page 371]{Bildhauer2011a}, respectively, -- show that it is possible in principle:

\eal
\ex
\gll [Weiterhin] [Hochbetrieb] herrscht am Innsbrucker Eisoval.\footnotemark\\
     \spacebr{}further \spacebr{}high.traffic reigns at.the Innsbruck icerink\\
\footnotetext{
\citew[\page 72]{BC2010a}
}
\glt ‘It’s still all go at the Innsbruck icerink.’
\ex Die Kinder haben eigene Familien gegründet und wohnen alle einigermaßen in der Nähe, so daß die Jubilarin ihre 19 Enkel- und 17 Urenkelkinder häufig sehen kann.\\
\gll "`[Alle] [gleichzeitig] können mich nicht besuchen, weil ich {gar nicht} so viel Platz habe"', lacht sie.\footnotemark\\
     \hspaceThis{"`[}all simultaneously can me not visit  because I not.at.all so much space have    laughs she\\      
\footnotetext{
DeReKo corpus, V99/JAN.02701. Quoted from \citew[\page 371]{Bildhauer2011a}.
}
\glt `The children raised their own families and live close enough so that the jubilarian can see her 19 grandchildren and 17 great"=grandchildren often. It is not possible that all grandchildren and great"=grandchildren visit me simultaneously
because I do not have that much space, she says laughingly.'\todostefan{check}
\zl






\subsection{Accusative objects and prepositional phrases}

In (\mex{1}), the prefield consists of a noun phrase and a prepositional phrase.
\eal 
\gll {}[Nichts] [mit derartigen Entstehungstheorien] hat es natürlich zu tun, wenn \ldots \label{bsp-nichts-mit-derartigen}\footnotemark\\
	 \spacebr{}nothing \spacebr{}with these.kind theories.of.origin has it of.course to do if\\      
\footnotetext{
        K. Fleischmann, \emph{Verbstellung und Relieftheorie}, München, 1973, p.\,72.
        quoted from \citew[\page 135]{vdVelde78a}.
        }
\glt `It has, of course, nothing to do these kinds of theories of origin, if \ldots'
\ex 
\gll {}[Zum zweiten Mal] [die Weltmeisterschaft] errang Clark 1965 \ldots\footnotemark\\
	   \spacebr{}to.the second time \spacebr{}the world.championship won Clark 1965 {}\\
\footnotetext{
        \citep*[\page 162]{Benes71}
      }\label{bsp-zum-zweiten-mal-die-Weltmeisterschaft}
\glt `Clark won the world championship for the second time in 1965.'
\ex\label{die-kinder-nach-stuttgart}
\gll {}[Die Kinder] [nach Stuttgart] sollst du bringen.\footnotemark\\
     \spacebr{}the children \spacebr{}to Stuttgart should you bring\\
\footnotetext{
        \citep[\page 81]{Engel70a}
    }
\glt `You should take the children to Stuttgart.'
\zl



\noindent
In (\ref{bsp-nichts-mit-derartigen}), we are dealing with \emph{cohesion}\footnote{%
			See \citew[\page 77]{Bech55a} for more on the term cohesion.
}: The word \emph{nichts} `nothing' is a semantic fusion of  \emph{nicht} `not'
and \emph{etwas} `something'.  \emph {etwas} is the accusative object. The \emph{mit} PP is a
complement of \emph{zu tun haben} `to do have'.
The PP \emph{zum zweiten Mal} `for the second time' in \pref{bsp-zum-zweiten-mal-die-Weltmeisterschaft} is, on the other hand,
an adjunct.


\begin{comment}
Reviewer: raus

\citet[\page 69]{Fanselow93a} diskutiert das folgende Beispiel:
\ea
In Hamburg eine Wohnung hätte er sich besser nicht suchen sollen.
\z
Bei diesem Beispiel handelt es sich aber wahrscheinlich im NP-interne Voranstellung, wie sie
\zb von \citet[\page 68]{Fortmann96a-unread-gekauft} für (\mex{1}) in Erwägung gezogen wird:
\ea
Mit der Bahn eine Reise ist nicht geplant.
\z
\citet[\page 133]{Abb94} analysiert solche Voranstellungen als DP-interne Topikalisierungen.
Er ordnet auch folgende Beispiele als umgangsprachlich möglich ein:
\eal
\ex Übermorgen das Spiel gegen Kaiserslautern würde ich gern live sehen.
\ex Der die Karten hat, der Mann, soll gleich kommen.
\ex An der Wand das Bild kommt mir bekannt vor.
\zl
Bei (\mex{0}b) sieht man besonders deutlich, daß es sich nicht um eine Mehrfachbesetzung
des Vorfelds handeln kann, da der Relativsatz ja allein nicht vorfeldfähig ist. Solche Beispiele
sollen in diesem Aufsatz nicht behandelt werden.
\end{comment}

% M89/910.39160: Mannheimer Morgen, 21.10.1989, Politik; Jürgen Möllemanns Hindernislauf
% PS: Sicher nicht ganz unwahr ist, daß Möllemann seit Mai vorsichtshalber für jede Kabinettssitzung einen Fotografen einbestellt hat, der den großen Augenblick für die Nachwelt im Bild festhalten soll.




\subsection{Accusative objects and adverbs}

In (\mex{1}), we are dealing with sentences where the accusative object occurs in initial position together with an adverb 
or an adjective used as an adverb.
\eal
\label{bsp-mehrfach-vf-adv-acc}
\ex 
\gll {}[Gezielt] [Mitglieder] [im Seniorenbereich] wollen die Kendoka allerdings nicht werben.\label{bsp-gezielt-mitglieder}\footnotemark\\
       \spacebr{}specifically \spacebr{}members \spacebr{}in pensioner.area want the Kendoka PRT not gain\\
\footnotetext{
        taz, 07.07.1999, p.\,18
      }
\glt `The kendoka are not looking to gain members specifically in the pensioner demographic.'
\ex
\gll {}[Dauerhaft] [mehr Arbeitsplätze] gebe es erst, wenn sich eine Wachstumsrate von  mindestens 2,5 Prozent über einen Zeitraum von drei oder vier Jahren halten lasse.\footnotemark\\ 
       \spacebr{}constantly \spacebr{}more jobs gives it first when \textsc{refl} a growth.rate of  at.least 2.5 percent over a time.period of three or four years hold let\\
\footnotetext{
        taz, 19.04.2000, p.\,5. %taz Nr. 6123 vom 19.4.2000 Seite 5
}\label{bsp-dauerhaft-mehr-arbeitsplaetze}
\glt `In the long run, there will only be more jobs available when a growth rate of at least 2.5 percent 
can be maintained over a period of three of four years.'	
\ex 
\gll {}[Kurz] [die Bestzeit] hatte der Berliner Andreas Klöden [\ldots] gehalten.\footnotemark\\
	 \spacebr{}briefly \spacebr{}the best.time had the Berliner Andreas Klöden {} held\\
\footnotetext{
        Märkische Oderzeitung, 28./29.07.2001, p.\,28.
}\label{bsp-kurz-die-bestzeit}     
\glt `Andreas Klöden from Berlin had briefly held the best time.'
%% \ex 
%% \gll {}[Noch entschiedener] [prädikativen Charakter] hat das Adj., wenn [\ldots]\footnotemark\\
%% 	 \spacebr{}still more.deciding \spacebr{}predicative character has the adj. if\\
%% \footnotetext{
%%         In the main text of \citew[\page 52]{Paul1919a}.
%%     }\label{bsp-praedikativen-charakter}
%% \glt  `Even more decidingly, the adjective has a predicative character, if \ldots'
\zl



In (\ref{bsp-gezielt-mitglieder}), the prefield is possibly even occupied by three elements since it is more likely that the
prepositional phrase refers to \emph{werben} `to solicit' rather than to \emph{Mitglieder} `members'. The sentence does not have the interpretation
that they want to gain `members in the pensioner demographic' but rather that the people who the advertising measures
are trying to attract are in fact seniors -- that is, they are advertising to the `demographic of seniors'.

The example (\ref{bsp-gezielt-mitglieder}) cannot be analyzed in the same way that \citet{Jacobs86a} suggested for
sentences such as (\mex{1}) since \emph{gezielt} `specifically' only has scope over \emph{werben} `to solicit' but not over the modal verb.
\ea
\label{bsp-vermutlich}
\gll {}[Vermutlich] [Brandstiftung] war die Ursache für ein Feuer in einem Waschraum in der Heidelberger Straße.\footnotemark\\
	   \spacebr{}supposedly \spacebr{}arson was the cause for a fire in a washroom in the Heidelberger Street\\
\footnotetext{
Mannheimer Morgen, 04.08.1989, Lokales; Pflanzendieb.
}
\glt `Arson was supposedly the cause of a fire in a washroom in the Heidelberger Straße.'
\z
In Jacob's analysis, \emph{gezielt} `specifically' would be connected to the rest of the sentence
and one  would therefore get a structure where the adverbial has scope over the modal verb.


\subsection{Präpositinalobjekt und Adverb}

(\mex{1}) shows an example of a fronting of an adverb together with a prepositional object:
\ea
\gll {}[Besonders] [an Profil] gewinnt Kathrin Passig allerdings in der Auseinandersetzung mit Jonathans Franzens technikkritischen Essays.\footnotemark\\
  \spacebr especially \spacebr at profile wins Kathrin Passing but in the argument with Jonathans Franzen's tecnics.critically essays\\ 
\footnotetext{
 taz 20./21.07.2019, p.\,16
}
\glt `Kathrin Passig gains profile especially in the competition with those essays of Jonathans
Franzen that are critical of technology.' 
\z



\subsection{Dative objects and prepositional phrases}

(\mex{1}) is an example of simultaneous fronting of a dative object and a prepositional object.
\begin{sloppypar}
\ea\iw{gratulieren}
\gll {}[Der Universität] [zum Jubiläum] gratulierte auch Bundesminister Dorothee Wilms, die in den fünfziger Jahren in Köln studiert hatte.\footnotemark\\
	   \spacebr{}the university \spacebr{}to.the anniversary congratulated also state.minister Dorothee Wilms who in the fifties years in Cologne studied had\\
\footnotetext{
        Kölner Universitätsjournal, 1988, p.\,36, quoted from \citew[\page 87]{Duerscheid89a}.
}
\glt `State minister Dorothee Wilms -- who studied in Cologne in the 1950s -- also congratulated the university on its anniversary.' 
\z
\end{sloppypar}


\subsection{Dative and accusative object}
\label{sec-dat-acc-vf}

The following examples are constructed examples from the literature that show that dative NPs can be
fronted together with accusative NPs:
\eal
\ex 
\gll Der Maria einen Ring glaube ich nicht, dass er je schenken wird.\footnotemark\\
	 the Maria a ring believes I not that he ever give will\\
\footnotetext{
\citew[\page 67]{Fanselow93a}.
}
\glt `I dont think that he would ever give Maria a ring.'
\ex 
\gll Ihm den Stern hat Irene gezeigt.\footnotemark\\
	 him the star has Irene shown\\
\footnotetext{
  \citew[\page 412]{Eisenberg94a}.%
}
\glt `Irene showed him the star.'
\ex 
\gll (Ich glaube) Kindern Bonbons gibt man besser nicht.\footnotemark\\
     \hspaceThis{(}I think children candy gives one better not\\
\footnotetext{
        G.\ \citew[\page 260]{GMueller98a}.
}
\glt `I think it's better not to give candy to children.'
\zl
\ea 
\gll Studenten einem Lesetest unterzieht er des öfteren.\\
     students a reading.test subjects.to he the often\\
\glt `He often makes his students do a reading comprehension test.'	
\z
(\mex{0}) is due to Anette Frank (p.\,c.\ 2002).

I discussed these sentences in \citew{Mueller2005d}. Back then I did not have any attested
examples apart from the one in (\mex{1}), which involves an idiom. 
\ea
\label{bsp-zeitgeist} 
\gll {}[Dem Zeitgeist] [Rechnung] tragen im unterfränkischen Raum die privaten, städtischen und kommunalen Musikschulen.\footnotemark\\
      \spacebr{}the.\dat{} Zeitgeist \spacebr{}account carry in.the lower.Franconian area the private, urban and communal music.schools\\
\footnotetext{
        Fränkisches Volksblatt, quoted from Spiegel, 24/2002, p.\,234.
    }
\z
But a more systematic corpus exploration by \citet{Bildhauer2011a} resulted in attested examples like the one in (\mex{1}a). (\mex{1}b) was
found by chance by Arne Zeschel.
\eal
\ex
\label{ex-dem-saft-eine-kraeftige-farbe}
\gll Dem Saft eine kräftige Farbe geben Blutorangen.\footnotemark\\
     the.\dat{} juice a.\acc{}   strong   color give blood.oranges\\
\footnotetext{
\citet{BC2010a} found this example in the \emph{Deutsches Referenzkorpus} (DeReKo), hosted at Institut
für Deutsche Sprache, Mannheim: \url{http://www.ids-mannheim.de/kl/projekte/korpora}
}
\glt `Blood oranges give the juice a strong color.'
\ex\label{bsp-ihnen-für-heute}
\gll {}[Ihnen] [für heute] [noch] [einen schönen Tag] wünscht Claudia Perez.\footnotemark\\
  \spacebr{}you.\dat{} \spacebr{}for today \spacebr{}still \spacebr{}a.\acc{} nice day wishes Claudia Perez\\
\footnotetext{
  Claudia Perez, Länderreport, Deutschlandradio.
}%
\glt `Claudia Perez wishes you a nice day.'
\zl
(\mex{1}) also from \citew[\page 369]{Bildhauer2011a} again involves an idioimatic example:
\ea
\gll {}[Den Kölnern] [einen Bärendienst] erwies nach etwas mehr als einer Stunde ausgerechnet Nationalspieler Podolski,
der wegen einer Fußblessur zunächst auf der Bank Platz nehmen musste.\footnotemark\\
       \spacebr{}the inhabitants.of.Cologne \spacebr{}a disservice did after some more than an
       hour of.all.people national.player Podolski who because.of his foot.wound initially on the bench
       seat take must\\
\footnotetext{
  \url{http://www.haz.de/Nachrichten/Sport/Fussball/Uebersicht/FC-Augsburg-gelingt-Coup-gegen-acht-Koelner},
  10.02.2010.
}
\glt `Podolski did a disservice to the Cologne team after a little more than an hour since he had to
seat himself on the bench due to a foot wound.'
\z

See (\ref{ex-multiple-nps-collocation-idiom}) and (\ref{bsp-weiterhin-derjugend}) for further examples that involve the
fronting of idiom/""collocation parts or parts of support verb constructions.

These examples show that such frontings may include two NPs and hence the syntax has to account for
such structures. This does not mean that all structures involving two fronted NPs will be predicted
to be possible. For instance frontings like the one in (\mex{1}) which I discussed in
\citew{Mueller2005d} are hardly possible without a context.
\ea[?*]{
\label{ex-maria-peter-stellt-max-vor}
\gll Maria Peter stellt Max vor.\\
     Maria Peter introduces Max \particle\\
\glt `Max introduces Peter to Maria.'
}
\z
As \citet[\page 48]{Winkler2014a} suggested the markedness of examples like (\mex{0}) is probably
due to the lack of case marking of the noun phrases. Because of this it is unclear who
introduces whom to whom. The sentence greatly improves if determiners are used with nouns since the
determiners are case marked and help to identify which noun fills which grammatical role.
\eal
\ex[?]{
\gll Die Maria dem Peter stellt der Max vor.\\
     the.\acc{} Maria the.\dat{} Peter introduces the.\nom{} Max \particle\\
\glt `Max introduces Maria to Peter.'
}
\ex[?]{ 
\gll Der Maria den Peter stellt der Max vor.\\
     the.\dat{} Maria the.\acc{} Peter introduces the.\nom{} Max \particle\\
\glt `Max introduces Peter to Maria.'
}
\zl
So, the unacceptability of (\ref{ex-maria-peter-stellt-max-vor}) may be due to processing
difficulties.\todostefan{Say something about *Ich das Wienerschnitzel habe bestellt. Should the
  discussion about NPs that was in the problem section be reflected somewhere else?}


\subsection{Instrumental prepositional phrases and temporal prepositional phrases}

In (\mex{1}), there is both a temporal prepositional phrase as well as an instrumental prepositional phrase 
in the pre-field.
\ea
\label{bsp-instrument}
\gll {}[Zum letzten Mal] [mit der Kurbel] wurden gestern die Bahnschranken an zwei Übergängen im Oberbergischen Ründeroth geschlossen.\footnotemark\\
	 \spacebr{}to.the last time \spacebr{}with the crank were yesterday the train.barriers at two crossings in Oberbergisch Ründeroth closed\\
\footnotetext{
        Kölner Stadtanzeiger, 26.04.1988, p.\,28, quoted from \citew[\page 107]{Duerscheid89a}.
}
\glt `The barriers at a train station in Ründeroth, Oberberg were closed using a crank for the last time yesterday.'
\z

I have also found many other examples for most of the types of examples discussed here. Furthermore, we find multiple fronting
with adjectives used adverbially and directional/local prepositional phrases, noun phrases in copula constructions with adverbials, 
prepositional phrases in copula constructions with adverbs, predicative conjunction phrase with adverbs, directional prepositional 
phrases with adverbs as well as local prepositional phrases with adverbs. For space considerations, not all examples have been included
here. A comprehensive discussion of the data can be found in \citep{Mueller2003b}, which appeared in \emph{Deutsche Sprache}.
Further data from newspaper can be found at
\url{http://hpsg.fu-berlin.de/~stefan/Pub/mehr-vf-ds.html}. A more systematic data collection was
done in the project A~6 of the SFB~632. The database is documented in \citew{Bildhauer2011a}. The
database is hosted at the IDS Mannheim and can be accessed via \url{http://hpsg.fu-berlin.de/Resources/MVB/}.



\subsection{Support verb constructions and idiomatic usages}
\label{sec-phraseolog}

In examples \fromto{\mex{1}}{\mex{3}}, we are dealing with support verb constructions/idiomatic usages, where
either a set phrase or some fixed lexical element has been fronted together with a complement or adjunct.
In (\mex{1}), there is an element in the prefield which is not part of the phraseologism. On the
other hand, there are only parts of a phraseologism in the prefield in example (\mex{2}). The most notable feature of the examples in 
(\mex{3}) is that more than two constituents are occupying the prefield.


\eal%
\label{pvp-fvg-idioms}%
\ex 
\gll {}[Den Kürzungen] [zum Opfer] fiel auch das vierteljährlich erscheinende Magazin \emph{aktuell}, das seit Jahren als eines der kompetentesten in Sachen HIV und Aids gilt.\footnotemark\\
	  \spacebr{}the cuts \spacebr{}to.the victim fell also the quarterly appearing magazine \emph{aktuell} which since years as one of.the most.competent in things HIV and Aids counts \\
\footnotetext{
         zitty, 8/1997, p.\,36.
}
\glt `The magazine \emph{aktuell}, which appears quarterly and has for years had a reputation as being one of the most competent when it comes to HIV and Aids, has also fallen victim to the cuts.'
\ex 
\gll  {}[Eine lange Kolonialgeschichte] [hinter sich] hat das einst britische Warenhaus Lane Crawford\footnotemark\\
	   \spacebr{}a long colonial.history \spacebr{}behind \textsc{refl} has the once British warehouse Lane Crawford \\
\footnotetext{
        Polyglott-Reiseführer "`Hongkong Macau"', München 1995, p.\,28.
      }
\glt `The former British warehouse Lane Crawford has a long colonial history behind it'
\ex %Wenn es darauf ankommt, wirkt ein CDU-Politiker, der gegen die SPD-Regierung schimpft, eben einfach überzeugender. 
\gll {}[Ernsthaft] [in Schwierigkeiten] geriet Koch deshalb nur am Anfang, als es um den drohenden Irakkrieg ging.\footnotemark\\
	  \spacebr{}seriously  \spacebr{}in difficulties came Koch therefore only at.the start when it around the threatening Iraq.war went\\
\footnotetext{
        taz, 28.01.2003, p.\,6.
   }
\glt `Koch therefore only encountered serious problems at the start when dealing with the impending Iraq war.'
\ex 
\gll {}[Ihm] [zur Seite] steht als stellvertretender Vorstandschef Gerd Tenzer.\footnotemark\\
	 \spacebr{}him \spacebr{}to.the side stands as temporary committee.boss Gerd Tenzer\\
\footnotetext{
        taz, 18.07.2002, p.\,7.
}
\glt `Gerd Tenzer is on his side as temporary head of the committee.'
\ex 
\gll Sex ist je besser, desto lauter. [Am lautesten] ["`zur Sache"'] gehe es in Köln und Düsseldorf mit einem Spitzenwert von jeweils 25\,\%.\footnotemark\\
	 Sex is the better, the louder \spacebr{}at.the loudest \hspaceThis{["`}to.the thing goes it in Cologne and Düsseldorf with a top.value of each 25\,\%\\
\footnotetext{
taz, 19.04.2000, p.\,11.% April
}
\glt `When it comes to sex: the better, the louder. The loudest when ``getting down to business'' can be found in Cologne and Düsseldorf with both topping 25\,\%.'
% Marga Reis -> Gradpartikel
% \ex {}[Erst recht] [auf Touren] brachte er sie, als das Regime im Jahr darauf den 
% Liedermacher Wolf Biermann ausbürgerte und zumeist Intellektuelle gegen 
% diese Willkür protestierten.\footnote{
%         Spiegel 44/2000, p.\,272
% }\label{bsp-erst-recht-auf-touren-brachte}
\ex 
\gll {}[Damit] [im Zusammenhang] steht auch eine Eigenschaft der paarweisen Konjunkte\footnotemark\\
	 \spacebr{}with.it \spacebr{}in.the relation stands also a property of.the in.pairs conjuncts\\
\footnotetext {
        In the main text of \citew[\page 40]{Haider88a}.
}
\glt `A property of the conjuncts in pairs is also related to this.'
\ex 
\gll {}[Endgültig] [auf den TV"=Geschmack] kam Anne Will bei den olympischen Spielen 2000.\footnotemark\\
	 \spacebr{}finally \spacebr{}on the TV"=taste came Anne Will at the Olympic Games 2000\\
\footnotetext{
        taz, 16.03.2001, p.\,12.
}
\glt `Anne Will finally got a taste of television at the 2000 Olympic Games.'
% Marga Reis sagt, das sei sowas wie ein Fokuspartikel
% \ex {}[Zunehmend] [Spaß] hat Michael Jordan mit seinen Washington Wizards: [\ldots]\footnote{
%         taz, 14.12.2001, p.\,19
% }
%
\ex 
\gll {}[Stark] [unter Druck] geriet der Pharmawert Schering.\footnotemark\\
	 \spacebr{}strong \spacebr{}under pressure came the pharmaceutical Schering\\
\footnotetext{
        taz, 28./29.09.2002, p.\,9 (dpa).
}
\glt `The pharmaceutical company Schering came under extreme pressure.'
\zl
%-------------------------------------------------------------------------------------------
\eal
\ex\label{mit-den-huehnern-ins-bett}
\gll {}[Mit den Hühnern] [ins Bett] gehen sie dort.\footnotemark\\
     \spacebr{}with the chickens \spacebr{}in.the bed go they there\\
\footnotetext{
    \citew[\page 81]{Engel70a}. Engel discusses this example in connection with \pref{die-kinder-nach-stuttgart}.
    Engel views \emph{ins Bett} and \emph{nach Stuttgart} as inner frame elements
    and notes that the ability to front a constituent with an inner frame element is restricted. Engel also
	classifies adjectives in copula constructions as inner frame elements. Fronting of adjectives with dependent
	elements behaves completely normally however.
    See Section~\ref{sec-keine-mehr-vf-pvp}.%
}
\glt `They go to bed very early there.'
% {}[Mit den Hühnern] [ins Bett] gehen sie in diesem fernen Land.\footnote{
%         \citet[p.\,192]{Engel94} schreibt zu diesem Beispiel: \emph{Gelegentlich
% werden sogar mehrere -- teilweise umfangreiche -- Satzglieder mit einer infiniten Verbform
% zusammen ins Vorfeld übernommen}. In (\ref{mit-den-huehnern-ins-bett}) liegt aber keine infinite
% Verbform vor. Auf p.\,195 wird dasselbe Beispiel zusammen mit (i) diskutiert:
%         \ea
%         Ans Meer gefahren sind wir erst im September.
%         \z
% Man kann das nur so verstehen, daß angenommen wird, daß in (\ref{mit-den-huehnern-ins-bett}) eine
% zu (i) parallele Konstruktion vorliegt. Das ist die Analyse, die ich im Abschnitt~\ref{sec-analyse-mf}
% vorschlagen werde.%
%
% Mit den Hühnern ins Bett pflegt er zu gehen. Engel82a:227
%
%}
\ex 
\gll {}[Öl] [ins Feuer] goß gestern das Rote-Khmer-Radio\footnotemark\\
	 \spacebr{}oil \spacebr{}in.the fire poured yesterday the Rote-Khmer-Radio \\
\footnotetext{
        taz, 18.06.1997, p.\,8.
}
\glt `Rote-Khmer-Radio fanned the flames yesterday'
\ex\iw{setzen!das Tüpfel aufs i $\sim$}
\gll {}[Das Tüpfel] [aufs i] setze der Bürgermeister von Miami, als er am Samstagmorgen von einer schändlichen Attacke der US-Regierung sprach.\footnotemark\\
	  \spacebr{}the dot  \spacebr{}on.the i put the mayor of Miami as he on Saturday.morning from a shameful attack of.the US-government spoke\\
\footnotetext{
        taz, 25.04.2000, p.\,3. %taz Nr. 6126 vom 25.4.2000 Seite 3
    }
\glt `On Saturday morning, the icing on the cake was when the mayor of Miami spoke of the shameful attack by the US government.'
\ex 
\gll {}[Ihr Fett] [weg] bekamen natürlich auch alte und neue Regierung [\ldots]\footnotemark\\
	  \spacebr{}their fat \spacebr{}away got of.course also old and new government\\ 
\footnotetext{
        Mannheimer Morgen, 10.03.1999, Lokales; SPD setzt auf den "`Doppel-Baaß"'. %M99/903.16159 Mannheimer Morgen, 10.03.1999, Lokales; SPD setzt auf den "Doppel-Baaß"
      }
\glt `Both the old and new governments were taken to task \ldots'
\ex 
\gll {}[Den Finger] [mitten in die Wunde] legte jetzt eine findige Gruppe Internetexperten aus Österreich: [\ldots]\footnotemark\\
	 \spacebr{}the finger \spacebr{}middle in the wound laid now a clever group internet.experts from Austria\\
\footnotetext{
        taz, 04./05.11.2000, p.\,30.
}
\glt `A clever group of internet experts from Austria have now rubbed salt into the wounds \ldots'
%Partikel steht noch rechts -> gesamte Verb kann nicht in V1 stehen
% Mit rechten Dingen geht es hingegen bei den deutschen Turnieren zu.\footnote{
%         taz, 19.07.2001, p.\,19
% }
\ex 
\gll {}[Heiß] [her] geht es dagegen beim Thema "`Kundenbewertungen"'  -- einem Herzstück der Online"=Börse.\footnotemark\\
	  \spacebr{}hot \spacebr{}to.here goes it on.the.other.hand by.the topic customer.reviews {} a centrepiece of.the online"=market\\
\footnotetext{
        Spiegel, 1/2003, p.\,123.
}
\glt `On the other hand, it gets rather heated when it comes to `customer reviews' -- a crucial part of the online market.'
\ex 
%Merkwürdig: Im Konzerthaus stieg gestern Abend eine Benefiz-Gala - "Cinema for Peace". Geladen hatte Roger Moore (Ex-007, jetzt Unicef), erwartet wurde Hollywoodpersonal à la George Clooney. Am Gendarmenmarkt aber flackert seitdem, als "Zeichen gegen den Krieg", ein Ableger der "Welt-Friedensflamme". Entzündet hat sie Christopher Lee. Und was von diesem Herrn zu erwarten ist, weiß man spätestens seit "The Two Towers".
% Absatz
\gll {}[Übles] [im Schilde] führten auch zwei mit Schußwaffen ausgestattete Maskierte, die am frühen Montagmorgen eine Kneipe in Neukölln überfielen und mit den Tageseinnahmen flüchteten.\footnotemark\\
     \spacebr{}bad.things  \spacebr{}in.the shield led also two with guns equipped masked.men who on early monday.morning a pub in Neukölln held.up and with the daily.takings fled\\
\footnotetext{
        taz berlin, 11.02.2003, p.\,20.
}
\glt `Two masked men carrying guns were also up to no good as they held up a pub in Neukölln and made off with that day's takings.'
\zl


\eal
\label{drei-und-mehr-idiomatisch}
\ex 
\gll {}[Endlich] [Ruhe] [in die Sache] brachte die neue deutsche Schwulenbewegung zu Beginn der siebziger Jahre.\footnotemark\\
	  \spacebr{}finally  \spacebr{}peace  \spacebr{}in the matter brought the new German gay.movement to beginning of.the seventy years\\ 
\footnotetext{
        taz, 07.11.1996, p.\,20.
}
\glt `The new German gay movement finally brought peace to the matter in the early 70s.'
\ex
% Den Arbeitern in der Tischlerei gefällt's, auch wenn ein ganz Junger offen zugibt, daß er "den Mann nur vom Namen her kennt und ihn Politik überhaupt nicht interessiert".
\gll {}[Wenig] [mit Politik] [am Hut] hat auch der Vorarbeiter, der sich zur Aussage hinreißen läßt, "`daß der Sausgruber das falsche anhat"'.\footnotemark\\
	 \spacebr{}little  \spacebr{}with politics  \spacebr{}on.the hat has also the foreman who \textsc{refl} to.the statement carry.away lets that the Sausgruber the wrong.one wears\\
\footnotetext{
        Vorarlberger Nachrichten, 03.03.1997, p.\,A5.
}\label{bsp-wenig-mit-politik-am-hut-hat}
\glt `The foreman also cares little about politics and got so carried away he claimed that Sausgruber was wearing the wrong thing.'
\ex
\gll {}[Wenig] [mit den aktuellen Ereignissen] [im Zusammenhang] steht die Einstellung der Produktion bei der Montlinger Firma Mega-Stahl AG auf Ende November.\footnotemark\\
	 \spacebr{}little  \spacebr{}with the recent events  \spacebr{}in relation stands the
         cancellation of.the production from the Montlingen company Mega-Stahl AG on end November\\
\footnotetext{
St. Galler Tagblatt, 26.10.2001 ; Sparsam auf bessere Zeiten wartend.
}
\glt `The suspension of production until the end of November at the company Mega-Stahl AG in Montlingen has little to do with recent events.'
% \ex 
% % Ihren frühen Arbeitszeitbeginn -- Prammer startet gegen sieben Uhr -- behält sie übrigens bei, egal, wie spät es in der vorangegangenen Nacht wurde. 
% Wenig mit Sport am Hut hat auch Unterrichtsministerin Elisabeth Gehrer (VP).
% X00/JAN.03127 Oberösterreichische Nachrichten, 25.01.2000; Sauna, Schokobananen und viel zu wenig Schlaf
% P92/AUG.24714 Die Presse, 20.08.1992; Ein Kokoschka für Bürgermeister Zilk, ein Attersee für Pasterk
% Nun hängt's, verkehrt rum, in seinem Pressebüro. Wenig mit Bildern am Hut, pardon: an der Wand, hat auch Stadtrat Johann Hatzl.
\zl
%
%-------------------------------------------------------------------------------------------
%
The examples in (\mex{1}) show that the verbal part of the idiom, i.e. the functional verbal complex,
does not necessary have to be adjacent to the fronted elements.
\eal
\label{bsp-idioms-nicht-adjazent}
\ex 
\gll {}[Öl] [ins Feuer] dürfte auch die Ausstrahlung eines Interviews gießen, das die US-Fernsehstation ABC in der vergangenen Woche mit Elián führte.\label{bsp-oel-ins-feuer-duerfte}\footnotemark\\
     \spacebr{}oil \spacebr{}in.the fire may also the broadcast of.an interview pour that the US-TV.station ABC in the last week with Elián led\\
\footnotetext{
        taz, 28.03.2000, p.\,9 %28.3.2000 Seite 9.
}
\glt `The broadcast of an interview with Elián carried out last week by the US Network ABC should also fan the flames somewhat.'
\ex\iw{kommen!in Berührung $\sim$}\label{bsp-zum-ersten-mal-mit-punk}
\gll {}[Zum ersten Mal] [persönlich] [in Berührung mit Punk und New Wave] bin ich über Leute gekommen, die in meiner Lehrklasse waren.\label{bsp-zum-ersten-mal-persoenlich-in-beruehrung-bin-ich-gekommen}\footnotemark$^,$\footnotemark\\
	   \spacebr{}the first time \spacebr{}peronsally \spacebr{}in contact with Punk and New Wave be I over people come who in my vocational apprenticeship.class were\\
\footnotetext{
        Toster in an interview in Ronald Galenza und Heinz Havemeister (eds).
        {\em Wir wollen immer artig sein \ldots{} Punk, New Wave, HipHop,
        Independent"=Szene in der DDR 1980--1990\/}, Berlin:
        Schwarzkopf \& Schwarzkopf Verlag, 1999, p.\,309.
        }
\footnotetext{
If one analyzes \emph{in Berührung kommen} as a support verb construction, then one has to view the
\emph{mit} PP as an extraposed argument of the support verb construction. As a result, one would have four constituents in the pre-field in \pref{bsp-zum-ersten-mal-persoenlich-in-beruehrung-bin-ich-gekommen}.
If one were to analyze \emph{in Berührung mit Punk und New Wave} ìn contact with Punk and New Wave'
as a single prepositional phrase, (\ref{bsp-zum-ersten-mal-mit-punk}) would still have three fronted
constituents.%
}
\glt `I first came into contact with Punk and New Wave through people in the apprenticeship class.'
\ex\label{wirklich-in-bed} 
\gll {}[wirklich] [in Bedrängnis] hatte die Konkurrenz den Texaner nämlich auch gestern nicht bringen können.\footnotemark\\
       \spacebr{}really  \spacebr{}in trouble had the competition the Texan actually also yesterday not bring could\\
\footnotetext{
        taz, 24.07.2002, p.\,19.
}
\glt `In fact, the competition couldn't pin the Texan into a corner yesterday either.'
\ex\label{ein-bisschen-wasser-in-den-wein}
\gll Allerdings: [Ein bißchen Wasser] [in den Wein] muß ich schon gießen, [\ldots]\footnotemark\\
     nevertheless  \spacebr{}a bit water  \spacebr{}in the wine must I PRT pour\\
\footnotetext{
%obwohl ich nicht die Probleme mit dem Empfang habe, von dem (sic!) die Hauptstadtpresse schon nach einem Tag in fetten Schlagzeilen zu berichten weiß: [\ldots]
taz, 05.03.2003, p.\,18.
}
\glt `Nevertheless, I will have to add a bit of water to the wine.'
\zl
In the examples in (\mex{0}), the finite verb is a modal verb or a perfect auxillary verb. (\mex{1})
presents an example with the phraseologism \emph{eine gute Figur machen} `to cut a fine figure', where the finite verb occupies
the left sentential bracket but is, however, not adjacent to \emph{Figur} but rather separated from it by
the heavy \emph{bei} prepositional phrase.
\ea
\gll {}[Die beste Figur] [beim ersten Finalspiel um die Basketball"=Meisterschaft in der Berliner Max"=Schmeling"=Halle] machte ohne Zweifel Calvin Oldham.\footnotemark\\
	   \spacebr{}the best figure \spacebr{}at.the first final.game for the basketball"=championship in the Berlin Max"=Schmeling"=Halle made without doubt Calvin Oldham.\\
\footnotetext{
       taz, 22.05.2000, p.\,17.
     }
\glt `It was Calvin Oldham who, without doubt, made the biggest impression during the first round of the final of the basketball championship in the Max Schmeling Halle in Berlin.'
\z
(\ref{bsp-gezielt-mitglieder}) and %(\ref{bsp-erstmals-in-Hongkong}),
(\ref{bsp-kurz-die-bestzeit}) are examples of multiple fronting without idioms where the verb on which the
constituents are dependent is not in initial position.
Analyses which assume that multiple fronting is only possibly when the verb on which the constituents
are dependent is in initial position, are therefore inadequate.

The examples in (\mex{1}) show that it is certainly possible for two noun phrases to occupy the pre-field.
\eal
\label{ex-multiple-nps-collocation-idiom}
\ex
\label{bsp-zeitgeist-zwei} 
\gll {}[Dem Zeitgeist] [Rechnung] tragen im unterfränkischen Raum die privaten, städtischen und kommunalen Musikschulen.\footnotemark\\
      \spacebr{}the Zeitgeist \spacebr{}account carry in.the lower.Franconian area the private, urban and communal music.schools\\
\footnotetext{
        Fränkisches Volksblatt, quoted from Spiegel, 24/2002, p.\,234.
    }
\glt `The private, urban and communal music schools in the lower Franconian area account for the
\emph{zeitgeist}.'
\ex
\gll [Dem Frühling] [ein Ständchen] brachten Chöre aus dem Kreis Birkenfeld im Oberbrombacher Gemeinschaftshaus.\footnotemark\\
\hspaceThis{[}to.the spring \hspaceThis{[}a little.song brought choirs from the county Birkenfeld
    in.the Oberbrombach municipal.building\\
\footnotetext{
 \sigle{RHZ02/JUL.05073}.
}
\glt `Choirs from Birkenfeld county welcomed (the arrival of) spring with a little song in the Oberbrombach municipal building.'\label{fruehling}
\ex
\gll [Dem Ganzen] [ein Sahnehäubchen] setzt der Solist Klaus Durstewitz auf\footnotemark\\
     \hspaceThis{[}to.the everything \hspaceThis{[}a little.cream.hood puts the soloist Klaus
         Durstewitz on\\
\footnotetext{
 \sigle{NON08/FEB.08467}.
}
\glt `Soloist Klaus Durstewitz is the cherry on the cake.'\label{bsp-sahne}
\zl
See also (\ref{ex-dem-saft-eine-kraeftige-farbe}) for a non-idiomatic example.


\subsection{Fronting of three or more constituents}
\label{sec-fronting-more-than-two}

\citet[\page 11]{Luehr85a}\iadata{Lühr} presents examples with more than two fronted elements:\footnote{
	She also discusses other combinations of elements in the prefield which occur in Feuchtwanger's texts.
	She arrives, however, at the conclusion that the order of elements is a conscious style choice on the part
	of the author designed to mirror camera movements in films. The examples are rather deviant in standard German.

\citet{Lee75a} discusses several examples from Kleist where sometimes up to four constituents have been fronted.%
}

\eal
\ex
\gll Im Schnellzug, nach den raschen Handlungen und Aufregungen der Flucht und der 
      Grenzüberschreitung, nach einem Wirbel von Spannungen und Ereignissen, Aufregungen
      und Gefahren, noch tief erstaunt darüber, daß alles gut gegangen war, sank Friedrich
      Klein ganz und gar in sich zusammen.\footnotemark\\
      in.the express.train after the swift action and excitement of.the escape and the border.crossing after a whirlwind of
tensions and events commotions and danger still deeply shocked about that all good gone was sank Friedrich Klein whole and done
in REFL together\\
\footnotetext{
        Herman Hesse. Klein und Wagner. In {\em Gesammelte Werke Band 5\/}. Frankfurt/M. 1970.
}
\glt 'In the express train, following the swift events and action of the escape and the border crossing, after a whirlwind of tensions and events, commotion and danger and still deeply shocked that everything turned out well, Friedrich Klein slumped down completely into himself.'
\ex Mit seinen großen Buchstaben, quer über die letzte Schreibmaschinenseite des Gesuches,
      langsam mit rotem Stift malt Klenk: "`Abgelehnt K."'.\footnote{
        Lion Feuchtwanger. \emph{Erfolg. Drei Jahre Geschichte einer Provinz}. Frankfurt/M. 1981, p.\,114.
}\todoandrew{Übersetzung fehlt}
\zl

See also (\ref{bsp-gezielt-mitglieder}) for a further example with more than two elements in the prefield. The examples in
(\ref{drei-und-mehr-idiomatisch}) constitute idiomatic usages (support verb constructions) which also have more than two fronted
constituents.

The following examples are taken from newspapers:\footnote{
  I thank Felix Bildhauer\aimention{Felix Bildhauer} for these examples.
}
\eal
\ex\label{bsp-ebenfalls-positiv} 
\gll {}[Ebenfalls] [positiv] [auf die Kursentwicklung] wirkte sich die Ablehnung einer Zinserhöhung durch die
Bank of England aus.\footnotemark\\
\spacebr{}also \spacebr{}positive \spacebr{}on the market.trend affected \refl{} the rejection of.a
rate.hike by the Bank of England \partic\\
\footnotetext{
Tiroler Tageszeitung, 18.05.1998, Ressort: Wirtschaft; Frankfurt in fester Verfassung; I98/MAI.19710.
}
\glt `The rejection of a rate hike by the bank of England also affected the market trend positively.'
\ex 
\gll {}[Zum ersten Mal]    [ein Trikot]    [in der Bundesliga]    hat    Chen Yang angezogen, und zwar das des Aufsteigers Eintracht Frankfurt.\footnotemark\\
    \spacebr{}to.the first time \spacebr{}a jersey in the Bundesliga has Chen Yang on.put  und
    namely that the promoted.team Eintracht Frankfurt\\
\footnotetext{
Frankfurter Rundschau, 24.08.1998, S. 13, Ressort: FRANKFURTER, R98/AUG.67436.
}
\glt `Chen Yang put on a jersey in the Bundesliga for the first time, namely one of the jerseys of
the promoted team Eintracht Frankfurt.'
\ex\label{bsp-weiterhin-derjugend}
\gll {}[Weiterhin]    [der Jugend]    [das Vertrauen]    möchte    man beim KSK Klaus schenken.\footnotemark\\
     \spacebr{}still  \spacebr{}the.\dat{} youth \spacebr{}the.\acc{} trust wants one at.the KSK Klaus give.as.a.present\\
\footnotetext{
Vorarlberger Nachrichten, 26.09.1997, S. C4, Ressort: Sport; Die Ländle-Staffeln wollen Serie halten,    V97/SEP.48951.
}
\glt `People at the KSK Klaus want to continue to trust the youth.'
\zl

(\ref{bsp-ihnen-für-heute}) -- repeated here as (\mex{1}) for convenience -- is the most extreme example I know of with four constituents before the finite verb:\footnote{
  I thank Arne Zeschel\aimention{Arne Zeschel} for this example.
}
\ea\label{bsp-ihnen-für-heute-zwei}
\gll {}[Ihnen] [für heute] [noch] [einen schönen Tag] wünscht Claudia Perez.\footnotemark\\
  \spacebr{}you.\dat{} \spacebr{}for today \spacebr{}still \spacebr{}a.\acc{} nice day wishes Claudia Perez\\
\footnotetext{
  Claudia Perez, Länderreport, Deutschlandradio.
}%
\glt `Claudia Perez wishes you a nice day.'
\z



\subsection{Non-cases of multiple fronting}
\label{sec-keine-mehr-vf-pvp}

This section explores examples that were discussed in connection with multiple frontings but behave
different in important respects. Subsection~\ref{sec-pvp-is-not-mf} deals with complex \vfs that
include a verb, Subsection~\ref{sec-left-dislocation-hanging-topic} deals with left dislocation and
hanging topic and Subsection~\ref{sec-np-internal-frontings} deals with NP-internal
frontings.\todostefan{Selting93a: Linksversetzung = Mehrfache Vorfeldbesetzung}

\subsubsection{Partial verb phrase fronting}
\label{sec-pvp-is-not-mf}

In connection with cases of multiple fronting, certain examples have been discussed with supposed
cases of fronted nonfinite verbs or adjectives as well as elements dependent on them
\citep{VogelgesangDoncer2004a}. Examples of this kind of fronting are shown in (\mex{1}): 
\eal
\label{bsp-pvp}
\ex 
\gll Besonders Einsteigern empfehlen\iw{empfehlen} möchte ich Quarterdeck Mosaic, dessen gelungene grafische Oberfläche und Benutzerführung auf angenehme Weise über die ersten Hürden hinweghilft, obwohl sich die Funktionalität auch nicht zu verstecken braucht.\footnotemark\\
     especially beginners recommend want.to I Quarterdeck Mosaic whose well.designed graphic surface and user.interface on pleasant way over the first hurdles help.over although \textsc{refl} the functionality also not to hide needs\\
\footnotetext{
        c't, 9/1995, p.\,156.
}\label{bsp-besonders-einsteigern}
\glt `I would particularly recommend Quarterdeck Mosaic for beginners due to its well-designed graphic surface and user interface, which can give a helping hand over those first few hurdles. This should not, however, raise any doubts about its functionality.'
\ex 
\gll Der Nachwelt hinterlassen\iw{hinterlassen} hat sie eine aufgeschlagene \emph{Hör zu} und einen kurzen Abschiedsbrief: \ldots\footnotemark\\
     the afterworld left has she an opened \emph{Hör zu} and a short suicide.note:\\
\footnotetext{
        taz, 18.11.1998, p.\,20.
}\label{bsp-der-nachwelt-hinterlassen}
\glt `She left the rest of the world an open copy of \emph{Hör zu} and a short suicide note.'
\ex 
\gll Viel anfangen\iw{anfangen mit} konnte er damit nicht.\footnotemark\\
	 much begin could he with.it not\\
\footnotetext{
        Wochenpost, 41/1995, p.\,34.
        }
\glt `It was lost on him.'
\ex 
\gll Bei der Polizei angezeigt\iw{anzeigen} hatte das Känguruh ein Autofahrer, nachdem es ihm vor die Kühlerhaube gesprungen war und dabei fast angefahren wurde.\footnotemark\\
	 at the police reported had the kanagroo a motorist after it him before the bonnet jumped was and there.by almost run.over was\\
\footnotetext{
        taz, 18./19.01.1997, p.\,32.
      }
\glt `A motorist informed the police of the kangaroo after it had jumped in front of his car and was nearly hit.'
\ex 
\gll Aktiv am Streik beteiligt\iw{beteiligen} haben sich "`höchstens zehn Prozent"':\footnotemark\\
     active on.the strike took.part have \textsc{refl} \hspaceThis{"`}at.most ten percent\\%
\footnotetext{    
        taz, 11.12.1997, p.\,7.
}\label{bsp-aktiv-am-streik}%
\glt `Only a ``maximum of ten percent'' actively took part in the strike action:'
\zl



These kinds of constructions have been investigated extensively and there is now some consenus about the fact that
there is exactly one constituent present in the prefield.\todostefan{references}
However, one also finds suggestions like Gunkel's \citeyearpar[\page 170--171]{Gunkel2003b} to analyse sentences such as (\mex{1}) as verb-third
clauses with a flat structure. He does not, however, offer any explanation for the linearization contraints
for such clauses. If one were to analyze examples such as (\mex{1}) with a completely flat structure and with
three fronted constituents, it is not possible to explain why the constituents preceding the finite verb
act as if they also contained a middlefield, right verbal bracket and a postfield.
\ea
\gll Den Kunden sagen, daß die Ware nicht lieferbar ist, wird er wohl müssen.\\
	 the customer say that the product not deliver.able is will he PRT must\\
\glt `He will presumably have to tell the customers that the product cannot be delivered.' 
\z
On the other hand, if one assumes that the three constituents preceding the finite verb form
a verbal projection, then the individual elements in the verbal projection can be assigned to
topological fields and the order of the constituents do not require any special explanation.
See \citew[\page 82]{Reis80a}.
	
Regardless of the question whether the words preceding the finite verb have constituent status
\citep{Kathol95a} or not \citep{Gunkel2003b}, analyses which attempt to explain (\mex{0}) via local reordering
cannot account for examples such as (\mex{1}).
\eal
\ex 
\gll Das Buch gelesen glaube ich nicht, dass er hat.\footnotemark\\
     the book read believe I not that he has\\
\footnotetext{
\citep[\page 82]{Sabel2000a}.
}
\glt `I don't think that he has read the book.'
\ex 
\gll Angerufen denke ich, daß er den Fritz nicht hat.\footnotemark\\
     called think I that he the Fritz not has\\
\footnotetext{
\citep{Fanselow2002a}.
}
\glt `I don't think he has called Fritz.'
\zl
In (\mex{0}), we have elements preceding the finite verb which clearly originate in the
embedded clause and therefore cannot have reached their current position by local reordering.%
\todostefan{gb4e should use (i) in footnotes}



I have shown in \citew[\page 93--94]{Mueller2002b} that the fact that \emph{den Wagen} `the car' in (\mex{1}) bears
accusative case could not be explained if one had two independent constituents in the prefield. 
\eal 
\ex[]{
\gll Den Wagen zu reparieren wurde versucht.\\
	 the.\acc{} car to repair was tried\\
\glt `They tried to repair the car.'\todoandrew{geht denn für a: It was tried to repair the car.}%
\footnote{
      The original German sentence is actually a passive: `The car was tried to be repaired'.
	  As such a construction is impossible in English, I have translated it with an active sentence.%
}}
\ex[*]{
\gll Der Wagen     zu reparieren wurde versucht.\\
	 the.\nom{} car   to repair was tried\\
}
\zl
In constructions with the so-called `remote passive'\is{passive!remote}, the object can most certainly appear in the
nominative as is shown by (\mex{1}a).\footnote{
       Evidence for the long-distance passive from corpora can be found in \citew[\page
         136--137]{Mueller2002b} and in \citew{Wurmbrand2003a}.%
}
It is clear from (\mex{1}b) that it is possible to front the nominative NP on its own.
\eal
\ex[]{
\gll weil der Wagen zu reparieren versucht wurde\\
	 because the.\nom{} car to repair tried was\\
\glt `because they tried to repair the car'
}
\ex[]{
\gll Der Wagen wurde zu reparieren versucht.\\
	 the.\nom{} car was to repair tried\\
\glt `They tried to repair the car.'
}
\zl
The infintival construction with \emph{zu} can also be fronted on its own as shown in (\mex{1}):
\ea
\gll Zu reparieren wurde der Wagen versucht.\\
	 to repair was the car tried\\
\z
The NP \emph{der Wagen} has to bear nominative case in this kind of construction. If (\mex{-2}) were an example of fronting
the infintive and the noun phrase as a single constituent, we would also expect the nominative to be possible here, which
is in fact not what we observe.

%\subsection{Prädikative Adjektive und Adverbialien}


% Hier handelt es sich wohl um einen Spezifikator
% \ea
% Am ehesten börsentauglich dürfte der Fernverkehr sein, der schon heute auf einigen
% Strecken gute Gewinne einfährt.\footnote{
%         taz, 16.10.1999, p.\,9
% }
% }

% Im folgenden Beispiel scheint sich die PP allerdings auf das Verb \emph{zeigen} und nicht auf das
% Adjektiv zu beziehen:
% \ea
% {}[Kauf"|freudig] [im betrachteten vierten Quartal] zeigten sich aber auch Azubis/""Zivildienstleistende (21,2 Prozent)
% und Angestellte (19,6 Prozent).\footnote{
%         c't, 1/2003, p.\,69, Umfrage zum PC-Kauf zu Weihnachten 2002
% }
% \z





\subsubsection{Left dislocation and free topics}
\label{sec-left-dislocation-hanging-topic}
              
Other authors have discussed examples with left dislocation or `free topics' as cases of multiple
fronting.\todostefan{R1: add examples} These kinds of movement
have been discussed in detail by \citet{Altmann81a}. I assume that left-dislocated constituents and free topics do not move to the
prefield, but rather -- as suggested by \citet[\page 329]{Hoehle86} -- that they occupy another topological position. For this reason,
they are not relevant to the present discussion.

\subsubsection{NP-internal frontings}
\label{sec-np-internal-frontings}


\citet[\page 456]{Speyer2008a} treats examples like those in (\mex{1}) as instances of multiple fronting:
\ea
\label{ex-inzuepfners-box-der-mercedes}
\gll {}[[In Züpfners Box] [der Mercedes]] bewies, dass Züpfner zu Fuß gegangen war.\footnotemark\\
       \hspaceThis{[[}in Züpfners box \spacebr{}the Mercedes proofed that Züpfner by foot went was\\
\footnotetext{
Böll, Heinrich (1963): \emph{Ansichten eines Clowns}. Köln: Kiepenheuer \& Witsch. Quoted from
\citew[\page 456]{Speyer2008a}.
}
\glt `The Mercedes in Züpfners box was proof of Züpfner's walking.'
\z

A similar example is also discussed by \citet[\page 69]{Fanselow93a} in the context of multiple frontings:
\ea
\gll In Hamburg eine Wohnung hätte er sich besser nicht suchen sollen.\\
     in Hamburg a    flat    had  he \self{} better not search should\\
\glt `It would have been better for him not to rent/buy a flat in Hamburg.'
\z
I exclude these examples from the present discussion since they are probably best analyzed as
NP-internal frontings as suggested for instance by \citet[\page 68--69]{Fortmann96a-u} for (\mex{1}):
\ea
\gll Mit der Bahn eine Reise ist nicht geplant.\\
     with the train a journey is not planned\\
\glt `A journey by train is not planned.'
\z
\citet[\page 133]{Abb94} also treats such examples as DP-internal frontings. He remarks that the
following examples are possible in colloquial speech:
\eal
\ex 
\gll Übermorgen das Spiel gegen Kaiserslautern würde ich gern live sehen.\\
     the.day.after.tomorrow the game against Kaiserslautern would I like.to live see\\
\glt `I would like to see the game against Kaiserslautern tomorrow live.'
\ex 
\gll Der die Karten hat, der Mann, soll gleich kommen.\\
     who the tickets has the man shall soon come\\
\glt `The man with the tickets is supposed to come soon.'
\ex 
\gll An der Wand das Bild kommt mir bekannt vor.\\
     on the wall the picture comes me known \particle\\
\glt `I think I know the picture on the wall.'
\zl
The example (\mex{0}b) clearly shows that an analysis like Speyer's \citeyearpar{Speyer2008a} would
fail on such sentences since relative clauses cannot be fronted independent of the noun they modify:
\ea[*]{
\gll Der die Karten hat, soll der Mann gleich kommen.\\
     who the tickets has shall the man soon come\\
\glt Intended: `The man with the tickets is supposed to come soon.'
}
\z

% Na irgendwie gibt es ja doch Subjekte.
%Note also that subjects in apparent multiple frontings are rather rare (see Subsection~\ref{sec-subj-mf}).


\subsection{Impossible multiple frontings (Same Clause Constraint)}
\label{sec-ausgeschlossen-mvf}\label{sec-clause-mates}

As noted by Fanselow \citeyearpar[\page 99]{Fanselow87a}; \citeyearpar[\page 67]{Fanselow93a}, the constituents preceding the finite verb have to belong to
the same clause. Simultaneous fronting of several constituents from different clauses is not possible:
\eal
\label{ex-mult-front-same-verb}
\ex[]{
\gll Ich glaube  dem Linguisten nicht, einen Nobelpreis  gewonnen zu haben.\\
     I   believe the linguist   not    a     Nobel.prize won      to have\\
\glt  `I don't believe the linguist's claim that he won a Nobel prize.'
}
\ex[*]{
\gll Dem Linguisten einen Nobelpreis  glaube  ich nicht gewonnen zu haben.\\
     the linguist   a     Nobel.price believe I   not   won      to have\\
}
\ex[]{
\gll Ich habe den Mann gebeten, den Brief  in den Kasten zu werfen.\\
     I   have the man  asked    the letter in the box    to throw\\
\glt `I asked the man to post the letter in the letterbox.'
}
\ex[*]{
\gll Den Mann in den Kasten habe ich gebeten, den Brief  zu werfen.\\
     the man  in the box    have I asked      the letter to throw\\
}
\zl
This observation was verified with 3.200 examples of apparent multiple fronting that were collected by \citet{Bildhauer2011a}
in the DFG project \emph{Theorie und Implementation einer Analyse der Informationsstruktur im Deutschen unter besonderer Berücksichtigung der linken Satzperipherie} (MU 2822/1-1 and SFB 632, A6).



\subsection{Multiple frontings of idiom parts and restrictions on separate frontings}
\label{sec-idiom-parts-mf}

Many of the examples in \pref{pvp-fvg-idioms} support the claim that multiple fronting is actually fronting of a single
projection which contains part of the predicate complex. If we were to assume -- as in \citew{Mueller2000d} -- that in these
cases two independent constituents have been fronted, we would also have to assume that each of these constituents can be fronted
individually, which would be difficult to reconcile with the ungrammaticality of (\mex{1}):

\eal
\ex[*]{
\gll Ins Feuer goß gestern das Rote-Khmer-Radio Öl.\\
     in.the fire poured yesterday the Rote-Khmer-Radio oil\\
}
\ex[*]{
\gll Aufs i setze der Bürgermeister von Miami das Tüpfel, als er am Samstagmorgen von einer schändlichen 
Attacke der US-Regierung sprach.\\
     \spacebr{}on.the i put the mayor of Miami the dot  as he on Saturday.morning from a shameful attack of.the US-government spoke\\
}
\ex[*]{
\gll Weg bekamen natürlich auch alte und neue Regierung ihr Fett.\\
     away got of.course also old and new government their fat\\ 
}
% Die gehen, wenn man die Adverbien ganz wegläßt.
% \ex[*]{
% Unter Schock stehen (schwer) deshalb (schwer) zur Zeit (schwer) zwei der hervorragendsten Kräfte ihrer Branche (schwer).
% }
% \ex[*]{
% Ins Gericht gehen (hart) die Wirtschaftsforscher (hart) zudem (hart) mit der sogenannten Fluggastgebühr (hart).
% }
\ex[*]{
\gll Rechnung tragen im unterfränkischen Raum die privaten, städtischen und kommunalen Musikschulen dem Zeitgeist.\\
     account carry in.the lower.Franconian area the private, urban and communal music.schools the Zeitgeist\\
}
\zl
One would have to formulate complex constraints which would ensure that, for example,
\emph{Rechnung} `account' could
only be fronted if \emph{dem Zeitgeist} `the Zeitgeist' were also fronted. All in all, this sort of explanation would turn out
to be more complicated than one which assumes that part of a predicate complex is fronted.

\subsection{Scope of negation and fronting}

Furthermore, Fanselow notes that negation has scope over everything preceding 
the finite verb.
\eal
\ex  
\gll Nicht der Anna einen Brief hätte er schreiben sollen, sondern der Ina eine Postkarte.\\
     not the Anna a letter had he write should rather the Ina a postcard\\
\glt `He shouldn't have sent Anna a letter, but rather Ina a postcard.'
\ex 
\gll Nicht am Sonntag einen Brief hätte er schreiben sollen, sondern am Samstag seinen Vortrag für Potsdam.\\
     not on Sunday a letter had he write should rather on Saturday his presentation for Potsdam.\\
\glt `He shouldn't have written a letter on Sunday, he should have written his presentation for Potsdam on Saturday.'
\zl

The data discussed here can be easily accounted for if one assumes that the fronted elements are arguments of
an empty head or that they modify some kind of empty head. This null head has the properties of a verb in the
remaining sentence, which explains the fact that the fronted constituents cannot be dependents of different verbs.
Corresponding suggestions in this direction have been made by \citet{Fanselow93a} and \citet[\page 1634]{Hoberg97a},
although they did not work out the details of these suggestions.

As the examples in (\mex{0}) show, the negation cannot be analyzed as constituent negation. It
follows that \emph{nicht} is a separate constituent in (\mex{0}) and not part of an NP. Again this is entirely
unproblematic in approaches that assume that \emph{nicht} is part of a larger verbal constituent
\emph{nicht der Anna einen Brief} `not the Anna a letter'.

%% Note also that fronting of negation alone is rather restricted, but as \citet{Ulvestad75a} showed in
%% a careful corpus study, attested examples can be found.
%% \ea[*]{
%% \gll Nicht ging Peter einkaufen.\\
%%      not went Peter shopping\\
%% \glt `Peter didn't go shopping.'
%% }
%% \z
%% The following two examples were provided by \citet{Reis80a} and
%% \citet{Hoberg81a} and the examples in (\mex{2}) are attested examples that show that the fronting of
%% the negation is possible. The examples (\ref{ex-nicht-weht}) and (\ref{ex-nicht-etwa-haben}) can
%% also be found in \citew[\page 348]{Mueller99a}.
%% \eal
%% \ex Das alles erwähnte der Autor. Nicht hat er hingegen berücksichtigt,
%%               daß \ldots{}\footnote{
%% \citep*[\page 72]{Reis80a}
%% }
%% \ex Nicht ahnten wir, daß Franz von Papen das Spiel der Intrigen fortspann 
%%               -- \ldots{}\footnote{
%%   \citep*[\page 161]{Hoberg81a}
%% }
%% \zl
%% \eal
%% \ex Nicht aber ist der abtrennbare Teil des Verbs auch stets ein Satzglied.\footnote{
%%  Im Haupttext von \citep[\page 365]{Stechow79}.
%% }
%% \ex Noch nicht erhalten w[i]r dagegen (2).\footnote{
%% Im Haupttext von \citep[\page 440]{Stechow79}.
%% }
%% %       \item Vorläufig noch nicht schaffe ich allerdings zwei Lesarten für
%% %       "`Jedermann liebt jemanden"'.    (Im Haupttext von
%% %              \citep[S.\,464]{Stechow79})     
%% \ex Du bist, wie ich sehe, ein Mann des Buches, und nicht frommt es dir, dem Einsamen,
%%         obdachlos in Bettlerkleidung zu stromern.\footnote{
%% Michail Bulgakow, \emph{Der Meister und
%%           Margarita}. München: Deutscher Taschenbuch Verlag. 1997, S.\,420}

%% \ex\label{ex-nicht-weht} Der Wind ist auch mal aus der Puste, aber nicht weht er nur selten.\footnote{
%% taz, taz-mag, 07./08.98, S. 16. als Antwort auf die Frage: "`Was macht
%%                der Wind, wenn er nicht weht?"'
%% }
%% \ex\label{ex-nicht-etwa-haben} In der Logik dieses Modells liegt, daß die von Tovee befragten Studenten
%%               alle aus reinem Fortpflanzungsinteresse denselben "`Playmate"'-Körperbau
%%               bevorzugen. Nicht etwa haben Modeplakate oder Hochglanzmagazine den Blick
%%               der Männer auf diese Modelmaße geeicht.\footnote{
%% Spiegel 47/98, S.\,238}

%% \ex Die Athleten jedoch haben eine feine Antenne für das, was
%%                 in diesem Verband geht, und auch für das, was nicht geht.
%%               Nicht geht ein in welcher Form auch immer geratenes Mitspracherecht
%%               ging aus der Einaldung hervor: [\ldots]\footnote{
%% taz, 25.09.2003, S.\,13}

%% \ex Auch nicht hilft die später lancierte Behauptung, die SPD-Linken hätten ihn zu sehr unter Druck
%% gesetzt - die beteuern ihm nun seit Wochen ihr allergrößtes Vertrauen.\footnote{
%% taz, 08.06.2005, S.\,6}
%% \zl



\subsection{The order of fronted constituents}
\label{sec-abfolge}

As was noted by \citealp[\page 6--7]{Luehr85a}, \citet[\page 412--413]{Eisenberg94a}, and \citet[\page
  1625--]{Hoberg97a},\todostefan{Check Lühr and Hoberg} the order of the fronted constituents is relatively fixed. If the order of the
elements in \pref{bsp-alle-traeume-gleichzeitig} and \pref{bsp-dauerhaft-mehr-arbeitsplaetze} is
changed as in the following examples, the result is sentences that are degraded in aceptability:
\eal
\ex[?*]{
\gll Gleichzeitig   alle Träume lassen sich       nur selten verwirklichen.\\
     simultaneously all  dreams let    themselves only seldom realize\\
\glt `Very rarely, all dreams can be realized simultaneously.'
}
%% \ex[*]{
%% Nach der Zugehörigkeit Personen bezeichnen auch \emph{Gesellschafter}, \emph{Gewerkschafter} \ldots.
%% }
\ex[?*]{
\gll Mehr Arbeitsplätze dauerhaft   gebe es erst, wenn \ldots.\\
     more jobs          permanently gives it first when \\
}
\zl
The observation that the order in apparent multiple frontings corresponds to the unmarked order in
the \mf was verified with 3.200 examples of apparent multiple fronting that were collected by \citet{Bildhauer2011a}
in the DFG project \emph{Theorie und Implementation einer Analyse der Informationsstruktur im
  Deutschen unter besonderer Berücksichtigung der linken Satzperipherie} (MU 2822/1-1 and SFB 632,
A6).\footnote{
  The database is available at \url{https://clarin.ids-mannheim.de/SFB632/A6}.
}


These differences can be explained if one assumes that there is a single verbal projection (the
projection of a single verbal head) present in the prefield. The verbal projection contains a
middle-field, right verbal bracket occupied by the empty head, and even a postfield in certain
cases. The order of the fronted elements is therefore subject to the same restrictions that are
known for the ordering of elements in the middle-field/postfield:

\eal
\ex[]{
\gll weil sich nur selten alle Träume gleichzeitig verwirklichen lassen\\
	 because \textsc{refl} only seldom all dreams simultaneously realise let\\
\glt `because only seldom can all of your dreams be realised at the same time'
}
\ex[??]{
\gll weil sich nur selten gleichzeitig alle Träume verwirklichen lassen\\
     because \textsc{refl} only seldom simultaneously all dreams realise let\\
}
\zl
\eal
\ex[]{
\gll weil    es dauerhaft mehr Arbeitsplätze erst gebe, wenn \ldots.\\
     because it constantly more jobs PRT give if\\
\glt `because there will only be a constant supply of jobs if/when \ldots'
}
\ex[?*]{
\gll weil es mehr Arbeitsplätze dauerhaft erst gebe, wenn \ldots.\\
     because it more jobs constantly PRT give if\\
}
\zl



\subsection{Summary of the data discussion}

I have shown that various kinds of constituents can co-occur in the prefield: arguments, adjuncts and predicatives
can be fronted together with another constituent. The number of constituents preceding the finite verb is by no
means limited to two.

The sequence of the fronted elements corresponds to the order the constituents would have in the middle-field.
This supports an analysis which assumes that multiple fronting involves a complex verbal projection, which contains its own
topological fields: middlefield, right verbal bracket and postfield. The right verbal bracket is occupied by a silent
verbal head.

I showed that multiple fronting with idioms is quite common and that certain parts of phraseologisms cannot be
fronted individually. The constituent parts of a phraseologism can be realised inside this projection, but individual fronting
is not possible.

The observation that only elements from the same clause can be fronted together can also be explained by the assumption of a silent
verbal head.
 





\section{The analysis}
\label{sec-analyse-mf}\label{sec-analyse-mf-vn}\label{sec-vn}

A prerequisite for the analysis of apparent multiple frontings are the following
sub-analyses: 1) an analysis of V1-order derived by verb movement, 2) an analysis of the verbal
complex by means of argument attraction and 3) an analysis of fronting as a long-distance dependency.
These three ingredients have already been provided in Chapter~\ref{chap-german-sentence-structure}
and I will show in Subsection~\ref{sec-mult-front-complex-formation} how they interact in the
analysis of apparent multiple frontings. Section~\ref{sec-left-dislocation} discusses a potential problem with left
dislocation, Subsection~\ref{sec-extraposition} talks about extraposition in complex prefields and Subsection~\ref{sec-unwanted-traces}
deals with traces in unwanted positions.

\subsection{Multiple frontings as lexical rule and predicate complex formation}
\label{sec-mult-front-complex-formation}

It was shown in the data discussion in Section~\ref{sec-ausgeschlossen-mvf} that elements can only be fronted together
if they are dependent on the same head/predicate complex.\footnote{
		The examples from Jacobs with sentence adverbs behave differently. It is certainly
		possible that there are cases where focus particles or sentence adverbs and a constituent
		from an embedded clause occur together before the finite verb.%
}
		
%
% Ist nicht so stichhaltig, wenn man Reanalyze annimmt.
% \citet[\page 246]{BH2001a} diskutieren die Beispiele in (i).
% \eal
% \ex[]{
% Sogar [gegen die Regierung]$_i$ hat sie [eine Proklamation \_$_i$] unterzeichnet.
% }
% \ex[*]{
% {}[Eine Proklamation sogar gegen die Regierung]$_i$ hat sie  \_$_i$  unterzeichnet.
% }
% \zl
% (i.b) zeigt, daß es nicht sinnvoll ist, anzunehmen, daß \emph{sogar} die PP \emph{gegen die Regierung}
% modifiziert.



\citet{Fanselow93a} and \citet[\page 1634]{Hoberg97a} have therefore suggested positing a silent head which
can then be combined with the arguments and adjuncts which actually belong to the verb.
In what follows, I will attempt to formalize and define this analysis more precisely.  
Like Hoberg, I assume that the silent head is a part of the predicate complex and that fronting is analogous
to partial fronting of a predicate complex.
Example (\ref{bsp-zum-zweiten-mal-die-Weltmeisterschaft}) would therefore have the following structure:
\ea
\label{ex-zum-zweiten-anal}%
\gll {}[\sub{VP} [Zum zweiten Mal] [die Weltmeisterschaft] \_\sub{V} ]$_i$ errang$_j$ Clark 1965 \_$_i$ \_$_j$.\\
     {}          \spacebr{}to.the second time  \spacebr{}the world.championship {} {} won Clark 1965\\
\z
\_$_j$ represents the movement trace, which is left behind by the verb \emph{errang} in initial position.
\_$_i$ is the trace of the extraction of \emph{zum zweiten Mal die Weltmeisterschaft}
`for the second time the world's championship',  which also binds it.
\_\sub{V} stands for the silent verbal head in the prefield.
\citet[\page 69]{Fanselow93a} suggests treating this empty head in a similar way to the empty elements present
in gapping constructions and argues against a fronting analysis with verb trace with the following examples
of particle verbs:
\eal
\ex[*]{
\gll Die Anette an sollte man lieber nicht mehr rufen.\\
	 the Anette on should one rather not more call\\
\glt Intended: `It's probably better if you don't call Anette.'
}
\ex[*]{
\gll Mit dem Vortrag auf sollte er lieber hören.\\
	 with the presentation on should he rather stop\\
\glt Intended: `It would be better if he were to end his presentation.'
}
\ex[*]{
\gll Dem Minister einen Aufsichtsratsposten zu hätte er niemals schanzen sollen.\\
	 the minister a supervisory.board.post to had he never ensure should\\
\glt Intended: `He should have never made sure that the minister got a position on the supervisory board.'
}
\zl
Fanselow argues that an analysis which treats fronting of multiple constituents as including movement of a 
corresponding trace should predict that the sentences in (\mex{0}) are grammatical.\footnote{
		Although see \citew{Fanselow2003c} for an analysis of particle fronting
		as pars"=pro"=toto movement.%
}
As these sentences are clearly ungrammatical, Fanselow assumes that these kinds of movement analyses are not
adequate. However, the following examples in (\mex{1}) show that particles can indeed occur with other constituents
in the prefield.

\eal\label{ex-adv-particle}
\ex\iw{zurechtkommen}%
\gll Gut \emph{zurecht} \emph{kommt} derjenige, der das Leben mit all seinen Überraschungen annimmt und dennoch verantwortungsvoll mit sich umgeht.\label{ex-gut-zurecht}\footnotemark\\
	 good to.right comes the.one who the life with all its surprises accepts and PRT responsibly with \textsc{refl} treats\\
\footnotetext{
        Balance, broschure of TK-series for healthy living, Techniker Kran\-ken\-kas\-se. 1995.
      }
\glt `Those who accept life with all its little surprises, yet still act responsibly, are the ones who will cope best.'
\ex\iw{klarkommen} 
\gll Ich bin alleinstehende Mutter, und so gut \emph{klar} \emph{komm} ich nicht.\footnotemark\\
	 I   am single mother and so good clear come I not\\
\footnotetext{
        radio show, 02.07.2000, I would like to thank Andrew McIntyre\ia{MacIntyre@McIntyre} for this example.
    }
\glt `I am a single mother and I am really not coping that well.'
\ex 
\gll Den Atem \emph{an} \emph{hielt} die ganze Judenheit des römischen Reichs und weit hinaus über die Grenzen.\footnotemark\\
	 the breath in held the whole Jewish.people of.the Roman empire and wide further over the borders\\
\footnotetext{
        Lion Feuchtwanger, \emph{Jud Süß}, p.\,276, citied in \citew[\page 56]{Grubacic65a}.
}
\glt `The entire Jewish population of the Roman Empire held their breath and the same was true far past its borders.'
\ex\iw{umhinkönnen}%
\gll Nicht \emph{umhin} \emph{konnte} Peter, auch noch  einen Roman über das Erhabene zu schrei\-ben.\label{ex-grewendorf-nicht-umhin}\footnotemark\\
	not around could Peter also \textsc{prt} a novel over the sublime to write\\
\footnotetext{
        \citep*[\page 90]{Grewendorf90a}.
      }
\glt `Peter couldn't get around writing another novel about the sublime.'
\ex\iw{herauskommen}%
\gll Die Zeitschrift \frq Focus\flq{} hat vor einiger Zeit auch die Umweltdaten deutscher Städte miteinander verglichen. Dabei \emph{heraus} \emph{kam} u.\,a., daß Halle an der Saale die leiseste Stadt Deutschlands ist.\footnotemark\\
	 the magazine \frq Focus\flq{} has before some time also the environmental.data German cities with.eachother compared there.at out came amongst.other.things that Halle an der Saale the quietest city Germany's is\\	 
\footnotetext{
        Max Goldt, \emph{Die Kugeln in unseren Köpfen}. Munich:\ Wilhelm Heine Verlag. 1997, p.\,18.
}
\glt `Not too long ago, the magazine \emph{Focus} compared environmental data on various German cities. As a result, they found out, among other things, that Halle an der Saale was the quietest city in Germany.'
\ex\label{ex-los-damit} 
\gll \emph{Los} damit \emph{geht} es schon am 15. April.\footnotemark\\
	  off there.with goes it PRT on 15. April\\
\footnotetext{
        taz, 01.03.2002, p.\,8.
    }
\glt `The whole thing starts on the 15th April.'
\ex 
\gll Sein Vortrag wirkte [\ldots] ein wenig arrogant, nicht zuletzt wegen seiner Anmerkung, neulich habe er bei der Premiere des neuen "`Luther"'"=Films in München neben Sir Peter Ustinov und Uwe Ochsenknecht gesessen. Gut \emph{an} \emph{kommt} dagegen die Rede des Jokers im Kandidatenspiel: des Thüringer Landesbischofs Christoph Kähler (59).\footnotemark\\
    his presentation seemed {} a bit arrogant not lastly because.of his comment recently has he at the premiere of.the new \hspaceThis{"`}Luther.film in Munich next.to Sir Peter Ustinov and Uwe Ochsenknecht sat good on comes there.against the speech of.the joker in candidate.game of.the Thüring state.bishop Christoph Kähler (59)\\
\footnotetext{
        taz, 04.11.2003, p.\,3.%
}
\glt `His presentation came across somewhat arrogant. Not least because of his comment that he recently sat next to Sir Peter Ustinov and Uwe Ochsenknecht at the premiere of the new Luther film. What did get a good reception was the speech by the wild card in the election race: the Thüringen state bishop Christoph Kähler (59).' 
\ex\iw{hinzukommen}
\gll Erschwerend \emph{hinzu} \emph{kommt} der Leistungsdruck, dem auch die Research"=Abteilungen unterliegen.\label{ex-adv-non-true-particle}\\
	 difficultly there.to comes the pressure.to.perform that also the Research"=departments underlie\\
\glt `What makes it even more difficult is the pressure to perform, which the research departments are also under.'
\ex\iw{vornliegen}
\gll Immer  noch  mit  Abstand  \emph{vorn} \emph{liegt} Reiseunternehmer Kuoni.\footnotemark\\
	 always still PRT   with distance  in.front lies travel.company Kuoni\\
\footnotetext{
        \citep[\page 126]{CT75a-u}.
      }
\glt `The travel company Kuoni is always ahead by some distance.'
\ex  
\gll Den Umschwung im Jahr 1933 stellt Nolte als "`Volkserregung"' und "`Volksbewegung"' dar. (\ldots) Nicht \emph{hinzu} \emph{setzt}\iw{hinzusetzen} Nolte Zeugnisse republiktreuer Sozialdemokraten  und Zentrumsleute, die im Januar 1933 von lähmendem Entsetzen befallen (\ldots) waren.\footnotemark\\
	 the turnaround in.the year 1933 presents Nolte as excitment.of.the.people and \hspaceThis{"`}people's.movement PRT {} not here.to places\iw{hinzusetzen} Nolte testimonies loyal.to.the.repbulic social.democrats and centre.people who in January 1933 of paralyising horror struck {} were\\
\footnotetext{
        Die Zeit, 19.03.1993, p.\,82. Cited in \citew[\page 1633]{Hoberg97a}.
      }\label{ex-nicht-hinzu-setzt}
\glt `Nolte presents the turnaround in 1933 as `animation of the people' and a `people's movement'. Nolte does not include testimonies of Social Democrats and people positioned the centre of the political spectrum, who were struck by paralysing horror in January 1933.'
\zl



These data show that structures with a fronted particle cannot be ruled out in general. I assume that such structures have to be made
available by syntax in general and that there are certain stipulations for fronting which are responsible for Fanselow's examples
being ungrammatical. For more on fronting of verb particles and further data, see \citew{Mueller2002b,Mueller2002d}.

I will assume then that there is an ordinary verb trace in the prefield and will follow Hoberg in assuming that the example of fronting in (\ref{ex-zum-zweiten-anal})
should be analyzed parallel to the fronting of a partial projection of a verbal complex. Hoberg describes the idea for her analysis in a footnote and does
not go into any details. In particular, it remains unexplained how the trace in (\ref{ex-zum-zweiten-anal}) is licensed.

In what follows, I wish to delve a little deeper into the details of the analysis. I will start with a discussion of the less complex examples in (\mex{1}).
\eal
\ex\label{dass-clark} 
\gll dass Clark 1965 zum zweiten Mal die Weltmeisterschaft errungen hat\\
	 that Clark 1965 to.the second time the world.championship won has\\
\glt `that Clark won the world championship for the second time in 1965'
\ex\label{zum-zweiten-mal-hat} 
\gll {}[\sub{VP} [Zum zweiten Mal] errungen]$_i$  hat$_j$ Clark die Weltmeisterschaft 1965 \_$_i$ \_$_j$.\\
     {}          \spacebr{}to.the second time   won has Clark the world.championship 1965\\
\ex\label{zum-zweiten-mal-errungen} 
\gll {}[\sub{VP} [Zum zweiten Mal] [die Weltmeisterschaft] errungen]$_i$ hat$_j$ Clark 1965 \_$_i$ \_$_j$.\\
     {}          \spacebr{}to.the second time \spacebr{}the world.championship  won has Clark  1965\\
\zl



In (\ref{dass-clark}), the relations between the various elements should be clear. The auxiliary
verb \emph{hat} `has' selects the participle \emph{errungen} `won' and they together form a verbal complex. The arguments of the verbal complex can be permutated in the middle-field and adjuncts
can appear between the arguments. In (\ref{zum-zweiten-mal-hat}), the auxiliary is in initial position. The verb with which
\emph{hat} would have normally formed a complex is now in the prefield. The extraction trace \_$_i$ has the same arguments
as the verb in initial position, namely \emph{Clark} and \emph{die Weltmeisterschaft} `the world championship'. The verb trace \_$_j$, which corresponds to
\emph{hat} in initial position, forms a verbal complex with the extraction trace \_$_i$, which then requires these two arguments.
For this reason, \emph{Clark} and \emph{die Weltmeisterschaft} can now appear in the middle-field. In (\ref{zum-zweiten-mal-errungen}),
the extraction trace \_$_i$ corresponds to the verb phrase \emph{zum zweiten Mal die
  Weltmeisterschaft errungen} `for the second time the world championship'. When the auxiliary is
combined with this trace, it is not possible for a further complement to be attracted, since \emph{die Weltmeisterschaft} is already a
complement of \emph{errungen}. Therefore, only the subject of \emph{errungen} can appear in the middle-field.
(\ref{ex-zum-zweiten-anal}) can be explained as follows: I assume an empty verb in the prefield, which takes \emph{die Weltmeisterschaft} as
complement and \emph{zum zweiten Mal} as an adjunct. The properties of this head are determined by the other material in the main clause, i.e.
the arguments of \emph{errang} which occur in the middle-field cannot be realised in the prefield -- and adjuncts which occur in the prefield
must be compatible with the semantic properties of \emph{errang}. Sentences such as (\mex{1}) are not possible:
\eal\label{bsp-zu-viele-komplemente-adjunkte}
\ex[*]{
\gll Zum zweiten Mal die Weltmeisterschaft errang Clark 1965 die Goldmedaille.\\
	 to.the second time the world.championship won Clark 1965 the gold.medal\\
\glt Intended: `Clark won the gold medal for the second time during the world championships in 1965.'
}
\ex[*]{
\gll Drei Stunden lang die Weltmeisterschaft errang Clark 1965.\\
	 the hours long the world.championship won Clark 1965\\
\glt Intended: `Clark won the world championship for three hours in 1965.'
}
\zl



In (\mex{0}a), both \emph{die Weltmeisterschaft} `the world championship' and \emph{die
  Goldmedaille} `the gold medal' would fulfil the role of object and in (\mex{0}b),
the adjunct \emph{drei Stunden lang} is not compatible with \emph{errang}.

We can only explain this if we assume some relation between \emph{errang} (or the verb trace \_$_j$) and the extraction trace
\_$_i$ in (\ref{ex-zum-zweiten-anal}), repeated here as (\mex{1}). The extraction trace is in a filler"=gap relation to the
complex projection in the prefield. What is missing is a relation between the extraction trace \_$_i$ and the overt verb.
\ea
\label{ex-zum-zweiten-anal-zwei}%
\gll {}[\sub{VP} [Zum zweiten Mal] [die Weltmeisterschaft] \_\sub{V} ]$_i$ errang$_j$ Clark 1965 \_$_i$ \_$_j$.\\
      {}         \spacebr{}to.the second time \spacebr{}the world.championship {} {} won Clark 1965\\
\z 
It is for this reason that I suggest a lexical rule which licenses a further lexical item for each verb that is able to select a trace
with which it forms a predicate complex. The trace has to have the same valence as the original verb and all arguments which are
not realised together with the trace are attracted by the verb. (\mex{1}) shows the syntactic aspects of this lexical rule:

\eas
\label{lr-mult-vf}
Lexical rule for multiple fronting (preliminary version):\\
\begin{tabular}[t]{@{}l@{}}
\ms{
synsem$|$loc & \ibox{1} \ms{ cat$|$head & \ms[verb]{
                                          initial & \ibox{2}\\
                                          vform   & \ibox{3}\\
                                          }\\
% XCOMP<>
                           }\\
} $\mapsto$\\
\onems{
synsem$|$loc$|$cat \ms{ head & \ms[verb]{
                                          initial & \ibox{2}\\
                                          vform   & \ibox{3}\\
                                          }\\
               spr & \eliste\\
               comps & \ibox{4} $\oplus$ \sliste{ \onems{ loc$|$cat \onems{ head \ms[verb]{
                                                                                   dsl & \ibox{1}\\
                                                                                  }\\
                                                                    comps \ibox{4}\\
                                                                   }\\
                                                           lex +\\
                                                         }}\\
                       }\\
}
\end{tabular}
\zs



\noindent
The trace of the silent verbal head \_\sub{V} in \pref{ex-zum-zweiten-anal-zwei} is identical to
the trace which is responsible for verb movement in the analysis of verb-first order. The details
of verb movement are explained in Section~\ref{sec-v1}. There, I give the following entry for the
verb trace:
\ea
\label{le-verbspur2-mf}
Head movement trace as suggested by \citew[\page 207]{Meurers2000b}:
\onems{
phon \eliste\\
synsem$|$loc \ibox{1} \onems{ cat$|$head$|$dsl \ibox{1} }\\
}
\z

Figure~\vref{anal1} shows the analysis of \pref{ex-zum-zweiten-anal-zwei} when using this trace. I assume
that the ouput of the lexical rule in \pref{lr-mult-vf} forms the input of the verb-first lexical rule. The rule
for verb movement, which is also explained in detail in Section~\ref{sec-v1}, has the following
form:

\eas
\label{lr-verb-movement2-mf}
Lexical rule for verb in initial position:\\
\begin{tabular}[t]{@{}l@{}}
\ms{
synsem$|$loc & \ibox{1} \ms{ cat$|$head & \ms[verb]{ vform & fin\\
                                                     initial & $-$\\
                                             }\\
                  }\\
} $\mapsto$\\*
\onems{
synsem$|$loc \onems{ cat  \ms{ head & \ms[verb]{ vform & fin\\
                                                     initial & $+$\\
                                             }\\
                           spr   & \eliste\\
                           comps & \sliste{ \onems{ loc \onems{ cat \ms{ head & \ms[verb]{
                                                                                dsl & \ibox{1}\\
                                                                               }\\
                                                                        comps & \eliste\\
                                                                    }\\
                                                           cont \ibox{2}\\
                                                         }\\
                                              }}\\
                         }\\
                     cont \ibox{2}\\
             }\\
}
\end{tabular}
\zs



It is important that the verb trace on the far right corresponds to the right-hand side of the rule
in \pref{lr-mult-vf}. 
\begin{figure}
\resizebox{!}{\textheight-3\baselineskip}{%
%\resizebox{\textwidth}{!}{%
\begin{sideways}
\begin{forest}
sm edges
[V{[\comps \eliste]}
	[V\ibox{5}\,{[\comps \sliste{ \ibox{1} }]}
		[PP
			[zum zweiten Mal;to.the second time, roof]]
		[V{[\comps \sliste{ \ibox{1} }]}
			[\ibox{2} NP{[\textit{acc}]}
				[die Weltmeisterschaft; the world championship, roof]]
			[V{[\comps \sliste{ \ibox{1}, \ibox{2} }]}
				[\trace]]]]
	[V\feattab{\comps \eliste,\\
                   \textsc{slash} \sliste{ \ibox{5} }}
		[V{[\comps \sliste{ \ibox{6} }]}
			[V{[\comps \ibox{3} $\oplus$ \sliste{ \ibox{4} }]}, tier=np,edge label={node[midway,right]{V1-LR}}
				[V{[\comps \sliste{ \ibox{1}, \ibox{2} }]}, tier=trace,edge label={node[midway,right]{MF-LR}}
					[errang;won]]]]
		[\ibox{6} V\feattab{\comps \eliste,\\
                                    \textsc{slash} \sliste{ \ibox{5} }}
			[\ibox{1} NP{[\textit{nom}]}, tier=np
				[Clark;Clark]]
			[V\feattab{\comps \ibox{3},\\
                                   \textsc{slash} \sliste{ \ibox{5} }}
				[\ibox{4}\feattab{\textsc{loc} \ibox{5} {[\comps \ibox{3}]},\\
                                                  \textsc{slash} \sliste{ \ibox{5} }}, tier=trace
					[\trace]]
				[V{[\comps \ibox{3} \sliste{ \ibox{1} } $\oplus$ \sliste{ \ibox{4} }]}
					[\trace]]]]]]
\end{forest}
\end{sideways}
}
\caption{\label{anal1}Analysis of multiple frontings with an empty head}
\end{figure}\todostefan{Unbedingt noch H, A, CL-Anotation machen}


The verb trace in the prefield is combined with \emph{die Weltmeisterschaft} `the world
championship' as an argument and \emph{zum zweiten Mal} `for the second time' as an adjunct to form
the phrase \emph{zum zweiten Mal die Weltmeisterschaft} `for the second time the world
championship'. The entire phrase is the filler in a long-distance dependency that was
introduced by the extraction trace directly next to \emph{Clark}. The local properties of the filler
\iboxb{5} are identical to those of the extraction trace. The arguments of the extraction trace
attracted by the lexical entry for \emph{errang} are licensed by the lexical rule (\ref{lr-mult-vf})
(see \iboxt{3} in the trace for verb movement furthest to the right). Therefore, the \compsl of the
trace of verb movement and the extraction trace contain exactly those elements which cannot appear
as arguments of the verb trace in the prefield, namely \iboxt{1} in Figure~\ref{anal1}.

As we have seen from the discussion of (\ref{bsp-zu-viele-komplemente-adjunkte}), there has to be a connection between the trace in the prefield and
the verb in the remainder of the sentence. This connection is established in the same way as the connection between the verb in initial position and 
the verb trace at the end of sentence: the head feature \dsl is used to represent the required
information. Figure~\vref{anal2} shows the identity of the respective \dsl features \iboxb{7} in addition to the valence information and the \textsc{nonloc} information. 

\begin{figure}
\resizebox{!}{\textheight-3\baselineskip}{%
\begin{sideways}
\begin{forest}
sm edges
[V{[\comps \eliste]}
	[V\ibox{5}\,{[\dsl \ibox{7}, \comps \sliste{ \ibox{1} }]}
		[PP
			[zum zweiten Mal;to.the second time, roof]]
		[V{[\dsl \ibox{7}, \comps \sliste{ \ibox{1} }]}
			[\ibox{2} NP{[\textit{acc}]}
				[die Weltmeisterschaft;the world championship, roof]]
			[V\feattab{\dsl \ibox{7},\\
                                   \comps \sliste{ \ibox{1}, \ibox{2} }}
				[\trace]]]]
	[V\feattab{\comps \eliste,\\
                   \textsc{slash} \sliste{ \ibox{5} }}
		[V{[\comps \sliste{ \ibox{6} }]}
			[V{[\comps \ibox{3} $\oplus$ \sliste{ \ibox{4} }]}, tier=np,edge label={node[midway,right]{V1-LR}}
				[V{[\comps \sliste{ \ibox{1}, \ibox{2} }]}, tier=trace,edge label={node[midway,right]{MF-LR}}
					[errang;won]]]]
		[\ibox{6} V\feattab{\comps \eliste,\\
                                    \textsc{slash} \sliste{ \ibox{5} }}
			[\ibox{1} NP{[\textit{nom}]}, tier=np
				[Clark;Clark]]
			[V\feattab{\comps \ibox{3},\\
                                   \textsc{slash} \sliste{ \ibox{5} }}
				[\ibox{4} {[\begin{tabular}[t]{@{}l@{}}
                                           \textsc{loc} \ibox{5} [\begin{tabular}[t]{@{}l@{}}
                                                                       \dsl \ibox{7},\\
                                                                       \comps \ibox{3}\,],
                                                                       \end{tabular}\\
                                                   \textsc{slash} \sliste{ \ibox{5} }]
                                           \end{tabular}}, tier=trace
					[\trace]]
				[V{[\comps \ibox{3} \sliste{ \ibox{1} } $\oplus$ \sliste{ \ibox{4} }]}
					[\trace]]]]]]
\end{forest}
\end{sideways}
}
\caption{\label{anal2}Representation of valence information}
\end{figure}
The properties of the verb \emph{errang} are listed under \dsl in the \compsv of the item licensed
by the lexical rule in (\ref{lr-mult-vf}). The complement in the predicate complex \iboxb{4} is
realized by an extraction trace. The \localv of this trace \iboxb{5} is identical to the \localv
of the filler. Since \dsl is a head feature and therefore inside of the \localv, the \dsl value of
the complement of the verbal complex of \emph{errang} is identical to the \dsl value of the phrase 
\emph{zum zweiten Mal die Weltmeisterschaft}. As \dsl is a head feature, it is also ensured that the 
\dsl value is identical in all the projections of the verb trace in the prefield. In the verb trace
(\ref{le-verbspur2-mf}), the structure sharing between \local and \dsl ensures that the \compsv of the verb trace
matches the valence information under \dsl. In this way, we can ensure that the trace allows only those elements
which were required by the original verb.

The representation of meaning of the constituents in the prefield and in the trace is done in an analogous manner:
The semantic content (\iboxt{5}) in (\mex{1}) is taken over from the projection of the trace that is selected by the verb in
initial position. (\mex{1}) shows the corresponding modified lexical
rule:
\eas
\label{lr-mult-vf-zwei}
Lexical rule for multiple fronting:\\
\resizebox{\linewidth}{!}{%
\begin{tabular}{@{}l@{}}
\ms{
synsem$|$loc & \ibox{1} \onems{ cat$|$head \ms[verb]{
                                          initial & \ibox{2}\\
                                          vform   & \ibox{3}\\
                                          }\\
                              }\\
} $\mapsto$\\
\onems{
synsem$|$loc \onems{ cat \ms{ head & \ms[verb]{
                                          initial & \ibox{2}\\
                                          vform   & \ibox{3}\\
                                          }\\
                              spr & \eliste\\
                    comps & \ibox{4} $\oplus$ \sliste{\ms{ loc & \onems{ cat \ms{ head & \ms[verb]{
                                                                                            dsl & \ibox{1}\\
                                                                                           }\\
                                                                                  %spr & \eliste{}
                                                                                  %SPR ist leer,
                                                                                  %weil die finiten
                                                                                  %Verben immer
                                                                                  %leere SPR-Werte
                                                                                  %haben, deshalb
                                                                                  %kann auch über
                                                                                  %den Trace nie
                                                                                  %etwas anderes
                                                                                  %projiziert werden
                                                                                  %und wir müssen
                                                                                  %hier nichts
                                                                                  %sagen. Oder? 08.06.2015
                                                                                  comps & \ibox{4}\\
                                                                                  }\\
                                                                          cont \ibox{5}\\ 
                                                                     }\\
                                                               lex & $+$\\
                                                           }}\\
                            }\\
        cont \ibox{5}\\
      }\\
}
\end{tabular}}
\zs
Inside the trace in (\ref{le-verbspur2-mf}), a connection is made between the meaning of the original verb, which is represented
under \textsc{dsl}, and the meaning of the trace, which is represented under \local.
Figure~\vref{anal3} shows the aspects of the semantic representation with the modified lexical rule and the trace (\ref{le-verbspur2-mf}).

\begin{figure}
\resizebox{!}{\textheight-3\baselineskip}{%
\begin{sideways}
\begin{forest}
sm edges
[V{[\textsc{cont} \ibox{3}\,]}
	[V\ibox{5}\,{[\dsl \ibox{7}, \textsc{cont} \ibox{3}\, 2*erringen{(c,w)}]}
		[PP
			[zum zweiten Mal;to.the second time, roof]]
		[V{[\dsl \ibox{7}, \textsc{cont} erringen{(c,w)}]}
			[\ibox{2} NP{[\textit{acc}]}
				[die Weltmeisterschaft;the world championship, roof]]
			[V\feattab{\dsl \ibox{7},\\
                                   \textsc{cont} \ibox{2}\, erringen{(c,w)}}
				[\trace]]]]
	[V\feattab{\textsc{cont} \ibox{3},\\
                   \textsc{slash} \sliste{ \ibox{5} }}
		[V{[\comps \sliste{ \ibox{6} }, \textsc{cont} \ibox{3}\,]}
			[V{[\comps \sliste{ \ldots, \ibox{4} }, \textsc{cont} \ibox{3}\,]}, tier=np,edge label={node[midway,right]{V1-LR}}
				[V{[\textsc{cont} \ibox{2}\, erringen{(c,w)}]}, tier=trace,edge label={node[midway,right]{MF-LR}}
					[errang;won]]]]
		[\ibox{6} V\feattab{\textsc{cont} \ibox{3},\\ 
                                    \textsc{slash} \sliste{ \ibox{5} }}
			[NP{[\textit{nom}]}, tier=np
				[Clark;Clark]]
			[V\feattab{\textsc{cont} \ibox{3},\\
                                   \textsc{slash} \sliste{ \ibox{5} }}
				[\ibox{4}{[\begin{tabular}[t]{@{}l@{}}
                                           \textsc{loc} \ibox{5} [\begin{tabular}[t]{@{}l@{}}
                                                              \dsl \ibox{7},\\
                                                              \textsc{cont} \ibox{3}\,],
                                                              \end{tabular}\\
                                           \textsc{slash} \sliste{ \ibox{5}}]
                                           \end{tabular}}, tier=trace
					[\trace]]
				[V{[\comps \sliste{ \ldots, \ibox{4} }\textsc{cont} \ibox{3}\,]}
					[\trace]]]]]]
\end{forest}
\end{sideways}
}
\caption{\label{anal3}Representation of meaning contribution}
\end{figure}
The verb \emph{errang} `won' licensed by the lexical rule requires an empty head. This empty head contains the representation of the
syntactic and semantic properties of the original verb inside its \dslv -- importantly also its semantic content $erringen(x,y)$,
whereby $x$ is linked to the subject and $y$ to the object. This means that by assigning its arguments, $x$ refers to \emph{Clark}
(abbreviated to $c$), while $y$ refers to \emph{die Weltmeisterschaft} (abbreivated to $w$).
Since the \locv of the extraction trace is identical to the \locw of the filler and therefore its \dsl is located inside its
\textsc{loc}, the \dsl value of the extraction trace is also identical to the \dsl value of the filler. Since \dsl is a head
feature, it is present at all nodes inside of the verbal projection in the prefield and on the verb trace in the prefield. Inside the
verb trace, the \contv under \dsl is identified with the \contv of the trace itself. The computation and projection of the semantic 
content inside of the complex constituent in the prefield then follows via the normal principles of HPSG:
The combination of the trace with its complement \emph{die Weltmeisterschaft} results in the projection of the \contv of the head
($erringen(c,w)$). When then combined with the adjunct \emph{zum zweiten Mal} `for the second time', the semantic content of the adjunct (2*$erringen(c,w)$)
is projected.
The semantic representation of the filler is identical to the semantic representation of the extraction trace. Through specification in
our lexical rule, the semantic content of the verb is associated with the semantic content of the selected (projection of the) verb trace
\iboxb{5}, \ie the trace that stands for \emph{errang} adopts the semantic representation of the extraction trace (2*$erringen(c,w)$).
This meaning is then projected along the head chain up to the verb in initial position and from there it is projected to the entire clause.


As was shown by examples (\ref{bsp-gezielt-mitglieder}) and (\ref{bsp-kurz-die-bestzeit}) on page~\pageref{bsp-gezielt-mitglieder}
as well as the examples in (\ref{bsp-idioms-nicht-adjazent}) on page~\pageref{bsp-idioms-nicht-adjazent}, the elements in the prefield
do not have to be adjacent to the verb on which they are dependent. A modal or auxiliary verb can occupy the initial position. The verb which
selects the elements in the prefield is then located in the right verbal bracket.
Figure~\vref{anal4-vlast} shows how the example in (\mex{1}) (which conforms to this pattern) should be analyzed.
\ea
\gll Zum zweiten Mal die Weltmeisterschaft hat Clark 1965 errungen.\\
	 to.the second time the world.championship has Clark 1965 won\\
\glt `Clark won the world championship in 1965 for the second time.'
\z

\begin{figure}
\resizebox{!}{\textheight-3\baselineskip}{%
\begin{sideways}
\begin{forest}
sm edges
[V{[\comps \eliste]}
	[V \ibox{5}\,{[\dsl \ibox{7}, \comps \sliste{ \ibox{1} }]}
		[PP
			[zum zweiten Mal;to.the second time, roof]]
		[V{[\dsl \ibox{7}, \comps \sliste{ \ibox{1} }]}
			[\ibox{2} NP{[\textit{acc}]}
				[die Weltmeisterschaft; the world championship, roof]]
			[V\feattab{\dsl \ibox{7},\\
                                   \comps \sliste{ \ibox{1}, \ibox{2} }}
				[\trace]]]]
	[V\feattab{\comps \eliste,\\
                   \textsc{slash} \sliste{ \ibox{5} }}
		[V{[\comps \sliste{ \ibox{6}}]}
			[V{[\comps \ibox{3} $\oplus$ \sliste{ \ibox{8} }]}, tier=np,edge label={node[midway,right]{V1-LR}}
				[hat;has]]]
		[\ibox{6}\,V\feattab{\comps \eliste,\\ 
                                     \textsc{slash} \sliste{ \ibox{5} }}
			[\ibox{1} NP{[\textit{nom}]}, tier=np
				[Clark;Clark]]
			[V\feattab{\comps \ibox{3},\\
                                   \textsc{slash} \sliste{ \ibox{5} }}
				[\ibox{8}\,\feattab{\comps \ibox{3},\\
                                                    \textsc{slash} \sliste{ \ibox{5} }}
					[\ibox{4}\,{[\begin{tabular}[t]{@{}l@{}}
                                                    \textsc{loc} \ibox{5}\,{[\begin{tabular}[t]{@{}l@{}}
                                                                          \dsl \ibox{7},\\
                                                                          \comps \ibox{3}\,],
                                                                         \end{tabular}}\\
                                                    \textsc{slash} \sliste{ \ibox{5} }]
                                                    \end{tabular}}
						[\trace]]
					[V{[\comps \ibox{3}\, \sliste{ \ibox{1} } $\oplus$ \sliste{ \ibox{4} }]}, l sep+=2ex
						[V{[\comps \sliste{ \ibox{1}, \ibox{2} }]},edge label={node[midway,right]{MF-LR}}
							[errungen;won]]]]
				[V{[\comps \ibox{3}\,\sliste{ \ibox{1} } $\oplus$ \sliste{ \ibox{8} }]}
					[\trace]]]]]]
\end{forest}
\end{sideways}
}
\caption{\label{anal4-vlast}Analysis of \emph{Zum zweiten Mal die Weltmeisterschaft hat Clark 1965 errungen.} `Clark has won the world championship in 1965 for the second time.'}
\end{figure}



In contrast to the analysis discussed here, the lexical rule for putative multiple fronting is not applied
to the finite verb (which was present in initial position), but rather to the non-finite verb in final position.
The verb which is the output of the lexical rule requires a verbal complex (something that is \lex +) and attracts its previously non-realised
arguments \iboxb{3}. This verbal complex is realised as the extraction trace. The combination of the extraction trace
and \emph{errungen} `won' forms a verbal complex, which becomes the complement of the verb trace,
which corresponds to \emph{hat} `has' in initial position. The complex consisting of extraction trace, \emph{errungen} and verb trace is then combined with the
arguments, i.e. \emph{Clark}, not realised in the prefield. The percolation of the \textsc{slash} and \dsl values
proceeds parallel to the previously discussed example.


It still remains to be seen how we can rule out the following structure:
\ea[*]{
\gll dass Clark 1965 zum zweiten Mal die Weltmeisterschaft [\_\sub{V} hat]\\
     that Clark 1965 for.the second time the world.championship {}    has\\
}
\z
Without further restrictions, the silent head could be combined with the auxiliary \emph{hat} and take the
place of \emph{errungen}. This structure can however be ruled out under the assumption that all verbs directly specified
in the lexicon which are able to select other verbs require that the embedded verb should have \type{none} as its \dsl value.
In this way, it is ensured that the trace cannot be combined with the normal verb-final \emph{hat} `has', but rather only
with lexical items licensed by the lexical rule in (\ref{lr-mult-vf-zwei}).

The other data discussed in Section~\ref{sec-phenomenon-mult-front} can be analyzed entirely parallel to the examples discussed here:
adjuncts/arguments are linked to an empty verbal head, just as would be the case for their ordering in the middle-field and
their single fronting. The complex projection in the prefield enters a long-distance dependency with the extraction trace in the
verbal complex. If there is any motivation for analysing the data discussed in Section~\ref{sec-analyse-mfvorschlaege} as instances where
a non-verbal constituent precedes the finite verb, this would still be compatible with the analysis presented here. These examples would
have to be explained using the mechanisms presented in Chapter~\ref{chap-german-sentence-structure}, i.e. as standard fronting with a basic extraction trace.
My claim in Section~\ref{sec-analyse-mfvorschlaege} that these analyses cannot be applied to all the data presented in Section~\ref{sec-phenomenon-mult-front}
remains valid.

Finally, I would like to clarify one more point about the status of the lexical rule for multiple fronting. This rule is
entirely parallel to the verb movement rule, which is needed to derive the position of the finite verb. The verb-first rule
differs from the multiple fronting rule in that the verb-first rule mentions finiteness features and
the \textsc{initial} feature
relevant for its positioning. Furthermore, the \compsl of the embedded projection (\iboxt{4} in (\ref{lr-mult-vf-zwei})) is
instantiated as an empty list. This difference in the constraint of the \compsl corresponds to the difference between verbs
which form verbal complexes (the so-called coherent construction) and verbs which embed phrases (the
so-called incoherent construction).  

\begin{comment}
\section{Probleme}

Jacobs Negation: Neg CP

Ich kann Neg XP V nicht ausschließen und kriege immer Skopus über das am tiefsten eingebettete Verb. (darauf hat mich
Tibor hingewiesen).


Allerdings ist das folgende sicher nicht

Adv Neg CP sondern Adv Neg NP V

[Sicher] [nicht] [die letzte Aktion der BAW in diesem Zusammenhang] war am 30.\ Mai eine
      zweite Durchsuchung des Mehringhofes, bei der nochmals nach dem angeblichen Sprengstoffversteck
      gesucht wurde.


% Auch Marga Reis, 2003

Nur die Maria, die liebt jeder.

\end{comment}


\subsection{Left dislocation}
\label{sec-left-dislocation}

Marga Reis (p.\,c.\,2003) has pointed out that the following examples could pose a problem for the analysis
I have developed here:
\eal
\ex[]{
\gll Zum zweiten Mal die Weltmeisterschaft, die gewann Clark 1965.\\
	 to.the second time the world.championship.\fem{} that.\fem{} won Clark 1965\\
\glt `Clark won the world championship for the second time in 1965.'
}
\ex[*]{
\label{ex-zweiten-mal-weltmeisterschaft-das}
\gll Zum zweiten Mal die Weltmeisterschaft, das gewann Clark 1965.\\
	 to.the second time the world.championship.\fem{} that.\neu{} won Clark 1965\\
}
\zl
If a verb phrase is referred to in so-called left-dislocation structures, the pronoun 
\emph{das} (neuter) is obligatory:
\eal
\ex[]{
\gll Die Torte essen, das will Peter nicht.\\
	 the cake eat that.\neu{} wants Peter not\\
\glt `Peter doesn't want to eat the cake.'
}
\ex[*]{
\gll Die Torte essen, die will Peter nicht.\\
	 the cake eat that.\fem{} wants Peter not\\
}
\zl
If \emph{zum zweiten Mal} and \emph{die Weltmeisterschaft} were part of a verbal constituent, then -- just as
in (\mex{0}a) -- we would assume that \emph{das} is obligatory in left-dislocation. (\mex{-1}b) clearly shows
that this is not the case.



One could argue that this difference can be traced back to the fact that the pronoun  refers to an overt element.
The left-dislocated constituent could then be a verbal projecton, however, since this verbal projection does not
contain an overt verb and the closest overt phrase is the feminine NP \emph{die Weltmeisterschaft}, one has to use
the feminine demonstrative pronoun \emph{die}.\footnote{
		A reviewer from Linguistische Berichte pointed out the following kind of gapping data:
\ea
\gll Der Eva Buntstifte gekauft und der Rita Bauklötze, das hat Otto heute in der Stadt.\\
	 the Eva crayons bought and the rita building.blocks that.\neu{} has Otto today in the city\\
\glt `Otta went to town today and bought crayons for Eva and building blocks for Rita.'
\z
It is possible here to argue that the fronted constituent contains a verb. The verb is not in final position, but
still relevant for the anaphoric relation. Furthermore, the verb and pronoun do not have to be adjacent in cases of
extraposition: 
\ea
\gll Geschlafen in der Vorlesung, das hat sie nicht.\\
	 slept in the lecture.\fem{} that.\neu{} has she not\\
\glt `She didn't sleep during the lecture.'
\z
The overtly realised verb is however still anaphorically accessible.
}



Unfortunately, instances of multiple fronting do not show uniform behaviour when used in left-dislocation
constructions (as pointed out by a reviewer from \emph{Linguistische Berichte}). Examples like (\ref{bsp-dauerhaft-mehr-arbeitsplaetze}) %and (\ref{bsp-praedikativen-charakter})
optionally allow \emph{das}, whereas this is the only possibility with example (\ref{bsp-zeitgeist-zwei}):
\eal
\ex[]{
\gll {}Dauerhaft mehr Arbeitsplätze, das gebe es erst, wenn sich eine Wachstumsrate von  mindestens 2,5 Prozent über einen Zeitraum von drei oder vier Jahren halten lasse.\\
       constantly more jobs that.\neu{} gives it first when \textsc{refl} a growth.rate of  at.least 2.5 percent over a time.period of three or four years hold lets\\
\glt `In the long run, there will only be more jobs available when a growth rate of at least 2.5 percent 
can be maintained over a period of three or four years.'	
}
\ex[]{\label{ex-duaerhaft-arbeitsplaetze-die}
\gll {}Dauerhaft mehr Arbeitsplätze, die gebe es erst, wenn sich eine Wachstumsrate von  mindestens 2,5 Prozent über einen Zeitraum von drei oder vier Jahren halten lasse.\\
       constantly more jobs that.PL gives it first, when \textsc{refl} a growth.rate of  at.least 2.5 percent over a time.period of three or four years hold lets\\
}
%% \ex[?]{
%% \gll {}Noch entschiedener prädikativen Charakter, das hat das Adj., wenn [\ldots]\\
%% 	   still more.deciding predicative character.\mas{} that.\neu{} has the adj. if\\
%% \glt  `Even more decidingly, the adjective has a predicative character, if \ldots'
%% }
%% \ex[?]{
%% \gll {}Noch entschiedener prädikativen Charakter, den hat das Adj., wenn [\ldots]\\
%% 	   still more.deciding predicative character.\mas{} that.\mas{} has the adj. if\\
%% }
\ex[?]{\label{ex-zeitgeist-rechnung-das}
\gll {}Dem Zeitgeist Rechnung, das tragen im unterfränkischen Raum die privaten, städtischen und kommunalen Musikschulen.\\
      the Zeitgeist attention that.\neu{} carry in.the lower.Franconian area the private urban and communal music.schools\\
\glt `The private urban and communal music schools in the lower Franconian area account for the Zeitgeist.'
		}
\ex[*]{\label{ex-zeitgeist-rechnung-die}
\gll {}Dem Zeitgeist Rechnung, die tragen im unterfränkischen Raum die privaten, städtischen und kommunalen Musikschulen.\\
       the Zeitgeist attention.\fem{} that.\fem{} carry in.the lower.Franconian area the private urban and communal music.schools\\
}
\zl

The considerable deviance of (\ref{ex-zeitgeist-rechnung-die}) could be down to the fact that we are dealing with an idiomatic construction here and
that referring to individual parts of an idiom often results in ungrammaticality. As for why there is more than one possibility 
for the other examples, this will have to be shown by future research.

So if we would take the existence of clauses with \emph{das} as a criterion, the data in (\mex{0})
would support the analysis that treats the complex \vf as a unit since (\mex{0}a,c) show that reference with \emph{das} is indeed possible. The alternative realization of
\emph{die} in (\ref{ex-duaerhaft-arbeitsplaetze-die}) can be explained as a proximity effect where a
meaning corresponding to (\mex{1}) is taken up by the demonstrative pronoun.
\ea
\gll mehr dauerhafte Arbeitsplätze\\
     more constantly jobs\\
\z

\noindent
Note, however, that we are dealing with special cases of left dislocation anyway. According to the analysis
suggested here, the meaning of the empty verb in the fronted constituent corresponds to the meaning
of the overt verb in the remainder of the clause. For (\ref{ex-zweiten-mal-weltmeisterschaft-das}),
we would have \emph{zum zweiten Mal die Weltmeisterschaft} `for the second time the world
championship' meaning \emph{zum zweiten Mal die Weltmeisterschaft gewonnen} `for the second time the
world championship won'. This meaning is then referred to by \emph{das}. But such a meaning of
\emph{das} would be incompatible with \emph{gewann Clark 1965} `won Clark in 1965' since \emph{win}
selects for a competition and not an event of winning a competition (This was pointed out by
Joachim Jacobs in personal communication to Julia Winkler, see \citew[\page 39]{Winkler2014a}). Of course the
same argument applies to (\ref{ex-zeitgeist-rechnung-das}): in principle, this example should be
excluded as well. I guess what is happening here is that we are dealing with very marked structures
that cannot be processed according to usual grammar rules. So instead of a reading that would
correspond to (\mex{1}), \emph{dauerhaft mehr Arbeits\-plätze} may be perceived as a complex situation of a certain duration
in which there are more jobs and \emph{das} refers to this situation.
\ea[*]{
\gll Dauerhaft  mehr Arbeitsplätze geben, das gebe es erst, wenn \ldots\\
     constantly more jobs          give   this gives it first if\\
}
\z
% Freies Thema muss im Nominativ stehen. Altmann81a: 50
%
% Dauerhaft den Zuschlag, den gebe es erst

\subsection{Extraposition inside the complex prefield}
\label{sec-extraposition}

Tibor Kiss (p.\,c.\ 2002) has pointed out that the analysis with a verb trace allows sentences such as (\mex{1}):
\ea[*]{
\label{bsp-dem-mann-etwas-rs-hat}
\gll Dem Mann etwas \_\sub{V}, der dort steht, hat sie zugeflüstert.\\
	 the man something {} that there stands has she whispered\\
\glt `She whisphered something to the man standing over there.'
}
\z
In (\mex{0}), the silent verb head forms the right verbal bracket and the relative clause belonging to \emph{Mann}
is in the postfield of the verbal projection. These examples should be grammatical in the same way (\mex{1}) is:
\ea[]{
\label{bsp-dem-mann-etwas-zugefluestert}
\gll Dem Mann etwas zugeflüstert, der dort steht, hat sie.\\
     the.\dat{} man something whispered that there stands has she\\
\glt `She whispered something to the man standing over there.'
}
\z

This argument against the analysis with a verbal head in the \vf can be rejected right away since
there are examples like (\ref{ex-los-damit}) -- repeated here as (\mex{1}) -- that clearly show that extraposition in the complex \vf is possible:
\ea 
\gll {}[Los] [damit] geht es schon am 15. April.\footnotemark\\
	\spacebr{}off \spacebr{}there.with goes it \particle{} on 15. April\\
\footnotetext{
        taz, 01.03.2002, p.\,8.
    }
\glt `The whole thing starts on the 15th April.'
\z 
The particle \emph{los} marks the right sentence bracket and \emph{damit} is located inside the \nf
in the complex \vf.

Nevertheless, there remains the question why (\ref{bsp-dem-mann-etwas-rs-hat}) is impossible. First,
multiple fronting with indefinite pronouns like \emph{etwas} seems to be impossible. (\ref{bsp-dem-mann-etwas-rs-hat}) is ungrammatical even without extraposition of the relative clause:
\ea[*]{
\gll Dem Mann etwas hat sie zugeflüstert.\\
     the.\dat{} man something has she whispered\\
}
\z
If one modifies the preceding example so that one has two full noun phrases with a contrastive interpretation,
one observes an improvement in acceptability (and -- as noted in the Section~\ref{sec-dat-acc-vf} -- there are attested examples of this pattern):
\ea[?]{
\gll Dem Mann die Nachricht hat sie zugeflüstert.\\
	 the.\dat{} man the.\acc{} message has she whispered\\
\glt `She whispered the message to the man.'
}
\z
If we add a relative clause to one of the noun phrases, we see that the already marginally acceptable example becomes even
worse:
\ea[?*]{
\gll Dem Mann, der dort steht, die Nachricht hat sie zugeflüstert.\\
	 the.\dat{} man that there stands the message has she whispered\\
\glt `She whispered the message to the man standing there.'
}
\z
Our example becomes completely ungrammatical if we then try and extrapose the relative clause:
\ea[*]{
\gll Dem Mann die Nachricht, der dort steht, hat sie zugeflüstert.\\
     the.\dat{} man the.\acc{} message that there stands has she whispered\\
}
\z
%
Example (\ref{bsp-dem-mann-etwas-zugefluestert}) differs from (\mex{0}) in that \emph{dem Mann} `the
man' is stressed in (\ref{bsp-dem-mann-etwas-zugefluestert}), whereas \emph{etwas} is unstressed. Following the generalization proposed
by \ao, the elements involved in multiple fronting have to bear the same communicative importance, which is not the case
for (\ref{bsp-dem-mann-etwas-rs-hat}) and (\mex{0}). 

While further work is needed for the formalization of the respective constraints, it is clear that
extraposition inside of complex \vf{}s is possible and hence the assumptions of structures like the
one that is assumed in the current analysis is legitimate.


\subsection{Traces in undesired positions}
\label{sec-unwanted-traces}

The analysis in (\mex{1}) is ruled out by the fact that the second lexical item for \emph{errang} (licensed
by the rule in (\ref{lr-mult-vf-zwei})) selects a \textsc{lex}+ element.
\ea
\label{bsp-trace-mit-argumenten-im-mf}
\gll dass Clark 1965 [[zum zweiten Mal die Weltmeisterschaft \_\sub{V}] errang]\\
     that Clark 1965 \hspaceThis{[[}to.the second time the world.championship {} won\\
\glt `that Clark won the world championship for the second time in 1965'
\z
This structure is ruled out for the same reason as embedding of verbal projections in obligatorily coherent 
constructions.

There is however still the analysis in (\mex{1}), which is entirely parallel to verbal complex formation and therefore
cannot be ruled out by a \textsc{lex} feature.
\ea
\gll dass Clark 1965 zum zweiten Mal die Weltmeisterschaft     [[\_\sub{V} errungen] hat].\\
     that Clark 1965 to.the second time the world.championship {} won has\\
\z
Furthermore, we have not yet encountered anything that would rule out the possibility of a verbal trace in the prefield as a filler
for a long-distance dependency.
\ea
\gll \_$_V$ hat Clark 1965 zum zweiten Mal die Weltmeisterschaft [~\_$_i$ errungen]\\
     {}     has Clark 1965 to.the second time the world.championship {} won\\
\z
\citet[\page 100]{Fanselow87a} discussed cases with one fronted PP and noticed that such sentences
are ambiguous since they could be analyzed as structures in which a single constituent is fronted or
as structures in which a complex constituent containing one element is fronted.

As has already been suggested, there are various conditions for cases of supposed multiple fronting that rely on the thematic status
of the constituents preceding the finite verb. If we require that there be certain relations between such constituents, then the corresponding
constrains would prohibit any case where there are no constituents in the prefield, \ie where the verb trace does not project. (\mex{0}) and
also examples with a verb trace and a single constituent are also ruled out by general constraints
on putative cases of multiple fronting.\todostefan{Does the IS stuff restrict this? Should I include
the implicational constraint that enforces extraction?}


\section{Alternatives}
\label{sec-analyse-mfvorschlaege}

The problem posed by the present data for all theories assuming verb-second order cannot simply be solved
by marking problematic examples with `*' as \citet[\page 37]{Bungarten73a} does for examples like
(\ref{bsp-zum-zweiten-mal-die-Weltmeisterschaft}). There are just too many attested examples and for
this reason this data may not be ignored. There have been several proposals in the 1980ies and
1990ies and I will discuss each in turn.

\subsection{Movement of parts of the \mf and the verbal complex}

\citet*{Loetscher85a}\ia{Lötscher} has sketched the beginnings of a theory, which -- under certain conditions -- would allow
for an unlimited amount of constituents to be fronted.\footnote{
  Also see \citew[\page 412--413]{Eisenberg94a} for suggestion of a similar analysis.%
}
His proposal makes use of several rules, which have to be applied in a set order. These kinds of analyses are by their very nature
incompatible with theories based in a HPSG framework, since the principles of HPSG are unordered and hold equally for all structures.
Lötscher assumes that any chain in the left edge of the verbal complex can be fronted. These chains can contain verbs, which
would explain the fronting of partial projections. The adjacency of elements of the chain to the verbal complex could have come
about by movement operations in the middle-field. \citet[\page 92]{Duerscheid89a} has criticised Fanselow's \citeyearpar{Fanselow87a} approach,
and this criticism can also be applied to Lötscher's proposal: if fronting were in fact movement of any continuous chain from the left
periphery of a verbal complex into the initial position of a sentence, then (\mex{1}c) would be the underlying structure for the
fronting operation in (\mex{1}b).



\eal
\ex 
\gll dass ein Professor seinen Schüler nicht prüfen muss\\
     that a professor his student not test must\\
\glt `that a professor does not have to test his student'
\ex
\gll Seinen Schüler prüfen muss ein Professor nicht.\\
     his student test must a professor not\\
\ex 
\gll dass ein Professor nicht seinen Schüler prüfen muss\\
     that a   professor not   his student test must\\
\zl
The sentential negation precedes the verbal complex in example (\mex{0}a). In (\mex{0}c), the negation
has scope over \emph{seinen Schüler} `his student' and therefore does not correspond to the expected
base order for (\mex{0}b). According to \citet[\page 103]{Duerscheid89a}, a similar argumentation goes back to \citep{Thiersch86a}.

% Thiersch86a: 42-43

%
% \eal
% \ex{\iw{hinterlassen}
% [Der Nachwelt hinterlassen] hat sie [eine aufgeschlagene {\it Hör zu\/} und einen kurzen Abschiedsbrief]:\footnote{
%         taz, 18.11.1998, p.\,20.%
% }\label{bsp-der-nachwelt-hinterlassen-pred-compl}
% }
% \ex{
% weil sie der Nachwelt eine aufgeschlagene Hör zu und einen kurzen Abschiedsbrief hinterlassen hat.
% }
% \ex{
% weil sie eine aufgeschlagene Hör zu und einen kurzen Abschiedsbrief der Nachwelt hinterlassen hat.
% }
% \zl
%
% Sätze wie (\ref{bsp-vom-leutnant}) stellen keine Verletzung der Generalisierung dar,
% wenn man annimmt, daß es sich bei \emph{vom Leutnant zum Hauptmann}
% um eine Phrase handelt. Indizien für diese Annahme liefern Sätze wie
% (\mex{1}), in denen die Lokalangaben auch als eine Konstituente betrachtet
% werden können.
% \eal
% \ex[]{
% Wir befinden uns in Berlin.\iw{befinden}
% }
% \ex[]{
% Wir befinden uns in Berlin am Flußufer.
% }
% \ex[]{
% Wir befinden uns in Berlin am Flußufer unter der Brücke.
% }
% \ex[*]{
% Wir befinden uns.
% }
% \zl
% %Allerdings könnte man bei solche einem Ansatz nicht erklären, wieso die beiden Phrasen
% %verschiedene Rollen des Verbs füllen.
% %Allerdings können \emph{vom Leutnant} und \emph{zum Hauptmann} auch
% %getrennt im Mittelfeld auf"|treten.
% %\ea
% %Karl hat Peter vom Leutnant gestern zum Hauptmann befördert und nicht vorgestern.
% %} 

\subsection{Complex PPs formed from several PPs}

\citet*[\page 79]{Wunderlich84} suggested treating the fronted phrases in (\mex{1}) as a single constituent,
more specifically, a prepositional phrase.
\eal
\ex 
\gll {}[\sub{PP} [\sub{PP} Zu ihren Eltern] [\sub{PP} nach Stuttgart]] ist sie gefahren.\\
	{}       {}        to her parents   {}        to Stuttgart is she driven\\
\glt `She drove to Stuttgart to her parents.'
\ex\label{bsp-von-muenchen-nach}
\gll {}[\sub{PP} [\sub{PP} Von München]     [\sub{PP} nach Hamburg]]   sind es 900 km.\\
       {}        {} from Munich  {} to Hamburg are it 900 km.\\
\glt `It is 900 km from Munich to Hamburg.'	
\ex 
\gll {}[\sub{PP} [\sub{PP} Durch den Park]  [\sub{PP} zum Bahnhof]]    sind sie gefahren.\\
	{}       {}        through the park {}        to.the train.station are they driven\\
\glt `They drove through the park to the train station.'
\zl
Wunderlich assumes that the second PP in (\mex{0}) always modifies the first. This is possible when both
 PPs bear the same semantic role.\footnote{
        See \citew[\page 107--109]{Duerscheid89a} for a similar suggestion.%
}
In (\mex{0}a), both prepositional phrases denote the destination of some movement. Wunderlich admits that the
thematic roles in (\mex{0}b) and (\mex{0}c) are different (source, route or destination of movement) and tries to
subsume them under the broader heading of `localization of movement'.
This approach is not satisfactory, however, as it would be difficult for a HPSG grammar to reconstruct the individual roles 
related to each verb from the broader `localization of movement'. The examples in (\mex{0}) and also examples such as 
(\ref{bsp-vom-leutnant}) can only be analyzed in the way Wunderlich does if each prepositional
phrase is analyzed as modifier, that is, if they do not receive a semantic role from some verb.
\ea
\label{bsp-vom-leutnant}%
\gll [Vom Leutnant] [zum Hauptmann] wird Karl befördert.\iw{befördern}\\
	 \spacebr{}from lieutenant \spacebr{}to.the captain becomes Karl promoted\\
\glt `Karl is getting promoted from lieutenant to captain.'
\z
This is, in my opinion, not an adequate explanation.
%Eine genau Analyze solcher Konstruktionen steht also noch aus.

%Duerscheid89a:87
% zitiert allerdings Engel und nimmt die Beispiele aus



\citet[\page 62]{Riemsdijk78a} discusses data from Dutch, which are parallel to 
(\ref{bsp-von-muenchen-nach}). He suggests analysing the first PP as the specifier of the second.
The specifier analysis also runs into problems when both prepositional phrases are complements and
are independently associated with a verb.

\citet[\page 217--218]{Dowty79a} discusses (\mex{1}) in a different context:
\ea
John drives a car from Boston to Detroit.
\z
He suggests that \emph{Boston} as well as \emph{to Detroit} are complements of \emph{from}. This analysis
would not however be able to shed light on (\ref{bsp-vom-leutnant}). Furthermore, it is not compatible with
other cases of multiple fronting.

\subsection{Fronting and LF correspondence restrictions}

\ia{Haider|(}
\citet*[\page 17]{Haider82} formulated a condition similar to that of Wunderlich. According to Haider, the LF"=projection\is{Logical Form (LF)}
of the prefield has to correspond to a single LF"=constituent. LF stands for `Logical Form' in Government and Binding theory. Haider's condition
allows for the simultaneous fronting of adverbs and fronting of certain non-maximal verbal projections.

Haider discusses the contrast between the following examples in (\mex{1}):
\eal
\ex[]{
\gll Wann und wo hat sie sich mit ihm getroffen?\\
	 when and where has she \textsc{refl} with him met\\
\glt `When and where did she meet him?'
}
\ex[*]{
\gll Wann und wer hat sich mit ihm getroffen?\\
     when and who has \textsc{refl} with him met\\
}
\zl
He explains the difference by claiming that the \emph{wh}-words together bind a single empty adverbial position. This is not
possible for (\mex{0}b). He offers a similar explanation for (\mex{1}).
\ea
\label{bsp-gestern-am-strand-haider}
\gll Gestern am Strand hat sie sich mit ihm getroffen.\\
	 yesterday on.the beach has she \textsc{refl} with him met\\
\glt `She met him yesterday on the beach.'
\z
It is plausible to assume, as Haider does, that temporal and spatial adjuncts form a single constituent. In this case, instances of fronting such as
(\mex{0}) would be unproblematic. Nevertheless, we have seen in Section~\ref{sec-phenomenon-mult-front} that complements can be fronted along with adjuncts.
If we compare examples (\ref{bsp-nichts-mit-derartigen}) and (\ref{bsp-zum-zweiten-mal-die-Weltmeisterschaft}) with the previous examples, it is clear
that the coordination test does not really tell us much:



\eal
\ex[]{
\gll {}[Nichts] [mit derartigen Entstehungstheorien] hat es natürlich zu tun, \ldots\\
	    \spacebr{}nothing \spacebr{}with those.kind origin.theories has it of.ocurse to do\\
\glt `It of course has nothing to do with those kinds of theories of origin.'
}
\ex[]{
\gll Was hat das mit derartigen Entstehungstheorien zu tun?\\
	 what has that with those.kind origin.theories to do\\
\glt `What has that got to do with those kinds of theories of origin?'
}
\ex[]{
\gll Womit hat das nichts zu tun?\\
	 with.what has that nothing to do\\
\glt `What has that got nothing to do with?'
}
\ex[*]{
\gll Was und womit hat das zu tun?\\
	 what and with.what has that to do\\
}
\zl
\eal
\ex[]{
\gll {}[Zum zweiten Mal] [die Weltmeisterschaft] errang\iw{erringen} Clark 1965 \ldots\\
	    \spacebr{}to.the second time  \spacebr{}the world.championship won Clark 1965\\
\glt `Clark won the world championship for the second time in 1965.'
}
\ex[]{
\gll Zum wievielten Mal errang Clark 1965 die Weltmeisterschaft?\\
	 to.the how.many time won Clark 1965 the world.championship\\
\glt `How many times was it that Clark had won the world championship in 1965?'
}
\ex[]{
\gll Was errang Clark 1965 zum zweiten Mal?\\
	 what won Clark 1965 to.the second time\\
\glt `What did Clark win for the second time in 1965?'
}
\ex[*]{
\gll Was und zum wievielten Mal errang Clark 1965?\\
	 what and to.the how.many time won Clark 1965\\
	 }
\zl

There are also other combinations of adjuncts in the prefield, \eg (\ref{bsp-instrument}),
where assuming a single constituent of the Haider type is somewhat questionable. 
%% An anonymous reviewer from \emph{Linguistische
%% Berichte} pointed out that the question of why (\mex{1}) is worse than (\ref{bsp-gestern-am-strand-haider})
%% also remains unanswered.
%% \ea
%% \gll Gestern und am Strand hat sie sich mit ihm getroffen.\\
%% 	 yesterday and on.the beach has she \textsc{refl} with him met\\
%% \glt `She met him yesterday at the beach.'
%% \z



Furthermore, Haider's constraint exludes fronting of non-maximal projections which consist of a verb
and a dative object \citet*[\page 17]{Haider82}. Haider offers the following example, which he classes as ungrammatical:
\ea
\gll Seiner Tochter erzählen konnte er ein Märchen mit ruhiger Stimme.\\
	 his daughter.\dat{} tell could he.\nom{} a fairy.tale\acc{} with quiet voice\\
\glt `He could tell his daughter a fairy tale in a quiet voice.'
\z
The unacceptability of the sentence has nothing to do with its syntactic structure, but is rather to do with the information
structural requirements which must be fulfilled for a verbal projection to be fronted. If we change the lexical material in (\mex{0}),
the result is a perfectly acceptable sentence:
\ea
\gll Den Wählern erzählen sollte man so was nicht.\\
	 to the voters.\dat{} tell should one.\nom{} such a.thing.\acc{} not\\
\glt `One shouldn't tell the voters something like that.'
\z
Examples in (\ref{bsp-besonders-einsteigern}) and (\ref{bsp-der-nachwelt-hinterlassen}) are further cases of fronting a verb with its dative object:\footnote{
	The data in (\ref{bsp-pvp}) can also be found in \citep[\page 353--354]{Mueller99a}.
	 \citet[\page 91]{Thiersch82a}, \citet[\page 429]{Sternefeld85a},
%\citet[\page 103]{Scherpenisse86a}, 
\citet[\page 159]{Uszkoreit87a},
\citet[\page 459]{SS88a},
\citet[Chapter~1.5.3.3.1]{Oppenrieder91a}, \citet[\page 1301]{Grewendorf93} and G.\ \citet[\page 5]{GMueller98a}
offer their own examples of a dative complement being fronted together with its verb.%
}
Haider's constraint can therefore be rejected as being too restrictive.

As with Wunderlich's analysis, Haider's approach also struggles to explain (\ref{bsp-vom-leutnant}). 
\ia{Haider|)}

%Bei Sätzen wie (\ref{pvp-fvg-idioms}) handelt es sich um 
%Funktionsverbgefüge\is{Funktionsverbgefüge}\footnote{
%        Zur Analyze von Funktionsverbgefügen siehe
%\citep*{KE94}\iaf{Krenn}\iaf{Erbach} und \citep*{Kuhn95}\iaf{Kuhn}.
%} bzw.\ idiomatische Wendungen, \dh, es liegen
%komplexe Prädikate vor: {\it etwas in etwas bringen\/},\iw{bringen!etwas in etwas $\sim$} 
%{\it zum Opfer fallen\/}.\iw{fallen!zum Opfer $\sim$}
%Die Voranstellungen in (\ref{pvp-fvg-idioms}) haben das Muster der
%Voranstellung von Phrasenteilen, die in Chapter~\ref{pvp} beschrieben
%wird. Eine genaue Analyze für diese Sätze steht ebenfalls noch aus.

\subsection{Apparent multiple frontings as multiple frontings}
\label{sec-mueller-2000}\label{sec-speyer2008-syntax}


In an earlier proposal I assumed that multiple frontings are just multiple extractions
\citep{Mueller2000d}. The respective analysis is sketched in Figure~\vref{fig-mult-extraction}.
\begin{figure}
\centering
\begin{forest}
[{S[\slasch \eliste]}
  [C$_1$] 
  [{S[\slasch \sliste{ C$_1$ }]}
     [C$_2$] 
     [{S[\slasch \sliste{ C$_1$, C$_2$ }]}]
  ]]
\end{forest}
\caption{Multiple frontings as multiple extractions according to \citet{Mueller2000d}}\label{fig-mult-extraction}
\end{figure}%
A sentence with two gaps (C$_1$, C$_2$) is combined with appropriate fillers in two steps.

Similarly, \citet{Speyer2008a} suggests a Rizzi-style analysis of German (\citealp{Rizzi97a-u};
Grewendorf \citeyear[\page 85, 240]{Grewendorf2002a}; \citeyear{Grewendorf2009a}) in which he
assumes several functional projections for topic and focus before the finite verb. For instance,
Figure~\vref{fig-clause-structure-speyer} shows the analysis of (\mex{1}):
\ea
\gll Briefe hat Uller geschrieben.\\
     letters has Uller written\\
\glt `Uller has written letters.'
\z
\begin{figure}
\centerfit{%
\begin{forest}
[ ForceP
   [ SpecForceP ]
   [ Force$'$
     [Force ]
     [Top1P
        [SpecTop1P]
        [Top1$'$
           [Top]
           [FocP
             [SpecFocP\\
              \emph{Briefe}\sub{[foc] 2}]
             [Foc$'$
               [Foc\\
                {$\varnothing$,[foc]}]
               [Top2P
                 [SpecTop2P]
                 [Top2$'$
                   [Top2]
                   [FinP
                     [SpecFinP\\
                      t$_2$]
                     [Fin$'$
                       [Fin\\
                        \emph{hat}$_1$]
                       [IP
                        [Uller t$_2$ geschrieben t$_1$,roof]]]]]]]]]]]]
\end{forest}
}
\caption{Analysis of \emph{Briefe hat Uller geschrieben} `Uller has written letters.' according to
  \citet[\page 471]{Speyer2008a}}\label{fig-clause-structure-speyer}
\end{figure}
The subject and main verb and finite auxiliary are generated as part of the IP and then the finite
verb is moved to the head position of the Fin head. The object of \emph{geschrieben} `written' moves
to the specifier position of the Fin head and leaves a trace there when moving on to the specifier
position of an empty Focus head. In sentences in which a topic fills the \vf it is assumed that the
fronted element moves on from the SpecFinP position into the specifier position of a topic head. The
Topic and Focus projections are assumed to be present in the structure even if no focus or topic element is present
in the clause. The Force head is assumed to host features that are relevant for determining the
clause type.

In such approaches, a sentence like our example in
(\ref{bsp-zum-zweiten-mal-die-Weltmeisterschaft}) -- repeated here as (\mex{1}) -- has an analysis
in which there are two extraction traces in the \mf: one for \emph{zum zweiten Mal} and one for
\emph{die Weltmeisterschaft}.

\ea
\gll {}[Zum zweiten Mal]$_i$ [die Weltmeisterschaft]$_j$ errang \_$_i$ \_$_j$ Clark 1965 \ldots\\
	   \spacebr{}to.the second time \spacebr{}the world.championship won {} {} Clark 1965 {}\\%
\footnote{
        \citep*[\page 162]{Benes71}
      }\label{bsp-zum-zweiten-mal-die-Weltmeisterschaft-anal}%
\glt `Clark won the world championship for the second time in 1965.'
\z

\subsubsection{Same verb constraint}

This proposal has various problems: first, it cannot be explained why the elements in the \vf have
to depend on the same verb (see Section~\ref{sec-clause-mates}). The following example from
\citet[\page 57]{Fanselow87a} shows that more than one extraction can go on in German sentences.
\ea
\label{ex-radios-weiss-ich-nicht}
\gll Radios weiß ich nicht, wer repariert.\\
     radios know I not who.\nom{} repairs\\
\glt `I do not know who repairs radios.'
\z
The interrogative pronoun is in initial position of the interrogative clause, which is usually
analyzed as extraction since the interrogative phrase may depend on a deeply embedded
head. \emph{Radios} is the object or \emph{repariert} `repairs' and hence extracted from the
interrogative clause \emph{wer repariert} `who repairs'.

Now, the question is: why are sentences like Fanselow's sentences in
(\ref{ex-mult-front-same-verb}b,d) on page~\pageref{ex-mult-front-same-verb} impossible? The first
two of the sentences in (\ref{ex-mult-front-same-verb}) are repeated below for convenience:
\eal
\ex[]{
\gll Ich glaube  dem Linguisten nicht, einen Nobelpreis  gewonnen zu haben.\\
     I   believe the linguist   not    a     Nobel.prize won      to have\\
\glt  `I don't believe the linguist's claim that he won a Nobel prize.'
}
\ex[*]{
\gll Dem Linguisten$_i$ einen Nobelpreis$_j$  glaube \_$_i$ ich nicht [ \_$_j$ gewonnen zu haben].\\
     the linguist   a     Nobel.price believe {}    I   not   {} {}   won      to have\\
}
\zl
In the analysis presented in the previous section it is clear that sentences like (\mex{0}b) are
ruled out since the fronted material has to depend on the same verb. There is no such explanation
for the multiple extraction approach.

\subsubsection{Elements that cannot be extracted (idiom parts)}

Furthermore, as already explained in Section~\ref{sec-idiom-parts-mf}, idioms pose a challenge for the
multiple-extraction approach.

\eal
\ex[]{\label{ex-oel-ins-feuer}
\gll {}[Öl] [ins Feuer] goß gestern das Rote-Khmer-Radio\footnotemark\\
	 \spacebr{}oil \spacebr{}in.the fire poured yesterday the Rote-Khmer-Radio \\
\footnotetext{
        taz, 18.06.1997, p.\,8.
}
\glt `Rote-Khmer-Radio fanned the flames yesterday'
}
\ex[*]{\label{ex-ins-feuer-goss}
\gll [Ins Feuer] goß gestern das Rote-Khmer-Radio Öl.\\
     \spacebr{}in.the fire poured yesterday the Rote-Khmer-Radio oil\\
}
\zl 

\noindent
If both \emph{Öl} `oil' and \emph{ins Feuer} `in.the fire' are extracted in (\mex{0}a), it is difficult to see how
(\mex{0}b) can be ruled out. In the approach with an empty verbal head in the \vf, neither \emph{Öl} `oil'
nor \emph{ins Feuer} `in.the fire' is extracted but both phrases are just combined with an empty verbal head as
they are in sentences like (\mex{1}):
\ea
\gll Das Rote-Khmer-Radio goß gestern Öl ins Feuer.\\
     the Rote-Khmer-Radio poured yesterday oil in.the fire\\
\glt `Rote-Khmer-Radio fanned the flames yesterday`.'
\z
(\mex{1}) shows that \emph{Öl} `oil' can be extracted, but \emph{ins Feuer} `in.the fire' cannot be extracted as
(\ref{ex-ins-feuer-goss}) shows.
\ea
\gll Öl goss auch Lord O’Donnel ins Feuer.\footnotemark\\
     oil poured also Lord O'Donnel in.the fire\\
\glt `Lord O'Donnel also fanned the flames.'
\footnotetext{
\url{http://www.swp.de/ulm/nachrichten/politik/Brexit-ja-aber-nicht-so-fix;art1222886,3985964}, 26.09.2016.
}
\z
So, if (\ref{ex-oel-ins-feuer}) is analyzed as double extraction, one has to find ways to say that
\emph{ins Feuer} `in.the fire' can be extracted only if \emph{Öl} `oil' is extracted as well. It may be possible to do
this but it is highly likely that the system of constraints that is needed to pin that down formally
is highly complex.

%% Zusätzlich Öl ins Feuer goss, dass sich just in dem Moment in Indien ein todbringendes Erdbeben
%% ereignete, als die Statue enthüllt wurde.
%% http://derwaechter.net/superzelle-oder-uebernatuerliches-phaenomen-unglaublicher-sturm-wuetet-ueber-cern

%% Kräftig Öl ins Feuer goss dagegen heute Morgen der türkische Präsident Recep Tayyip Erdogan.
%% http://www.t-online.de/nachrichten/ausland/krisen/id_77442016/krieg-um-berg-karabach-aserbaidschan-macht-rueckzieher.html

\subsubsection{Order of elements in the \vf}

Finally, approaches that assume that individual items are extracted from the \mf and fronted
independently have to explain why the fronted material has to appear in the same order as it appears
in the unmarked order in the \mf. This is automatically explained if one assumes that the fronted
material is part of a verbal projection since then of course one would have all the verbal fields
available: \mf, right sentence bracket, and \nf. As the discussion above showed we need all these
topological fields: particles of particle verbs may fill the right sentence bracket inside a complex
\vf and pronominal adverbs may be extraposed in the complex \vf (\ref{ex-los-damit}), which is
evidence for a \nf. If the fronted material is part of a complex \vf that is the projection of a
verbal head, all facts are explained immediately.

One could try and derive the constraints on the order in the \vf from more general constraints that
are usually assumed in the literature. For instance, one could assume that there must not be any
crossing dependencies. However, there are sentences like (\ref{ex-radios-weiss-ich-nicht}) -- repeated here as (\mex{1}) -- in which
the object is realized before the subject although both subject and object are moved.
\ea
\label{ex-radios-weiss-ich-nicht-zwei}
\gll Radios weiß ich nicht, wer repariert.\\
     radios know I not who.\nom{} repairs\\
\glt `I do not know who repairs radios.'
\z
Note furthermore that such a general ban on extraction structures would be in conflict with Speyer's
original motivation for a Rizzi-style analysis. He agreed that my proposal for the analysis of
modern German is basically on the right track  but criticized the fact that it was not applicable to Early New High
German\il{Early New High German} \citep[\page 461]{Speyer2008a}. The interesting fact about earlier stages of German is that the constraint that the elements
in the \vf have to be in the same order as they would appear in the \mf in unmarked order does not
hold for Middle Low German\il{Middle Low German} \citep[Section~5.2]{Petrova2012a}. The following sentences from \citew[\page
  174--175]{Petrova2012a} illustrate:
\eal
%%\ex
%% hier sieht man keine Verbklammer
%% \gll [Dre sone] [he] leet\\
%%    \spacebr{}three sons \spacebr{}he left\\
%% \glt `He left behind three sons.’
\ex
\gll [Eine warheit] [ich] wille dir sagen\\
     \spacebr{}one truth \spacebr{}I want you-Dat tell\\
\glt `I want to tell you a certain truth.’
\ex
\gll [Sea] [en thegan] habda Joseph gimahlit\\
     \spacebr{}she.\acc{} \spacebr{}a man had Joseph married\\
\glt `A man [called] Joseph had married her.'
\zl
In both sentences the direct object is realized before the subject.

Petrova also assumes a Rizzi-style analysis of
Early New High German. So, since there are languages that allow the order of fronted elements to
differ from the normal order in the \mf, the restrictions that we observe in modern German cannot be
explained by reference to general constraints like the Shortest Move Constraint or the Minimal Link
Condition. It follows that the \vf-\mf correspondence would have to be stipulated for modern German
in Speyer's model, while it is entirely expected in any model with an empty verbal head.


\subsubsection{Assignment to Topic and Focus positions}
\label{sec-assignment-to-top-foc}

While the proposal in \citew{Mueller2000d} does not make any claims about information structure, the
proposal by \citet{Speyer2008a} assumes Rizzi-style functional projections. According to
\citet[\page 482]{Speyer2008a} the \vf consists of elements that are moved to the specifier position
of a SceneP, a FocP, and a TopP that are ordered in this way. He assumes that the upper topic phrase
in the analysis of \citet{Rizzi97a-u} and \citet{Grewendorf2002a} is specialized to contain
frame-setting elements. So, the TopP, FocP, TopP sequence of the former models -- which is also
depicted in Figure~\ref{fig-clause-structure-speyer} -- is more constrained in Speyer's model. Speyer claims that the historical development from Early High German to
modern German resulted in this specialization. Speyer's proposal as presented in his paper
predicts that there can be at most three constituents in the \vf: one scene element, one focus
element, one topic. So if complex frontings with more than three elements are possible, Speyer's
theory is falsified. In Section~\ref{sec-fronting-more-than-two} I showed that frontings with three elements can be found and
provided Arne Zeschel's example (\ref{bsp-ihnen-für-heute-zwei}) with a complex \vf containing four
elements. The example is repeated here as (\mex{1}) for convenience:
\ea\label{bsp-ihnen-für-heute-drei}
\gll {}[Ihnen] [für heute] [noch] [einen schönen Tag] wünscht Claudia Perez.\footnotemark\\
  \spacebr{}you.\dat{} \spacebr{}for today \spacebr{}still \spacebr{}a.\acc{} nice day wishes Claudia Perez\\
\footnotetext{
  Claudia Perez, Länderreport, Deutschlandradio.
}%
\glt `Claudia Perez wishes you a nice day.'
\z

\noindent
Note that Rizzi and Grewendorf assume that the
Topic projections are recursive. So in principle there could be as many topic positions as needed,
followed by one optional focus position, followed by arbitrarily many topic positions. This could
account for multiple elements in the \vf provided some of them are topics. Note, however, that none
of the fronted constituents in (\mex{0}) are topics. The radio speaker announced the program for the
next week and said good buy to the hearers. The reason for the fronting was to emphasize the name of
the speaker. The fronted material is not put in the \vf because it has a certain information
structural function like topic or contrastive focus, it is moved out of the way to make other material more
prominent. \citet{BC2010a} discussed a similar construction, which they called Presentational Multiple Fronting. It
will be discussed in more detail in Section~\ref{sec-presentational-MF}. An example of this
construction is (\mex{1}b), with (\mex{1}a) and (\mex{1}c) providing some context:
\eal
\ex Spannung pur herrschte auch bei den Trapez-Künstlern. [\ldots]  Musika\-lisch begleitet wurden die einzelnen Nummern vom Orchester des Zir\-kus Busch [\ldots]\\
    `It was tension pure with the trapeze artists. [\ldots] Each act was musically accompanied by Circus Busch's own orchestra.'
\ex
\gll [Stets] [einen Lacher] [auf ihrer Seite] hatte \textit{die} \textit{Bubi} \textit{Ernesto} \textit{Family}\lindex{i}.\\
 \hspaceThis{[}always  \hspaceThis{[}a laugh  \hspaceThis{[}on their side had the Bubi Ernesto Family\\
\glt `Always good for a laugh was the Bubi Ernesto Family.'
\ex Die Instrumental-Clowns\lindex{i} zeigten ausgefeilte Gags und Sketche [\ldots]\\
    `These instrumental clowns presented sophisticated jokes and sketches.'\\\sigle{M05/DEZ.00214} 
\zl
This example will be discussed in more detail on page~\pageref{pres}. What is important here is that
the material in the \vf is moved out of the way in order to present the NP in the \mf, which is then
the topic of the following clause. Speyer's analysis fails on examples like this and on the
Claudia Perez example in (\ref{bsp-ihnen-für-heute-drei}). In general, feature driven accounts that
assume that movement is triggered by features that have to be checked\is{feature checking} \citep{Chomsky95a-u}
fail on this data since the movement that is required here is altruistic movement\is{movement!altruistic}, that is, movement
that takes place for the benefit of some other element. See also \citet{Fanselow2003b} on other
cases of altruistic movement.

Speyer's proposal assumed Rizzi/Grewendorf structures and this aspect was criticized in this
section. Speyer works in the framework of Stochastic Optimality Theory in order to explain the
markedness and rareness of the phenomenon in Modern Standard German and in order to explain the
historical development from Early High German. I will turn to the discussion of the OT aspects in Section~\ref{sec-speyer-stochastic-OT}.


\subsection{V3 as adverb + clause}

For examples with sentences adverbs similar to (\ref{bsp-vermutlich}) -- repeated here as (\mex{1})
for convenience --, \citet[\page 111]{Jacobs86a} proposed a rule which combines
a verbal projection with an adverb. 
\ea
\label{bsp-vermutlich-zwei}
\gll {}[Vermutlich] [Brandstiftung] war die Ursache für ein Feuer in einem Waschraum in der Heidelberger Straße.\footnotemark\\
	   \spacebr{}supposedly \spacebr{}arson was the cause for a fire in a washroom in the Heidelberger Street\\
\footnotetext{
Mannheimer Morgen, 04.08.1989, Lokales; Pflanzendieb.
}
\z
Jacobs' rule also licenses the combination of a V2-clause with a sentence adverb and hence can be
used for the analysis of sentences like (\mex{0}). However, this approach encounters problems with
similar examples where the sentence adverb follows a preposed constituent:\footnote{ 
        The following examples are taken from \citet[\page 228]{Engel88}.%
}
\eal
\ex 
\gll Damit freilich muß er allein fertig werden.\\
     with.that simply must he alone finished become\\
\glt `He will simply have to come to terms with it himself.'
\ex 
\gll Ein paar Wochen immerhin ist noch Zeit.\\
     a few weeks nevertheless is still time\\
\glt `Well, we've still got a few weeks.'
\zl
\citet[\page 26]{Duerscheid89a} argues that these kinds of examples should also be analyzed as instances of multiple fronting, since the
sentence adverb refers to the entire sentence and not just to the fronted constituent.
In order to explain examples such as (\mex{0}), Jacobs would have to allow prepositional phrases or pronominal adverbs such as \emph{damit}
and NPs such as \emph{ein paar Wochen} to be combined with V2-clauses. This analysis is very similar
to the one discussed in Subsection~\ref{sec-mueller-2000} and therefore shares the previously discussed drawbacks of this analysis.


%% \subsection{Stochastic OT}
%% \label{sec-speyer-stochastic-OT}

%% \citew{Speyer2008a} developed an Optimality Theory account of multiple frontings. As outlined in
%% Section~\ref{sec-speyer2008-syntax}, he assumes that multiple fronting is a general option in the syntax of German. There
%% is a Scene, a Focus and a Topic position that can be filled by one element. Multiple fronting exists
%% as a phenomenon but it is rare in general. Optimality Theory assumes a component that generates
%% various sentences. These sentences are then compared and the optimal candidate is chosen. In
%% addition to general constraints on the well-formedness of the generated structures (which are often
%% not specified in OT papers) there are constraints that are used to rank the alternatives. Speyer
%% assumes a version of \xbart that is compatible with Rizzi/Grewendorf analyses of clause structure
%% and additionally specifies OT constraints. If these constraints were strict constraints including a
%% V2 constraint, the V2 constraint should dominated all other constraints and the result would be that
%% V3 sentences would always be filtered out since the competition model of OT would not allow for
%% orders that are less successful in terms of constraint rankings.

%% Due to this situation Speyer adopted the framework of Stochastic OT that works with probabilities:
%% even though a constraint is higher ranked than others there is certain variability in the rankings
%% and hence even lower ranked constraints may win from time to time \citep[\page 463]{Speyer2008a}. This allows for occasional V3 structures to be generated.

%% Speyer carefully examined Böll's book \emph{Ansichten eines Clowns} and states that it contains only
%% one example of multiple fronting, namely the sentence in (\ref{ex-inzuepfners-box-der-mercedes}) -- repeated here for convenience:
%% \ea
%% \label{ex-inzuepfners-box-der-mercedes-zwei}
%% \gll {}[[In Züpfners Box] [der Mercedes]] bewies, dass Züpfner zu Fuß gegangen war.\footnotemark\\
%%        \hspaceThis{[[}in Züpfners box \spacebr{}the Mercedes proofed that Züpfner by foot went was\\
%% \footnotetext{
%% Böll, Heinrich (1963): \emph{Ansichten eines Clowns}. Köln: Kiepenheuer \& Witsch. Quoted from
%% \citew[\page 456]{Speyer2008a}.
%% }
%% \glt `The Mercedes in Züpfners box was proof of Züpfner's walking.'
%% \z
%% Speyer states that this sentence is one out of approximately 6.000, which corresponds to a
%% percentage of 0,016\,\%. According to Speyer this rareness is reflected in the stochastic part of
%% his approach: in general the V2 constraint (1-VF: Exactly one constituent is placed in the \vf) is
%% ranked higher than other constraints but due to the stochastic nature, constraints like
%% Scene-Setting-VF (a frame-setting element is in the \vf) and Contrast-VF (a contrastive element is
%% in the \vf) may be ranked higher. Such deviant rankings allow the derivation of sentences like
%% (\mex{1}):
%% \ea
%% \gll Gestern Briefe hat Uller geschrieben.\\
%%      yesterday letters has Uller written\\
%% \glt `Uller wrote letters yesterday.'
%% \z
%% \citet[\page 466]{Speyer2008a} assumes that such a paradoxical ranking happened when Böll wrote the
%% sentence in (\ref{ex-inzuepfners-box-der-mercedes-zwei}).\footnote{
%% ``Nehmen wir an, dass die PP \emph{In Züpfners Box} ähnlich behandelt wird wie ein klassisches rahmenbildendes
%% Element (von dem es sich, wie angedeutet, nur durch den Skopus, also
%% ein strukturelles Merkmal, aber nicht intrinsisch unterscheidet). Dann kann man
%% annehmen, dass genau solch ein paradoxes Ranking vorgefallen sein muss, als
%% Böll diesen Satz generiert hat.''
%% }
%% I find this approach rather unappealing. It would entail that Nobel prices for literature are due to
%% chance. If it would really be a matter of chance whether certain candidates win in a competition or
%% not one would expect that authors who check their manuscript before submission and the page proofs
%% after acceptance would weed out highly unlikely (and therefore marked) sentences, since the chance
%% that the same constraint ranking happens again in the very moment they read their sentence again is
%% minimal to the extreme. In addition there is an editor who checks manuscripts and the constraint
%% ranking has to happen for him/her at exactly the moment when he/she reads the unlikely/marked sentence.

%% So rather than attributing all multiple frontings to paradoxical constraint violations I would like
%% to claim that there are certain reasons to deviate from the canonical German V2 structure. Writers
%% who master their language are capable to form sentences that are appropriate for certain contexts
%% and know when the respective structure is appropriate. We try to describe some of these situations
%% and provide the respective constraints on information structure properties of the sentences in Chapter~\ref{chap-is}.

%% Apart from these considerations I think that the example from Böll is not an instance of multiple
%% constituents in the \vf but an instance of NP-internal fronting (see Section~\ref{sec-np-internal-frontings}).



\subsection{Remnant movement}


The analyses which come closest to the one I will develop in the following section are those of
\citet{Fanselow93a} and G.\ \citet[Chapter~5.3]{GMueller98a}. Both authors assume that a sentence such as
(\ref{bsp-zum-zweiten-mal-die-Weltmeisterschaft}) has a structural representation as in (\mex{1}) (although
both authors make different assumptions about the nature of the verb trace in the prefield).
\ea
\gll {}[\sub{VP} [Zum zweiten Mal] [die Weltmeisterschaft] \_\sub{V} ]$_i$ errang$_j$ Clark 1965 \_$_i$ \_$_j$.\\
     {}          \spacebr{}to.the second time  \spacebr{}the world.championship {} {} won Clark 1965\\
\z
Fanselow claims that \_\sub{V} is a verb trace, similar to the one which plays a role in gapping. G. Müller,
on the other hand, assumes that \_\sub{V} is a normal verb trace and that cases such as (\mex{0}) should be analyzed
as (\emph{remnant movement}).\footnote{
        Analyses using remnant movement have a long tradition. They started with the work of
		Gert Webelhuth und Hans den Besten \citeyearpar{WdB87a}
        and Craig \citet{Thiersch86a}, which was sadly unpublished.%
}
\citet[Abschnitt~7]{Fanselow2002a} follows G. Müller's remnant movement analysis for cases of multiple fronting.

\citet[\page 281]{Haider93a}, \citet{deKuthy2002a}, \citet{dKM2001a} and
\citet{Fanselow2002a} have however shown that remnant movement analyses of discontinuous NPs and the fronting of incomplete
verbal and adjectival projections run into empirical problems. G.\ \citet{GMueller2014a-u} discusses the scrambling of indefinite problems, but ignores the other problems pointed out by the authors just cited. Therefore, I will pursue an analysis in which 
putative cases of multiple fronting are explained via argument attraction (this corresponds to reanalysis approaches
in the Principles and Parameters Framework). 




\section{Conclusion}
\label{sec-zusammenfassung}

In this chapter, I have presented data which had previously been neglected in many other works. Upon
closer inspection, however, it becomes clear that multiple fronting is in fact not that unusual and
that it is possible to identify clear patterns. This chapter was an attempt to integrate multiple
fronting into the current syntax of German. This chapter provides the analysis of the syntax of
apparent multiple frontings and explains how the interface to semantics works. Of course further
constraints on prosody and information structure are needed for a better understanding of the
phenomenon. I will turn to information structure in Chapter~\ref{chap-is} after having discussed clause types
in the following chapter.



%% \subsection{Multiple NPs preceding the finite verb}

%% The reason for the general unacceptability of (\mex{1}) should also be a focus of future research:
%% \ea[?*]{
%% \label{zwei-np-in-vf}
%% \gll Maria Peter stellt Max vor.\\
%%      Maria Peter introduces Max PRT\\
%% \glt `Max introduces Maria to Peter.'
%% }
%% \z
%% It is not possible to rule out these examples by implementing a general ban on two objects in the prefield
%% (as Haider does \citeyearpar[\page 15]{Haider82}), since these kinds of sentences do seem possible
%% as in the examples in (\mex{1}) and the example in (\mex{2}), which was presented to me by Anette Frank (p.\,c.\ 2002).
%% \eal
%% \ex 
%% \gll Der Maria einen Ring glaube ich nicht, dass er je schenken wird.\footnotemark\\
%% 	 the Maria a ring believes I not that he ever give will\\
%% \footnotetext{
%% \citew[\page 67]{Fanselow93a}.
%% }
%% \glt `I dont think that he would ever give Maria a ring.'
%% \ex 
%% \gll Ihm den Stern hat Irene gezeigt.\footnotemark\\
%% 	 him the star has Irene shown\\
%% \footnotetext{
%%   \citew[\page 412]{Eisenberg94a}.%
%% }
%% \glt `Irene showed him the star.'
%% \ex 
%% \gll (Ich glaube) Kindern Bonbons gibt man besser nicht.\footnotemark\\
%%      \hspaceThis{(}I think children candy gives one better not\\
%% \footnotetext{
%%         G.\ \citew[\page 260]{GMueller98a}.
%% }
%% \glt `I think it's better not to give candy to children.'
%% \zl
%% \ea 
%% \gll Studenten einem Lesetest unterzieht er des öfteren.\\
%%      students a reading.test subjects.to he the often\\
%% \glt `He often makes his students do a reading comprehension test.'	
%% \z
%% Attested examples with two noun phrases in the prefield were provided in
%% (\ref{ex-dem-saft-eine-kraeftige-farbe}), (\ref{bsp-zeitgeist}) and (\ref{bsp-weiterhin-derjugend}).


%% It might be possible to rule out sentences such as (\ref{zwei-np-in-vf}) by additional constraints, which take the
%% information structure of an utterance into account. It is known from research into the fronting of partial
%% verb phrases that factors such as the definiteness of noun phrases affects the acceptability of
%% respective frontings (\citealp[\page 45--46]{Kratzer84a}\ia{Kratzer};
%% \citealp{Haider90a}).

%% With regard to \pref{die-kinder-nach-stuttgart} and \pref{mit-den-huehnern-ins-bett}, \citet[\page 81]{Engel70a} notes
%% that multiple fronting is also used for means of constrastive focus. Multiple fronting is then often used to
%% focus several elements simultaneously. The examples discussed in Section~\ref{sec-phenomenon-mult-front} und in \citew{Mueller2003b}
%% show, however, that constrastive interpretation cannot be the only reason for multiple fronting.

%% \newcommand{\ao}{Avgustinova und Oliva\xspace}%
%% Avgustinova and Oliva \citeyearpar{AO95a,AO97a} examined constituents which can occur in front of clitics in Czech. Normally, there
%% is only exactly one constituent in this position. \ao discuss exceptions to the position of clitics as second in the clause and claim
%% that constituent groupings which may precede clitics also occur before the finite verb in V2"=languages such as German, Dutch and Swedish.
%% They investigated path topicalization, PP-iteration and fronting of various adverbials.
%% %
%% They reach the generalization that syntactic constituents with the same communicative weight assignment (the first "'considerably communicative"'
%% segment) can occur in fronting of the designated second position. In a similar vein, \citet[\page 1639]{Hoberg97a} appeals to a "'specific kind of
%%  minimal communicative entity"'.
 
%% This generalization is itself not sufficient to rule out cases of ungrammatical fronting such
%% as (\mex{1}) as an answer to the question \emph{Who ordered what?}:
%% \ea[*]{
%% \gll Ich das Wienerschnitzel habe bestellt.\footnotemark\\
%%      I   the wiener.schnitzel have ordered\\
%% \footnotetext{
%%         \citew[\page 316]{Lenerz86a}. Also see \citet[\page 32]{Duerscheid89a}.
%% }	 
%% \glt `I ordered the Wiener schnitzel.'
%% }
%% \z










\nocite{HN94a,Kiss95a}
%\bibliography{bib-abbr,biblio}
%\bibliographystyle{natbib.myfullname}


\if 0 



\fi


% Local variables:
% mode: lazy-lock
% End:


%      <!-- Local IspellDict: en_US-w_accents -->

%% -*- coding:utf-8 -*-

\chapter{Clause types}

This chapter is devoted to a description of the basic clause types and the integration of their
semantic contribution with their syntax.


\section{The phenomenon}
\label{sec-phenomenon-clause-types}

Most of the data that is covered in this chapter has been discussed in the previous chapters
already. German has interrogative clauses that are V1 clauses (\mex{1}a), assertive clauses that are
V2 clauses (\mex{1}b) and then there are verb-last clauses of various kinds.
\eal
\ex Kennt der Mann die Frau?
\ex Der Mann kennt die Frau.
\zl
\eal
\ex dass der Mann die Frau kennt
\ex Ich frage mich, welche Frau der Mann kennt.
\ex die Frau, die der Mann kennt
\zl
The example in (\mex{0}a) is a simple assertive embedded clause, (\mex{0}b) is an embedded
interrogative clause and (\mex{0}c) a relative clause. I assume that interrogative and relative
clauses are licensed by a schema that combines a filler that contains a \emph{wh} element or a
relative pronoun, respectively, with a sentence in which the respective element is missing. The
semantics is contributed by this schema. I will not discuss these clause types any further. What I
want to discuss here are the basic V2, V1 and VL patterns that are instantiated by (\mex{-1}a,b) and
(\mex{0}a).

The V1 pattern can also be observed in imperatives (\mex{1}a) although V2 is also a form that
imperatives can take (\mex{1}b):
\eal
\ex
\gll Gib mir das Buch!\\
     give me the book\\
\ex
\gll Jetzt gib mir das Buch!\\
     now   give me the book\\
\glt `Give me the book now!'         
\zl

Similarly questions are not restricted to V1 order. Yes/no-questions typically are V1. Other
questions are V2:
\eal
\ex 
\gll Wer kennt diese Frau?\\
     who knows this woman\\
\glt `Who does know this woman?'
\ex 
\gll Wen kennt dieser Mann?\\
     who knows this man\\
\glt `Who does this man know?'
\zl
However, with the right intonation a V2 clause can also be a yes/no question:
\ea
\gll Der Mann kennt die Frau?\\
     the man  knows the woman\\
\glt `Does the man know the woman?'
\z

To make things even more interesting German has a construction called \emph{Vorfeldellipse}
`pre-field ellipses' or \emph{Topic Drop}. A fronted element that is recoverable from the context
can be dropped. The following sentences from \citet{Huang84} in (\mex{1}) show that both subjects and objects
can be dropped.
\eal
\ex{
\gll {}[Ihn] hab' ich schon gekannt.\\
       ~him  have I yet known\\
\glt `I knew him.'
}
\ex{
\gll {}[Ich] hab' ihn schon gekannt.\\
     ~I      have him yet   known\\
}
\zl
The material in brackets may be omitted.

(\mex{1}) shows that adjuncts can also be omitted:
\ea
Die (die Pinguine) kommen so nah ran, daß man sie hätte streicheln können. Zum Fotografieren
zu nah -- und zu schnell, unmöglich da scharf zu stellen.\\
\gll [Da/Hier] Kann man ewig rumkucken.\footnotemark\\
     ~there/here can one eternally around.look\\
\glt `The penguins come so close that one could stroke them. One can look around eternally.'
\footnotetext{
        In an Email report from the south pole.
      }
\z
The generalization is that things that can be fronted can also be dropped in the \emph{Vorfeldellipse}.\footnote{
        This is a simplification: More oblique arguments drop less easily. Space limitations
        prevent me from going into a detailed discussion, but see the cited references.%
}

Finally, there are also conditional clauses like \emph{kommt Peter} `comes Peter' in (\mex{1}):
\ea
\gll Kommt Peter, komme ich nicht.\\
     comes Peter  come  I not\\
\glt `If Peter comes, I will not come.'
\z

Summarizing what we have seen so far, we can say that German has V1 and V2 clauses and both can be
questions (yes/no questions or \emph{wh} questions) and both can be declaratives (with topic drop
and without) and both can be imperatives. V1 clauses can function as conditionals in complex
sentences. This shows that there is no simple one to one mapping from topological mapping or clause structure to clause types. 







\section{The analysis}

Section~\ref{sec-analysis-v1-v2} provides the analysis of V1 and V2 clauses. A V1 clause is analyzed
as a combination of a finite verb in initial position that selects a clause with verb final order
from which it is missing. Sentences with a complementizer differ from the V1 sentences in that the
position of the finite verb is taken by the complementizer. So in the examples below \emph{kennt}
selects for \emph{der Mann die Frau \_} and \emph{dass} selects for \emph{der Mann die Frau kennt}:
\eal
\ex 
\gll Kennt der Mann die Frau \_?\\
     knows the man the woman\\
\glt `Does the man know the woman?'
\ex
\gll dass der Mann die Frau kennt\\
     that the man the woman knows\\
\zl
What has to be explained in this section is how Topic Drop is accounted for syntactically and how
all the constructions that we dealt with so far are paired with a semantics.

There are two options to account for Topic Drop: The first is to use an empty element \citet{Huang84} and the
second is to use a unary branching rule. The disadvantage of the solution with the empty element
is that it has to be ensured that it does not appear in other positions than the \vf. If the empty
element would be allowed in the \mf or \nf, all arguments could be omitted.\footnote{
This is only a small
disadvantage though since there are other elements as for instance the reflexive pronouns in
constructions with inherently reflexive verbs that cannot be put in the \vf:
\eal
\ex[]{ 
\gll Er erholt sich.\\
     he recreates \self\\
\glt `He recreates.'
}
\ex[*]{
\gll Sich erholt er.\\
     \self{} recreates he\\
}
\zl
For a general discussion of empty elements see Chapter~\ref{chap-empty}.
}

%% So far, we can distinguish between verb final and verb initial clauses by making reference 
%% to the value of the \initialf. The value of the \initialf of lexical items of verbs is `$-$'. Only
%% the result of the application of the verb movement lexical rule (see page~\pageref{lr-verb-movement2}) has an \initialv `+'.

%% Since verb first and verb second sentences are both \textsc{initial}+,
%% we need a further feature to be able to distinguish these clause types. I suggest naming
%% this feature \textsc{v2}. Normal verbal projections have the \textsc{v2} value $-$ and projections
%% that are the result of the head-filler-schema or the topic-drop-schema are \textsc{v2}+.

%% Since the \textsc{v2} feature is located inside of the \textsc{synsem} value of a sign, nouns like
%% those in (\ref{noun-scomp}) can select for verb second sentences.



So rather than an empty element, I use a schema that drops an element in \slasch. The analysis of
(\mex{1}) is shown in Figure~\vref{abb-kennt-er}.
\ea
\gll Kennt er.\\
     knows he\\
\glt `He knows him/her/it.'
\z
\begin{figure}
\begin{forest}
sm edges
[V\feattab{\comps \eliste,\\ 
           \textsc{inh}|\textsc{slash} \eliste}
	[V\feattab{\comps \eliste,\\
                   \textsc{inh}|\textsc{slash} \sliste{ \ibox{1} },\\
                   \textsc{to-bind}|\textsc{slash} \sliste{ \ibox{1} }}
		[V{[\comps \sliste{ \ibox{2} }]}
			[V{[\comps \sliste{ \ibox{3}, \ibox{4} }]}, tier=trace,edge label={node[midway,right]{V1-LR}}
				[kennt;knows]]]
		[\ibox{2} V\feattab{\comps \eliste,\\
                                    \textsc{inh}|\textsc{slash} \sliste{ \ibox{1} },\\
                                    \textsc{to-bind}|\textsc{slash} \eliste}
			[\ibox{4} \feattab{\textsc{local} \ibox{1},\\
                                                  \textsc{inh}|\textsc{slash} \sliste{ \ibox{1} }}, tier=trace
					[\trace]]
			[V\feattab{\comps \sliste{ \ibox{3} },\\
                                   \textsc{inh}|\textsc{slash} \sliste{ \ibox{1} },\\
                                   \textsc{to-bind}|\textsc{slash} \eliste}
				[\ibox{3} NP{[\textit{nom}]}
				   [er;he]]
				[V{[\comps \sliste{ \ibox{3}, \ibox{4} }]}
					[\trace]]]]]]
\end{forest}     
\caption{Analysis of \emph{Kennt er.} `He knew him/her/it.'}\label{abb-kennt-er}
\end{figure}
The analysis is completely parallel to the analysis of (\mex{1}), which was provided in
Figure~\ref{abb-das-buch-kennt} on page~\pageref{abb-das-buch-kennt}:
\ea
\gll Das Buch kennt er.\\
     the book knows he\\
\glt `He knows the book.'
\z


The top-most node in Figure~\ref{abb-kennt-er} is licensed by the following Schema:

\begin{schema}[Topic-Drop Schema]
\textit{topic-drop-phrase} \impl\\
\onems{ head-dtr \ms{ synsem & \ms{ local$|$cat \ms{ head & \ms[verb]{ vform   & fin \\
                                                                       initial & \upshape +\\
                                                                     }\\
                                                     comps &  \eliste\\
                                                   }\\
                                     } \\
                             nonloc & \ms{ inher|slash   & \sliste{ \ibox{1} }  \\
                                           to-bind|slash & \sliste{ \ibox{1} } \\ 
                                      }\\ 
                    }\\
  non-head-dtrs  \eliste\\
}
\end{schema}

\noindent
This schema projects a projection of a finite verb in initial position with an element
in \textsc{slash} and binds off this element in \textsc{slash}: Pollard and Sag's
nonlocal feature principle ensures that the \textsc{inherited$|$slash} value of the resulting projection is the
empty set. The semantic/discourse effects of this rule are ignored, but of course it is clear where
the additional constraints would be located in a fully specified grammar: the constraints would be
attached to the schema above. The semantics of the head daughter is enriched by the semantics that
is contributed by the construction.

The schema is similar to the Filler-Head Sschema that was introduced on page~\ref{hf-schema}.
%% suggested by other authors for
%% German verb second sentences (\citealp[\page 293]{Pollard90a-Eng}; \citealp[p.\,97]{Mueller99a}). 
The only difference is that there is no non-head-daughter since the \textit{Vorfeld} is not filled.
The commonalities of the two schemata are captured in the hierarchical organization 
of dominance schemata without the reference to surface linearization.



The discussion of the data in Section~\ref{sec-phenomenon-clause-types} showed that the clause types cannot simply be derived
from the position of the verb since, for instance, a V1 clause can be a clause with topic drop, a
yes/no question or a conditional. What I suggest here is different: because of the passing on of
information about the extracted elements in a tree, the information whether an element is missing in
a tree is directly accessible. For instance the verb in Figure~\ref{abb-das-buch-kennt} on
page~\pageref{abb-das-buch-kennt} selects the sentence [ \trace{} er \trace{} ]. This sentence
contains an element in \slasch and hence it is clear that the combination of \emph{kennt} `knows'
and \emph{er} `he' has to be part of a V2 clause or a clause with topic drop.

Therefore we can formulate an implicational constraint that says that verbal projections with a
finite verb and something in \slasch must be imperatives, questions or assertions.\footnote{
  The empty tag \etag\is{\etag} stands for some value which is not shared anywhere in the description.
}

\ea
\ms[verb-initial-lr]{
synsem|nonloc|inher|slash & ne\_list \\
} \impl\\
\flushright\ms{
synsem|loc|cont|rels & \sliste{ [ \type{imperative-interrogative-assertion} ] } $\oplus$ \etag\\
}
\z
The lexical rule that was given on page~\pageref{lr-verb-movement2} is modified in a way that includes a relation that
represents the clause type.
\eanoraggedright
\begin{tabular}[t]{@{}l@{}}
Lexical rule for verbs in initial position (including relation for clause\\types):\\
\ms{
synsem$|$loc & \ibox{1} \ms{ cat$|$head & \ms[verb]{ vform & fin\\
                                                     initial & $-$\\
                                                   }\\
                             cont & \onems{ hook \ms{ ltop & \ibox{2} \\
                                                      ind  & \ibox{3} \\
                                                    }\\
                                            rels  \ibox{4}\\
                                       }\\
                  }\\
} $\mapsto$\\*
\onems{
synsem$|$loc \onems{ cat  \ms{ head & \ms[verb]{ vform   & fin\\
                                                 initial & $+$\\
                                             }\\
                               spr & \eliste\\
                           comps & \liste{ \onems{ loc \onems{ cat \ms{ head & \ms[verb]{
                                                                                     dsl & \ibox{1}\\
                                                                                 }\\
                                                                         spr & \eliste\\
                                                                         comps & \eliste\\
                                                                    }\\
                                                                cont|hook \ibox{4}\\
                                                              }\\
                                                  } }\\
                         }\\
                   cont \ms{ hook & \ibox{4}\\
                             rels & \sliste{ \ms{ arg0 & \ibox{2} \\
                                                  arg3 & \ibox{3} }  } $\oplus$ \ibox{4}\\
                           }\\     
             }\\
}
\end{tabular}
\label{lr-verb-movement-clause-type}
\z


%%  (\mex{1}a). 
%% \eal
%% \ex V1-LR:[\sliste{ XP/Y }] \impl [\textsc{sem} imperative\_or\_interrogative\_or\_assertion ]
%% \ex V1-LR:[\sliste{ XP   }] \impl [\textsc{sem} imperative\_or\_interrogative\_or\_conditional ]
%% \zl
%% Gibt es kein Element in \slasch (\mex{0}b), so muss es sich um einen Imperativ, einen
%% Interrogativsatz oder einen Konditionalsatz handeln. Zwei weitere Implikationen, die auf die
%% morphologische Form des Verbs Bezug nehmen erzwingen als Bedeutung imperativ bzw.\ schließen
%% Imperativ aus.

This means that we can infer possible clause types from the knowledge about the presence of an
extracted element. The actual clause type remains underspecified though since imperatives,
interrogatives and assertions can be V2 clauses. 
In order to fully determine the clause type, one has to refer to the intonation pattern of the
clause, one has to have information about the presence or absence of an interrogative pronoun in the
\vf. I do not go into the details of intonation here, but since HPSG represents phonological
information in every complex linguistic object and not just at the terminal nodes it is clear that
phonological information can be used in implicational constraints as well. It is possible to
formulate constraints saying: if the phonological representation has the properties X and Y, the
semantics/information structure has to contain Z. For information on how phonological constraints
are represented in HPSG see \citew{BK94b,Hoehle99a-u,Bildhauer2008a}. 

While we can see in the lexical item whether an element is extracted or not, we cannot see whether
the filler of the nonlocal dependency contains a \emph{wh} element or not. The reason for this is
that the information about \emph{wh} elements is treated as nonlocal information in order to be able
to account for pied-piping.
\ea
\gll Von welchem Musiker hat Peter geschwärmt?\\
     from which musician has Peter enthused\\
\glt `Which musician thrilled Peter?'
\z
The phrase \emph{von welchem Musiker} contains a \emph{w} word, but it is deeply embedded as the
determiner of a noun phrase that is part of a PP. The information about the interrogative element is
passed up in the tree as it is common for other nonlocal information. The feature that is used for
this kind of nonlocal dependency ist the \quef. The information that is passed up is the semantic
index of the interrogative pronoun. In comparison only locally relevant information is passed up in
\slasch, that is, information about part of speech, valence, case and semantic
information. Information about other nonlocal dependencies as for instance the \quev is not
contained in \slasch. Therefore it is impossible to determine from within the phrase \emph{kennt jeder} `knows everybody'
whether the constituent in the Vorfeld contains a \emph{w} element or not.\footnote{
  \citet{HN94b} suggest an analysis in which complete signs are elements of \slasch. This makes a
  completely lexical determination of clause types possible, since both the local information and
  the nonlocal information of the fronted constituent can be addressed from within the partial
  clause. I nevertheless assume the more restrictive analysis that is usually assumed in HPSG.%
} Hence the clause type determination has to happen with reference to the constituent in the
\vf. There are several ways to do this in HPSG. One is suggested by \citet{Sag2010b} for the
analysis of extraction structures in English:\footnote{
  See also \citew{Jacobs2016a} for a suggestion that can be transferred into HPSG and that would be
  parallel to Sag's proposal.%
} Sag uses schemata for various types of sentences (relative clauses and interrogative clauses) to
be able to account for the idiosyncratic distribution of \emph{wh} pronouns. Each schema corresponds
to a specific type. Types are arranged in type hierarchies and more specific types inherit
constraints from their supertypes. This makes it possible to capture generalizations. For instance,
Sag assumes a general type for filler-head structures and then assumes subtypes of this type for the
specific cases he discusses. Rather than enumerating all the syntactic patterns and associating them
with types, I would like to suggest that there is just one schema for the combinatoin of filler and
head in German V2 clauses and that the semantic information regarding the sentence type is dependent
on the form of the element in the prefield. If the element contains a \emph{w} element, the clause
is an interrogative clause, if it does not, the clause is a declarative clause. Formally this can be
expressed by implicational constraints that have a complex structure with or without \emph{w}
element as antecedent and which specify in the consequence the semantic relation that is contributed
by the respective utterance. Figure~\vref{abb-imp-interrogativ} shows the implication in tree notation.
\begin{figure}
\centerline{%
\begin{forest}
[{}
       [\textsc{que} \sliste{ [ ] }]  [\vphantom{t}]
       ]
\end{forest}\hspace{1em}\raisebox{\baselineskip}{\impl}\hspace{1em}
\begin{forest}
[{}
       [\vphantom{t}]  [int(x)]
       ]
\end{forest}
}
\caption{\label{abb-imp-interrogativ}Implicational constraint for interrogative clauses}
\end{figure}
If we have a tree structure with a \emph{w} element in initial position, the second daughter has to
contribute an interrogative semantics.\todostefan{R1: What if there are several elements in \que?}
The good thing about the representational format of HPSG is
that tree structures are also modeled by feature structures. Since we can use complex feature
descriptions in antecedents of implicational constraints, implications like the one sketched in
Figure~\ref{abb-imp-interrogativ} can be formulated.

The implication in Figure~\ref{abb-imp-interrogativ}  is a simplification. In addition one has to
require in the antecedence that the interrogative semantics is possible at all since otherwise
sentences like (\mex{1}b) -- quoted from \citet[\page
    113]{RR92a-u} -- would result in a contradiction, since the imperative form of the verb enforces
an imperative meaning.
\eal
\ex 
\gll Sag mal, wem du die Rezension anvertraut hast!\\
     say once who you the review trust have\\
\glt `Who did you trust the review with?'
\ex 
\gll Wem sag mal, dass du die Rezension anvertraut hast!\\
     who say once that you the review trust have\\
\glt `Who did you trust the review with?'
\zl\todostefan{add formal version of the implication}




\section{Alternatives}

In what follows I briefly discuss two alternatives. Section~\ref{sec-schema-based-ct} compares the
implication"=based proposal that was suggested here with proposals that attach the respective
constraints to very specific dominance schemata. This is a rather abstract discussion, concrete schema-based
suggestions are discussed in Chapter~\ref{chap-alternatives}.
Section~\ref{sec-functional-projections} deals with recent suggestions within the Minimalist
Programm that rely on Rizzi-style functional projections \citep{Rizzi97a-u}.

\subsection{Schema-based analyses}
\label{sec-schema-based-ct}

I suggested an analysis in which the relation that is needed for the clause type is introduced by a
lexical rule (a unary branching schema). The alternative is a phrasal view that refers to a certain configuration.

The approaches can be depicted as in Figure~\ref{abb-konstruktion-implikation}.
\begin{figure}
\hfill
%\begin{tabular}{cc}
\subfloat[Phrasal construction]{
\makebox[.4\textwidth]{
\begin{forest}
%baseline
[\textsc{sem} f(x) (y)
       [\textsc{sem} y]  [\textsc{sem} x [\vphantom{\textsc{sem}},no edge]]
       ]
\end{forest}}}
\hfill
\subfloat[Implication + lexical construction]{
\makebox[.4\textwidth]{
\begin{forest}
%baseline
[\textsc{sem} f(x) (y)
       [\textsc{sem} y]  [\textsc{sem} f(x) [\textsc{sem} x] ]
       ]
\end{forest}}}\hfill\mbox{}
\caption{\label{abb-konstruktion-implikation}constructional, phrasal approach and approach with
  implicational constraint}
\end{figure}%
The semantic contribution at the mother node in Figure~\ref{abb-konstruktion-implikation}a is not
derived compositionally from the daughters since it is not the combination of $x$ and $y$ but rather
the combination of $f(x)$ and $y$. The function $f$ is contributed by the construction. In contrast
the additional meaning component is contributed lexically in
Figure~\ref{abb-konstruktion-implikation}b, that is, there already is a function that is applied to
$x$. The combination of $f(x)$ and $y$ is compositional. The exact content of $f$ depends from the
environment in which the verb is realized. An example for a constraint that determines the function
was given in Figure~\ref{abb-imp-interrogativ}, which shows the implication that constraints the
semantic contribution of interrogative clauses.





\subsection{Functional projections}
\label{sec-functional-projections}

This section compares the analysis of clause types that was developed in this chapter with an
analysis that was suggested within the framework of the Minimalist Program \citep{Chomsky93b-u,Chomsky95a-u}. The analysis of V2
clauses that is developed in this book can be sketched as in Figure~\vref{Abbildung-Fernabhaengigkeiten-HPSG}.
\begin{figure}
\centering
\begin{forest}
sm edges
[VP
	[NP
		[diesen Mann$_i$;this man,roof]]
	[VP/NP
		[V
			[V
				[kennt$_k$;knows]]]
		[VP/NP
			[NP/NP
				[\trace$_i$]]
			[V$'$
				[NP
					[jeder;everyone]]
				[V
					[\trace$_k$]]]]]]
\end{forest}
\caption{\label{Abbildung-Fernabhaengigkeiten-HPSG}Analysis of long"=distance dependencies in HPSG}
\end{figure}%
This analysis is pretty similar to what \citet{Haider93a} and \citet{FL2011a} assume. The analysis
is compatible with current Minimalist assumptions: the combination of heads with their arguments are
licenced by the Head-Complement Schema and by the Head-Filler Schema. As I have shown in
\citew{MuellerUnifying} these schemata correspond to the operations \emph{Move} and
\emph{Merge}, which are assumed in Minimalism. Self-induced technical problems with Labelling and so
on that Chomsky's analyses \citeyearpar{Chomsky2008a,Chomsky2013a} are plagued with do not exist in
the proposal advocated here. In comparison to the analysis of \citet{Lohnstein2007a} -- which is
depicted in Figure~\ref{abb-satz-lohnstein} -- the analysis that is developed in this book is minimal.\footnote{
  Lohnstein's analyse is a simplification of Rizzi's analysis \citeyearpar{Rizzi97a-u}. Rizzi and
  also \citet[\page 70]{Grewendorf2002a} assume a Force head and a Typ head, respectively.
}$^,$\footnote{
Chomsky emphazises in several of his publications and talks that the Minimalist Program cannot be
criticized for not being minimal or minimalisitic, since it is a program and not a theory and the
goals of the program correspond to usual scientific goals (\eg \citew[\page 38]{Chomsky2013a}). I am
not criticizing the program here, but -- as many others before me -- a specific analysis, which was
suggested within the program.
} In what follows I want to explain why I do not consider Rizzi-style analyses minimalistic in the
sense of the Minimalist Program.
\begin{figure}
\oneline{%
\newlength\mytextheight
\settototalheight{\mytextheight}{XpX$^0$X$'$}
\begin{forest}
  delay={
    where content={}{
      content={\phantom{X}}
    }{},
  },
  for tree={
    text height=\mytextheight,
    fit=band,
    parent anchor=south,
    child anchor=north,
    s sep=-2mm,
  }
[TopP
       [SpecT [left\\dislocated\\elements, tier=word] ] 
       [T$'$ [T$^0$ ] 
         [FocP
           [SpFoc [{[+wh]-phrases}, tier=word] ]
           [Foc$'$ [Foc$^0$ ] 
             [TopP 
               [SpecT [{[$-$wh]-phrases}, tier=word] ] 
               [T$'$ [Top$^0$ ] 
                 [AgrP, s sep=-20mm [SpecAgr ]
                   [Agr$'$, s sep=-20mm
                     [MoodP
                       [{} ]
                       [Mood$'$, s sep=-5mm
                         [TenseP
                           [{} ]
                           [Tense$'$ 
                             [vP [theta-layer, tier=word] ]
                             [Tense$^0$ ] ] ] 
                         [Mood$^0$ [verbal mood\\factive vs.\\epistemic, tier=word] ] ] ]
                      [Agr$^0$ ] ] ] ] ] ] ] ] ]
\end{forest}%
}
\caption{\label{abb-satz-lohnstein}Rizzi-style analysis of the German clause according to \citet[\page 84]{Lohnstein2007a}}
\end{figure}
The goal of the Minimalist Program \citep{Chomsky95a-u} is to explain language evolution. Structures
should be simple so that their evolution and their repeated acquisition by speakers of succeeding
generations is plausible. Chomsky admits the possibility that the innate language-specific knowledge
that is necessary for this is minimal (\citealt*{HCF2002a}; \citealt[\page 4]{Chomsky2007a}).
If one compares Figure~\ref{Abbildung-Fernabhaengigkeiten-HPSG} with
Figure~\ref{abb-satz-lohnstein}, it is obvious that there are several tree positions in the latter
figure that do not exist in the former: there is no distinction between FocP and TopP. Clauses
always are verbal projections. This is what is visible as far as syntactic categories are
concerned. Focus and topic are part of the information structure of a sentence and are modelled
separately from syntactic categories like verb(al projection), noun or noun phrase, and so on.
In Rizzi-style\nocite{Rizzi97a-u} analyses like the one suggested by Lohnstein the topic or focus
position may be empty in clauses of the appropriate kind. Such empty positions do not exist in my
analysis.
Children have to learn that certain clauses have a topic element in the \vf and others have a focus
element. Children do not have to learn that there are clauses in which there is a topic, but the
focus element is empty both phonologically and semantically. I consider the use of topic and focus
projections an aberration in a syntax-centered research context that realized that language cannot
be described adequately with syntactic categories alone and that models that assume highly separate modules
like syntax, semantics, and information structure and pose a linear sequence of such modules with
limited interaction are inadequate. These insights resulted in a proliferation of semantically and
information structurally motivated functional categories.\footnote{
 \citet{Rizzi97a-u} and \citet[\page 70]{Grewendorf2002a} assume ForceP, TopP, FocP  and
 \citet[\page 31]{Poletto2000a-u} assumes HearerP and SpeakerP,  \citet*{WHBH2007a-u} suggest TopP, ForceP and OuterTopP
\citet[\page 96, 99]{Cinque94a-u} assumes Quality, Size, Shape, Color, Nationality.
See \citet[\page 76]{Webelhuth95a} for an overview. He also lists Honorific and Predicate. For a
recent overview see \citew[Abschnitt~4.6.1]{MuellerGT-Eng1}.
}$^,$\footnote{
Some projections are also motivated by the presence of morphemes in other languages. Such an
argumentation is only sound if simultaneously a rich UG is assumed since monolingual children do not
have information about topic and focus morphemes in other languages
\citep[Section~2]{MuellerCoreGram}.% 
} It is clear that relevant semantic distinctions have to be modelled but this has to take place on
respective semantic and pragmatic levels, that are related to the syntactic level. This can be
established by the semantic or information structural contribution of single lexemes or of phrasal
configurations. What has been shown in this section is how the semantics of clause types can be
integrated into the architecture of grammar without mixing the semantic categories with the
syntactic ones. We will deal with information structure in Chapter~\ref{chap-is}.



%      <!-- Local IspellDict: en_US-w_accents -->

%% -*- coding:utf-8 -*-
\chapter{Information structure constraints on multiple frontings}
\label{chap-is}

% \newcommand{\lindex}[1]{$_{#1}$}

Chapter~\ref{chapter-mult-front} provided the syntactic aspects of the analysis of apparent multiple frontings. Of course this analysis vastly overgenerates: it admits structures that are not well-formed. This chapter discusses information structure constraints on multiple frontings and shows how they can be formalized in HPSG. The analysis is based on \citew{BC2010a}, which is one result of the project A6 in the Collaborative Research Center SFB 632 on information structure. Felix Bildhauer and Philippa Cook are co-authors of this chapter.

\section{A note on terminology}

There is no general terminological consensus about information structural categories. The definitions of such categories tend to vary across different research traditions, and sometimes they are not even used consistently within the same paradigm of research (see \citealp{KruijffSteedman2003} for an overview of the evolution and interdependencies of such terms). In what follows, we adopt the view that topic-comment and focus-background are distinct, orthogonal dimensions of information structure, along the lines of \cite{Krifka2007a-u}. Thus, we think of utterances as being structured along both of these two dimensions, which serve different purposes: A focus evokes a set of alternatives and selects a particular one among them \citep{Rooth85a-u,Rooth92a-u}. On the other hand, a topic singles out a specific discourse referent as an ``address'' in the mental representation of the discourse (``aboutness topic'') or it narrows down the domain within which the comment is supposed to hold at all (``framesetting topic''; see also \citealp{Jacobs2001a-u-platte} for discussion). The kind of topic we will be dealing with in this chapter is of the ``aboutness''-type. 



\section{The phenomenon}



As we saw earlier in Chapter~\ref{sec-v2-phen}, German is classed as V2 language, that is, normally
exactly one constituent occupies the position before the finite verb in declarative main clauses. In
what have been claimed to constitute rare, exceptional cases, however, more than one constituent
appears to precede the finite verb, as illustrated in the attested examples that were discussed in
Section~\ref{sec-phenomenon-mult-front}. Some attested examples with two fronted objects are
repeated here for ease of reference in (\mex{1}):\footnote{Unless otherweise indicated, corpus examples in this
  chapter were extracted from the German Reference Corpus \citep{DeReKo}.} 
\eal
\ex\label{ex:saft}
    \gll [Dem Saft] [eine kräftigere Farbe] geben Blutorangen.\footnotemark\\
         \hspaceThis{[}the.\dat{} juice \hspaceThis{[}a.\acc{} more.vivid colour give blood.oranges\\
\footnotetext{
\sigle{R99/JAN.01605}.
}
    \glt `What gives the juice a more vivid colour is blood oranges.'
%\footnotemark
%\footnotetext{\fnpaper{Frankfurter Rundschau}{08/01/1999}{Alles Orange: Pomeranzen, Salusianas, Kumquats}}
\ex
\gll [Dem Frühling] [ein Ständchen] brachten Chöre aus dem Kreis Birkenfeld im Oberbrombacher Gemeinschaftshaus.\footnotemark\\
\hspaceThis{[}the.\dat{} spring \hspaceThis{[}a.\acc{} little.song brought choirs from the county Birkenfeld
    in.the Oberbrombach municipal.building\\
\footnotetext{
 \sigle{RHZ02/JUL.05073}.
}
\glt `Choirs from Birkenfeld county welcomed (the arrival of) spring with a little song in the Oberbrombach municipal building.'\label{fruehling-zwei}

\ex
\gll [Dem Ganzen] [ein Sahnehäubchen] setzt der Solist Klaus Durstewitz auf\footnotemark\\
     \hspaceThis{[}the.\dat{} everything \hspaceThis{[}a.\acc{} little.cream.hood puts the soloist Klaus
         Durstewitz on\\
\footnotetext{
 \sigle{NON08/FEB.08467}.
}
\glt `Soloist Klaus Durstewitz is the cherry on the cake.'\label{ex:sahne}
\zl

There has been ongoing debate in the theoretical literature concerning the status of examples seemingly violating this V2 constraint. The examples in (\ref{fanselow}) (from \citealp{Fanselow93a}) and (\ref{gmueller}) (from G.\ \citealp{GMueller2004a}), are similar to (\ref{ex:saft})--(\ref{ex:sahne}) in that both objects of a ditransitive verb are fronted. The grammaticality judgments given by these authors diverge and, as can be seen from G.\ Müller's assessment of the data, such constructed examples tend to be deemed at best marginal, or even ungrammatical if presented without context.

\ea
\gll [Kindern] [Heroin] sollte man besser nicht geben.\\
     \hspaceThis{[}children.\dat{} \hspaceThis{[}heroin.\acc{} should one better not give\\
\glt `One shouldn't give heroin to children.'\label{fanselow}
\z
\eal\label{gmueller}
\ex[??]{
\gll [Kindern] [Bonbons] sollte man nicht geben.\\
      \hspaceThis{[}children.\dat{} \hspaceThis{[}candies.\acc{}  should one not give\\
\glt `One shouldn't give candies to children.'
}
\ex[*]{
\gll [Dieses billige Geschenk] [der Frau] sollte man nicht geben.\\
      \hspaceThis{[}this.\acc{} cheap present \hspaceThis{[}the.\dat{} woman should one not give\\
\glt `One shouldn't give the woman this cheap present.'
        }

\zl

%On the basis of corpus data, \citet{Mueller2003b,Mueller2005d} shows that a large variety of syntactic categories, grammatical functions and semantic classes can occur preverbally in such Multiple Frontings (MFs). 
Chapter~\ref{sec-analyse-mf} provided the syntactic aspects of the analysis that treats the fronted constituents as dependents of an empty verbal head, thus preserving the assumption that the preverbal position is occupied by exactly one constituent (namely a VP):\footnote{For simplicity, we continue to refer to this phenomenon as `multiple fronting', but in the light of the analysis given in Chapter~\ref{sec-analyse-mf}, the term is exchangeable with `apparent multiple fronting' or `fronting of a VP that has an empty head'. Interestingly, multiple fronting rivals regular VP fronting in frequency for certain combinations of lexical material. For a comparison of multiple fronting and regular VP fronting (i.\,e., fronting of a VP with a lexically filled head), see \citealp[Section~4]{MBC2012a}.}  

\begin{exe}
  \ex {[\sub{VP} [Dem Saft] [eine kräftigere Farbe] \_\sub{V}]\lindex{i} geben\lindex{j} Blutorangen \_\lindex{i} \_\lindex{j}.}\label{saft-struc}
\end{exe}

While this account by itself correctly predicts certain syntactic properties of MFs, such as the fact that the fronted parts must depend on the same verb, it is in need of further refinement. In particular, multiple fronting seems to require very special discourse conditions in order to be acceptable (which is why out-of-context examples often sound awkward). %  e there are cases of highly questionable acceptability that are not ruled out in the analysis as it stands, viz.\ (\ref{schoko})
% \begin{exe}
%   \ex[???]{\gll [Der Student] [Schokolade] \textbf{mag} gerne.\\
%         \hspaceThis{[}The student.nom \hspaceThis{[}chocolate.acc likes very.much\\
%         Intended: `The student likes chocolate very much.'}\label{schoko}
% \end{exe}
Relying on findings from a corpus of naturally occurring data, we have identified two different information-structural environments in which MFs are licensed. Section~\ref{description} briefly sketches these two patterns, which in Section~\ref{analysis} we will analyze as being licensed by two related but distinct constructions, each of them instantiating a specific pairing of form, meaning and contextual appropriateness.

%In particular, we will show that MFs do not correspond to a single information structural configuration but, in fact, are motivated by different pragmatic considerations.  We  analyze these different configurations (two of which we briefly sketch below) as a set of related but distinct constructions, each of them instantiating a specific pairing of form, meaning and contextual appropriateness.  

\subsection{Multiple Fronting in Context}\label{description}

In this section we examine two possible contexts of MF: Section~\ref{sec-presentational-MF} deals with what we term \emph{Presentational MF} and Section~\ref{sec-propositional-ass-mf} with \emph{Propositional Assessment MF}.

\subsubsection{Presentational MF}
\label{sec-presentational-MF}

One of the configurations in which MF is well attested in naturally occurring data is illustrated in
(\ref{clown}), (\ref{deflorian}) and (\ref{eisoval}), where the (b) line contains the MF structure and the (a) and
(c) lines provide the context before and after it, respectively. We call this type
\textit{Presentational Multiple Fronting}.


%\ealnoraggedright
% \label{clown} should refer to the level of exe, not xlist
\begin{exe}\exnrfont
\ex\label{clown}
\begin{xlist}[iv.]

\ex Spannung pur herrschte auch bei den Trapez-Künstlern. [\ldots]  Musika\-lisch begleitet wurden die einzelnen Nummern vom Orchester des Zir\-kus Busch [\ldots]\\
    `It was tension pure with the trapeze artists. [\ldots] Each act was musically accompanied by Circus Busch's own orchestra.'
\ex\label{pres}
\gll [Stets] [einen Lacher] [auf ihrer Seite] hatte \textit{die} \textit{Bubi} \textit{Ernesto} \textit{Family}\lindex{i}.\\
 \hspaceThis{[}always  \hspaceThis{[}a laugh  \hspaceThis{[}on their side had the Bubi Ernesto Family\\
\glt `Always good for a laugh was the Bubi Ernesto Family.'\label{clown-b}
\ex Die Instrumental-Clowns\lindex{i} zeigten ausgefeilte Gags und Sketche [\ldots]\\
    `These instrumental clowns presented sophisticated jokes and sketches.'\\\sigle{M05/DEZ.00214} 
\zl


%\ealnoraggedright 
% \label{deflorian} should refer to the level of exe, not xlist
\begin{exe}\exnrfont
\ex\label{deflorian}
\begin{xlist}[iv.]

    \ex {}[\ldots] wurde der neue Kemater Volksaltar [\ldots] geweiht. Die Finan\-zie\-rung haben
    die Kemater Basarfrauen übernommen.
    Die Altar\-wei\-he bot auch den würdigen Rahmen für den Einstand von Msgr.\ Walter Aichner als Pfarrmoderator von Kematen.\\
       `\ldots\ the new altar in Kemate \ldots\ was consecrated. It was financed by the Kemate bazar-women. The consecration of the altar also presented a suitable occasion for Msgr.\ Walter Aichner's first service as Kematen's parish priest'

 \ex \gll [Weiterhin] [als Pfarrkurator] wird \textit{Bernhard} \textit{Deflorian}$_i$ fungieren. \\
            \hspaceThis{[}further      \hspaceThis{[}as  curate        will Bernhard Deflorian function. \\
 \glt `Carrying on as curate, we have Bernhard Deflorian.'\label{deflorian-b}

\ex Ihn$_i$ lobte Aichner besonders für seine umsichtige und engagierte Füh\-rung der pfarrerlosen Gemeinde. Er$_i$ solle diese Funktion weiter aus\-üben, ,,denn die Entwicklung, die die Pfarrgemeinde Kematen genommen hat, ist sehr positiv''.\\
  `Aichner praised him especially for his discreet and committed leading of the priestless congregation. He should carry on with his work, ``for the development of the Kematen congregation has been very positive.'''\\ \sigle{I97/SEP.36591} 
\zl

We take Presentational MF to be a topic shift strategy. What is typical for this construction, we claim, is that a new entity (in italics in the examples (\ref{clown-b}), (\ref{deflorian-b}) and (\ref{eisoval-b})) is first introduced into the discourse and can then better serve as a topic in the continuation of the discourse or text. We argue that this introduced element benefits from first being `presented' in a construction such as Presentational MF before then functioning as an aboutness topic precisely because at the moment it is introduced into the discourse it bears some features that are not typical for topics (e.g. focus, discourse newness). The position for this `presentation' to take place is late in the clause, where the main accent typically falls in German declaratives. Presentational MF is never obligatory though; we are simply highlighting here why a speaker or an author might choose this construction in a particular context. Conversely, this kind of presentation is also found in canonical sentences not involving multiple fronting. In the corpus data we looked at, the presented entity is frequently a subject, but not always. We have also found experiencer objects and locative dependents. Our account below is intended to capture this observed distribution of presented entities. 


What is it then that unites (agentive) subjects on the one hand and (non-subject) experiencer or
locative dependents on the other and makes them candidates for being presented in such a
construction? On the basis of a close examination of a large quantity of naturally occurring
data\footnote{In the context of the Collaborative Research Center SFB 632, a reasonably large
  database of multiple frontings (containing more than 2400 instances, most of them extracted from
  the German Reference Corpus \citep{DeReKo}) was compiled and annotated by the authors. Annotations
  include topological fields, syntactic function and various information structure categories. The
  collection is publically accessible through a search interface at
  \url{https://hpsg.fu-berlin.de/Resources/MVB/}.},\todofelix{In footnote below: Is it a good idea
  to point to the site at FU Berlin? For how how much longer will this be available?}
we suggest that this presented entity
corresponds to the dependent of the verb that is -- in general -- the most topic-worthy of all the
verb's dependents and is thus most likely to actually be realized as a topic in some particular
discourse context. We will refer to this dependent as the verb’s `designated topic' (DT) -- a term
intended to apply to a verb’s most likely topical dependent outside of any particular discourse
context. This element does not \textit{have to} instantiate the topic, but it is the \textit{most
  likely} candidate to instantiate topic. Agents are dependents which typically are the designated
topic (DT) of their predicates but when the subject is semantically a theme (e.\,g.\ with
unaccusatives or some psych verbs), then we find that it is the experiencer or a locative dependent
that has a closer affinity with topic (cf. \cite{vanOosten84a-u} for similar observations about
topic prototypicality, but without the notion of DT).

As mentioned above, since focus and newness are not prototypical topic features cross-linguistically, cf. again \cite{Krifka2007a-u}, it has been argued that new entities often have to be first `presented' before they can function as aboutness topics and we claim this is what is happening here (cf.\ \citealp{Lambrecht94a-u}, for whom the type of phrases introducing brand new referents into the discourse are lowest on the scale of `Topic Accessibility'). Interestingly, then, rather than checking/spelling out a discourse function of the fronted material, a motivating factor in Presentational MF is the tendency to realize certain material external to the post-verbal domain in order to maximize the presentational effect lower down in the clause. Note that the pattern is not characterized adequately if the description makes reference to the subject rather than to the `designated topic'. The reason is, as mentioned above, that the presented element need not be the subject in all cases, as illustrated in (7b): here, the subject is actually part of the fronted material, while the newly introduced entity is coded as a locative PP. Our analysis in terms of designated topic accommodates these data since the locative phrase, rather than the subject, plays this role in the case of \emph{herrschen} `to reign' (in the relevant `existential' reading). 
%We take Presentational MF to be a topic shift strategy. A new entity (in italics) is introduced into
%the discourse and serves as a topic in the continuation. On the basis of a close examination of a
%large quantity of naturally occurring data, we suggest that this presented entity corresponds to the
%dependent (argument or adjunct) of the verb that is most topic-worthy and is thus most likely to be
%realized as a topic in other circumstances. We will refer to it as the verb's `designated topic',
%and it is, typically, the grammatical subject, but non-subjects may take on this role -- as we
%illustrate immediately below -- in the case of e.\,g.\ unaccusatives/psych verbs which presumably
%favor spatio-temporal or experiencer topics. Since focus and newness are not prototypical topic
%features cross-linguistically, it has been argued that brand new/focal entities often have to be
%first ``presented'' before they can function as aboutness topics \cite[cf.][for whom the type of
%  phrases introducing brand new referents into the discourse are lowest on the scale of `Topic
%  Accessibility']{Lambrecht94a-u}. Interestingly, then, rather than checking/spelling out a discourse
%function of the fronted material, the motivating factor here is the need to shift material away from
%the post-verbal domain to maximize the presentational effect.  Note that the pattern is not
%characterized adequately if the description makes reference to the subject, rather than to the
%`designated topic'. The reason is that the presented element need not be the subject in all cases,
%as illustrated in (\ref{eisoval}): here, the subject is actually part of the fronted material, while
%the newly introduced entity is coded as a locative PP. Our analysis in terms of designated topic
%accommodates these data, since the locative phrase, rather than the subject, plays this role in the
%case of \textit{herrschen} `to reign' (in the relevant ``existential'' reading). 
%
It also predicts that a subject can occur among the fronted material in a MF construction if it is not the verb's designated topic. %\todofelix{I guess this could be extended a bit.}
 

%\ealnoraggedright
%  \label{eisoval} should refer to the level of exe, not xlist
\begin{exe}\exnrfont
\ex\label{eisoval}
\begin{xlist}[iv.]
\ex Gesucht? Schnelle Sprinter\\
    `Wanted: fast sprinters'
\ex
\gll [Weiterhin] [Hochbetrieb] herrscht am \textit{Innsbrucker} \textit{Eisoval}.\\
    \hspaceThis{[}further \hspaceThis{[}high.traffic reigns at.the Innsbruck icerink\\
\glt `It's still all go at the Innsbruck icerink.'\label{eisoval-b}
\ex Nach der Zweibahnentournee am Dreikönigstag stehen an diesem Wochenende die Österreichischen Staatsmeisterschaften im Sprint am Pro\-gramm.\\
 `Following the two-rink tournament on Epiphany-Day there's now the Austrian National Championship in Sprinting coming up at the weekend.' \sigle{I00/JAN.00911}
\zl

%  \begin{exe}
%    \ex
%    \begin{xlist}
%      \ex Mit tieferen Kursen als zuletzt eröffnete Wall Street: Der Dow-Jones-Index rutschte innerhalb der ersten dreißig Minuten nach Börseröffnung um 9,39 Punkte auf 3052,33 Einheiten. (\ldots)
% \ex{\gll [Weiterhin] [freundliche Stimmung] \textbf{herrscht} indes an der Tokioter Börse. \\
%           \hspaceThis{[}further \hspaceThis{[}friendly climate reigns though at the Tokyo {stock market}\\
%           {\glt `At the Tokyo stock market, the climate is still buisiness-friendly.'}} 
%  \ex Der Nikkei-Index schloß mit 24.439,85 Punkten und lag damit 105,18 Zähler dem Mittwoch. \\
%    {\glt `The Nikkei-index closed at 24439.85, thus 105.18 points higher than on Wednesday.'}\\
% \sigle{N91/OKT.16548}

%    \end{xlist}
% \label{nikkei}
%  \end{exe}




\subsubsection{Propositional Assessment MF}
\label{sec-propositional-ass-mf}

The second configuration in which MF occurs is best described as \textit{Propositional Assessment MF}. Examples (\ref{knecht}) and (\ref{berliner}) illustrate this type of structure. 

\ealnoraggedright
\ex Bauern befürchten Einbußen\\
   `Farmers fear losses'
\ex{\gll [Nach Brüssel] [zum Demonstrieren] ist Gerd Knecht \textit{nicht} gefahren\\
          \hspaceThis{[}to Brussels   \hspaceThis{[}to  demonstrate  is Gerd Knecht not gone\\
   {\glt `G. K. did not go to Brussels for the demo'}}
%\footnotetext{\fnpaper{Mannheimer Morgen}{26/02/1999}{Umlandseite(n)}}
\ex aber gut verstehen kann der Vorsitzende des Lampertheimer Bauernverbands die Proteste der Kollegen.\\
    `but the president of the Lampertheim Farmers' Association can well understand his colleagues' protest.' \sigle{M99/FEB.12802} \label{knecht}
\zl

\ealnoraggedright
\ex Im Schlussabschnitt war den Berlinern das Bemühen durchaus an\-zu\-mer\-ken, vor ausverkauftem Haus ein Debakel zu verhindern.\\
    `During the last phase of the match, it was clearly visible that the Berlin players were struggling to fight off a debacle in the packed arena.'
\ex
\gll [Dem Spiel] [eine Wende] konnten sie aber \emph{nicht} mehr geben.\\
      \hspaceThis{[}to.the match \hspaceThis{[}a turn could they however not more give\\
\glt `However, they didn't manage to turn the match around.'

\ex  Rob Shearer (46.) traf noch einmal den Pfosten, das nächste Tor erzielten aber wieder die Gäste.\\
     In the 46th minute, Rob Shearer hit the post again, but it was the guests who scored the next goal.'  \sigle{NUZ07/MAI.01360} 
\label{berliner}
\zl

We analyze Propositional Assessment MF as involving a Topic-Comment structure plus an assessment of the extent to which the Comment holds of the Topic. More precisely, we are dealing with an inverted Topic-Comment configuration, in which the fronted material constitutes (part of) the Comment, while the Topic is instantiated by a discourse-given element in the middlefield (\emph{Gerd Knecht} in (\ref{knecht}), \emph{sie} in (\ref{berliner})). Also in the middlefield, we regularly find an `evaluative' expression, generally an adverb or particle, frequently but not exclusively negation. It must be prosodically prominent (i.\,e., it must probably receive the main stress of the sentence), and it expresses/highlights the degree to which the Comment holds for the Topic. Besides \textit{nicht} `not', particles/adverbs frequently found in \textit{Propositional Assessment MF} include \textit{nie} `never', \textit{selten} `rarely', \textit{oft} `often'. 
%We analyze Propositional Assessment MF as involving an inverted Topic-Comment structure. The fronted material constitutes (part of) the Comment, while the Topic is instantiated by a discourse-given element in the middlefield. A stressed evaluative particle (\textit{nicht} `not') in the middlefield expresses/highlights the degree to which the Comment holds for the Topic. Other such evaluative particles include \textit{nie} `never', \textit{selten} `rarely', \textit{oft} `often' etc. 


\section{The analysis}\label{analysis}

Before we turn to the analysis of the interaction between syntax and information structure in apparent multiple frontings in Section~\ref{sec-inf-struc-and-mf}, we have to introduce the notation that we use for representing constraints on information structure (Section~\ref{sec-information-structure-general}). Before we can do this, we have to introduce the representational format of Minimal Recursion Semantics (Section~\ref{sec-intro-MRS}). MRS is particularly well-suited for
modelling information structure constraints since embedding of predicates is not done in the representation directly, but rather elementary predications are represented in a list and embedding is expressed by pointers.

\subsection{Introduction to Minimal Recursion Semantics}
\label{sec-intro-MRS}

This introduction is divided into two parts: first, we introduce the basic representation of semantic information and explain how scope can be represented in an underspecified way and then we turn to the analysis of non-compositional constructions in which some semantic information is contributed by a certain phrasal pattern itself.

\subsubsection{Basic representation and compositional semantics}
 
(\mex{1}) shows the examples for the semantic contribution of a noun and a verb in Minimal Recursion
Semantics (MRS):
\ea
\label{le-buch}
\begin{tabular}[t]{@{}l@{~}ll@{~}l@{}}
a. & \emph{dog}  & b. & \emph{chases} \\
   & \ms[mrs]
           { ind  & \ibox{1} \ms[index]{ per & 3 \\
                                  num & sg \\
                                } \\
             rels & \liste{ \ms[dog]{ inst & \ibox{1} \\ }} \\
           } & & 
\ms[mrs]
           { ind  & \ibox{1} event \\
             rels & \liste{ \ms[chase]{ event & \ibox{1} \\
                                        agent & index \\
                                        patient & index \\ }} \\
           }
\end{tabular}
\z

\noindent
An MRS consists of an index, a list of relations, and a set of handle constraints, which will be
introduced below. The index can be a referential index\footnote{
  Phrases like \emph{no dog} also have a referential index in this sense. These referential indices
  are like variables.%
} of a noun (\mex{0}a) or an event variable (\mex{0}b). In the examples above the lexical items contribute the \relation{dog} relation and the \relation{chase} relation. The relations can be modeled with feature structures by turning the semantic roles into features. The semantic index of nouns is basically a variable, but it comes with an annotation of person, number, and gender since this information is important for establishing correct pronoun bindings.
%
%\subsection{Linking}

The arguments of each semantic relation (e.g. agent, patient) are linked to their syntactic realization (e.g. NP[nom], NP[acc]) in the lexicon. (\mex{1}) shows an
example. NP[\type{nom}]\ind{1} stands for a description of an NP with the semantic index identified with \ibox{1}. The semantic indices of the arguments are structure shared with the arguments of the semantic relation \relation{chase}.
\ea
\label{le-chase}
\emph{chase}:\\
\onems
{ synsem|loc \ms{ cat & \ms{ head & \ms[verb]
                                    { vform & fin \\} \\
                             arg-st & \liste{ NP[\type{nom}]\ind{1}, NP[\type{acc}]\ind{2}   } \\
                           } \\
                  cont &  \ms{ ind  & \ibox{3} event \\
                               rels & \liste{ \ms[chase]{ event   & \ibox{3} \\
                                                          agent   & \ibox{1} \\
                                                          patient & \ibox{2} \\ }} \\
                             }\\
               }\\
}
\z
Generalizations over linking patterns can be captured elegantly in inheritance hierarchies (see
%Section~\ref{sec-generalizations} on inheritance hierarchies and
\citew{Davis96a-u,Wechsler91a-u,DK2000b-u} for further details on linking in HPSG).


Before turning to the compositional analysis of (\mex{1}a), we want to introduce some additional machinery that is needed for the underspecified representation of the two readings in (\mex{1}b,c).
\eal
\label{every-dog-chased}
\ex\label{ex-every-dog-chased}
Every dog chased some cat.
\ex $\forall x (dog(x) \to \exists y (cat(y) \wedge chase(x,y)))$
\ex $\exists y (cat(y) \wedge\forall x  (dog(x) \to chase(x,y)))$
\zl
Minimal Recursion Semantics assumes that every elementary predication comes with a
label. Quantifiers are represented as three place relations that relate a variable and two so-called handles. The handles point to the restriction and the body of the quantifier, that is, to two labels of other relations. (\mex{1}) shows a (simplified) MRS representation for (\mex{0}a).
\ea
$\langle$ h0, \{ \begin{tabular}[t]{@{}l@{}}
                  h1: every(x, h2, h3), h2: dog(x), h4: chase(e, x, y), \\
                  h5: some(y, h6, h7), h6:  cat(y) \} $\rangle$\\
                  \end{tabular}
\z
The three-place representation is a syntactic convention. Formulae like those in
(\ref{every-dog-chased}) are equivalent to the results of the scope resolution process that is described below.

The MRS in (\mex{0}) can best be depicted as in Figure~\ref{fig-dominance-graph-chanse}. h0 stands for the top element. This is a handle that dominates all other handles in a dominance graph. The restriction of \emph{every} points to \emph{dog} and the restriction of \emph{some} points to \emph{cat}. The interesting thing is that the body of \emph{every} and \emph{some} is not fixed in (\mex{0}). This is indicated by the dashed lines in Figure~\ref{fig-dominance-graph-chanse} in contrast to the straight lines connecting the restrictions of the quantifiers with elementary predications for \emph{dog} and \emph{cat}, respectively.
\begin{figure}
\centering
\begin{tabular}{@{}ccc@{}}
                               & \mybox[h0]{h0}                & \\[8ex]
\mybox[h1]{h1:every(x, \mybox[h1h3]{h2}, \mybox[h1h2]{h3})}      &                              & \mybox[h5]{h5:some(y, \mybox[h5h7]{h6}, \mybox[h5h6]{h7})}\\[8ex]
\mybox[h3]{h2:dog(x)}           & ~~~~~\mybox[h7]{h6:cat(y)}         & \\[6ex]
                               & \mybox[h4]{h4:chase(e, x, y)}\\
\end{tabular}
\begin{tikzpicture}[overlay,remember picture] 
\draw[dashed](h0.south)--(h1.north); 
\draw[dashed](h0.south)--(h5.north);
\draw[dashed](h5h6.south)--(h4.north);
\draw[dashed](h1h2.south)--(h4.north);
\draw(h1h3.south)--(h3.north);
\draw(h5h7.south)--(h7.north);
\end{tikzpicture}
%% {\psset{linestyle=dashed}%
%% \ncline{h0}{h5}%
%% \ncline{h0}{h1}%
%% \ncline{h5h6}{h4}%
%% \ncline{h1h2}{h4}%
%% }%
%% \ncline{h1h3}{h3}%
%% \ncline{h5h7}{h7}%
\caption{\label{fig-dominance-graph-chanse}Dominance graph for \emph{Every dog chases some cat.}}
\end{figure}
There are two ways to plug an elementary predication into the open slots of the quantifiers:
\eal
\ex Solution one: h0 = h1 and h3 = h5 and h7 = h4.\\
  (\emph{every dog} has wide scope)

\ex Solution two: h0 = h5 and h7 = h1 and h3 = h4.\\
  (\emph{some cat} has wide scope)
\zl
The solutions are depicted as Figure~\ref{fig-forall} and Figure~\ref{fig-exists}.

\begin{figure}
\centering
\begin{tabular}{@{}ccc@{}}
                               & \mybox[h0]{h0}                & \\[8ex]
\mybox[h1]{h1:every(x, \mybox[h1h3]{h2}, \mybox[h1h2]{h3})}      &                              & \mybox[h5]{h5:some(y, \mybox[h5h7]{h6}, \mybox[h5h6]{h7})}\\[8ex]
\mybox[h3]{h2:dog(x)}           & ~~~~~\mybox[h7]{h6:cat(y)}         & \\[6ex]
                               & \mybox[h4]{h4:chase(e, x, y)}\\
\end{tabular}
\begin{tikzpicture}[overlay,remember picture] 
\draw(h0.south)--(h1.north); 
\draw(h5h6.south)--(h4.north);
\draw(h1h2.south) .. controls +(0,-1) and +(-1,1).. (h5.north);
\draw(h1h3.south)--(h3.north);
\draw(h5h6.south)--(h4.north);
\draw(h5h7.south)--(h7.north);
\end{tikzpicture}
\caption{\label{fig-forall}%
%$\forall x (dog(x) \to \exists y (cat(y) \wedge chase(x,y)))$
every(x,dog(x),some(y,cat(y),chase(x,y)))
}
\end{figure}

\begin{figure}
\centering
\begin{tabular}{@{}ccc@{}}
                               & \mybox[h0]{h0}                & \\[8ex]
\mybox[h1]{h1:every(x, \mybox[h1h3]{h2}, \mybox[h1h2]{h3})}      &                              & \mybox[h5]{h5:some(y, \mybox[h5h7]{h6}, \mybox[h5h6]{h7})}\\[8ex]
\mybox[h3]{h2:dog(x)}           & ~~~~~\mybox[h7]{h6:cat(y)}         & \\[6ex]
                               & \mybox[h4]{h4:chase(e, x, y)}\\
\end{tabular}
\begin{tikzpicture}[overlay,remember picture] 
\draw(h0.south)--(h5.north); 
\draw(h5h6.south).. controls +(0,-1) and +(-1,1).. (h1.north);
\draw(h1h2.south)--(h4.north);
\draw(h1h3.south)--(h3.north);
\draw(h5h7.south)--(h7.north);
\end{tikzpicture}
\caption{\label{fig-exists}%$\exists y (cat(y) \wedge\forall x  (dog(x) \to chase(x,y)))$\newline
some(y,cat(y),every(x,dog(x),chase(x,y)))}
\end{figure}
There are scope interactions that are more complicated than those we have been looking at so far. In
order to be able to underspecify the two readings of (\mex{1}) both slots of a quantifier have to stay open.
\eal
\ex Every nephew of some famous politician runs.
\ex every(x, some(y, famous(y) ∧ politician(y), nephew(x, y)), run(x))
\ex some(y, famous(y) ∧ politician(y), every(x, nephew(x, y), run(x)))
\zl
In the analysis of example (\ref{ex-every-dog-chased}), the handle of \relation{dog} was identified
with the restriction of the quantifier. This would not work for (\mex{0}a) since either
\relation{some} or \relation{nephew} can be the restriction of \relation{every}. Instead of direct
specification so-called handle constraints are used (\type{qeq} oder $=_q$). A qeq constraint
relates an argument handle and a label: \mbox{h $=_q$ l} means that the handle is filled by the label
directly or one or more quantifiers are inserted between \emph{h} and \emph{l}. Taking this into
account, we can now return to our original example. A more accurate MRS representation of
(\ref{ex-every-dog-chased}) is given in (\mex{1}).
\ea
$\langle$ h0, \{ \begin{tabular}[t]{@{}l@{}}
                  h1:every(x, h2, h3), h4:dog(x), h5:chase(e, x, y), \\
                  h6:some(y, h7, h8), h9:cat(y) \}, \{ h2 $=_q$ h4, h7 $=_q$ h9 \} $\rangle$\\
                  \end{tabular}
\z
The handle constraints are associated with the lexical entries for the respective
quantifiers. Figure~\vref{fig-every-dog-chases-a-cat-syntax} shows the analysis. 
\begin{figure}
\resizebox{!}{\textheight}{%
\begin{sideways}
\begin{forest}
sm edges
[V\feattab{
         \spr \sliste{ },\\
         \comps \eliste\\
         \rels  \relliste{ h1:every(x, h2, h3), h4:dog(x), h5:chase(e, x, y), h6:some(y, h7, h8), h9:cat(y) },\\
         \hcons \relliste{ h2 $=_q$ h4, h7 $=_q$ h9 } }
          [\ibox{1} NP\feattab{
                        \rels  \relliste{ h1:every(x, h2, h3), h4:dog(x) },\\
                        \hcons \relliste{ h2 $=_q$ h4 } }
               [Det\feattab{
                     \rels  \relliste{ h1:every(x, h2, h3) },\\
                     \hcons \relliste{ h2 $=_q$ h4 } } [every] ] 
               [N\feattab{
                   \rels  \relliste{ h4:dog(x) },\\
                   \hcons \eliste } [dog] ] ]
          [V\feattab{
              \spr \sliste{ \ibox{1} },\\
              \comps \eliste\\
              \rels  \relliste{ h5:chase(e, x, y), h6:some(y, h7, h8), h9:cat(y) },\\
              \hcons \relliste{ h7 $=_q$ h9 } }
              [V\feattab{
                  \spr  \sliste{ \ibox{1} },\\
                  \comps \sliste{ \ibox{2} },\\
                  \rels  \relliste{ h5:chase(e, x, y) },\\
                  \hcons \eliste } [chases] ]
              [\ibox{2} NP\feattab{
                            \rels  \relliste{ h6:some(y, h7, h8), h9:cat(y) },\\
                            \hcons \relliste{ h7 $=_q$ h9 } }
                [Det\feattab{
                      \rels  \relliste{ h6:some(y, h7, h8) },\\
                      \hcons \relliste{ h7 $=_q$ h9 } } [some] ]
                [N\feattab{
                     \rels  \relliste{ h9:cat(y) },\\
                     \hcons \eliste } [cat] ] ] ] ]
\end{forest}%
\end{sideways}
} %oneline
\caption{\label{fig-every-dog-chases-a-cat-syntax}Analysis for \emph{Every dog chases some cat.}}
\end{figure}
For compositional cases as in Figure~\ref{fig-every-dog-chases-a-cat-syntax}, the \relsv of a sign
is simply the concatenation of the \relsvs of the daughters. Similarly the \hconsv is a
concatenation of the \hconsvs of the daughters. 


\subsubsection{The Analysis of ``Non-Compositional'' Constructions}
\label{sec-non-compositional}

\citew*{CFPS2005a} extended the basic analysis that concatenates \rels and \hcons to cases in which
the meaning of an expression is more than the meaning that is contributed by the daughters in a
certain structure. They use the feature \feat{c-cont} for the representation of constructional
content. While usually the semantic functor (the head in head argument combinations and the adjunct
in head adjunct structures) determines the main semantic contribution of a phrase, the \ccontf can
be used to specify a new main semantic contribution. In addition relations and scope constraints may
be introduced via \ccont. The feature geometry for \ccont is given in (\mex{1}):
\ea
\ms[c-cont]{
hook  & \ms{ 
        index & event-or-index\\
        ltop  & handle\\
        }\\
rels  & list of relations\\
hcons & list of handle constraints\\
}
\z
The \feat{hook} provides the local top for the complete structure and a semantic index, that is a
nominal index or an event variable. In compositional structures the \hookv is structure shared with
the semantic contribution of the semantic functor and the \relsl and the \hconsl is the empty
list. As an example for a non-compositional combination \citew{CFPS2005a} discuss determinerless
plural NPs in English. For the analysis of \emph{tired squirrels} they assume an analysis using a unary branching
schema. Their analysis corresponds to the one given in (\mex{1}):\footnote{
    We do not assume a unary branching schema for bare plurals but an empty determiner, since using
    an empty determiner captures the generalizations more directly: while the empty determiner is
    fully parallel to the overt ones, the unary branching schema is not parallel to the binary
    branching structures containing an overt determiner. See also \citew{AB2012a} for a similar
    point regarding relative clauses in Modern Standard Arabic\il{Modern Standard Arabic} with and without a complementizer.
}
\ea
\onems{
synsem|loc|cont \ms{ hook  & \ibox{1}\\
                     rels  & \ibox{2} $\oplus$ \ibox{3} \\
                     hcons & \ibox{4} $\oplus$ \ibox{5} \\
                   }\\
c-cont   \ms{ hook & \ibox{1} \ms{ ind & \ibox{0}\\
                                   } \\
                rels & \ibox{2} \relliste{ \ms[udef-rel]{
                                    arg0 & \ibox{0}\\
                                    restr & \ibox{6}\\
                                    body  & handle\\
                                   } } \\[10mm]
                hcons & \ibox{4} \relliste{ \ms[qeq]{
                                     harg & \ibox{6}\\
                                     larg & \ibox{7}\\
                                     } }
              }\\
head-dtr  \onems{ synsem|loc|cont  \ms{ ind & \ibox{0}\\
                                       ltop & \ibox{7}\\
                                     }\\
                rels    \ibox{3} \relliste{ \ms[tired]{ lbl  & \ibox{7}\\
                                                arg1 & \ibox{0}\\
                                              },
                                     \ms[squirrel]{ lbl  & \ibox{7}\\
                                                    arg0 & \ibox{0}\\
                                                  } }\\
                hcons  \ibox{5} \eliste\\
              }\\
}
\z
The semantic content of the determiner is introduced constructionally in \ccont. It consists of the
relation \relation{udef-rel}, which is a placeholder for the quantifier that corresponds to
\emph{some} or \emph{every} in the case of overt determiners. The \rels and \hconsvs that are
introduced constructionally (\ibox{2} and \ibox{4}) are concatenated with the \rels and \hconsvs of
the daughters (\ibox{3} and \ibox{5}).

The Semantics Principle can now be specified as follows:
\begin{principle}[Semantics Principle]
The hook value of a phrase (containing the main index and the local top) is identical to the value of \textsc{c-cont|hook}. The
\relsv is the concatenation of the \relsv in \ccont and the concatenation of the \rels values of the
daughters. The \hconsv is the concatenation of the \hconsv in \ccont and the concatenation of the \hcons values of the
daughters. 
\end{principle}







\subsection{Information structure features}
\label{sec-information-structure-general}

Various approaches to information structure have been proposed within HPSG, differing both in the features that are assumed to encode aspects of IS, and in the sort of objects these features take as their value \cite[among others,][]{EV96a,Wilcock2001a,deKuthy2002a,Paggio2005a-u,Webelhuth2007a-u}. The representation we use here is based on \cite{Bildhauer2008b}. As mentioned above, we take topic/comment and focus/background to be two information structural dimensions that are orthogonal to one another. We thus introduce both a \textsc{topic} and a \textsc{focus} feature, bundled under a \textsc{is} feature on \textit{synsem}-objects.\footnote{Information-structure should be inside \textit{synsem} because at least information about focus must be visible to elements (such as focus sensitive particles) that select their sister constituent via some feature (\textsc{mod}, \textsc{spec}, \textsc{comps}/\textsc{subcat}). Possibly, the situation is different with topics: we are not aware of data showing that topicality matters for selection by modifiers or heads. We leave open the question whether \textsc{topic} is better treated as an attribute of, say, \textit{sign} rather than \textit{synsem}.} These take as their value a list of lists of \textit{elementary predications}. In the basic case, i.\,e.\ in a sentence with a single topic and a single focus, the \textsc{topic} and
\textsc{focus} lists each contain one list of \textit{EPs}, which are structure shared with elements on the sign's \textsc{rels}-list. In other words, we are introducing pointers to individual parts of a sign's semantic content. By packaging the \textit{EP}s pertaining to a focus or topic in individual lists, we are able to deal with multiple foci/topics. The feature architecture just outlined is shown in (\ref{arch}), and (\ref{arch-exe}) illustrates a possible instantiation of the
\textsc{topic}, \textsc{focus} and \textsc{cont} values.

%multiple fronting must be appropriately constrained. e properties Our account involves an IS (information structure) feature on signs along with an appropriate subtyping of the value it takes, illustrated in (\ref{is-sig}). 

\ea
\label{arch}
\ms[sign]{
  synsem & \ms{ loc & local\\
                nonloc & nonloc\\
                is & \ms[is]{
                      topic & list\\
                      focus & list\\
                }\\
  }\\
}
\z
\ea
\label{arch-exe}
\ms[sign]{
  synsem & \onems{ is \ms[is]{
                        topic & \sliste{ \sliste{ \ibox{1} } }\\
                        focus & \sliste{ \sliste{ \ibox{2}, \ibox{3} }, \sliste{ \ibox{4} } }\\
                       }\\
                   loc|cont|rels \sliste{ \ibox{1}, \ibox{2}, \ibox{3}, \ibox{4}, \ibox{5} }\\
                 }\\
}
\z

Next, we introduce a subtyping of \textit{is}, given in Figure~\vref{fig-types-is}. These subtypes can then be used to refer more easily to particular information-structural configurations, that is, to specific combinations of \textsc{topic} and \textsc{focus} values.\footnote{These types are thus used as abbreviations or labels for specific combinations of attributes and their values. From a technical perspective, they are not strictly necessary, but we use them here for clarity of
  exposition.}
The subtypes that are relevant for our purpose are \textit{pres} (`presentational')
and \textit{a-top-com} (`assessed-topic-comment', a subtype of the more general
\textit{topic-comment} type. 
\begin{figure}
\begin{forest}
  typehierarchy
  [is
    [pres
      [\ldots]
      [\ldots]]
    [topic-comment
      [a-top-com]
      [\ldots]]
      [\ldots]]
\end{forest}
\caption{\label{fig-types-is}Type hierarchy of information structure types}
\end{figure}


Those \textit{head-filler} phrases that are instances of multiple fronting can then be restricted to have an \textsc{is}-value of an appropriate type, as shown in (\ref{constraint-mf}).

\ea
\label{constraint-mf}
\ms[head-filler-phrase]{
  non-hd-dtrs & \sliste{ [ head|dsl  \type{local} ] }\\
  } \impl
   \ms{
     is & pres $\vee$ a-top-com $\vee$ \ldots
   }
\z

\subsection{Information structure and apparent multiple frontings}
\label{sec-inf-struc-and-mf}

Having introduced MRS and the general representation of information structure constraints, we can
now go on and demonstrate how two of the MF patterns that we identified can be modeled in
HPSG. Section~\ref{sec-identifying-mf} highlights the syntactic property of MF structures, which can
be used to enforce information structure constraints, Section~\ref{sec-presentationl-mf} discusses Presentational MFs
and Section~\ref{sec-propositional-assessment-mf} Propositional Assessment MFs.

\subsubsection{Identifying cases of MF}
\label{sec-identifying-mf}

%For HPSG analysis, we build on \cites{Mueller2005d} syntactic analysis and enrich it with discourse information as needed.
To account for the multiple fronting data within HPSG, it is necessary to appropriately constrain syntactic, semantic, and information-structural properties of a sign whenever it instantiates a multiple fronting configuration. Thus, in order to be able to specify any constraints on their occurrence, instances of multiple fronting must be identified in the first place. Since we base our proposal on \citew{Mueller2005d} syntactic analysis of multiple fronting, this is not a major
problem: on this approach, the occurrence of elements in the preverbal position in general is
modeled as a filler-gap-relation, where the non-head daughter corresponds to the preverbal material
(prefield) and the head daughter corresponds to the rest of the sentence (in the topological model
of the German sentence, this would be the finite verb, the middlefield, and the right bracket, and
the final field). In the analysis of multiple frontings that is presented in Section~\ref{sec-analyse-mf}, filler daughters in multiple fronting
configurations (and only in these) have a \textsc{head|dsl} value of type \type{local}, that is,
conforming to the analysis sketched in (\ref{saft-struc}) above, they contain information about an
empty verbal head, as shown in (\mex{1}).


\ea
\ms[head-filler-phrase]{
  non-hd-dtrs & \liste{ [ \textsc{head|dsl} \type{local} ] }\\
}
\z
This specification then allows us to pick out exactly the subset of \textit{head-filler}-phrases we are interested in, and to formulate constraints such that they are only licensed in some specific information-structural configurations, to which we turn next.

%  out of a small set of possible ones (the dots stand for further values that we have not discussed here). The relevant constraint is given in (\ref{constraint}).      



\subsubsection{Modeling Presentational MF}
\label{sec-presentationl-mf}

In order to model Presentational MF, we introduce a pointer to the designated topic as a head feature of the verb that subcategorizes for it. The feature \textsc{DT} takes a list (empty or singleton) of \type{synsem}-objects as its value, and it states which element, if any, is normally realized as the Topic for a particular verb. This is not intended to imply that the designated topic must in fact be realized as the topic in all cases. Rather, it merely encodes a measurable preference in topic realization for a given verb. The statement in (\ref{ex:verbs}) is intended as a general constraint, with further constraints on verbs (or classes of verbs) determining which element on \textsc{arg-st} is the Designated Topic.


\ea
\label{ex:verbs}
\type{verb-stem} \impl \ms{ head|dt & \eliste } $\vee$ \ms{ head|dt &  \sliste{ \ibox{1} }\\
                                                           arg-st  & \etag $\oplus$  \sliste{ \ibox{1} } $\oplus$ \etag\\
                                                         }
\z
                                      
The constructional properties of Presentational MF are defined in (\ref{ex:hfill}): the designated topic must be located within the non-head daughter and must be focused. % (and thus bear an accent, which is independently enforced).
 Figure \ref{clown-analysis} shows the relevant parts of the analysis of sentence (\ref{clown}) above.


\ea\label{ex:hfill}
\ms[head-filler-phrase]{
  is & pres\\
} \impl
%\flushright
\ms{
  synsem|l|cat|head|dt \sliste{ [ l|cont|rels \ibox{1} ] }\\
  head-dtr|synsem|is|focus \sliste{ \ibox{1} }\\
}
\z

\begin{figure}
\centerfit{
    \begin{forest}
      for tree={
        parent anchor=south,
        child anchor=north,
        anchor=north,
        align=center
      }
      [ {\ms[head-filler-phrase]{
          phon & \phonliste{ stets einen Lacher auf ihrer Seite hatte die Bubi Ernesto Family }\\
          synsem & \ms{ is & \ms[pres]{
                              focus & \sliste{ \ibox{1} }\\
                             }\\
                        loc & \onems{ cat|head|dt \sliste{ \ibox{4} [ l|cont|rels \ibox{1} ] }\\
                                      cont|rels \ibox{3} $\oplus$ \ibox{2} $\oplus$ \ibox{1} \\
                                    }\\
                      }\\
          }}
        [ \onems{
          phon \phonliste{ stets einen Lacher auf ihrer Seite }\\
          synsem|loc  \onems{ cat|head|dsl \type{local}\\
                              cont|rels \ibox{3} \\
            }\\
        } ]
        [, phantom, calign with current]
        [ \ms{
          phon & \phonliste{ hatte die Bubi Ernesto Family }\\
          synsem & \ms{ is|focus & \sliste{ \ibox{1} }\\
                        loc & \ms{ cat|head|dt \sliste{ \ibox{4} }\\
                                   cont|rels \ibox{2} $\oplus$ \ibox{1} \\
                                 }\\
                      }\\
          },
        % http://tex.stackexchange.com/questions/255104/aligning-nodes-at-the-right-periphery-of-the-text-area-in-forest?noredirect=1#comment610300_255104
      parent anchor=east,
      anchor=north east,
      for descendants={
        where n'=1{
          calign with current,
          anchor=north east,
        }{},
      },
      before drawing tree={
        parent anchor=south,
        for descendants={
          if n'=1{
            child anchor=north,
            parent anchor=south,
          }{}
        }
      }
          [ \ms{
             phon & \phonliste{ hatte }\\
             synsem & \onems{ is|focus  \sliste{ \ibox{1} }\\
                              loc  \onems{ cat \ms{ head|dt & \sliste{ \ibox{4} }\\
                                                    subcat  & \sliste{ \ibox{4}, \ldots } }\\
                                           cont|rels \ibox{2}  \\
                                         }\\
                            }\\
          } ]
          [ \ms{
             phon & \phonliste{ die Bubi Ernesto Family }\\
             synsem & \ibox{4} \onems{ is|focus \sliste{ \ibox{1} }\\
                                       loc|cont|rels \ibox{1} \\
                                 }\\
          } ] ] ]
\end{forest}
}
\caption{Sample analysis of \textit{Presentational Multiple Fronting}}\label{clown-analysis}
\end{figure}



\subsubsection{Modeling Propositional Assessment MF}
\label{sec-propositional-assessment-mf}

For Propositional Assessment MF, we use a special subtype of \type{topic-comment}, namely
\type{a(ssessed)-top-com}. We then state that the designated topic must in fact be realized as the
topic, and that it must occur somewhere within the head daughter (which comprises everything but the
prefield). Most importantly, the head-daughter must also contain a focused element that has the
appropriate semantics (i.\,e.\ one which serves to spell out the degree to which the comment holds
of the topic; glossed here as \textit{a-adv-rel}). However, the mere presence of such an element on
the \textsc{rels} list does not guarantee that it actually modifies the highest verb in the clause
(e.\,g., it could modify a verb in some embedded clause as well.) Therefore, the construction also
adds a handle constraint specifying that the focused element takes scope over the main verb. This
handle constraint needs to be added rather than just be required to exist among the head-daughter's
handle constraints because the \textit{outscoped} relation need not be an immediate one, i.\,e.,
there can be more than one scope-taking element involved. An appropriate handle constraint can be
introduced via the \textsc{c\_cont}-feature, i.\,e.\ as the construction's contribution to the overall
meaning. If the relevant element does not in fact outscope the main verb, the MRS will contain
conflicting information and cannot be scope-resolved. In that case, the phrase's semantics will not
be well-formed, which we assume will exclude any unwanted analysis due to focussing of the wrong
element. The necessary specifications are stated in (\ref{a-top-com}). A sample analysis of sentence
(\ref{knecht}) above is given in Figure \ref{knecht-analysis}.\todostefan{Alignment number AVM is broken}
%(More precisely, the constraint below says that the designated topic must be the topic of the head daughter, in order to ensure that it is contained within the head daughter, not the filler daughter. However, \textsc{topic} values are shared between mother and head daughter due to principles that cannot be discussed here for reasons of space).


\eas%
%\begin{tabular}[t]{@{}p{\linewidth}@{}}
\ms[head-filler-phrase]{
  is & a-top-com\\
} \impl\\
\flushright
\onems{
  synsem \onems{ l|cat|head|dt \sliste{ [ l|cont|rels \ibox{1} ] }\\[1mm]
                   is \ms{ topic & \sliste{ \ibox{1} }\\
                           focus & \sliste{ \sliste{ \ibox{3} } }\\
                   }\\
               }\\
  c\_cont|hcons \liste{ \ms[qeq]{
                        harg & \ibox{5}\\
                        larg & \ibox{4}\\
                        } }\\
  head-dtr|synsem|cont \ms{ ltop & \ibox{4} \\
    rels & \liste{ \ibox{3} \ms[a-adv-rel]{
                            arg & \ibox{5} \\
                            }} $\bigcirc$ \ibox{1}  $\bigcirc$ \etag \\
                 \\
                          }
}%
%\end{tabular}
\label{a-top-com}
\zs









\begin{figure}
\centerfit{
  \begin{forest}
      for tree={
        parent anchor=south,
        child anchor=north,
        anchor=north,
        align=center
      }
    [
      \onems{
        phon \phonliste{ nach Brüssel zum Demonstrieren ist Gerd Knecht nicht gefahren }\\
        synsem \onems{ l \onems{ cat|head|dt \sliste{ \ibox{1} [ l|cont|rels \ibox{2} ] }\\[1mm]
                                 cont|rels \ibox{8} $\oplus$ \ibox{7} $\oplus$ \ibox{2} $\oplus$ \liste{ \ibox{3} \ms[nicht-rel]{
                                              arg & \ibox{5} \\
                            } } $\oplus$ \ibox{6} \\
                            }\\
                       is \ms{ topic & \sliste{ \ibox{1} }\\
                               focus & \sliste{ \sliste{ \ibox{3} } }\\
                             }\\
                     }\\
        c\_cont|hcons \liste{ \ms[qeq]{
                        harg & \ibox{5}\\
                        larg & \ibox{4}\\
                        } }\\
       }
      [ \onems{
        phon \phonliste{ nach Brüssel zum Demonstrieren }\\
        cat|head|dsl \type{local} \\
        cont|rels  \ibox{8}  \\
      } ]
      [, phantom, calign with current]
      [   \onems{
        phon \phonliste{ ist Gerd Knecht nicht gefahren }\\
        cat|head|dt \sliste{ \ibox{1} }\\[1mm]
                                 cont \ms{ ltop & \ibox{4}\\
                                           rels & \ibox{7} $\oplus$ \ibox{2} $\oplus$ \liste{ \ibox{3} \ms[nicht-rel]{
                                                                                                           arg & \ibox{5} \\
                                                                                                          } } $\oplus$ \ibox{6}  \\
                                             }\\
        },
        % http://tex.stackexchange.com/questions/255104/aligning-nodes-at-the-right-periphery-of-the-text-area-in-forest?noredirect=1#comment610300_255104
      parent anchor=east,
      anchor=north east,
      for descendants={
        where n'=1{
          calign with current,
          anchor=north east,
        }{},
      },
      before drawing tree={
        parent anchor=south,
        for descendants={
          if n'=1{
            child anchor=north,
            parent anchor=south,
          }{}
        }
      }
        [ \onems{
        phon \phonliste{ ist  }\\
        cat|head|dt \sliste{ \ibox{1} }\\[1mm]
        cont|rels  \ibox{7}  \\
                               }
         ]
        [ \onems{
        phon \phonliste{ Gerd Knecht nicht gefahren }\\
        cont|rels   \ibox{2} $\oplus$ \sliste{ \ibox{3} } $\oplus$ \ibox{6} \\
        is|focus \sliste{ \sliste{ \ibox{3} } }\\
         }
          [ \onems{
        phon \phonliste{ Gerd Knecht }\\
        synsem \ibox{1} [ l|cont|rels  \ibox{2} ] \\
          } ]
          [ \onems{
        phon \phonliste{ nicht gefahren }\\
        cont|rels \sliste{ \ibox{3} } $\oplus$ \ibox{6} \\
                       is|focus \sliste{ \sliste{ \ibox{3} } }\\
            }
            [ \onems{
        phon \phonliste{ nicht  }\\
        cont|rels \sliste{ \ibox{3} } \\
        is|focus \sliste{ \sliste{ \ibox{3} } }\\
        } ]
            [ \onems{
        phon \phonliste{ gefahren }\\
        cat|head|dt \sliste{ \ibox{1} }\\[1mm]
        cont|rels \sliste{ \ibox{3} } $\oplus$ \ibox{6}\\
        } ] ] ] ] ]
\end{forest}
            

}
\caption{Sample analysis of \textit{Propositional Assessment MF}}\label{knecht-analysis}
\end{figure}



%%% What remains to be done is ensure that all instances of multiple fronting actually have one of these \type{is} values. In \citew{Mueller2005d} formalization, filler daughters in multiple fronting configurations (and only in these) have a \textsc{dsl} value of type \type{local}.\footnote{The \textsc{dsl} (`double slash') feature is needed to model the HPSG equivalent of verb movement from the sentence final position to the V2 position. Cf.\ the indices in example (\ref{saft-struc}) above.} The constraint in (\ref{constraint}) will thus enforce that instances of multiple fronting will have one out of a small set of possible \textsc{is} values (the dots stand for further values that we have not discussed here).


% \begin{figure}[h]
%   \begin{exe}
% \ex\label{constraint}
%     \begin{avm}
%       \[\tp{head-filler-phrase}\\
%         non-hd-dtrs & \< \[head|dsl & local \]\>\]
%     \end{avm}~~$\Rightarrow$\\[2ex]
%     \begin{avm}
%       \[is & pres $\vee$ a-top-com $\vee$ \ldots\]
%     \end{avm}
%   \end{exe}
% \end{figure}



\section{Conclusion}

In the way outlined above, the relative freedom of the fronted material in the analysis of
multiple frontings that was provided in Chapter~\ref{sec-analyse-mf} is appropriately restricted
with respect to the contexts in which multiple frontings can felicitously
occur. While we are not claiming to have identified these contexts exhaustively, the two
configurations modeled here, if taken together, account for the majority of naturally occurring
examples in our database. In sum, then, this chapter underlines the importance of examining
attested examples in context and demonstrates that it is possible to further constrain a syntactic
phenomenon which in the past has even been deemed ungrammatical in many (decontextualized) examples.

%      <!-- Local IspellDict: en_US-w_accents -->



%% -*- coding:utf-8 -*-
\chapter{Alternatives}
\label{chap-alternatives}%

\label{sec-local-frontings-alternatives}% Wetta, GO2009, Kathol95a

This chapter discusses alternative proposals of German sentence structure. The phenomena that have
to be explained by all proposals are the placement of the (finite) verb in initial or final
position, the possibility of scrambling of arguments, the fact that German is a V2 language that
allows to front an arbitrary constituent even if the constituent is dependent on a deeply embedded
head and the fact that sometimes there seem to be more than one constituent in the position before
the finite verb.\todostefan{add \citew{Kasper94a}}

Existing approaches can be classified along the following dimensions:
\begin{itemize}
\item phrase structure"=based vs.\ dependency-based
\item flat structures vs.\ binary branching structures
\item discontinuous vs.\ continuous constituents
\item linearization vs.\ ``head movement''
\item linearization-based approaches vs.\ ``movement''
\end{itemize} 
Movement and head movement are put in quotes since I include GPSG, HPSG, and Dependency Grammar
analyses among the movement analyses although technically, there is no movement in any of these
frameworks, but there are special relationships between fillers and gaps. 
In the following I will explore proposals from various frameworks (GPSG, HPSG, Dependency Grammar)
that differ along these dimensions.

The first proposal I want to look at is a GPSG proposal that does not assume a head-movement
mechanism.


\section{Flat structures and free linearization of the verb}
\label{sec-flat-free-linearization-of-verb-gpsg}

\citet{Uszkoreit87a} has developed a GPSG grammar for German which assumes that a verb is realized with
its arguments in a local tree. As the verb and its arguments are dominated by the same node, they can -- under
GPSG assumptions -- exhibit free ordering as long as certain theory-specific linearization
constraints are respected. For instance there is a rule for ditransitive verbs that states that a
sentence (V3) may consist of a verb (the head, abbreviated as H) and three NPs:
\ea
V3 $\to$ H[8], N2[\textsc{case} dat], N2[\textsc{case} acc], N2[\textsc{case} nom] 
\z
Each lexical item of a verb comes with a number which is associated with its valence and regulates
into which kind of phrase a verb can be inserted. The example in (\mex{0}) shows a rule for
ditransitive verbs. Since this rule does not restrict the order in which the elements at the right
hand side of the rule have to be realized, verb initial and verb final orderings are
possible. Furthermore all six permutations of the NPs can be derived.

\citet{Pollard90a} has adapted Uszkoreit's approach for his HPSG analysis of sentence structure in
German.

These kinds of analyses have the advantage of not needing empty heads to describe the position
of the verb. However, there does not seem to be any possibility of expressing the generalizations that are captured
in the analysis of apparent multiple frontings that was presented in the previous chapter in a flat
linearization model. In head-movement analyses it is possible to assume that the verb trace forms a
constituent with other nonverbal material, but this option is simply excluded in approaches like the
GPSG one for the simple reason that there is no empty verbal head.

Of course one could assume an empty element int the \vf as I did in
\citew{Mueller2002f,Mueller2002c,Mueller2005d}, but this empty element would be a special empty
element that would not be needed in any other part of the grammar and it would be stipulated with
the only purpose of getting an analysis of apparent multiple frontings.

GPSG is famous for its non-transformational treatment of non-local dependencies \citep{Gazdar81}
and the tools that were developed by Gazdar for extraction in English were used by
\citet{Uszkoreit87a} for the analysis of V2 sentences in German. However, some researchers assume
that such mechanisms are not necessary for simple sentences. They see the possible orderings as a
simple reordering of elements that depend on the same head. Such proposals are discussed in the
following section. 


\section{Flat structures and no extraction in simple sentences}

This section deals with approaches that assume that the constituent orders in (\mex{1}) are just
linearization variants of each other:
\eal
\ex
\gll Der Mann kennt die Frau.\\
     the.\nom{} man  knows the.\acc{} woman\\
\ex
\gll Die Frau kennt der Mann.\\
     the.\acc{} woman knows the.\nom{} man\\
\ex
\gll Kennt der Mann die Frau?\\
     knows the.\nom{} man the.\acc{} woman\\
\ex
\gll {}[dass] der Mann die Frau kennt\\
     \spacebr{}that the.\nom{} man the.\acc{} woman knows\\     
\zl
(\mex{0}) shows two V2 sentences and one V1 and one VL sentence. While most theories assume that
\emph{der Mann} in (\mex{0}a) and \emph{die Frau} in (\mex{0}b) are extracted, there are some
researchers that assume that these two sentences are just possible linearizations of the dependents
of \emph{kennt} `knows'. Such linearization proposals have been made in HPSG
(\citealp[Chapter~6.3]{Kathol95a}; \citealp{Wetta2011a,Wetta2014a-u})\footnote{
  \citet{Kathol2001a} revised his treatment and assumes a uniform analysis of V2 phenomena in German.}
and in Dependency Grammar. In what follows, I discuss the Dependency Grammar proposal in more detail.

One option in a Dependency Grammar analysis would be to allow for discontinuous constituents and
assume that dependents of deeply embedded heads can be serialized in the \vf even if the head is not
adjacent to the \vf. However, such radical approaches are difficult to constrain
\citep[]{MuellerGT-Eng1} and are hardly ever proposed in Dependency Grammar. Instead Dependency
Grammarians like \citet{Kunze68a-u}, \citet{Hudson97a,Hudson2000a}, \citet*{KNR98a},
and \citet{GO2009a}
%\todostefan{S: rather Hudson 2000, Khanae et al. 1998, \citew{DD2001a-u}, etc.} 
suggested analyses in which dependents of a head rise
to a dominating head for those cases in which a discontinuity would arise otherwise. The approach is
basically parallel to the treatment of non-local dependencies in GPSG, HPSG, and LFG, but the
difference is that it is only assumed for those cases in which discontinuity would arise
otherwise. However, there seems to be a reason to assume that fronting should be treated by special mechanisms even in cases
that allow for continuous serialization. In what follows I discuss three phenomena that provide
evidence for a uniform analysis of V2 sentences: scope of adjuncts, coordination of simple and
complex sentences, and apparent multiple frontings that cross clause boundaries.

\subsection{Scope of adjuncts}

The ambiguity or lack of ambiguity of the examples in (\ref{ex-oft-liest-er-das-buch-nicht}) from
page~\pageref{ex-oft-liest-er-das-buch-nicht}--repeated here as (\mex{1})--cannot be explained in a straightforward way:
\eal
\ex\label{ex-oft-liest-er-das-buch-nicht-zwei} 
\gll Oft liest er das Buch nicht.\\
     often reads he the book not\\
\glt `It is often that he does not read the book.' or `It is not the case that he reads the book
often.'
\ex
\gll dass er das Buch nicht oft liest\\
     that he the book not often reads\\
\glt `It is not the case that he reads the book often.'
\ex
\gll dass er das Buch oft nicht liest\\
     that he the book often not reads\\
\glt `It is often that he does not read the book.'
\zl
The point about the three examples is that only (\mex{0}a) is ambiguous. Even though (\mex{0}c) has
the same order as far as \emph{oft} `often' and \emph{nicht} `not' are concerned, the sentence is
not ambiguous. So it is the fronting of an adjunct that is the reason for the ambiguity. The
dependency graph for (\mex{0}a) is shown in Figure~\vref{fig-oft-liest-er-das-buch-nicht-dg}.
\begin{figure}
\centering
\begin{forest}
dg edges
[V
  [Adv, dg adjunct [oft;often] ] 
  [liest;reads] 
  [N [er;he] ]
  [N 
    [Det [das;the] ]
    [Buch;book] ]
  [Adv, dg adjunct [nicht;not]] ]
\end{forest}
\caption{\label{fig-oft-liest-er-das-buch-nicht-dg}Dependency graph for \emph{Oft liest er das Buch
    nicht.} `He does not read the book often.'}
\end{figure}%
Of course the dependencies for (\mex{0}b) and (\mex{0}c) do not differ, so the graphs would be the
same only differing in serialization. Therefore, differences in scope could not be derived from the
dependencies and complicated statements like (\mex{1}) would be necessary:
\ea
If a dependent is linearized in the \vf it can both scope over and under all other adjuncts of the
head it is a dependent of.
\z
\citet[\page 320]{Eroms85a} proposes an analysis of negation in which the negation is treated as the head,
that is, the sentence in (\mex{1}) has the structure in Figure~\vref{dg-adv-head}.\footnote{
But see \citew[Section~11.2.3]{Eroms2000a}.
}
\begin{figure}
\begin{forest}
dg edges
[Adv 
  [V [N [er;he]]
     [kommt;comes]]
  [nicht;not]] 
\end{forest}
\caption{\label{dg-adv-head}Analysis of negation according to \citet[\page 320]{Eroms85a}}
\end{figure}%
% S:
%this clearly not a syntactic structure. You must use different conventions. See for instance the semantic graph of MTT.
%Moreover it is strange to propose a direct interface between semantics and word order. In mots DGs, semantics is lnked to an unordered dependency tree and this tree to the linear order.
%The necessary seperation between the syntactic dependencies and the linear order is extensively discussed in the begining of TEsnière's book.
%
This analysis is equivalent to analyses in the Minimalist Program that assume a NegP\is{NegP} and it
has the same problem: The category of the whole object is Adv, but it should be V. This is a problem
since higher predicates may select for a V rather than an Adv. See for instance the analysis of
embedded sentences like (\ref{ex-dass-er-nicht-singen-darf}) below.

The same is true for constituent negation or other scope bearing elements. For example, the analysis of (\mex{1})
would have to be the one in Figure~\vref{dg-alledged-murderer}.
\ea
\gll der angebliche Mörder\\
     the alleged murderer\\
\z
\begin{figure}
\begin{forest}
dg edges
[Adj
    [Det, no edge, name=det, l+=3\baselineskip, [der;the] ]
  [angebliche;alleged]
  [N, name=n [Mörder;murderer]]]
\draw (n.south)--(det.north);
\end{forest}
\caption{\label{dg-alledged-murderer}Analysis that would result if one considered all scope-bearing adjuncts
  to be heads}
\end{figure}%
This structure would have the additional problem of being non-projective. Eroms does treat the determiner
differently from what is assumed here, so this type of non"=projectivity may not be a problem for
him. However, the head analysis of negation would result in non"=projectivity in so"=called coherent
constructions in German. The following sentence has two readings: in the first reading the negation
scopes over \emph{singen} `sing' and in the second one over \emph{singen darf} `sing may'.
\ea\label{ex-dass-er-nicht-singen-darf} 
\gll dass er nicht singen darf\\
     that he not sing may\\
\glt `that he is not allowed to sing' or `that he is allowed not to sing'
\z
The reading in which \emph{nicht} scopes over the whole verbal complex would result in the
non-projective structure that is given in Figure~\vref{dg-nicht-singen-darf}.
\begin{figure}
\begin{forest}
dg edges
[Subj
  [dass;that]
  [Adv
    [N, no edge, name=n, tier=n [er;he]]
    [nicht;not]
    [V,name=v 
      [V, tier=n [singen;sing]]
      [darf;may]]]]
\draw (v.south)--(n.north);
\end{forest}
\caption{\label{dg-nicht-singen-darf}Analysis that results if one assumes the negation to be a head}
\end{figure}%
Eroms also considers an analysis in which the negation is a word part (`Wortteiläquivalent'), but
this does not help here since first the negation and the verb are not adjacent in V2 contexts like
(\ref{ex-oft-liest-er-das-buch-nicht}) and even in verb final contexts like
(\ref{ex-dass-er-nicht-singen-darf}). Eroms would have to assume that the object to which the negation
attaches is the whole verbal complex \emph{singen darf}, that is, a complex object consisting of two
words.

So, this leaves us with the analysis provided in Figure~\ref{fig-oft-liest-er-das-buch-nicht-dg} and
hence with a problem since we have one structure with two possible adjunct realizations that
correspond to different readings, which is not predicted by an analysis that treats the two possible
linearizations simply as alternative orderings.
% Eroms 2000: 159 nicht is an adjunct, später dann Ergänzungskanten

Thomas Groß (p.\,c.\ 2013) suggested an analysis in which \emph{oft} does not depend on the verb but
on the negation. This corresponds to constituent negation in phrase structure approaches. The
dependency graph is shown at the left-hand side in Figure~\vref{fig-oft-liest-er-das-buch-nicht-dg-constituent-negation}.
\begin{figure}
\hfill
\begin{forest}
dg edges
[V
  [Adv, edge=dashed [oft;often] ] 
  [liest;reads] 
  [N [er;he] ]
  [N 
    [Det [das;the] ]
    [Buch;book] ]
  [Adv\sub{g}, dg adjunct [nicht;not]] ]
\end{forest}
\hfill
\begin{forest}
dg edges
[V, l sep+=5pt
  [N [er;he] ]
  [N 
    [Det [das;the] ]
    [Buch;book] ]
  [Adv, dg adjunct=4pt [nicht;not]
    [Adv [oft;often] ] ] 
  [liest;reads] ]
\end{forest}
\hfill\mbox{}
\caption{\label{fig-oft-liest-er-das-buch-nicht-dg-constituent-negation}Dependency graph for \emph{Oft liest er das Buch
    nicht.} `He does not read the book often.' according to Groß and verb-final variant}
\end{figure}%
The figure at the right-hand side shows the graph for the corresponding verb-final sentence. The
reading that corresponds to constituent negation can be illustrated with contrastive
expressions. While in (\mex{1}a) it is just the \emph{oft} `often' that is negated, it is \emph{oft
  gelesen} `often read' that is in the scope of negation in (\mex{1}b).
\eal
\ex 
\gll Er hat das Buch nicht oft gelesen, sondern selten.\\
     he has the book not often read     but seldom\\
\glt `He did not read the book often, but seldom.'
\ex
\gll Er hat das Buch nicht oft gelesen, sondern selten gekauft.\\
     he has the book not often read     but seldom bought\\
\glt `He did not read the book often but rather bought it seldom.'
\zl
These two readings correspond to the two phrase structure trees in Figure~\vref{fig-er-das-buch-nicht-oft-liest-psg}.
\begin{figure}
\hfill
\begin{forest}
sm edges
  [V
    [N [er;he] ]
    [V
      [NP 
        [Det [das;the] ]
        [N [Buch;book] ] ]
      [V 
        [Adv [nicht;not]] 
        [V [Adv [oft;often] ] 
           [V [liest;reads] ] ] ] ] ]
\end{forest}
\hfill
\begin{forest}
sm edges
  [V
    [N [er;he] ]
    [V
      [NP 
        [Det [das;the] ]
        [N [Buch;book] ] ] 
      [V 
        [Adv 
           [Adv [nicht;not] ]
           [Adv [oft;often] ] ]  
        [V [liest;reads] ] ] ] ]
\end{forest}
\hfill\mbox{}
\caption{\label{fig-er-das-buch-nicht-oft-liest-psg}Possible syntactic analyses for \emph{er das
    Buch  nicht oft liest} `He does not read the book often.'}
\end{figure}%
Note that in an HPSG analysis, the adverb \emph{oft} would be the head of the phrase \emph{nicht oft}
`not often'. This is different from the Dependency Grammar analysis suggested by Groß. Furthermore,
the Dependency Grammar analysis has two structures: a flat one with all adverbs depending on the
same verb and one in which \emph{oft} depends on the negation. The phrase structure"=based analysis
has three structures: one with the order \emph{oft} before \emph{nicht}, one with the order
\emph{nicht} before \emph{oft} and the one with direct combination of \emph{nicht} and
\emph{oft}. The point about the example in (\ref{ex-oft-liest-er-das-buch-nicht}) is that one of the
first two structures is missing in the Dependency Grammar representations. This probably does not make it
impossible to derive the semantics, but it is more difficult than it is in constituent"=based approaches.

\subsection{Coordination of simple and complex sentences}


A further argument against linearization approaches for simple sentences can be based on the following coordination
example:
\ea
\gll Wen$_i$ kennst \_$_i$ du  und glaubst du, dass \_$_i$ jeder kennt?\\
     who     knows  {} you and believe you that {} everybody  knows\\
\glt `Who do you know and do you believe that everybody knows?'
\z
The classical analysis of Across the Board Extraction\is{Across the Board Extraction} by
\citet{Gazdar81} assumes that two slashed clauses are coordinated. If one assumes that simple
clauses are analyzed via linearization of one element into the \vf while long-distance dependencies
are analyzed with a special mechanism (for instance the \slasch mechanism of GPSG/HPSG or special
dependencies \citew{Hudson2000a}), the two coordinated clauses in (\mex{0}) would differ
fundamentally in their structure and all coordination theories would fail. The conclusion is that
coordination forces us to treat the frontings in the sentences in (\mex{1}) in the same way:
\eal
\ex
\gll Wen kennst du?\\
     who     knows  you\\
\glt `Who do you know?'
\ex
\gll Wen glaubst du, dass jeder kennt?\\
     who     believe you that everybody knows\\
\glt `Who do you believe that everybody knows?'
\zl
Either both sentences are analyzed via linearization or both are analyzed using a special
mechanism for extraction. Since linearization analyses of (\mex{0}b) are either very complicated
(in HPSG) or open Pandora's box (in Dependency Grammar, see \citealp[Section~11.7.1]{MuellerGT-Eng1}), extraction-based
analyses with a special mechanism for both sentences should be preferred.


\subsection{Apparent multiple frontings}

Furthermore, note that models that directly relate dependency graphs to topological fields will not be able to
account for sentences like (\mex{1}).
\ea
\gll Dem Saft eine kräftige Farbe geben Blutorangen.\footnotemark\\
     the juice a   strong   color give blood.oranges\\
\footnotetext{
\citet{BC2010a} found this example in the \emph{Deutsches Referenzkorpus} (DeReKo), hosted at Institut
für Deutsche Sprache, Mannheim: \url{http://www.ids-mannheim.de/kl/projekte/korpora}
}
\glt `Blood oranges give the juice a strong color.'
\z
The dependency graph of this sentence is given in Figure~\vref{fig-dem-saft-eine-kraefitge-farbe-dg}.
\begin{figure}[htb]
\centerline{
\begin{forest}
dg edges
[V
  [N [Det,tier=eine [dem;the]]
   [Saft;juice] ]
  [N, l sep+=3pt 
      [Det,tier=eine [eine;a] ]
      [Adj, dg adjunct=4pt  [kräftige;strong]]
      [Farbe;color]]
  [geben;give] 
  [N [Blutorangen;blood.oranges] ] ]
\end{forest}
}
\caption{\label{fig-dem-saft-eine-kraefitge-farbe-dg}Dependency graph for \emph{Dem Saft eine kräftige Farbe geben Blutorangen.} `Blood oranges give the juice a strong color.'}
\end{figure}%

Such apparent multiple frontings are not restricted to NPs. As was shown in Section~\ref{sec-phenomenon-mult-front}, various types of dependents can be
placed in the \vf. Any theory that is based on dependencies alone and that does not allow for
empty elements is forced to give up the restriction that is commonly assumed in the analysis of V2 languages, namely that the verb is in second position. 
In comparison, analyses like \gb and those \hpsg variants that assume an empty verbal head can
assume that a projection of such a verbal head occupies the \vf. This explains why the material in
the \vf behaves like a verbal projection containing a visible verb: Such \emph{Vorfelds} are
internally structured topologically, they may have a filled \nf and even a particle that fills the
right sentence bracket (Examples with verbal particle and \mf or \nf are given in
(\ref{ex-adv-particle}) on page~\pageref{ex-adv-particle}). The equivalent of the analysis in Gross
\& Osborne's framework \citeyearpar{GO2009a} would be something like the graph that is shown 
in Figure~\vref{fig-dem-saft-eine-kraefitge-farbe-empty-dg}, but note that \citet[\page 73]{GO2009a}
explicitly reject empty elements and in any case an empty element that is stipulated just to get the
multiple fronting cases right would be entirely ad hoc.\footnote{
  I stipulated such an empty element in a linearization-based variant of HPSG allowing for
  discontinuous constituents \citep{Mueller2002c},
  but later modified this analysis so that only continuous constituents are allowed and verb
  position is treated as head-movement and multiple frontings involve the same empty verbal head as
  is used in the verb movement analysis. The revised theory is presented in this book.
}
\begin{figure}
\centerline{
\begin{forest}
dg edges
[V\sub{g}
  [V,edge=dashed [N [Det [dem;the]]
      [Saft;juice] ]
     [N [Det [eine;a] ]
        [Adj, dg adjunct=4pt [kräftige;strong]]
        [Farbe;color]]
     [ \trace ] ]
  [geben;give] 
  [N [Blutorangen;blood.oranges] ] ]
\end{forest}
}
\caption{\label{fig-dem-saft-eine-kraefitge-farbe-empty-dg}Dependency graph for \emph{Dem Saft eine
    kräftige Farbe geben Blutorangen.} `Blood oranges give the juice a strong color.' with an empty
  verbal head for the \vf}
\end{figure}%
It is important to note that the issue is not solved by simply dropping the V2 constraint and
allowing dependents of the finite verb to be realized to its left, since the fronted constituents do
not necessarily depend on the finite verb as the examples in (\ref{bsp-gezielt-mitglieder}) and
(\ref{bsp-kurz-die-bestzeit}) from page~\pageref{bsp-gezielt-mitglieder} -- repeated
here as (\mex{1}) -- show:
\eal
\ex
\gll [Gezielt] [Mitglieder] [im     Seniorenbereich]       wollen  die Kendoka allerdings nicht werben.\footnotemark\\
    \spacebr{}targeted \spacebr{}members     \spacebr{}in.the senior.citizens.sector want.to the Kendoka however    not   recruit\\
\glt `However, the Kendoka do not intend to target the senior citizens sector with their member recruitment strategy.'%
\label{bsp-gezielt-mitglieder-zwei}
\footnotetext{
        taz, 07.07.1999, p.\,18. Quoted from \citew{Mueller2002c}.
      }
\ex 
\gll {}[Kurz] [die Bestzeit] hatte der Berliner Andreas Klöden [\ldots] gehalten.\footnotemark\\
	 \spacebr{}briefly \spacebr{}the best.time had the Berliner Andreas Klöden {} held\\
\footnotetext{
        Märkische Oderzeitung, 28./29.07.2001, p.\,28.
}\label{bsp-kurz-die-bestzeit-zwei}     
\glt `Andreas Klöden from Berlin had briefly held the record time.'
\zl
And although the respective structures are marked, such multiple frontings can even cross clause boundaries:
\eal
\label{ex-der-maria-einen-ring}
\ex 
\gll Der Maria einen Ring glaube ich nicht, daß er je schenken wird.\footnotemark\\
	 the Maria a ring believes I not that he ever give will\\
\footnotetext{
\citew[\page 67]{Fanselow93a}.
}
\glt `I dont think that he would ever give Maria a ring.'
\ex 
\gll Kindern Bonbons denke ich daß  man besser nicht gibt.\footnotemark\\
     children candy  think I   that one better not   gives\\
\footnotetext{
        G.\ \citew[\page 261]{GMueller98a}.
}
\glt `I think it's better not to give candy to children.'
\zl
If such dependencies are permitted it is really difficult to constrain them. As was discussed in
Section~\ref{sec-ausgeschlossen-mvf}, I started with an approach that admitted several elements in \slasch, but the
disadvantage was that it was difficult to explain why certain parts of idioms could not be
extracted. Furthermore, it would be difficult to represent the fact that the fronted elements have
to be clausemates (see also Section~\ref{sec-ausgeschlossen-mvf}).

So far I only discussed Dependency Grammar approaches but all the issues mentioned in this and the
previous subsections are also problematic for Wetta's approach
\citeyearpar{Wetta2011a,Wetta2014a-u}. Wetta, working in the framework of linearization"=based HPSG 
\citep{Reape94a,Kathol95a,Kathol2001a,Mueller99a,Mueller2002b}, assumes that sentences in which
the\todostefan{Rui has some further suggestions, check whether published}
fronted elements belong to a verb in the same clause are simply reordering variants of sentences
with the verb in initial or final position. For the analysis of apparent multiple frontings \citet{Wetta2011a}
assumes a relational constraint that takes arbitrarily many preverbal objects and forms a new complex
one:\footnote{
  I added a \textsc{dom} feature in the first domain element of the mother, which was missing in the
  original.
}
\begin{exe}
\ex Discourse prominence constructions for German according to \citet[\page 264]{Wetta2011a}:
\begin{xlist}
\ex doms$_\bigcirc$(\sliste{ [\textsc{dom} X$_1$], \ldots, [\textsc{dom} X$_n$] }) $\equiv$ X$_1$
$\bigcirc$ \ldots{} $\bigcirc$ X$_n$
\ex \type{prom-part-compact-dom-cxt} \impl\\
    \ms{
    mtr & \ms{ dom \sliste{ \ms{ prom & \normalfont +\\
                                 dom  & \normalfont doms$_\bigcirc$(L$_1$) } } $\bigcirc$
      \normalfont doms$_\bigcirc$(L$_2$) \\
                               }\\
    dtrs & \upshape L$_1$:\type{list}( [\textsc{prom}+] ) $\bigcirc$ L$_2$:\type{list}\\
    }
\zl
The list L$_1$ in (\mex{0}) is a list of elements that are marked as prominent (\textsc{prom}+). The
idea is that these elements are compacted into one element which is then the element that is placed
in the \vf. The daughters that are not marked as prominent are collected in L$_2$.%
% Könnte man machen, indem man Linearisierungsbeschränkungen sagen lässt, dass es nur ein
% PROM+-Element pro Domäne geben darf.
\footnote{
  Actually the fact that L$_2$ does contain \textsc{prom}$-$ elements is not specified in
  (\mex{0}b). It may follow from other constraints in the theory. They were not given in the paper though.
}
Since they are combined with $\bigcirc$ with the prominent element, they can appear in any order
provided no linearization constraint is violated.
Figure~\ref{fig-v3-wetta2011} shows what (\mex{0}b) does.\footnote{%
  Note that \citet[\page 259]{Wetta2011a} assumes an unusual definition of compaction and hence that the \domv of \emph{die Weltmeisterschaft} is a single object, while
  all other domain"=based approaches assume that \emph{die Weltmeisterschaft} has two \domain
  objects: \emph{die} and \emph{Weltmeisterschaft} \citep{KP95a,Kathol2001a,Babel,Mueller99a}. Without these rather unusal assumptions the call
  of doms$_\bigcirc$ would be unnecassiry and L$_1$ and L$_2$ could be used directly.
}
\begin{figure}
\oneline{%
\begin{forest}
sn edges
[{\ms{ d & \sliste{ \ms{ prom & \normalfont + \\
                          f    & \phonliste{ zum, zweiten, Mal, die Weltmeisterschaft } },
                     \ms{ f    & \phonliste{ errang } },
                     \ms{ f    & \phonliste{ Clark } } } } } 
 [{\ms{ d & \sliste{ \ms{ f  & \phonliste{ Clark } \\
                            s  & \normalfont NP[\type{nom}] } }\\ }}]
 [{\ms{ d & \sliste{ \ms{ f  & \phonliste{ die, Weltmeisterschaft } \\
                            s  & \normalfont NP[\type{nom}] } }\\ }}]
 [{\ms{ d & \sliste{ \ms{ f  & \phonliste{ zum, zweiten, Mal } \\
                            s  & \normalfont PP } }\\ }}]
 [{\ms{ d & \sliste{ \ms{ f  & \phonliste{ errang } \\
                            s  & \normalfont V[\type{fin}] } }\\ }}]
]
\end{forest}}
\caption{Vn with partial compaction according to \citet[\page 264]{Wetta2011a}}\label{fig-v3-wetta2011}
\end{figure}
The daughters of the complete construction are shown in the figure. \emph{zum zweiten Mal} and \emph{die Weltmeisterschaft} will
be marked \textsc{prom}+ and their \domvs will be combined via $\bigcirc$ and the result will be the
\domv of the first element in the \doml of the mother. This begs the question what the properties of this new object would be. This is not made explicit
in Wetta's paper but he assumes that domain objects are of type \type{sign}, so this object has to
have syntactic and semantic properties. Note that HPSG grammars are descriptions of
models. Descriptions are usually partial. Everything that is not specified can vary as long as no
appropriateness conditions on types are violated \citep{King99a-u}. For example, a theory of German could leave the
actual case value of the German noun \emph{Frau} `woman' unspecified since it is clear that the
type \type{case} has the four subtypes \type{nom}, \type{gen}, \type{dat}, \type{acc}. In a model, the
value can only vary in this limit, that is, the actual case has to be one of these four values \citep[Section~2.7]{MuellerLehrbuch1}. If
this is applied to Wetta's theory we get infinitely many models since the syntactic properties of
the first domain element are not specified. Since valence lists are part of syntactic descriptions
and since they may be arbitrarily long in principle, there are infinitely many models for Wetta's
structures. To fix this, he would have to specify the category of the element in the \vf. But what
could the category be? In other partial compaction theories the category of the created element is
the category of the head \citep{KP95a,Kathol2001a,Babel,Mueller99a}, but in Wetta's theory there is no common head for the compacted
objects. He could stipulate that the newly created object would be something like the object one gets in an analysis with an empty
verbal head. But then a relational constraint is used that creates structure out of nothing
instead of the assumption of an empty head that basically behaves like a normal visible verb. The
constraint introducing this verbal element would be something unseen elsewhere in the grammar. Nothing is gained by such an analysis.

\citet{Wetta2014a-u} drops the relational constraint that compacts several fronted elements into one
consituent and just states that arbitrarily many constituents can appear in front of the finite verb
in German. This has the advantage that no stipulation of the properties of the preverbal constituent
is needed but it leaves the fact unexplained that the material in front of the finite verb behaves
like a verbal projection with all topological fields in them (see below). 

Neither \citet{Wetta2011a} nor \citew{Wetta2014a-u} addresses data like (\ref{ex-der-maria-einen-ring}) and indeed it would be difficult to integrate such data into
his picture, since he does assume that nonlocal frontings are handled via the \slasch
mechanism. \citet{Wetta2018a} suggested a modification of the Filler-Head Schema that allows
multiple elements in \slasch and inserts the \slasch elements into the order domain of the
head.\footnote{%
  I adapted the schema and put the C inside of the angle brackets.
}
\ea
\label{schema-f-g-wetta}
\type{filler-s-p-cxt} \impl\\*
\ms{
mtr & \ms{ syn \ms{ cat & y \\
                    val & \eliste\\
                    gap & \eliste\\
                  }\\
         }\\
dtrs & \sliste{ H } $\bigcirc$ L$_1$ $\bigcirc$ L$_2$ $\bigcirc$ \sliste{ C }\\
hd-dtr & \upshape H: \ms{ syn \ms{ cat & Y: \ms{ vf & fin\\
                                      }\\
                          val & L$_1$ $\oplus$ \sliste{ C: \ms[clause]{
                                                           syn & \ms{ gap & L$_2$\\
                                                                    } } }\\
                       }\\
}
} 
\z
H is the head of the matrix clause, the finite verb, C is the clausal complement of H, L$_1$ is the
list of other arguments of H and L$_2$ is the list of gaps coming up from the embedded clause. The
\dtrsl of the complete construction consists of the shuffeling of lists containing the head and the
complement clause and L$_1$ and L$_2$. Shuffling means that the elements of the involved lists can be
ordered in any order as long as the relative order of elements in the lists remains constant \citep{Reape94a}. That
is, elements from L$_1$ can be ordered before or after any elements from the other lists as long as
the order of elements in L$_1$ is not changed.\footnote{%
  In fact this makes wrong predictions as far as the order of arguments in L$_1$ is
  concerned. German is a scrambling language and hence all orders of arguments of a head are allowed
  for in principle. For example both the orderings of the subject and object of \emph{gebeten} in
  (i) are possible.
\eal
\ex 
\gll [Über dieses Thema]$_i$ hat noch niemand den Mann gebeten,~~~~~~~~~~~~~ [[einen Vortrag \_$_i$ zu halten].\\
     \spacebr{}about this topic  has yet  nobody.\nom{}  the man.\acc{}  asked    \hspaceThis{[[}a
         talk {} to give\\
\glt `Nobody asked the man yet to give a talk about this topic.'
\ex 
\gll [Über dieses Thema]$_i$ hat den Mann noch niemand gebeten,~~~~~~~~~~~~~ [[einen Vortrag \_$_i$ zu halten].\\
     \spacebr{}about this topic  has the man.\acc{} yet  nobody.\nom{}    asked    \hspaceThis{[[}a
         talk {} to give\\    
\zl
  Since the schema in (\ref{schema-f-g-wetta}) can account for only one of these orders, it is not empirically
  adequate. The same problem applies to several schemata in \citew{Wetta2014a-u}, for example to the
  schema he gives on page 164: the elements of L can be scrambled but the schema does not account
  for this. The problem can be solved by assuming a special constraint that maps
  a list to all lists containing permutations of its elements. Rather than combining L$_1$ with the other lists in
  (\ref{schema-f-g-wetta}), one would then combine permutations(L$_1$) with the lists. Of course
  stipulating such a constraint that is not used anywhere else in Wetta's grammar adds to the complexity of his approach.
}

This makes it possible to account for sentences like (\ref{ex-der-maria-einen-ring}) but this analysis does not explain that the elements that
appear in front of the finite verb have to be clause mates. \citew[\page 171]{Wetta2014a-u} captures
the same-clause restriction by analyzing Vn orders as local reorderings. Since in the 2014 approach nonlocal dependencies
are not involved in Vn constructions, it follows that the elements have to be clause mates
(dependents of the verbs in the highest clause). In order to deal with examples like (\ref{ex-der-maria-einen-ring}), \citet{Wetta2018a} drops the constraint that relates Vn to local reorderings. But as soon as
nonlocal dependencies are allowed in Vn constructions, the problem of mixing material depending on
different heads creeps back in again. Rui Chaves (p.\,c.\,2018) suggested that
one could fix this problem by information-structural constraints on the fronted material. He
suggested that all fronted elements are marked as [\textsc{prominent}+] and that all \textsc{prom}+
elements have to depend on the same semantic predicate. Note that such a constraint would be violated
in sentences in which a V2 sentence is embedded into another V2 sentence:
\ea
\gll Peter denkt, Klaus kommt morgen.\\
     Peter thinks Klaus comes tomorrow\\
\glt `Peter thinks that Klaus is coming tomorrow.'
\z
\emph{Peter} is fronted and hence \textsc{prom}+ and \emph{Klaus} is also fronted and
\textsc{prom}+. Both depend on different verbs and both are fronted but within separate
clauses. The fronted elements do not share a common \vf. So, the constraint that is supposed to rule
out (\mex{1}) would also rule out (\mex{0}).
\ea[*]{
\gll Peter Klaus denkt, kommt morgen.\\
     Peter Klaus thinks comes tomorrow\\
}
\z
There may be ways to formalize the intuition behind the original proposal but this cannot be done
exclusively on the semantic/""information structural level. One would have to find ways to know which
\vf one is talking about, that is, whether the constituents are in the same \vf or in different
ones. The empty verbal head seems to be better suited to capture such constraints than any other
device one may think of.
%
% Rui 20.01.2018
% 2. A way to solve the problem you raise wrt to my
%% information-structural hypothesis and examples in (15) and (16) would
%% be to relativize it to each clause: for a given clause either all
%% fronted elements are linked to the same local verb, or all fronted
%% elements are linked to the same verb via SLASH.  Hence (15) would be
%% fine because the rule applies twice, for each clause.  Another way
%% would be to distinguish between scrambling prominence and extraction
%% prominence and require that fronted elements be of the same kind of
%% prominence (i.e. no need to refer to the verb).
%
% St: OK, may work. But how do you ensure that the SLASHes are coming from the same verb?

Note that information structural approaches that assume that several independent items are fronted
and that these fronted elements have certain special information structure functions also fail on examples
like the Claudia Perez sentence in (\ref{bsp-ihnen-für-heute-drei}) on page~\pageref{bsp-ihnen-für-heute-drei}, which was discussed in
Section~\ref{sec-assignment-to-top-foc}. Sentences like (\ref{ex-los-damit}) from page~\pageref{ex-los-damit}--repeated here as
(\mex{1})--are also problematic since it is not the case that the fronted elements have a special 
information structural status. It is not appropriate to just label \emph{los} and \emph{damit} as
\textsc{prom}+ and state that all \textsc{prom}+ elements are ordered before the finite verb.  It is the fronted phrase that has to be taken care of. Without the
assumption of a phrase there, there is no straight-forward way to do this.
\ea
\label{ex-los-damit-zwei} 
\gll \emph{Los} damit \emph{geht} es schon am 15. April.\footnotemark\\
	  off there.with goes it PRT on 15. April\\
\footnotetext{
        taz, 01.03.2002, p.\,8.
    }
\glt `The whole thing starts on the 15th April.'
\z
Note also that the order of the elements before the finite verb corresponds to the order that we
would see without fronting. \emph{los} is a right sentence bracket and \emph{damit} is
extraposed. Any theory that assumes that \emph{los} and \emph{damit} are in the same order domain as
the finite verb would run into deep trouble since it would have to assume a right sentence bracket
(\emph{los} `off') and a \nf (\emph{damit} `there.with') to the left of the left sentence bracket (\emph{geht} `goes').\footnote{
  There is a technical solution to the problem: one could set up linearization constraints that only apply to constituents with the same
  \textsc{prom} value. \textsc{prom}+ would be the \vf, \textsc{prom}$-$ the rest of the clause. In
  order to avoid the problems with information structural properties of the elements in the \vf, the
  \textsc{prom} feature could be renamed into \textsc{vorfeld}. 
\eal
\ex Mittelfeld [\textsc{vorfeld}  \ibox{1}] $<$ right bracket [\textsc{vorfeld}  \ibox{1}]
\ex right bracket [\textsc{vorfeld}  \ibox{1}] $<$ Nachfeld [\textsc{vorfeld}  \ibox{1}]
\zl         
See \citew*{KKP95} on linearization constraints with structure sharing.
}

Another problem that \slasch-based approaches have is also present again for Wetta's revised
proposal: it is difficult to restrict the fronting of idiom parts. It is possible to construct
nonlocal frontings that are parallel to the examples that were discussed in Section~\ref{sec-phraseolog}.
\ea 
\gll {}[Öl] [ins Feuer] behauptete der Vorsitzende, dass gestern das Rote-Khmer-Radio gegossen habe.\\
	 \spacebr{}oil \spacebr{}in.the fire claimed the chairman that yesterday the
         Rote-Khmer-Radio poured has \\
\glt `The chairman claimed that Rote-Khmer-Radio fanned the flames yesterday.'
\z
As was explained in Section~\ref{sec-idiom-parts-mf}, fronting of \emph{ins Feuer} `in.the fire'
without \emph{Öl} `oil' results into a literal reading. It is not obvious how this constraint can be formalized in an
extraction-based approach while the restriction that certain idiom parts want to stick together in a
verbal projection falls out immediately from an approach using an empty verbal head. I argued that the head
is the same head as is used in the verb movement analysis. So no new empty elements are needed to
get the data discussed in this section.

In what follows I have a look at other verb movement analyses that have been suggested in HPSG.


\section{Binary branching and linearization domains}

\citet{Kathol2000a} suggests an analysis with binary branching structures in which all arguments are
inserted into a linearization domain and can be serialized there in any order provided no LP rule is
violated. Normally one would have the elements of the \compsl in a fixed order, combine the head
with one element from the \compsl after the other, and let the freedom in the \doml be responsible
for the various attested orders. So both sentences in (\mex{1}) would have analyses in which the
verb \emph{erzählt} is combined with \emph{Geschichten} first and then \emph{Geschichten erzählt} is
combined with \emph{den Wählern}. Since the verb and all its arguments are in the same linearization
domain they can be ordered in any order including the two orders in (\mex{1}):
\eal
\label{ex-waehlern-geschichten-erzaehlt}
\ex weil er den Wählern Geschichten erzählt
\ex weil er Geschichten den Wählern erzählt
\zl
The problem with this approach is that examples like (\mex{1}) show that grammars have to account
for combinations of any of the objects to the exclusion of the other:
\eal
\label{ex-geschichten-erzaehlen}
\ex Geschichten erzählen sollte man den Wählern nicht.
\ex Den Wählern erzählen sollte man diese Geschichten nicht.
\zl
\citet{Kathol2000a} accounts for examples like (\mex{0}) by relaxing the order of the objects in the
valence list. He uses the shuffle operator in the valence representation:
\ea
\sliste{ NP[nom] } $\oplus$ \sliste{ NP[dat] } $\bigcirc$ \sliste{ NP[acc] }
\z
This solves the problem with examples like (\mex{-1}) but it introduces a new one: sentences like
(\ref{ex-waehlern-geschichten-erzaehlt}) now have two analyses each. One is the analysis we had before and another one is the one
in which \emph{den Wählern} is combined with \emph{erzählt} first and the result is then combined
with \emph{Geschichten}. Since both objects are inserted into the same linearization domain, both
orders can be derived. So we have too much freedom: freedom in linearization and freedom in the
order of combination. The proposal that I suggested has just the freedom in the order of combination
and hence can account for both (\ref{ex-waehlern-geschichten-erzaehlt}) and
(\ref{ex-geschichten-erzaehlen}) without spurious ambiguities.

\section{Binary branching in different directions}


\mbox{}\citet[\page 159]{Steedman2000a-u}, working in the framework of Categorial Grammar, proposed
an analysis with variable branching for Dutch\il{Dutch}, that is, there are two lexical entries for
\emph{at} `eat': an initial one with its arguments to the right, and another occupying final
position with its arguments to its left. 

\eal
\ex \emph{at} `eat' in verb-final position: (s\sub{+SUB}$\backslash$np)$\backslash$np
\ex \emph{at} `eat' in verb-initial position: (s\sub{$-$SUB}/np)/np
\zl
Steedman uses the feature \textsc{sub} to differentiate between subordinate and non-subordinate
sentences. Both lexical items are related via lexical rules.\is{lexical rule}

Such approaches were criticized by \citet{Netter92} since the branching in verb-initial sentences
is the mirror image of verb-final sentences. The scope facts in sentences like (\ref{bsp-absichtlich-nicht-anal-v1}) on
page~\pageref{bsp-absichtlich-nicht-anal-v1} cannot be explained easily, while they fall out
automatically in a verb movement approach as is shown in the examples in (\mex{1}):
\eal
\label{bsp-absichtlich-nicht-anal-v1-zwei}
\ex 
\gll Er lacht$_i$ [absichtlich [nicht \_$_i$]].\\
     he laughs \spacebr{}intentionally \spacebr{}not\\
\glt `He is intentionally not laughing.'
\ex 
\gll Er lacht$_i$  [nicht [absichtlich \_$_i$]].\\
     he laughs \spacebr{}not \spacebr{}intentionally\\
\glt `He is not laughing intentionally.'
\zl\is{scope}
Now, it has to be said that the scoping is the same in SVO languages like French\il{French} even though no
movement took place. So there may be a more general analysis of adjunct scope that covers both SVO
languages and the two verb placements that are possible in V2 languages with SOV order.

Independent of the scope question is the analysis of apparent multiple frontings: if there is no
empty head it is not obvious how the phenomenon that was discussed in
Chapter~\ref{chapter-mult-front} can be analyzed. The proposals with binary branching structures and
different branching directions are basically similar to the GPSG proposal with flat structures and
two alternative serializations of the finite verb. See Section~\ref{sec-flat-free-linearization-of-verb-gpsg}.


\section{Alternative verb-movement analyses}
\label{sec-alternative-verbbewegung}
\label{sec-kopfbewegung-vs-fernabhaengigkeit}

The rule for verb-first placement in German proposed here is similar to that of
\citet{KW91a}, \citet[Chapter~2.2.4.2]{Kiss95a} and \citet{Frank94}. However, there are differences
and these will be discussed in what follows.

\citet{Kiss95a} views \dsl not as a head feature (as I do here), but rather as a
\nonloc-feature. His head trace has the following form, which is parallel to the extraction trace:

\ea
Head trace \citep[\page 72]{Kiss95a}:\\*
\ms{
synsem & \onems{ loc \ibox{1}\\
                 nonloc$|$inher$|$dsl \menge{ \ibox{1} }\\
               }\\
}
\z
Kiss uses the same percolation mechanism for head movement as for extraction, namely percolation
via \textsc{nonloc|inher}.

The lexical rule which licenses the verb in initial position is represented as follows:\footnote{
		I have omitted a superfluous structure sharing between the \textsc{head} value in the input
		of the rule and the \textsc{head} value of the element in \subcat. The respective
		restrictions follow on from the specification of the trace.%
}

\eas
\label{kiss-dsl-lr}
\ms{ synsem$|$loc & \ibox{3} \onems{ cat$|$head \ibox{1} \ms{ vform & fin \\
                                                       } \\
                              } \\
   } $\mapsto$ \\
\ms{ synsem & \ms{ loc & \onems{ cat \ms{ head & \ibox{1} \\
                                         subcat & \liste{ \onems{ loc  \onems{ cat$|$subcat \liste{} \\
                                                                                    cont \ibox{2} \\
                                                                                  }  \\
                                                                          nl$|$inher$|$dsl  \{ \ibox{3} \} \\
                                                                        }
                                                        } \\
                                       } \\
                              cont \ibox{2} \\
                             } \\
                    nl & \ms{ to-bind$|$dsl & \textrm{\upshape \{ \ibox{3} \}} \\
                                } \\
                 } \\
}
\zs

\noindent
\citet{Frank94} has criticized Kiss' analysis as it does not predict the locality
restrictions of head movement.\footnote{
		For further discussion see \citew[\page 231--234]{Kiss95b}. Kiss proposes
		the exclusion of sentences such as (\mex{1}) by stating that complementizers
		always require that embedded sentences have an empty list as the value
		\textsc{nl$|$inher$|$dsl}. It is assumed in \citew{BFGKKN96a} that \dsl is  
		a \textsc{nonloc} feature, but that it is projected along a head path only.%
}
Without further assumptions, a sentence such as (\mex{1}a) 
would be predicted to be grammatical:

\eal
\ex[*]{
\gll Kennt$_i$ Peter glaubt,  dass Fritz Maria \_$_i$?\\
     knows     Peter believes that Fritz Maria\\
}
\ex[]{
\gll Glaubt$_i$ Peter \_$_i$,  dass Fritz Maria kennt?\\
     believes   Peter {}      that Fritz Maria knows\\
}
\zl
In the incorrect analysis of (\mex{0}a), the lexical rule (\ref{kiss-dsl-lr}) is applied
to \emph{kennt}. It is however not ensured that the element in \textsc{dsl} that is bound off by
\emph{kennt} (\,\ibox{3} in (\mex{-1})) is the head of the verbal projection that is selected by
\emph{kennen} `to know'. In the analysis of the well-formed (\mex{0}b) \emph{glaubt} is combined
with \emph{Peter \_$_i$,  dass Fritz Maria kennt} and the verb trace \_$_i$ is in the same local
domain as the verb \emph{glaubt}: \_$_i$ is the head of the clause that is combined with
\emph{glaubt}; it is the head of both \emph{Peter} and \emph{dass Fritz Maria kennt}. In the analysis of (\mex{0}a) the information about the verb trace crosses a clause
boundary. There is nothing that prevents the percolation of \dsl information from a more deeply
embedded clause. 

Frank has developed an analysis which creates a finer-grained distinction between functional
and lexical elements and suggests therefore the following solution for the locality issue:
the semantic content of the input for a lexical rule is identified with the semantic content
in the output of the lexical rule. When applied to Kiss' analysis, it would look like this:

\ea
\label{kiss-dsl-lr-2}
\begin{tabular}[t]{@{}l@{}}
\ms{ synsem$|$loc & \ibox{3} \onems{ cat$|$head  \ibox{1} \ms{ vform & fin \\
                                                       } \\
                                  cont  \ibox{2}\\
                              } \\
   } $\mapsto$ \\
\ms{ synsem & \ms{ loc & \onems{ cat \ms{ head & \ibox{1} \\
                                         subcat & \liste{ \onems{ loc  \onems{ cat$|$subcat \liste{} \\
                                                                                    cont \ibox{2} \\
                                                                                  }  \\
                                                                          nl$|$inher$|$dsl  \{ \ibox{3} \} \\
                                                                        }
                                                        } \\
                                       } \\
                              cont \ibox{2} \\
                             } \\
                    nl & \ms{ to-bind$|$dsl & \textrm{\upshape \{ \ibox{3} \}} \\
                                } \\
                 } \\
}
\end{tabular}
\z
This analysis fails, however, as soon as we have to deal with adjuncts. These are combined
with the verb trace and the \contv of the verb trace projection is therefore no longer
identical to the the \contv contained in \textsc{dsl}. See Figure~\vref{fig-verb-movement-adjunkt-sem}
for the exact representation of this.

The most simple solution to restrict verb movement to head domains is to make the corresponding
information a head feature, and for this reason only available along the head projection. \citet{Oliva92b}
and \citet{Frank94,Frank94b} have suggested representing valence information under \textsc{head} and
accessing this information inside the verb trace. As shown in Section~\ref{sec-v1}, valence information
alone is not enough to model verb movement correctly and we should therefore, as \citet{Kiss95a} suggests, 
assume that all local information , i.e. semantic content as well as syntactic information, percolates.


Furthermore, placing the head movement information inside of \textsc{local} features is
necessary for the analysis of cases of supposed multiple fronting as a verb trace is present
in initial position in such cases, \ie the verb trace is part of a filler in a long-distance dependency. A \textsc{dsl} value
which is percolated inside of \textsc{nonloc} in the constituent in initial position could not be checked at the extraction
site since only the features under \textsc{local} are shared by the extraction trace and filler.


\section{V1 via argument composition}

\citet{Jacobs91a}, working in Categorial Grammar, and \citet{Netter92}, working in HPSG, suggest an
analysis in which an empty head selects for the arguments of the verb and the verb itself. This
analysis is basically using the technique of argument composition that is also used for the analysis
of verbal complexes in German (see Section~\ref{sec-pred-compl}). The analysis of the example sentence in (\mex{1}) is shown in
Figure~\vref{fig-verb-head-argument-composition}.
\ea
\gll Isst er ihn?\\
     eats he him\\
\glt `Does he eat it/him.'
\z
\begin{figure}
\begin{forest}
sm edges
[{V[ \subcat \eliste ]}
  [ {\ibox{3} V[\subcat \sliste{ \ibox{1}, \ibox{2} } ]}
    [ \ibox{3} [isst;eats]] ]
  [ {V[ \subcat \sliste{ \ibox{3} } ]}
    [ \ibox{2} {NP[\type{nom}]} [ er;he ] ]
    [ {V[ \subcat \sliste{ \ibox{2}, \ibox{3} } ]}
      [ \ibox{1} {NP[\type{acc}]}[ ihn;him ] ]
      [ {V[ \subcat \sliste{ \ibox{1}, \ibox{2}, \ibox{3} } ]}
        [ \trace ]]]]]
\end{forest}
\caption{\label{fig-verb-head-argument-composition}Analysis of verb-initial sentences according to Jacobs and Netter}
\end{figure}
The trace is the head in the entire analysis: it is first combined with the accusative object and then with the subject. In a final step,
it is combined with the transitive verb in initial-position. A problem with this kind of analysis is that the verb \emph{isst} `eats', as well as \emph{er} `he' and
\emph{ihn} `him'/`it', are arguments of the verb trace in (\mex{1}).
\ea
\gll Morgen [isst [er [ihn \_]]]\\
	 tomorrow \spacebr{}eats \spacebr{}he \spacebr{}him\\
\glt `He will eat it/him tomorrow.'
\z
Since adjuncts can occur before, after or between arguments of the verb in German, one would expect that \emph{morgen} `tomorrow' can occur before the verb
\emph{isst}, since \emph{isst} is just a normal argument of the verbal trace in final position. As adjuncts do not change the categorial status of a projection, the phrase \emph{morgen isst er ihn} `tomorrow he eats him' should be able to
occur in the same positions as \emph{isst er ihn}. This is not the case, however. If we replace
\emph{isst er ihn} by \emph{morgen isst er ihn} in (\mex{1}a) the result is (\mex{1}b), which is ungrammatical.
\eal
\ex[]{
\gll Deshalb isst er ihn.\\
     therefore eats he him\\
\glt `Therefore he eats it/him.'
}
\ex[*]{
\gll Deshalb morgen isst er ihn.\\
	 therefore tomorrow eats he him\\
}
\zl
If one compares the analysis in Figure~\ref{fig-verb-head-argument-composition} with the one
suggested in this book it is clear how this problem can be avoided: in the analysis suggested in Section~\ref{sec-v1},
the verb in initial position is the head that selects for a projection of the empty verb in final
position. Since adjuncts attach to head-final verbs only, they cannot attach to \emph{isst er ihn}
`eats he him' in a normal head-adjunct structure. The only way for an adjunct to be combined with
\emph{isst er ihn} is as a filler in a V2 structure.


\section{V1 as underspecification}

\citet{Frank94} has suggested to eliminate the lexical rule for verb-placement and instead use
underspecification and model both order variants in the type system. The advantage of this would be
that one would not have to claim that one order is more basic and the other one is derived from
it. Frank's starting point is a version of the V1 lexical rule as it was developed by Tibor Kiss in
his dissertation \citep[\page 144]{Kiss93}. This version is given in (\mex{1}):
\eas
\onems{ loc  \ibox{3} \ms{ cat & \ms{ head \ibox{1} \ms{ vform & fin\\
                                                     }\\
                                  }\\
                         cont & \ibox{2} \\
                       }\\
   }
$\mapsto$\\
\onems{ loc  \ms{ cat & \ms{ head & \ibox{1}\\
                           subcat & \liste{ \ms{ loc$|$cat$|$head \ibox{1}\\
                                               nonloc$|$inher$|$dsl \{ \ibox{3} \}\\
                                             } }\\
                         }\\
                cont & \ibox{2}\\
              }\\
    nonloc$|$to-bind$|$dsl \{ \ibox{3} \}\\
}
\zs
Frank develops a type hierarchy in which there is a general type that both subsumes lexical verbs as
they are used in verb-final sentences and lexical verbs as they would be used in verb-initial
sentences. That is the result of the lexical rule application is encoded as a type. The lexical
entries for verbs would contain an underspecified description and since all feature structures in
actual models have to be maximal, it is ensured that actual instantiations of the lexical entries in
the lexicon are either verb-initial or verb-final verbs. (\mex{1}) shows the two AVMs that result if
information from the subtypes is filled in.
\eal
\ex \locv of the verb-final version of \emph{kennen} `to know':
\ms{ cat & \ms{ head   & \ms[verb]{ vform  & fin \\
                                    subcat & \sliste{ \ibox{1} NP[\type{nom}]\ind{2}, \ibox{3} NP[\type{acc}]\ind{4} }\\
                                  } \\
                subj   & \sliste{ \ibox{1} }\\
                comps  & \sliste{ \ibox{3} }\\
              } \\
    cont &  \ms[kennen]{
             arg1 & \ibox{2}\\
             arg2 & \ibox{4}\\
             }\\
}
\ex 
\begin{tabular}[t]{@{}l}
\locv of verb-initial version of \emph{kennen}:\\
\ms{ cat & \ms{ head & \ibox{3} \ms[verb]{ vform & fin \\
                                           subcat & \sliste{ \ibox{1} NP[\type{nom}]\ind{2}, \ibox{3} NP[\type{acc}]\ind{4} }\\
                                         } \\
                subj  & \eliste\\
                comps & \liste{ \ms{ loc$|$cat$|$head \ibox{1}\\
                                      nonloc$|$inher$|$dsl \{ \ldots{} \}\\
                                             } }}\\
         cont & \ms[kennen]{
             arg1 & \ibox{2}\\
             arg2 & \ibox{4}\\
             }\\
       }
\end{tabular}
\zl
The \dslv is not given in (\mex{0}b) since it is identical to (\mex{0}a). Frank assumes that there
is a separate head feature \subcat, which contains all arguments. Such a feature is also used in more
recent versions of HPSG, but it is called \argst and it is usually not a head feature.

Now, the problem with this approach, as with Kiss' original formalization of the lexical rule is
that the \contv that is contributed by the projection of the verb trace may differ from the
contribution of the verb (compare the analysis in Figure~\ref{fig-verb-movement-adjunkt-sem} on
page~\pageref{fig-verb-movement-adjunkt-sem}). This means that the semantics of the verb in initial
position has to be taken over from the element that is selected via \comps. This leaves us in the
rather unpleasant state that the argument-linking cannot be stated at a common supertype, since the
\contv of (\mex{0}a) is different from the \contv of (\mex{0}b).

It may be possible to rescue this analysis if one assumes a sort of default inheritance which allows overwriting information in subtypes \citep{LC99a}.
These kinds of defaults are however not compatible with all assumptions about the formal principles
of HPSG and in the case at hand, it would lead to a ``misuse'' anyway, as we want to express that there are
always two different \contvs, which means that we are not dealing with one general case which does not
hold true for certain exceptions.  
%% For the example discussed here, we would certainly need a special kind of
%% default treatment, which allows values to be overwritten in the syntax by another type. With the
%% default unification described in \citep{LC99a}, the necessary concessions for the \contvs of
%% subtypes cannot be made.

Another possibility to rescue the underspecification analysis comes in the form of the introduction of a feature
\textsc{cont2} for general types. The linking would be done with respect to the \textsc{cont2} value. The verb-final type would have a \contv identical to
\textsc{cont2}. The \contv of the verb-initial type would be independent of the \textsc{cont2}
value and hence conflicts would be avoided.

\citet{Frank94b} discusses the problem that adjuncts pose and notes that the adjunct problem is not shared by
approaches that assume an underspecified semantics and a modified Semantics Principle\is{Semantics Principle} which does
not project the meaning of the mother node from the daughter of the head, but rather combines lists
with the semantic contribution of all daughters (Frank uses Underspecified DRS \citep{FR95a-u}, but using MRS
as suggested in the previous Chapter would be an alternative option). The adjunct problem does not arise because the semantic
content of adjuncts is included in the VP, which is in turn combined with the verb in first
position. The verb in initial position contributes the meaning encoded in the lexicon. For this to
work, the actual relation that is contributed by the verb has to be represented outside of the
\contv that is shared with the projection of the verbal trace and it has to be ensured that only the
event variable is shared.

All these solutions fail however when one considers the coordination data discussed in
footnote~\vref{fn-koord-vm}, which is repeated here for convenience:
\ea
\gll Karl kennt und schätzt diesen Mann.\\
     Karl knows and values this man\\
\glt `Karl knows and values this man.'
\z
The example shows that it is not sufficient to develop accounts that explain the placement of single
verbs in initial position. To assume that (\mex{0}) is analyzed involving the coordination of V1 versions of lexical items like (\mex{-1}b) is not
appropriate, since the semantics of the initial verb has to be connected to the semantics of the
verb trace. In the original proposals the complete semantic representation of the verb was shared
with the trace, in approaches with underspecified semantics it would be an event variable that is
shared. If V1 versions of \emph{kennt} and \emph{schätzt} would be coordinated in the analysis of
(\mex{0}), the event variables of the two verbs would be wrongly identified. What is needed instead is an
event variable that refers to the conjoined event that includes both the \emph{kennen} and the
\emph{schätzen} event. This event variable is then present at the verb trace and adjuncts can refer
to it.

So either single verbs or arbitrarily complex coordinations of single verbs
can be placed in initial position. As was explained in the footnote referenced above, this can be
captured by a unary projection that relates single verbs or coordinations of single verbs to the
properties that are required for elements in initial position. If one uses a single underspecified
type for the description of lexical verbs that are supposed to be used either in initial or in final
position, this will never extend to complex coordinations as the one in (\mex{0}).


\section{A little bit of movement}
\label{crysmann}

In \citew[Chapter~11.5.2]{Mueller99a} and \citew{Mueller2004b}, I suggested that systematic bottom-up
processing is rather costly for grammars with empty verb heads due to the fact that any number of phrases
can be combined with empty verb heads. This follows from the fact that the valence and semantic content of
the verb trace remains unknown up to the point where its projection is combined with the verb in initial
position.
Berthold Crysmann took the grammar I developed as part of the \verbmobil-project \citep{MK2000a} and modified
it to improve it from a processing perspective \citep{Crysmann2003b}. Furthermore, he removed the unary-branching grammatical rules
which mimic the verb trace (see Chapter~\ref{chap-empty}) and -- rather than for an analysis with uniform right-branching -- opted for a left-branching 
analysis when the right verbal bracket is empty, and a right-branching one when the right bracket is occupied. The sentences in  
(\mex{1}) would have structures with different directions of branching:
\eal
\ex 
\gll {}[[[Gibt er] dem Mann] das Buch]?\\
      \hspaceThis{[[[}gives he the man the book\\
\glt `Is he going to give the man the book?'
\ex 
\gll {}[Hat [er [dem Mann [das Buch gegeben]]]]?\\
     \spacebr{}has \spacebr{}he \spacebr{}the man \spacebr{}the book given\\
\glt `Has he given the man the book?'
\zl
% Leider läßt sich die Linksversetzung wohl nicht auf einfache pronominale Referenz zurückführen.
% Das folgende Beispiel zeigt, daß Argumente, die zur linksversetzten Phrase gehören, im Satz mit
% dem Demonstrativpronomen realisiert werden können.
% \ea
% Artig finden, das kann Jan den Langweiler nicht.\footnote{
%         \citet[\page 33]{Neeleman94a}\iafdata{Neeleman} gibt ein analoges
%         niederländisches Beispiel.
% }
% \z
%}
In this sense, there is verb movement in Crysmann's analysis when there is a verbal complex in the sentence.
There is no verb movement, however, if the right verbal bracket is not filled. For similar suggestions, see
\citew*[\page 225]{KW91a} and \citew*{SRTD96a}. This avoids the processing problems that
an empty verb head brings with it, but then we are no longer able to explain the cases of supposed
multiple fronting by means of an empty verb head.

Instead of modifying the analysis of verb position, one should, for practical applications, turn to statistical components 
which predict the position of verb traces \citep{BFGKKN96a,FBCKS2003a}. If one processes the traces according
to their probability, one gets first readings quickly and dispreferred readings later. The structures which use traces
classified as `improbable' by the statistical component will be computed last.\footnote{%
	Berthold Crysmann has pointed out that the changes to the grammar he proposed
	have reduced the running time by a factor of 14, whereas the techniques described
	in \citew{BFGKKN96a} only resulted in a reduction of 46\,\% (less than a factor of 2)
	for the grammar they were using.

        However, the grammar that was used for the experiments done by \citew{BFGKKN96a} had a
        smaller coverage than grammars like the one that was developed in Saarbrücken by me, Walter
        Kasper and Berthold Crysmann and BerliGram, which is used in the CoreGram project. Therefore
        the use of a statistical component that was described by Batliner and colleagues probably
        would result in an even higher factor in the reduction of the run time. However, this would
        have to be studied experimentally. It seems unlikely though that a factor of 14 will be reached.
	
	It could then be the case that one still gets an overall slower system despite the application of the processing
	methods above than if one had modified the grammar.
}

Crysmann argues that his analysis ``leads to a more general grammar, if the
formalism does not support empty categories.'' He reduced the number of grammar rules which were needed for the
implementation of the LKB"=system \citep{Copestake2002a-unlinked} for verb movement (see Section~\ref{sec-verb-movement-LKB}) from 24 to 6.
The 24 rules were needed in the grammar for the exact reasons that empty elements were not allowed. The decision
to outlaw empty elements is, in that sense, a conscious decision on the part of the developer of the system and
is not necessarily driven by linguistic or computational necessities. As the implementation of the
analysis that is described here
in the TRALE system \citep*{MPR2002a-u,Penn2004a-u} demonstrates,\footnote{
        The grammar is freely available at \url{http://hpsg.fu-berlin.de/Fragments/b-ger-gram.html}.
        } it is most certainly possible to use an empty
head in the implementation of the verb-movement analysis that was developed in \verbmobil.

In order to describe verb movement, one empty element is required and one lexical rule. This kind of
grammar is therefore more compact than that of Crysmann, who needs six rules to achieve
this. Processing is unproblematic as empty elements are automatically removed from the grammar
before parsing while still remaining transparent for the developer of the grammar. The result of the
compilation of the grammar is identical to what developers who use grammar development systems such
as the LKB system had to produce tediously by hand. For more on empty elements, see
Chapter~\ref{chap-empty}.



\section{Special valence features for arguments forming a complex}

A special valence feature (\textsc{gov}) has been suggested by \citet*{Chung93a} for Korean\il{Korean} and
\citet*{Rentier94} for Dutch\il{Dutch} which is used for the selection of elements which form a 
verbal complex with their head. This approach was adopted by \citet{Kathol98b,Kathol2000a} and \citet{Mueller97c,Mueller99a}
for German. In \citew{Mueller2002b}, I expanded my earlier analysis to include resultative
constructions and subject and object predicatives of the \emph{jemanden für etwas/jemanden halten} `consider somebody for somebody/something' kind. 
Embedded predicates are also seen as being selected by a special valence feature (\vcomp) in this analysis.


The theory suggested here does not require this kind of additional feature. This has the advantage
that optional coherence can be analyzed as a special case of coherence as suggested by \citet{Kiss95a}.
We only need one lexical entry for verbs such as \emph{versprechen} `to promise' rather than two, which would be
needed for both coherent and incoherent constructions. 

By reducing the number of valence features, it is possible to considerably simplify the
analysis of multiple fronting. In \citew{Mueller2005d}, I suggest a lexical rule for
sentences such as (\ref{bsp-smvfb}) on page~\pageref{bsp-smvfb}, which is parallel to the verb-movement
rule in (\ref{lr-verb-movement2}) on page~\pageref{lr-verb-movement2}. 
Previous multiple fronting analyses of mine \citep{Mueller2002f,Mueller2002c}
have made use of the special valence feature \vcomp and this was the reason why the parallels of both of these verb-movement rules
remained hidden. With the feature geometry used here, cases of putative multiple fronting can be understood as
an optional variant of simple verb movement, which forms a complex. The details of the analysis are
discussed in Chapter~\ref{chapter-mult-front}.







%      <!-- Local IspellDict: en_US-w_accents -->

%% -*- coding:utf-8 -*-
%%%%%%%%%%%%%%%%%%%%%%%%%%%%%%%%%%%%%%%%%%%%%%%%%%%%%%%%%
%%   $RCSfile: grammatiktheorie-include.tex,v $
%%  $Revision: 1.13 $
%%      $Date: 2010/11/16 08:40:32 $
%%     Author: Stefan Mueller (CL Uni-Bremen)
%%    Purpose: 
%%   Language: LaTeX
%%%%%%%%%%%%%%%%%%%%%%%%%%%%%%%%%%%%%%%%%%%%%%%%%%%%%%%%%


\chapter{Empty elements}
\label{chap-empty}


%\citew[Chapter~6.2.5.1]{Mueller2002b}; \citew{Mueller2004e}



In some frameworks there is a dogma that empty elements should not be used in analyses. The argument
is that they are invisible and hence cannot be acquired from the input. I think this argumentation
is not correct in general since some empty elements correspond to visible entities and hence the
knowledge that is required to deal with such ommisions can be acquired. I distinguish between good
and bad empty elements: the good ones are the ones that correspond to visible entities in the
langauge under considerations and the bad ones are those that are semantically empty or that are
motivated by overt items in other languages. Empty expletives are suggested in GB and Minimalism and
empty functional heads like AgrO and other categories have been suggested for German on the basis of
evidence from Basque. I think for the latter examples the criticism by proponents of Construction
Grammar is fully legitimate, but I want to argue that this criticism went too far in throwing out the
good empty elements with the bath water.


In this chapter, I want to discuss the relation of grammars with empty elements to those
without empty elements. This will enable us to compare the solution with an empty verb head 
to solutions without empty elements.


\section{Empty elements in the German NP}

I want to start with a simple example and motivate the use of empty elements in the German noun
phrase. Consider the following nominal structures:

\eal
\ex 
\gll die Frauen\\
     the women\\
\ex 
\gll die klugen Frauen\\
     the smart  women\\
\ex 
\gll die klugen Frauen aus Greifswald\\
     the smart  women  from Greifswald\\
\ex 
\gll Frauen\\
     women\\
\ex\label{bsp-kluge-frauen} 
\gll kluge Frauen\\
     smart women\\
\ex 
\gll die klugen\\
     the smart\\
\glt `the smart ones'
\ex 
\gll die klugen aus Greifswald\\
     the smart  from Greifswald\\
\glt `the smart ones from Greifswald'
\ex 
\gll kluge aus Greifswald\\
     smart from Greifswald\\
\glt `smart ones from Greifswald'
\ex 
\gll kluge\\
     smart\\
\glt `smart ones'
\zl
As in English, the determiner may be omitted in the plural and with mass nouns. In addition, the
noun may be omitted in elliptical structures:
\eal
\ex 
\gll Ich kenne die klugen.\\
     I know the smart\\
\glt `I know the smart ones.'
\ex 
\gll Ich kenne kluge aus Greifswald.\\
     I know smart from Greifswald\\
\glt `I know smart ones from Greifswald.'
\ex 
\gll Ich kenne kluge.\\
     I know smart\\
\glt `I know smart ones.'
\zl
I think that the description I just gave, namely that the noun or the determiner or both may be
omitted is the most straightforward description of the phenomenon. This is what children have to
acquire. Of course the omitted elements do have a meaning. The noun may only be omitted, if the
whole nominal expression refers to somthing/somebody. If one uses the phrase \emph{kluge aus
  Greifswald} `smart from Greifswald', women, man or children or something else that can be smart
have to be mentioned in the preceeding discourse. Formally this can be represented in the following
small grammar:\footnote{
The grammar predicts that all bare determiners can function as full NPs, which is
not empirically correct:
\begin{exe}\exi{(i)}\begin{xlist}[iv.]
\ex[]{
\gll Ich helfe den Männern.\\
     I help the men\\
}
\ex[*]{
\gll Ich helfe den.\\
     I   help  the\\
}
\ex[]{
\gll Ich helfe denen.\\
     I help those\\
}
\zllast
}
\ea
\begin{tabular}[t]{@{}l@{ $\to$ }l}
NP    & Det \nbar\\
\nbar & Adj \nbar\\
\nbar & \nbar PP\\
\nbar & \trace\\
Det   & \trace\\
Det & die\\
Adj & klugen\\
\nbar & Frauen\\
\end{tabular}
\z
\nbar is an abbreviation for nouns that require a determiner and the rules \nbar $\to$
\trace{} and Det $\to$ \trace{} state that \nbar and Det may be omitted. The grammar is not
complete. Lexical entries and rules for the PP are missing. Furthermore, the grammar is not precise
enough since all inflectional information is left out. But it is sufficient for the discussion of
the advantages of empty elements.

The grammar licenses for instance the structures in Figure~\ref{abb-np}.
\begin{figure}
\oneline{\begin{forest}
sm edges
[NP
       [Det [die;the] ]
       [\nbar [Frauen;women] ] ]
\end{forest}
\hfill
\begin{forest}
sm edges
[NP
       [Det [die;the] ]
       [\nbar 
         [Adj [klugen;smart] ]
         [\nbar [Frauen;women] ] ] ]
\end{forest}
\hfill
\begin{forest}
sm edges
[NP
       [Det [die;the] ]
       [\nbar 
         [Adj [klugen;smart] ]
         [\nbar [\trace] ] ] ]
\end{forest}
\hfill
\begin{forest}
sm edges
[NP
       [Det [\trace] ]
       [\nbar 
         [Adj [kluge;smart] ]
         [\nbar [Frauen;women] ] ] ]
\end{forest}
\hfill
\begin{forest}
sm edges
[NP
       [Det [\trace] ]
       [\nbar 
         [Adj [kluge;smart] ]
         [\nbar [\trace] ] ] ]
\end{forest}}

\caption{\label{abb-np}Various nominal structures}
\end{figure}
\citet*[\page 153, Lemma~4.1]{BHPS61a} developed a procedure to transform grammars that use empty
elements into grammars without empty elements. To that end one has to insert all rules that have the
form X $\to$ \trace{} into all rules in which X appears on the right hand side. This results in new
rules, into which new empty elements may be inserted. If one does this long enough, one gets a
grammar without empty elements. For the grammar in (\mex{0}) one gets:\footnote{
  In principle the grammar in (\mex{0}) allows for completely empty NPs. This has to be blocked by
  features in the grammar. \citep[\page 81, Exercise~3]{MuellerGT-Eng1}.
}
\ea
\begin{tabular}[t]{@{}l@{ $\to$ }l}
NP    & Det \nbar\\
NP    & Det\\
NP    & \nbar\\
\nbar & Adj \nbar\\
\nbar & Adj\\
\nbar & \nbar PP\\
\nbar & PP\\
Det & die\\
Adj & klugen\\
\nbar & Frauen\\
\end{tabular}
\z
This grammar licenses among others the structures in Figure~\ref{abb-np2}.
\begin{figure}
%\oneline
{\begin{forest}
sm edges
[NP
       [Det [die;the] ]
       [\nbar [Frauen;women] ] ]
\end{forest}
\hfill
\begin{forest}
sm edges
[NP
       [Det [die;the] ]
       [\nbar 
         [Adj [klugen;smart] ]
         [\nbar [Frauen;women] ] ] ]
\end{forest}
\hfill
\begin{forest}
sm edges
[NP
       [Det [die;the] ]
       [\nbar 
         [Adj [klugen;women] ] ] ]
\end{forest}
\hfill
\begin{forest}
sm edges
[NP
       [\nbar 
         [Adj [kluge;smart] ]
         [\nbar [Frauen;women] ] ] ]
\end{forest}
\hfill
\begin{forest}
sm edges
[NP
       [\nbar 
         [Adj [kluge;smart] ] ] ]
\end{forest}}

\caption{\label{abb-np2}Verschiedene Nominalstrukturen ohne leere Elemente}
\end{figure}
The branches with empty elements were simply omitted. Comparing the two grammars it can be noted
that the grammar without empty elements contains more rules. It contains seven rules, whereas the one
with empty elements contains only three rules.
Even if one includes the lexical items for empty elements in the counting, one gets a proportion of
seven to five. In the end the grammar with empty elements is a more compact description of the
phenomenon and it covers directly what has to be acquired: the noun and the determiner can be
left unpronounced under certain circumstances.

Several attempts were made to account for noun phrases without empty elements. For inctance
\citet[\page 78]{Michaelis2006a} suggested a special lexical rule for nouns in the plural. The
plural items that are licensed by this lexical rule differ from other lexical items for nouns in not selecting for a
determiner. Thus one would have two lexical items for \emph{Frauen} `women': one of category \nbar
and one of category NP. The problem is that \emph{Frauen} `women' can be modified by \emph{kluge}
`smart' (\ref{bsp-kluge-frauen}) even when no determiner is present. If one admits adjectives to
modify NPs, phrases like (\mex{1}) cannot be excluded any longer:\footnote{
  See also \citew*[\page 265, Problem~2]{SWB2003a}.
}
\ea[*]{
\gll kluge die Frauen\\
     smart the women\\
}
\z





\section{Empty elements for verb movement}
\label{sec-verb-movement-LKB}


To demonstrate more clearly what the consequences of trace elemination are, I want to discuss
a transformation of the grammar that I suggest in this book for the German sentence structure: a grammar that uses a trace
for extraction and trace for verb movement.
\citet[p.\,92]{Kathol2000a} argues against head movement approaches for the verb position,
claiming that traceless accounts are not possible.
However, this is not correct as the following transformation of (\mex{1}) into (\mex{2}) shows:
\ea
\label{ex-grammar-eps-head}
\begin{tabular}[t]{@{}ll@{}}
\baro{v}  $\to$ \mbox{np}, v\\
v $\to$ $\epsilon$\\
\end{tabular}
\z
\ea
\label{ex-grammar-head}
\baro{v}   $\to$ \mbox{np}, v\\
\baro{v}   $\to$ \mbox{np}
\z
Instead of using a verb trace as in (\mex{0}) one can fold it into the rule. 
If we assume binary branching structures for head"=argument combination, head"=adjunct combination
and head"=cluster combination, such a trace elemination results in three new schemata in which no head daughter is present
since it was removed due to the elemination of the verbal trace.

Eliminating extraction traces from a phrase structure grammar works parallel to the
elemination of verb traces in (\mex{0}). For the grammar in (\mex{1}) we get (\mex{2}):
\ea
\begin{tabular}[t]{@{}ll@{}}
\baro{v}   $\to$ \mbox{np}, v\\
np $\to$ $\epsilon$\\
\end{tabular}
\z
\ea
\label{ex-grammar-trace-elim}
\baro{v}   $\to$ \mbox{np}, v\\
\baro{v}   $\to$ v
\z
In our HPSG grammar we get three new schemata
since arguments, adjuncts, and parts of the predicate complex can
be extracted. In the extraction case, the non"=head"=daughter is removed from the rule. The sentences
in (\mex{1}) are examples in the analysis of which these six rules will be needed:
\eal
\ex 
\gll Er$_i$ liest$_j$ t$_i$ ihn t$_j$.\\
     he     reads     {}    him {}\\
\glt `He reads it.'
\ex
\gll Oft$_i$ liest$_j$ er  ihn t$_i$ nicht t$_j$.\\
     often   reads     he  him {}    not   {}\\
\glt `He does not read it often.'
\ex
\gll Lesen$_i$ wird$_j$ er es t$_i$ müssen t$_j$.\\
     read      will     he it {}    must   {}\\      
\glt `He will have to read it.'
\zl
t$_j$ is the verb trace and t$_i$ is an extraction trace. In (\mex{0}a) the verb trace forms a constituent
with an argument, in (\mex{0}b) with an adjunct and in (\mex{0}c) with \emph{müssen}, which is a part of the predicate complex.
For these cases we need the first three rules. The second set of rules is needed for the combination with extraction
traces of respective types: In (\mex{0}a) the extracted element is an argument, in (\mex{0}b) it is an adjunct,
and in (\mex{0}c) it is a part of the predicate complex. 

If we look at grammars containg two traces we get the following situation:
\ea
\label{bsp-grammar-np-v-trace}
\begin{tabular}[t]{@{}ll@{}}
\baro{v}   $\to$ \mbox{np}, v\\
v $\to$ $\epsilon$\\
np $\to$ $\epsilon$\\
\end{tabular}
\z
Taking the rules from (\ref{ex-grammar-head}) and (\ref{ex-grammar-trace-elim}) we get:
\ea
\baro{v}   $\to$ \mbox{np}, v\\
\baro{v}   $\to$ \mbox{np}\\
\baro{v}   $\to$ v
\z
Due to the elemination of the extraction trace in (\mex{-1}) we got the rule \baro{v} $\to$ v,
but since we have the rule v $\to$ $\epsilon$ in (\mex{-1}) this means that \baro{v} can also be $\epsilon$.
\baro{v} is a new empty element that resulted from the combination of two other empty elements. To get
rid of all empty elements, this empty element has to be eliminated as well. This is done in the same
way as before. \baro{v} is removed from all righthand sides of rules were a \baro{v} appears.

For our HPSG grammar this means that we get nine new grammar rules: We have three new empty
elements that arise when a verb movement trace is directly combined with an extraction trace.
Since the extraction trace can be the non"=head daughter in the head"=argument structure (\mex{1}a),
head"=adjunct structure (\mex{1}b) or head"=cluster structure (\mex{1}c):
\eal
\ex 
\gll Er$_i$    [schläft$_j$ t$_i$ t$_j$].\\
     he        \spacebr{}sleeps\\
\glt `He sleeps.'
\ex 
\gll Jetzt$_i$ [schlaf$_j$ t$_i$ t$_j$]!\\
     now       \spacebr{}sleep\\
\glt `Sleep now!'
\ex 
\gll Geschlafen$_i$ [wird$_j$ t$_i$ t$_j$]!\\
     slept \spacebr{}is\\
\glt `Sleep!'
\zl
Due to these new three traces we need three aditional rules where each of the new traces is folded
into the rule instead of the argument daughter in the head"=argument schema.

For the examples in (\mex{1}) and (\mex{2}) we need six new rules, since the trace combinations
can function as heads in head"=argument structures (\mex{1}) and in head"=adjunct structures (\mex{2}):
\eal
\ex 
\gll Den Aufsatz$_i$ liest$_j$ [er t$_i$ t$_j$].\\
     the paper       reads     \spacebr{}he\\
\ex 
\gll Oft$_i$ liest$_j$ er [ihn t$_i$ t$_j$].\\
     often reads he \spacebr{}it\\
\glt `He reads it often.'
\ex 
\gll Lesen$_i$ wird$_j$ er [ihn t$_i$ t$_j$].\\
     read      will     he \spacebr{}it\\
\glt `He will read it.'
\zl
\eal
\ex 
\gll Den Aufsatz$_i$ liest$_j$ er [nicht t$_i$ t$_j$].\\
     the paper       reads     he \spacebr{}not\\
\glt `He does not read the paper.'
\ex 
\gll Oft$_i$ liest$_j$ er ihn [nicht t$_i$ t$_j$].\\
     often   reads     he it \spacebr{}not\\
\glt `He does not read it often.'
\ex 
\gll Lesen$_i$ wird$_j$ er ihn [nicht t$_i$ t$_j$].\\
     read      will    he it  \spacebr{}not\\
\zl


I applied this technique of epsilon elimination to the HPSG grammar that
was developed for the \verbmobil system \citep{MK2000a}, 
but there are processing systems, like Trale \citep*{MPR2002a-u},
that do such grammar conversion automatically \citep{Penn99b}.
The grammar in (\ref{bsp-grammar-np-v-trace}) and the corresponding
HPSG equivalent directly encode the claim that the np and v can be omitted, while
this information is only implicitly contained in the rules we get from specifying an
epsilon free grammar by hand. 
The same would be true for a grammar that accounts for copulaless sentences by
stipulating several constructions for questions and declarative sentences with
a missing finite verb.

Using grammar transformations to get epsilon"=free linguistic descriptions can yield rather
complicated rules that do not capture the facts in an insightful way. This is
especially true in cases where two or more empty elements are eliminated by
grammar transformation. While this is not a problem for computational algorithms
that deal with formally specified grammars, it is a problem for linguistic specifications.
For more discussion see Müller (\citeyear[Chapter~6.2.5.1]{Mueller2002b};
\citeyear{Mueller2005c}; \citeyear{Mueller2004e}; \citeyear[Chapter~19]{MuellerGT-Eng1}).



%% A reviewer remarked that empty elements are against the spirit of HPSG. The questions
%% to be asked is: ``What is the spirit of HPSG?'', ``Who defines it?'', ``Does it change?''.
%% %
%% \citet{Bender2000a} argued in her thesis for an empty copula in
%% African American Vernacular English. This empty copula is also discussed in
%% \citew*[p.\,464]{SWB2003a-unlinked}. Once we admit a single empty element in our grammars,
%% we cannot say that empty elements are ``not in the spirit of HPSG''.


\section{Conclusion}

This brief chapter showed that sometimes grammars that use empty elements can capture insights more
directly than grammars from which the empty elements were eliminated.


%      <!-- Local IspellDict: en_US-w_accents -->

%% -*- coding:utf-8 -*-
\chapter{Conclusion}


The discussion in Chapter~\ref{chap-alternatives} showed that approaches that rely on surface patterns only have
problems with accounting for the data in German. First there are elliptical sentences (Topic Drop),
in which the \vf is not filled and which are main declarative clauses nevertheless. Second there is the
big problem of apparently multiple frontings which runs afoul the V2 property of German. I suggested
using an empty head that is related to a verb in the remainder of the sentence. This captures the
fact that the elments in the \vf have to depend on the same head and preserves the generalization
that German is a V2 language. I showed in Chapter~\ref{chap-empty} that grammars with empty elements
may be more compact and capture the insights more directly than grammars without empty elements.

The theory that is represented in this book is implemented in the TRALE system
\citep*{MPR2002a-u,Penn2004a-u}. The grammar was developed in 2003 and is now part of the grammar
that is maintained in the CoreGram project \citep{MuellerCoreGramBrief,MuellerCoreGram}. It can
be downloaded at \url{http://hpsg.fu-berlin.de/Fragments/Berligram/} and is also distributed with the Grammix Virtual Machine \citep{MuellerGrammix}. For a list of positive and negative example sentences see Appendix~\ref{chap-tsdb-examples}.

%The fragment developed in \citew{Mueller2005c} is available in electronic form 
%at \url{http://hpsg.fu-berlin.de/Fragments/mehr-vf.html}. The version that is presented in this book is part of the grammar developed in the CoreGram project \citep{MuellerCoreGramBrief,MuellerCoreGram} .



%% I have shown that a theory that requires positions to be filled for certain clause types
%% is problematic. It cannot cope with elliptic patterns where no finite verb is present or
%% where an element in the \textit{Vorfeld} is omitted. The only possibility to get the data described in
%% such models is to stipulate several constructions that correspond to the observable patterns.
%% The number of constructions that had to be stipulated in a construction-based approach
%% would be higher than the number of empty heads that are needed in more traditional approaches
%% and generalizations regarding combinations of syntactic material would be missed.

%% As an alternative, I suggested that clause types are determined with reference to features that get
%% instantiated in immediate dominance schemata. Furthermore I provided an HPSG analysis for Topic Drop in German. 

%% The discussion showed that an entirely surface-based syntax cannot capture regularities that can be
%% observed in the data in an insightful way. I therefore suggest returning to more traditional
%% approaches to German clausal syntax.


\appendix

%% -*- coding:utf-8 -*-

\chapter{List of phrases covered/""rejected by the grammar}
\label{chap-tsdb-examples}

\section*{Verb last order}

\ea[]{
\gll dass der Mann der Frau das Buch gibt\\
    that the man the woman the book gives\\
\glt `that the man gives the woman the book'
}
\z
\section*{Verb initial order}

\ea[]{
\gll Gibt der Mann der Frau das Buch?\\
    gives the man the woman the book\\
\glt `Does the man give the woman the book?'
}
\z
\section*{Scrambling}

\ea[]{
\gll dass der Mann das Buch der Frau gibt\\
    that the man the book the woman gives\\
\glt `that the man gives the woman the book'
}
\z
\ea[]{
\gll dass das Buch der Mann der Frau gibt\\
    that the book the man the woman gives\\
\glt `that the man gives the woman the book'
}
\z
\ea[]{
\gll dass das Buch der Frau der Mann gibt\\
    that the book the woman the man gives\\
\glt `that the man gives the woman the book'
}
\z
\ea[]{
\gll dass der Frau der Mann das Buch gibt\\
    that the woman the man the book gives\\
\glt `that the man gives the woman the book'
}
\z
\ea[]{
\gll dass der Frau das Buch der Mann gibt\\
    that the woman the book the man gives\\
\glt `that the man gives the woman the book'
}
\z
\section*{Adjunct position}

\ea[]{
\gll dass jetzt der Mann das Buch der Frau gibt\\
    that now the man the book the woman gives\\
\glt `that the man gives the woman the book now'
}
\z
\ea[]{
\gll dass der Mann jetzt das Buch der Frau gibt\\
    that the man now the book the woman gives\\
\glt `that the man gives the woman the book now'
}
\z
\ea[]{
\gll dass der Mann das Buch jetzt der Frau gibt\\
    that the man the book now the woman gives\\
\glt `that the man gives the woman the book now'
}
\z
\ea[]{
\gll dass der Mann das Buch der Frau jetzt gibt\\
    that the man the book the woman now gives\\
\glt `that the man gives the woman the book now'
}
\z
\section*{V2}

\ea[]{
\gll Der Mann gibt der Frau das Buch.\\
    the man gives the woman the book\\
\glt `The man gives the woman the book.'
}
\z
\ea[]{
\gll Der Frau gibt der Mann das Buch.\\
    the woman gives the man the book\\
\glt `The man gives the woman the book.'
}
\z
\ea[]{
\gll Das Buch gibt der Mann der Frau.\\
    the book gives the man the woman\\
\glt `The man gives the woman the book.'
}
\z
\ea[]{
\gll Jetzt gibt der Mann der Frau das Buch.\\
    now gives the man the woman the book\\
\glt `The man gives the woman the book now.'
}
\z
\section*{V2 + scope}

\ea[]{
\gll Oft liest er das Buch nicht.\\
    often reads he the book not\\
\glt `It is often that he does not read the book. or It is not the case that he reads the book often.'
}
\z
\section*{Verbal complex}

\ea[]{
\gll dass er das Buch wird lesen müssen\\
    that he the book will read must\\
\glt `that he will be obliged to read the book'
}
\z
\ea[]{
\gll dass es ihm ein Mann zu lesen versprochen hat\\
    that it him a man to read promised has\\
\glt `that a man promised him to read it'
}
\z
\section*{Partial verb phrase fronting}

\ea[]{
\gll Der Frau geben wird er das Buch.\\
    the woman give will he the book\\
\glt `He will give the woman the book.'
}
\z
\ea[]{
\gll Das Buch geben wird er der Frau.\\
    the book give will he the woman\\
\glt `He will give the woman the book.'
}
\z
\ea[]{
\gll Geben wird er der Frau das Buch.\\
    give will he the woman the book\\
\glt `He will give the woman the book.'
}
\z
\ea[]{
\gll Der Frau geben wird er das Buch müssen.\\
    the woman give will he the book must\\
\glt `He will be obliged to give the woman the book.'
}
\z
\ea[]{
\gll Das Buch geben wird er der Frau müssen.\\
    the book give will he the woman must\\
\glt `He will be obliged to give the woman the book.'
}
\z
\ea[]{
\gll Geben wird er der Frau das Buch müssen.\\
    give will he the woman the book must\\
\glt `He will be obliged to give the woman the book.'
}
\z
\section*{Multiple frontings}

\ea[]{
\gll Der Frau das Buch gibt er nicht.\\
    the woman the book gives he not\\
\glt `He doesn't give the woman the book.'
}
\z
\ea[]{
\gll Dem Saft eine kräftige Farbe geben Blutorangen.\\
    the juice a strong color give blood.oranges\\
\glt `Blood oranges give the jiuce a strong color.'
}
\z
\ea[]{
\gll Der Frau den Aufsatz muss er geben.\\
    the woman the paper must he give\\
\glt `He has to give the woman the paper.'
}
\z
\ea[*]{
\gll Der Frau der Aufsatz gibt er.\\
    the woman the paper gives he\\
}
\z
\ea[*]{
\gll Der Frau der Aufsatz muss er geben.\\
    the woman the paper must he give\\
}
\z




%\input{gt-index-cross}

%%%%%%%%%%%%%%%%%%%%%%%%%%%%%%%%%%%%%%%%%%%%%%%%%%%%
%%%                                              %%%
%%%             Backmatter                       %%%
%%%                                              %%%
%%%%%%%%%%%%%%%%%%%%%%%%%%%%%%%%%%%%%%%%%%%%%%%%%%%%
\backmatter

%\bibliography{bib-abbr,biblio,crossrefs}
\printbibliography[heading=references] 
\cleardoublepage

\small
   

\phantomsection%this allows hyperlink in ToC to work
\addcontentsline{toc}{chapter}{Index} 
\addcontentsline{toc}{section}{Name index}
\ohead{Name index}
%with biblatex
%\printindex
%without it 
\printindex 
  
\phantomsection%this allows hyperlink in ToC to work
\addcontentsline{toc}{section}{Language index}
\ohead{Language index} 
\printindex[lan] 
  
\phantomsection%this allows hyperlink in ToC to work
\addcontentsline{toc}{section}{Subject index}
\ohead{Subject index}
\printindex[sbj]


\end{document} 

%%%%%%%%%%%%%%%%%%%%%%%%%%%%%%%%%%%%%%%%%%%%%%%%%%%%
%%%                                              %%%
%%%                  END                         %%%
%%%                                              %%%
%%%%%%%%%%%%%%%%%%%%%%%%%%%%%%%%%%%%%%%%%%%%%%%%%%%%





%      <!-- Local IspellDict: en_US-w_accents -->
